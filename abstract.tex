
\begin{abstract}
   An uncontested plea allows criminal defendants to self-convict without requiring the state to prove its case against them. Uncontested pleas may be inculpatory, exculpatory, or non-inculpatory. Guilty pleas are inculpatory uncontested pleas. When a defendant pleads guilty sincerely, they formally take responsibility for the offence and accept the consequences. Exculpatory and non-inculpatory uncontested pleas include best-interest pleas like \textit{Alford} and \textit{nolo contendere} pleas, respectively. When a defendant enters one of these pleas, they agree to self-convict without formally taking responsibility for the offence. Statutory language formally forbids exculpatory and non-inculpatory uncontested pleas like \textit{nolo contendere} pleas in Canada.
   
    In this thesis, I argue that the legal and ethical objections to these pleas and plea bargaining generally in Canada are largely misplaced. \textit{Nolo contendere} pleas open new avenues of plea bargaining for defendants and prosecutors to explore, creating new opportunities for certainty, factual accuracy, agency, and mutual advantage in otherwise highly adversarial proceedings. Although formally forbidden, defendants may still enter \textit{nolo contendere} informally and surreptitiously. I conclude by arguing that these pleas be formalized and suggesting ways to do so.
\end{abstract}