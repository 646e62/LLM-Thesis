
\begin{abstract}
   A no-contest plea is a way for an offender to self-convict without requiring the state to prove its case. No contest pleas may be inculpatory, exculpatory, or non-inculpatory. Inculpatory no-contest pleas are generally limited to guilty pleas. When a defendant pleads guilty, they may admit their actual culpability in the offence pleaded to and accept the consequences of doing so. Exculpatory and non-inculpatory no-contest pleas include best-interest pleas and \textit{nolo contendere} pleas, respectively. When a defendant enters one of these pleas, they agree to self-convict without taking responsibility for the offence.
   
   Although statutory language formally forbids \textit{nolo contendere} pleas in Canada, defendants may still enter them informally and surreptitiously. In this thesis, I argue that the legal and ethical objections to these pleas and plea bargaining generally in Canada are largely misplaced. \textit{Nolo contendere} pleas open new avenues of plea bargaining for defendants and prosecutors to explore, creating new opportunities for certainty, factual accuracy, agency, and mutual advantage in otherwise highly adversarial proceedings. I conclude by recommending ways these pleas could be adapted and used in the Canadian criminal context.
\end{abstract}