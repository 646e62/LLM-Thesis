\chapter{Introduction}

\section{Overview}

People accused of a criminal offence in Canada are presumed innocent until the state satisfies a judge or jury that they are guilty beyond a reasonable doubt. However, as long as the common law has given defendants\footnote{Here and throughout this thesis, I use the term ``defendants" to refer to persons charged with criminal offences rather than the term ``accused. This decision was entirely phonaesthetic. Although there is an argument to be made in favour of sticking with the governing statutory parlance, the clumsiness of the word ``accused" — especially apparent when dealing with phrases like ``multiple co-accuseds" — is justification enough.} the ability to contest their charges at trial, it has also allowed them to admit their wrongdoing and invite the court to convict them. Historically, this was done through a confession, a practice that eventually evolved into the modern guilty plea. However, a guilty plea is only one of several ways the common law has evolved to allow criminal defendants to forego a trial and agree to self-conviction. Throughout this thesis, I will refer to these pleas and plea procedures\footnote{As will be explained in greater detail below, some Canadian courts have adopted a process they refer to as a \textit{nolo contendere} procedure. To the extent that this procedure allows a criminal defendant to self-convict, it shares a key characteristic with guilty, \textit{nolo contendere}, and best-interest pleas and will be defined and discussed alongside them below.} collectively as ``no-contest pleas."\footnote{This term will be defined in detail below in §2.1: Types of no-contest pleas.} 

A defendant who enters a no-contest plea gives up their right to a trial. Because defendants are presumed innocent, and the burden of proof is on the state, common law courts insist that anyone giving up that right must do so \textit{knowingly} and \textit{voluntarily}. Canadian judges are also expected to ensure that such are entered \textit{unequivocally}.\footnote{Support for this requirement is often found in the influential dissent in \textit{Adgey v R}, 1973 CanLII 37 (SCC), [1975] 2 SCR 426. In 2002, Parliament codified the so-called ``plea inquiry" as section 606(1.1) of the \textit{Criminal Code}. In 2002, Parliament codified both the voluntariness and knowledge requirements as sections 606(1.1)(a) \& (b), respectively. Interestingly, Parliament did not appear to include the ``unequivocal" requirement in \textit{Criminal Code} s 606(1.1). \hl{The implications of this will be explored in greater detail below.}}

As self-convictions evolved from common-law confessions to codified criminal procedures, the practice of plea bargaining emerged and evolved alongside it. Plea bargaining occurs when a defendant enters a no-contest plea in exchange for some benefit, usually a reduced sentence, a plea to a lesser charge, having a charge withdrawn, or some other consideration. Although it is unclear exactly when the practice began, plea bargaining became standard throughout the United States and Canada during the 20\textsuperscript{th} century. The fundamental assumption underlying plea bargaining is that prosecutors and defendants risk losing something by going to trial but are guaranteed to gain something by agreeing to a conviction. Prosecutors avoid losing their cases outright at trial by securing a conviction. In order to convince defendants to agree to be convicted, the prosecutor will offer the defendant one or more inducements to plead. This arrangement is known as a ``plea deal."  On the other hand, defendants may mitigate their potential losses at trial by agreeing to a conviction on more favourable terms than they would otherwise receive. 

Today, plea bargains are arguably the dominant reason why far more criminal cases resolve without a trial than with one.\footnote{Some estimates place the percentage of guilty pleas as high as 97\%. See (the article that says this).} These plea deals require that prosecutors offer a defendant some inducement to plead guilty and are most effective when contrasted with more significant penalties upon conviction after trial. This practice has been heralded as a necessary component of a functioning modern justice system\footnote{See \textit{R v Anthony-Cook}, 2016 SCC 43 at paras 25, 46.} and derided as an efficiency-obsessed mechanism that inevitably results in wrongful convictions and involuntary pleas.\footnote{See Alschuler, Ireland.} Proponents of plea bargaining often note the immense time and resources that modern criminal trials require and the opportunities plea bargaining affords for defendants to min/max their "economic" risks and benefits. On the other hand, critics of plea bargaining argue that this stick-and-carrot-style practice puts improper pressure on all defendants to forfeit their right to trial\footnote{See An Offer You Can't Refuse.} and almost certainly results in a surfeit of wrongful convictions.\footnote{See Alschuler.}

Within this plea bargaining context, an obscure common-law plea known as \textit{nolo contendere} experienced a revival in the United States in the early 20\textsuperscript{th} century, having fallen into disuse in England some two centuries earlier. \textit{Nolo contendere} is a Latin phrase that translates to "I do not wish to contest." A defendant who enters a \textit{nolo contendere} plea invites the court to convict them but neither admits nor denies their involvement. They accept the consequences of a criminal conviction without taking responsibility for the underlying events. At common-law, and in most states, defendants who enter a \textit{nolo contendere} plea are usually not prevented from contesting subsequent civil suits stemming from the same allegations, a fact that likely contributed to the plea's early popularity in tax and corporate prosecutions.\footnote{See \hl{that early article discussing \textit{nolo contendere} in tax prosecutions.}} Typically, these same defendants may not have evidence of their \textit{nolo contendere} plea admitted at subsequent proceedings.\footnote{This is often true even in jurisdictions that do not otherwise recognize or allow \textit{nolo contendere} pleas. For example, North Dakota and Washington do not allow admitting evidence of \textit{nolo contendere} pleas in subsequent proceedings, despite not allowing local defendants to enter them. Other states, like Arkansas, take a very expansive view of subsequent inadmissibility, despite not otherwise recognizing \textit{nolo contendere} pleas.} In 1926, the US Supreme Court decided \textit{Hudson et al. v United States}.\footnote{\textit{Hudson et al v United States}, 272 US 451.} This case established that \textit{nolo contendere} pleas could be broadly employed and arguably led to the widespread popularity and eventual codification of these pleas across the United States.

Less than half a century after \textit{Hudson}, the American embrace of \textit{nolo contendere} led to the emergence and acceptance of the related (and controversial) ``\textit{Alford} plea." In 1970, the United States Supreme Court decided \textit{North Carolina v Alford}. In this case, a defendant entered a guilty plea to second-degree murder while claiming to be innocent of the offence. Alford had initially been charged with first-degree murder but agreed to a plea deal to avoid the death penalty. On appeal, Alford argued that his decision to plead guilty was involuntary, having been primarily motivated to do so out of fear. The majority in \textit{Alford} upheld his conviction, having found that the defendant had entered his plea knowingly and voluntarily and that there was a solid factual basis for the charge. That decision led to the proliferation of ``best-interest pleas"\footnote{Where a defendant pleads guilty despite actively protesting their innocence, it is generally safe to assume that they are doing so because it is in their best interests to do so. Were it not, the defendant would presumably plead not guilty and set their matter down for a trial.} in the United States, wherein a defendant is permitted to plead guilty to advance their interests without being required to admit their guilt on record. 

Like guilty pleas, \textit{nolo contendere} and best-interest pleas are both forms of self-conviction. Because of this, where these pleas are available, they may form part of the plea bargaining process and become an agreed-upon term in a plea deal. Both \textit{nolo contendere} and best-interest pleas provide defendants with certain advantages that guilty, and not guilty pleas do not. A defendant who enters a \textit{nolo contendere} plea may accept a plea bargain without having their plea used against them in subsequent civil proceedings. Meanwhile, a defendant who enters a best-interest plea may take advantage of a plea deal without having to compromise their conscience.

Unlike guilty pleas, which a defendant may reasonably enter as a matter of conscience, \textit{nolo contendere} and best-interest pleas do not involve any form of confession. As a result, these pleas may be more appealing to defendants who would otherwise be inclined to set their matter down for trial while having minimal appeal for defendants that want to self-convict as a matter of conscience. Nevertheless, because these defendants do not make any formal confession of guilt, the usual controversies that attend plea deals are amplified when those deals include these no-contest pleas. 

\section{Issues and academic treatment thus far}

The broad question I look to answer in this thesis is whether Canadian criminal law should expand to include \textit{nolo contendere} and best-interest pleas. However, no-contest pleas are an intrinsic part of the controversial practice of plea bargaining. As a result, plea bargaining must be considered when asking whether Canadian criminal law should include these particular no-contest pleas. Plea bargaining is widely discussed and controversial in American and Canadian academic circles. 

American academics have extensively discussed the history, use, and characteristics of \textit{nolo contendere} and best-interest pleas for over a century. However, in Canada, the academic conversation around plea bargaining rarely discusses these pleas. This lack of discussion on this subject results from our distinct common law traditions. The \textit{nolo contendere} plea emerged in 15\textsuperscript{th} century England but was last known to have been used there in 1702.\footnote{See \textit{R v Templeman}, 1 Salk 55 (QB 1702).} At the time, the British portion of ``Canada" consisted of Newfoundland and the parts of Rupert's Land that had been explored by the Hudson's Bay Company thus far. By the time \textit{nolo contendere} pleas re-emerged in force in the early 20\textsuperscript{th} century, Canada's criminal law was wholly developed and codified in the plea's absence.

The re-emergence of \textit{nolo contendere} pleas and the development of best-interest pleas in the United States did not appear to generate much interest in Canada either. My research into Canadian academic sources discussing these pleas in-depth produced only one brief article that examined \textit{nolo contendere} pleas and advocated for their incorporation into Canadian criminal law. Published in 1997, Frederick L. Forsyth's, ``A Plea for Nolo Contendere in the Canadian Criminal Justice System" argues that because \textit{nolo contendere} pleas offer defendants increased access to speedy charge resolution, Canadian criminal law ought to appropriate them. Since that article was published, I have been unable to find any substantive treatment of the subject. Since that article was published, Canadian appellate courts have begun to explicitly authorize a ``\textit{nolo contendere} procedure" that allows a defendant to self-convict after entering a not guilty plea. The Canadian academic literature on the subject is both sparse and incomplete.

\section{Importance of the current research}

As Forsyth's article's title alludes, Canadian criminal law does not \textit{formally} recognize \textit{nolo contendere} pleas. The \textit{Criminal Code of Canada} only permits defendants to enter a guilty plea, a not guilty plea, or one of the special (and rarely encountered) "double jeopardy" pleas listed in section 607. No other pleas are permitted.\footnote{See \textit{Criminal Code} s 606(1).} This provision formally precludes a defendant from entering a \textit{nolo contendere} plea in response to a criminal charge. Similarly, the common-law requirement that guilty pleas must be \textit{unequivocally} entered ostensibly precludes defendants from entering best-interest pleas like the one entered in \textit{Alford}.

But in the past decade, several Canadian appellate decisions have begun to authorize an \textit{informal} \textit{nolo contendere} plea, usually referred to as a ``\textit{nolo contendere} procedure." Others have signalled that a defendant's protestations of innocence after entering a guilty plea, no matter how earnest, are no guarantee that they will be allowed to withdraw it. Ambiguities in the statutory language and intersections between a few particular provisions have enabled Canadian courts to allow defendants to self-convict without formally admitting fault or contesting their matters. Similarly, appellate procedures and review standards have upheld self-convictions followed by subsequent protestations of innocence. These workarounds fulfill needs that a naive application of the statutory framework cannot but may cause unanticipated problems that the current statutory framework cannot resolve. As a result, a closer analysis is warranted.

Because all types of no-contest pleas and plea bargaining are inextricably linked, this thesis encounters and tackles the subject as a matter of necessity. Most discussions about plea bargaining accept a dichotomy where prosecutors, courts, public defenders and other similarly situated justice system participants exchange a defendant's right to fair and accurate proceedings for crass administrative expediency. Under this dichotomy, plea bargaining is justified and criticized for enticing defendants to plead guilty to ensure that the court system can continue to operate. Both sides of the debate generally accept that a trial by jury would be the ideal way to dispose of a case but disagree about how vital that ideal is versus the effective administration of justice. On the other hand, my approach sees a jury trial as an uncertain and unreliable method of truth-seeking, especially when contrasted with the truth and certainty gained through a well-conducted plea bargain. This observation has impacted how I answer whether and why alternative no-contest pleas should be allowed.

\section{Research questions, road map, and methodology}

In this thesis, I broadly ask whether Canadian criminal law should explicitly permit \textit{nolo contendere} and best-interest pleas. I answer this question by exploring the history and examining the nature of no-contest pleas, determining which no-contest pleas are compatible with Canadian criminal law and why, and addressing the legitimate concerns that these pleas and plea bargaining generally raise.

In Chapter 2, I define what no-contest pleas are and identify the role that they play in the plea bargaining process. From here, I conduct a high-level survey of the history of guilty, \textit{nolo contendere}, and best-interest pleas in England, the United States, and Canada to identify their convergences and divergences and to explicate the relationships between them. 

Chapter 3 then examines whether and to what extent no-contest pleas are compatible with Canadian criminal law. I identify the relevant statutory provisions in the \textit{Criminal Code} and examine each type of no-contest plea against it. Here I explore the informal \textit{nolo contendere} procedure that has developed in Canada and conclude by reviewing the uncertain legal standing of best-interest pleas.

Finally, in Chapter 4, I ask whether Canadian criminal law should expand and adapt to embrace no-contest pleas more fully and why. I first address the sort of cognitive dissonance that no-contest pleas can generate while demonstrating that the pleas are, in fact, internally coherent. Next, I look at the significant objections to plea bargaining and the concerns that specifically emerge with \textit{nolo contedere} and best-interest pleas. I then question the utility of the substantive/procedural law dichotomy, followed by a discussion of the advantages of increasing access to plea bargaining to defendants. I conclude this final substantial chapter by briefly discussing the benefits of formalization and codification.