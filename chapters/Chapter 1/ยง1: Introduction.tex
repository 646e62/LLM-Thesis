\chapter{Introduction}

\section{Overview}

Criminal defendants\footnote{Here and throughout this thesis, I use the term ``defendants" to refer to persons charged with criminal offences rather than the term ``accused. This decision was entirely phonaesthetic. Although there is an argument to be made in favour of sticking with the governing statutory parlance, the clumsiness of the word ``accused" — especially apparent when dealing with phrases like ``multiple co-accuseds" — is justification enough.} in Canada are presumed innocent until a prosecutor proves that they are guilty beyond a reasonable doubt. However the common law also allows them to admit confess invite a conviction. Historically, this confession evolved into the modern guilty plea. Today, a guilty plea is one of several ways criminal defendants may forego a trial and agree to self-conviction. Throughout this thesis, I will refer to these pleas and plea procedures\footnote{As will be explained in greater detail below, some Canadian courts have adopted a process they refer to as a \textit{nolo contendere} procedure. To the extent that this procedure allows a criminal defendant to self-convict, it shares a key characteristic with guilty, \textit{nolo contendere}, and best-interest pleas and will be defined and discussed alongside them below.} collectively as ``no-contest pleas."\footnote{This term will be defined in detail below in §2.1: Types of no-contest pleas.} 

A defendant who enters a no-contest plea gives up their right to a trial. Because they are presumed innocent, and the burden of proof is on the state, defendants waiving that right must do so \textit{knowingly}, \textit{voluntarily}, and \textit{unequivocally}.\footnote{See \textit{Adgey v R}, 1973 CanLII 37 (SCC), [1975] 2 SCR 426; \textit{R v Wong}, 2018 SCC 25, [2018] 1 SCR 696.}

As self-convictions evolved from common-law confessions to codified criminal procedures, the practice of plea bargaining emerged and evolved alongside it. Plea bargaining occurs when a defendant enters a no-contest plea in exchange for some benefit, usually a reduced sentence, a plea to a lesser charge, or some other consideration. The fundamental assumption underlying plea bargaining is that prosecutors and defendants risk losing something by going to trial and are guaranteed to gain something by agreeing to a conviction. Prosecutors avoid losing their cases outright by securing a conviction, while defendants mitigate their potential losses by agreeing to a conviction on favourable terms.

Today, plea bargains are arguably the dominant reason why far more criminal cases resolve without a trial than with one.\footnote{Some estimates place the percentage of guilty pleas as high as 97\%. See (the article that says this).} The practice has been heralded as a necessary component of a functioning modern justice system\footnote{See \textit{R v Anthony-Cook}, 2016 SCC 43 at paras 25, 46.} and derided as an efficiency-obsessed mechanism that inevitably results in wrongful convictions and involuntary pleas.\footnote{See Alschuler, Ireland.} Proponents of plea bargaining often note the immense time and resources that modern criminal trials require and the opportunities plea bargaining affords for defendants to minimax their risks and benefits. On the other hand, critics of plea bargaining argue that this stick-and-carrot-style practice improperly pressures defendants to forego trials\footnote{See An Offer You Can't Refuse.} and increases the likelihood of wrongful convictions.\footnote{See Alschuler.}

Within this plea bargaining context, an obscure common-law plea known as \textit{nolo contendere} experienced a revival in the United States in the early 20\textsuperscript{th} century, having fallen into disuse in England some two centuries earlier. \textit{Nolo contendere} is a Latin phrase meaning "I do not wish to contest." A defendant who enters a \textit{nolo contendere} plea accepts the consequences of a criminal conviction without taking responsibility for the underlying events. At common-law, and in most states, \textit{nolo contendere} pleas are usually inadmissible in subsequent actions stemming from the same allegations, a fact that likely contributed to the plea's early popularity in tax and corporate prosecutions.\footnote{See \hl{that early article discussing \textit{nolo contendere} in tax prosecutions.} This is often true even in jurisdictions that do not otherwise recognize or allow \textit{nolo contendere} pleas. For example, North Dakota and Washington do not allow admitting evidence of \textit{nolo contendere} pleas in subsequent proceedings, despite not allowing local defendants to enter them. Other states, like Arkansas, take a very expansive view of subsequent inadmissibility, despite not otherwise recognizing \textit{nolo contendere} pleas.} In 1926, the US Supreme Court decided \textit{Hudson et al. v United States},\footnote{\textit{Hudson et al. v United States}, 272 US 451.} which led to the widespread popularity and eventual codification of these pleas across the United States.\footnote{Discussed below.}

Like guilty pleas, \textit{nolo contendere} are self-convictions. Because of this, where these pleas are available, they may form part of the plea bargaining process and become an agreed-upon term in a plea deal. \textit{Nolo contendere} pleas provide defendants with certain advantages that guilty and not guilty pleas do not. A defendant who enters a \textit{nolo contendere} plea may accept a plea bargain without having their plea used against them in subsequent civil proceedings or take advantage of a plea deal without compromising their conscience.

Unlike guilty pleas, which a defendant may reasonably enter as a matter of conscience, \textit{nolo contendere} pleas do not involve any form of confession. As a result, these pleas may be more appealing to defendants who would otherwise be inclined to set their matter down for trial while having minimal appeal for defendants that want to self-convict as a matter of conscience.

\section{Issues and academic treatment thus far}

The broad question I look to answer in this thesis is whether Canadian criminal law should expand to include \textit{nolo contendere} pleas. However, no-contest pleas are an intrinsic part of the controversial practice of plea bargaining. As a result, plea bargaining must be considered when asking whether Canadian criminal law should include these particular no-contest pleas. Plea bargaining is widely discussed and controversial in American and Canadian academic circles. 

American academics have extensively discussed the history, use, and characteristics of \textit{nolo contendere} pleas for over a century. However, in Canada, the academic conversation around plea bargaining rarely discusses this plea. This lack of discussion on this subject results from our distinct common law traditions. The \textit{nolo contendere} plea emerged in 15\textsuperscript{th} century England but was last known to have been used there in 1702.\footnote{See \textit{R v Templeman}, 1 Salk 55 (QB 1702).} At the time, ``Canada" consisted of Newfoundland and the parts of Rupert's Land explored by the Hudson's Bay Company thus far. By the time \textit{nolo contendere} pleas re-emerged in force in the early 20\textsuperscript{th} century, Canada's criminal law had developed without it.

\textit{Nolo contendere}'s re-emergence did not appear to generate much interest in Canada either. My research into Canadian academic sources discussing these pleas in-depth produced only one brief article that examined \textit{nolo contendere} pleas and advocated for their incorporation into Canadian criminal law. Published in 1997, Frederick L. Forsyth's, ``A Plea for Nolo Contendere in the Canadian Criminal Justice System" argues that because \textit{nolo contendere} pleas offer defendants increased access to speedy charge resolution, Canadian criminal law ought to appropriate them. Since that article was published, I have been unable to find any substantive treatment of the subject, despite Canadian appellate courts having begun to explicitly authorize a ``\textit{nolo contendere} procedure." The Canadian academic literature on the subject is both sparse and incomplete.

\section{Importance of the current research}

As Forsyth's article's title alludes, Canadian criminal law does not \textit{formally} recognize \textit{nolo contendere} pleas. The \textit{Criminal Code of Canada} only permits defendants to enter a guilty plea, a not guilty plea, or one of the special (and rarely encountered) "double jeopardy" pleas listed in section 607. But in the past decade, several Canadian appellate decisions have begun to authorize an \textit{informal} \textit{nolo contendere} plea, usually referred to as a ``\textit{nolo contendere} procedure." This procedure's popularity suggests that it fulfills needs that naive applications of the statutory framework cannot. However, the procedure may also cause unanticipated problems that the current statutory framework cannot resolve. As a result, a closer analysis is warranted.

Because no-contest pleas and plea bargaining are inextricably linked, this thesis considers both as a matter of necessity. Most discussions about plea bargaining accept a dichotomy where prosecutors, courts, public defenders and other similarly situated justice system participants exchange a defendant's right to fair and accurate proceedings for crass administrative expediency. Under this dichotomy, plea bargaining is justified and criticized for enticing defendants to plead guilty to ensure that the court system can continue to operate.

\section{Research questions, road map, and methodology}

In this thesis, I broadly ask whether Canadian criminal law should explicitly permit \textit{nolo contendere}. I answer this question by exploring the history and examining the nature of no-contest pleas, determining which no-contest pleas are compatible with Canadian criminal law and why, and addressing the legitimate concerns that these pleas and plea bargaining generally raise.

In Chapter 2, I define what no-contest pleas are and identify the role that they play in the plea bargaining process. From here, I conduct a high-level survey of the history of guilty and \textit{nolo contendere} pleas in England, the United States, and Canada to identify their convergences and divergences and to explicate the relationships between them. 

Chapter 3 then examines whether and to what extent no-contest pleas are compatible with Canadian criminal law. I identify the relevant statutory provisions in the \textit{Criminal Code} and examine each type of no-contest plea against it. Here I explore the informal \textit{nolo contendere} procedure that has developed in Canada and conclude by reviewing the uncertain legal standing of best-interest pleas.

Finally, in Chapter 4, I ask whether Canadian criminal law should expand and adapt to embrace no-contest pleas more fully and why. I first address the sort of cognitive dissonance that no-contest pleas can generate while demonstrating that the pleas are, in fact, internally coherent. Next, I look at the significant objections to plea bargaining and the concerns that specifically emerge with \textit{nolo contedere}. I conclude this final substantial chapter by evaluating the benefits of formalization and codification.