\chapter{Introduction}

\section{Overview}

Criminal defendants\footnote{Here and throughout this thesis, I use the term ``defendants" to refer to persons charged with criminal offences rather than the term ``accused. This decision was entirely phonaesthetic. Although there is an argument to be made in favour of sticking with the governing statutory parlance, the clumsiness of the word ``accused" — especially apparent when dealing with phrases like ``multiple co-accuseds" — is justification enough.} in Canada are presumed innocent until a prosecutor\footnote{I also use the term ``prosecutor" to refer to prosecutors, rather than ``the Crown." Although this is common parlance both in Canadian courts and throughout the \textit{Criminal Code}, RSC 1985, c C-46 [\textit{Criminal Code}], the term ``Crown" is a broad term applied to anyone appearing on the government's behalf in court. Using the term ``prosecutor" makes for more precise usage.} proves they are guilty beyond a reasonable doubt. But defendants may also concede the state's evidence against themselves and invite a conviction. I refer to these self-convictions collectively as \textit{uncontested pleas}. By default, the state must prove its allegations against defendants at trial. Defendants are presumed innocent,\footnote{See \textit{Canadian Charter of Rights and Freedoms}, s 11(d), Part 1 of the \textit{Constitution Act, 1982}, being Schedule B to the \textit{Canada Act 1982} (UK), 1982, c 11 [\textit{Charter}]} have the right to remain silent and are secure against self-incrimination,\footnote{See \textit{ibid}, s 11(c). Both the right to silence and the right against self-incrimination flow from this provision.} but waive these rights when they enter an uncontested plea. Since defendants are under no obligation to do so, and because uncontested pleas preclude an acquittal, Canadian courts require such defendants to self-convict \textit{knowingly}, \textit{voluntarily}, and \textit{unequivocally}.\footnote{See \textit{R v Wong}, 2018 SCC 25, [2018] 1 SCR 696 [\textit{Wong}].}

Self-convictions began as common-law confessions and evolved into codified criminal procedures.\footnote{See Albert W. Alschuler, "Plea Bargaining and Its History" (1979) 79:1 Colum L Rev 1 at 7.} As they did, plea bargaining emerged and evolved alongside them. Plea bargaining occurs when a defendant enters an uncontested plea for a reduced sentence, a plea to a lesser charge, or some other consideration. These deals assume that prosecutors and defendants each risk losing something by going to trial but are guaranteed to gain something by agreeing to a conviction.

Today, plea bargains are both blamed and praised for the fact that far more criminal cases resolve without a trial than with one.\footnote{Some estimates place the percentage of guilty pleas as high as 97\%. See Innocence Project, ``Report: Guilty Pleas on the Rise, Criminal Trials on the Decline" (7 August 2018), online: \textless \url{https://innocenceproject.org/guilty-pleas-on-the-rise-criminal-trials-on-the-decline/}\textgreater. Reported statistics in Canada confirm that the majority of criminal charges that prosecutors proceed on resolve through guilty pleas: see e.g. Ontario Court of Justice, ``Criminal Statistics 2021", online: \textless \url{https://www.ontariocourts.ca/ocj/files/stats/crim/2021/2021-Offence-Based-Criminal.xlsx}\textgreater \hspace{1mm}[\textit{Ontario Court of Justice Criminal Statistics 2021}]. Although the numbers in Ontario are nowhere near the 97\% that the National Association of Criminal Defense Lawyers reports, it is clear that there are far fewer contested trials than guilty pleas in the Ontario provincial courts.} The practice has been heralded as a necessary component of a functioning modern justice system\footnote{See e.g. \textit{R v Anthony-Cook}, 2016 SCC 43, [2016] 2 SCR 204 at paras 25, 46 [\textit{Anthony-Cook}].} and derided as an efficiency-obsessed mechanism responsible for wrongful convictions and involuntary pleas.\footnote{See David Ireland, ``Bargaining for Expedience: The Overuse of Joint Recommendations on Sentence" (2015) 38:1 Man LJ 273 at 275f.} Proponents of plea bargaining often note the immense time and resources that modern criminal trials require and the opportunities plea bargaining affords for defendants to minimax their risks and benefits.\footnote{See e.g. Jenny Elayne Ronis, ``The Pragmatic Plea: Expanding Use of the Alford Plea to Promote Traditionally Conflicting Interests of the Criminal Justice System" (2010) 82:5 Temp L Rev 1389.} On the other hand, critics of plea bargaining argue that this stick-and-carrot-style practice improperly pressures defendants to forego trials and increases the likelihood of wrongful convictions.\footnote{See e.g. Joan Brockman, ``An Offer You Can't Refuse: Pleading Guilty When Innocent" (2010) 56:Issues 1 and 2 Crim LQ 116.}

Despite widespread criticism and controversy, plea bargaining quickly gained traction in early 20\textsuperscript{th} century North American criminal procedure.\footnote{See Alschuler, \textit{supra} note 6 at 6. See also Ireland, \textit{supra} note 7 at 280f.} As it did so, an obscure common-law plea known as \textit{nolo contendere} experienced a revival in the United States. \textit{Nolo contendere} is a Latin phrase meaning "I do not wish to contest." A defendant entering a \textit{nolo contendere} plea invites the criminal conviction's consequences without admitting responsibility for the offence. Like guilty pleas, \textit{nolo contendere} pleas are self-convictions and may play a part in resolving criminal charges. But \textit{nolo contendere} pleas also provide defendants with advantages that outright guilty and not guilty pleas do not. Most prominently, at common law and in all but a few states, \textit{nolo contendere} pleas are inadmissible in subsequent actions stemming from the same allegations. Most states that recognize these pleas allow defendants to enter them for all offences, authorizing the plea as an enticing bargaining chip to convince defendants to disavow responsibility without a trial.

\section{Academic history}

American academics have extensively discussed the history, use, and characteristics of \textit{nolo contendere} pleas for over a century.\footnote{See e.g. Paul E. Hadlick, Crim Prosecutions under the Sherman Anti-Trust Act (Washington, D.C.: Ransdell., 1939) at 131 — 138; Nathan B. Lenvin \& Ernest S. Meyers, ``Nolo Contendere: Its Nature and Implications" (1942) 51:8 Yale L J 1255; Patrick W. Healey, ``The Nature and Consequences of the Plea of Nolo Contendere" (1954) 33:3 Neb L Rev 428; Norman S. Oberstein, ``Nolo Contendere--Its Use and Effect" (1964) 52:2 Calif L Rev 408; Neil H. Cogan, ``Entering Judgment on a Plea of Nolo Contendere: A Reexamination of North Carolina v. Alford and Some Thoughts on the Relationship between Proof and Punishment" (1975) 17:4 Ariz L Rev 992; William C. Athanas, ``Criminal Law - Lack of Knowlege Concerning Deporation Consequences Does Not Invalidate Nolo Contendere Plea" (1995) 29:2 Suffolk U L Rev 607; Ramy Simpson, ``Nolo Contendere Convictions: The Effect of No Confession in Future Criminal Proceedings" (2019) 4:6 Crim L Prac 25.} However, in Canada, the academic conversation around plea bargaining rarely touches on \textit{nolo contendere}.\footnote{But see e.g. Brockman, \textit{supra} note 11 at 118.} This discursive dearth in Canada most likely results from our distinct common law traditions. The \textit{nolo contendere} plea emerged in 15\textsuperscript{th} century England but was last reported being used there in 1702.\footnote{See \textit{R v Templeman}, 1 Salk 55 (QB 1702).} When English courts last employed the \textit{nolo contendere} plea, ``Canada" was still more than a century and a half away. By the time \textit{nolo contendere} pleas re-emerged in force in the early 20\textsuperscript{th} century, Canadian criminal law had developed without it, and \textit{nolo contendere}'s American re-emergence did not initially generate academic or judicial interest in Canada.

\section{Importance of the current research}

Until recently, Frederick L. Forsyth's 1997 ``A Plea for Nolo Contendere in the Canadian Criminal Justice System" was a notable but seemingly lone exception to this silence.\footnote{Frederick L Forsyth, ``A Plea for Nolo Contendere in the Canadian Criminal Justice System" (1997) 40:2 Crim LQ 243.} But since then, court decisions in Ontario have authorized a ``\textit{nolo contendere} procedure" that allows defendants to enter these pleas informally. Canadian courts or legal scholars have yet to examine whether these informal pleas \textit{should} be allowed. As Forsyth's article's title alludes, Canadian criminal law does not \textit{formally} recognize \textit{nolo contendere} pleas. The \textit{Criminal Code of Canada} only allows defendants to plead guilty, not guilty, or one of the special ``double jeopardy" pleas listed in section 607.\footnote{\textit{Criminal Code} s 607 outlines the double jeopardy pleas. Each applies to defendants already punished for a crime who should not have to plead again. Since then, \textit{Charter}, \textit{supra} note 3, s 11(h) has provided the same guarantee at a constitutional level.} But the \textit{nolo contendere} procedure's emergence in Canada suggests that it fulfills needs that formulaic applications of the statutory framework cannot. However, the procedure is unrefined and unregulated, such that it causes problems that current legislation cannot resolve. As a result, a closer analysis is warranted.

\section{Research questions, methodology, and road map}

The broad question I answer in this thesis is whether Canadian criminal law should formalize \textit{nolo contendere} pleas. I argue that plea bargaining generally and \textit{nolo contendere} pleas specifically benefit the criminal justice system when implemented judiciously and regulated carefully. To support this answer, I examine the morality of plea bargaining, as all uncontested pleas are practically and ethically entwined in the controversial practice of plea bargaining.\footnote{See Stephanos Bibas, ``Harmonizing Substantive-Criminal-Law Values and Criminal Procedure:
The Case of Alford and Nolo Contendere Pleas" (2003) 88:5 Cornell L Rev 1361. I discuss Bibas's uniquely caustic view of \textit{nolo contendere} pleas in \S4 below.} Defendants frequently enter them due to plea-bargained deals they make with prosecutors, and it is reasonable to suppose that formalizing these pleas will result in more such agreements. Plea bargaining critics argue that these arrangements unfairly induce defendants to plead guilty, result in wrongful convictions, and improperly shift focus away from the criminal law's moral core. If true, these concerns should be considered when asking whether Canadian criminal law should expand plea bargaining's scope.

I found my argument in this thesis on an overview of legislation, case law, and academic literature in the United States and Canada. The English and American \textit{nolo contendere} plea began as an obscure common law creation, such that most early \textit{nolo contendere} plea discussions took place in recorded court decisions. In subsequent years, codification dominated criminal rules and procedure across the United States, leading to state legislatures formally recognizing or precluding \textit{nolo contendere} pleas through statute. This closely tracked Canadian criminal procedure, which had been statutorily regulated since Parliament introduced the first \textit{Criminal Code} in 1892.\footnote{See \textit{The Criminal Code}, 1892 (55-56 Vict), c 29 [\textit{Criminal Code, 1892}]} To locate these resources, I relied on WestLaw to find American case law, statutes, and court rules, as well as to gather early Canadian criminal reported decisions. I relied on CanLII for later Canadian decisions, statutes, and court rules. For academic articles and case commentaries, I primarily used HeinOnline. My years practicing law as a criminal defence attorney and my academic training in philosophy inform my normative propositions and conclusions.

In this thesis, I examine whether \textit{nolo contendere} pleas should be statutorily authorized by analyzing the nature of \textit{nolo contendere} pleas and addressing the legitimate concerns these pleas and plea bargaining raise. Beginning in Chapter 2, I define contested and uncontested pleas and outline their role in plea bargaining. From here, I conduct a brief high-level survey of the history of guilty and \textit{nolo contendere} pleas in England, the United States, and Canada to show how these pleas evolved and relate to one another. In Chapter 3, I examine whether and to what extent \textit{nolo contendere} pleas are compatible with Canadian criminal law. I identify the relevant statutory provisions in the \textit{Criminal Code}, examine \textit{nolo contendere} pleas against them, and outline the contours of the recently-developed and developing Canadian \textit{nolo contendere} procedure. I then go on in Chapter 4 to ask whether Canadian criminal law should formalize \textit{nolo contendere} pleas. I consider and respond to common and compelling objections to plea bargaining generally. I examine \textit{nolo contendere} pleas specifically and in light of these categorical objections and go on to identify several problems that formalization may address. I conclude in Chapter 5 that plea bargaining and \textit{nolo contendere} help facilitate normative goods for defendants and society and should be encouraged and carefully managed.