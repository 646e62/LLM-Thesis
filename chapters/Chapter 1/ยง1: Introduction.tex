\chapter{Introduction}

\section{Overview}

Criminal defendants\footnote{Here and throughout this thesis, I use the term ``defendants" to refer to persons charged with criminal offences rather than the term ``accused. This decision was entirely phonaesthetic. Although there is an argument to be made in favour of sticking with the governing statutory parlance, the clumsiness of the word ``accused" — especially apparent when dealing with phrases like ``multiple co-accuseds" — is justification enough.} in Canada are presumed innocent until a prosecutor\footnote{I also use the term ``prosecutor" to refer to prosecutors, rather than ``the Crown." Although this is common parlance both in Canadian courts and throughout the \textit{Criminal Code}, RSC 1985, c C-46 [\textit{Criminal Code}], the term ``Crown" is a broad term applied to anyone appearing on the government's behalf in court. Using the term ``prosecutor" makes for clearer usage.} proves that they are guilty beyond a reasonable doubt. But defendants may also concede the state's evidence against themselves and invite a conviction. I refer to these self-convictions collectively as ``uncontested pleas."

By default, a criminal defendant in Canada has the right to require the state to prove its allegations at trial. A defendant who enters an uncontested plea gives up this right. Defendants are presumed innocent, have the right to remain silent, are secure against self-incrimination, but waive these rights when they enter an uncontested plea. Because defendants are under no obligation to do so, and because doing so negates the possibility of an acquittal, Canadian courts require such defendants to do so \textit{knowingly}, \textit{voluntarily}, and \textit{unequivocally}.\footnote{See \textit{R v Wong}, 2018 SCC 25, [2018] 1 SCR 696 [\textit{Wong}].}

Self-convictions began as common-law confessions and evolved into codified criminal procedures.\footnote{See Albert W. Alschuler, "Plea Bargaining and Its History" (1979) 79:1 Colum L Rev 1 at 7.} As they did, plea bargaining emerged and evolved alongside it. Plea bargaining occurs when a defendant enters an uncontested plea for a reduced sentence, a plea to a lesser charge, or some other consideration. The fundamental assumption underlying plea bargaining is that prosecutors and defendants each risk losing something by going to trial but are guaranteed to gain something by agreeing to a conviction. Specifically, prosecutors avoid losing their cases outright by securing a conviction, while defendants mitigate their potential losses by agreeing to a conviction on favourable terms.

Today, plea bargains are both blamed and praised for the fact that far more criminal cases resolve without a trial than with one.\footnote{Some estimates place the percentage of guilty pleas as high as 97\%. See Innocence Project, ``Report: Guilty Pleas on the Rise, Criminal Trials on the Decline" (7 August 2018), online: \textless \url{https://innocenceproject.org/guilty-pleas-on-the-rise-criminal-trials-on-the-decline/}\textgreater. Some Canadian courts regularly report their plea statistics. See e.g. Ontario Court of Justice, ``Criminal Statistics 2021", online: \textless \url{https://www.ontariocourts.ca/ocj/files/stats/crim/2021/2021-Offence-Based-Criminal.xlsx}\textgreater. Although the numbers in Ontario are nowhere near the 97\% that the National Association of Criminal Defense Lawyers reports, it is clear that the number contested trials pales in comparison to the number of guilty pleas entered.} The practice has been heralded as a necessary component of a functioning modern justice system\footnote{See \textit{R v Anthony-Cook}, 2016 SCC 43, [2016] 2 SCR 204 at paras 25, 46 [\textit{Anthony-Cook}].} and derided as an efficiency-obsessed mechanism responsible for wrongful convictions and involuntary pleas.\footnote{See \S\S 4.1 \& 4.3 below for a detailed discussion and analysis of plea bargaining's opponents and proponents.} Proponents of plea bargaining often note the immense time and resources that modern criminal trials require and the opportunities plea bargaining affords for defendants to minimax their risks and benefits.\footnote{See e.g. Jenny Elayne Ronis, ``The Pragmatic Plea: Expanding Use of the Alford Plea to Promote Traditionally Conflicting Interests of the Criminal Justice System" (2010) 82:5 Temp L Rev 1389.} On the other hand, critics of plea bargaining argue that this stick-and-carrot-style practice improperly pressures defendants to forego trials and increases the likelihood of wrongful convictions.\footnote{See e.g. Joan Brockman, ``An Offer You Can't Refuse: Pleading Guilty When Innocent" (2010) 56:Issues 1 and 2 Crim LQ 116.}

Despite widespread criticism and controversy, plea bargaining quickly gained traction in early 20\textsuperscript{th} century North American criminal legal practice.\footnote{See Alschuler, \textit{supra} note 4 at 6.} As it did so, an obscure common-law plea known as \textit{nolo contendere} experienced a revival in the United States. \textit{Nolo contendere} is a Latin phrase meaning "I do not wish to contest." A defendant entering a \textit{nolo contendere} plea invites the consequences of a criminal conviction without admitting responsibility for the offence. Like guilty pleas, \textit{nolo contendere} are self-convictions. Because of this, where these pleas are available, they may form part of the plea bargaining process. 

\textit{Nolo contendere} pleas provide defendants with certain advantages that outright guilty and not guilty pleas do not. Most prominently, at common-law and in most states that allow them, \textit{nolo contendere} pleas are inadmissible in subsequent actions stemming from the same allegations. Unlike guilty pleas, which a defendant may reasonably enter as a matter of conscience, \textit{nolo contendere} pleas do not involve any form of confession. As a result, these pleas may appeal to defendants who would otherwise set their matter down for trial while having minimal appeal to defendants that want to self-convict as a matter of principle. Most states allow them for most offences, where they may be used as an enticing bargaining chip to convince defendants to disavow responsibility without a trial.

\section{Issues and academic treatment thus far}

The broad question I answer in this thesis is whether Canadian criminal law should expand to formally include \textit{nolo contendere} pleas. To answer this question, I also examine the morality of plea bargaining, as \textit{nolo contendere} pleas are ethically entwined in the controversial practice of plea bargaining. Defendants frequently enter them due to plea-bargained deals they make with prosecutors, and it is reasonable to suppose that formalizing these pleas will result in more such agreements. Plea bargaining critics argue that these arrangements unfairly induce defendants to plead guilty, result in wrongful convictions, and improperly shift focus away from the criminal law's moral core. If true, these concerns should be considered when asking whether Canadian criminal law should expand plea bargaining's scope.

American academics have extensively discussed the history, use, and characteristics of \textit{nolo contendere} pleas for over a century. However, in Canada, the academic conversation around plea bargaining rarely touches on \textit{nolo contendere}. This lack of discussion of \textit{nolo contendere} most likely results from our distinct common law traditions. The \textit{nolo contendere} plea emerged in 15\textsuperscript{th} century England but was last known to have been reported using there in 1702.\footnote{See \textit{R v Templeman}, 1 Salk 55 (QB 1702).} When English courts last used the \textit{nolo contendere} plea, ``Canada" was still more than a century and a half away.\footnote{See e.g. the map of North American European colonies in 1702 on Wikipedia, ``Queen Anne's War" (19 August 2022), online: \textless \url{https://en.wikipedia.org/wiki/Queen_Anne\%27s_War}\textgreater.} By the time \textit{nolo contendere} pleas re-emerged in force in the early 20\textsuperscript{th} century, Canadian criminal law had developed without it, and \textit{nolo contendere}'s American re-emergence did not initially generate much academic or judicial interest in Canada. Until recently, Frederick L. Forsyth's 1997 ``A Plea for Nolo Contendere in the Canadian Criminal Justice System" was a notable but seemingly lone exception to this silence.\footnote{Frederick L Forsyth, ``A Plea for Nolo Contendere in the Canadian Criminal Justice System" (1997) 40:2 Crim LQ 243.} But since then, court decisions in Ontario have authorized a ``\textit{nolo contendere} procedure" that allows defendants to enter these pleas informally. Whether these informal pleas \textit{should} be allowed has yet to be examined. 

\section{Importance of the current research}

As Forsyth's article's title alludes, Canadian criminal law does not \textit{formally} recognize \textit{nolo contendere} pleas. The \textit{Criminal Code of Canada} only allows defendants to plead guilty, not guilty, or one of the special ``double jeopardy" pleas listed in section 607.\footnote{\textit{Criminal Code} s 607 outlines these pleas. Each applies to defendants already punished for a crime who should not have to plead again. Since then, \textit{Canadian Charter of Rights and Freedoms}, s 11(h), Part 1 of the \textit{Constitution Act, 1982}, being Schedule B to the \textit{Canada Act 1982} (UK), 1982, c 11 [\textit{Charter}] has provided the same guarantee at a constitutional level.} But the emergence of the \textit{nolo contendere} procedure in Canada suggests that it fulfills needs that naive applications of the statutory framework cannot. However, the procedure may also cause unanticipated problems that current legislation cannot resolve. As a result, a closer analysis is warranted. Because uncontested pleas and plea bargaining are inextricably linked, this thesis considers both.

\section{Research questions, road map, and methodology}

In this thesis, I broadly ask whether Canadian criminal law should explicitly permit \textit{nolo contendere}. I answer this question by exploring the history and examining the nature of \textit{nolo contendere} pleas and addressing the legitimate concerns that these pleas and plea bargaining generally raise.

Beginning in Chapter 2, I define contested and uncontested pleas and outline their role in plea bargaining. From here, I conduct a brief high-level survey of the history of guilty and \textit{nolo contendere} pleas in England, the United States, and Canada to show how these pleas evolved and how they relate to one another. 

In Chapter 3, I examine whether and to what extent \textit{nolo contendere} pleas are compatible with Canadian criminal law. I identify the relevant statutory provisions in the \textit{Criminal Code}, examine \textit{nolo contendere} pleas against them, and outline the contours of the recently-developed and developing Canadian \textit{nolo contendere} procedure.

I then go on in Chapter 4 to ask whether Canadian criminal law should formalize \textit{nolo contendere} pleas. I begin by considering and responding to common and compelling objections to plea bargaining generally. I examine \textit{nolo contendere} pleas specifically and in light of these categorical objections. 

I conclude in Chapter 5 that plea bargaining and \textit{nolo contendere} help facilitate normative goods for defendants and society and should be encouraged and carefully managed.