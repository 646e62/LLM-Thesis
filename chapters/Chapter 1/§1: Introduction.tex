\chapter{Introduction}

\section{Overview}

Criminal defendants\footnote{Here and throughout this thesis, I use the term ``defendants" to refer to persons charged with criminal offences rather than the term ``accused. This decision was entirely phonaesthetic. Although there is an argument to be made in favour of sticking with the governing statutory parlance, the clumsiness of the word ``accused" — especially apparent when dealing with phrases like ``multiple co-accuseds" — is justification enough.} in Canada are presumed innocent until a prosecutor proves that they are guilty beyond a reasonable doubt. But the law also allows defendants to concede the state's evidence against themselves and invite a conviction. I refer to these self-convictions collectively as ``uncontested pleas."

By default, a criminal defendant in Canada has the right to contest the state's allegations at a trial. A defendant who enters a uncontested plea gives up this right. Defendants are presumed to be innocent, have the right to remain silent, and may not be forced to incriminate themselves, and they waive these rights when they enter a uncontested plea. Because defendants are under no obligation to do so, and because doing so negates the possibility of an acquittal, Canadian courts require that defendants waiving their right to trial do so \textit{knowingly}, \textit{voluntarily}, and \textit{unequivocally}.\footnote{See \textit{Adgey v R}, 1973 CanLII 37 (SCC), [1975] 2 SCR 426; \textit{R v Wong}, 2018 SCC 25, [2018] 1 SCR 696. See also \hl{section} below for further details about how these requirements evolved at common law, and the unique Canadian position.}

As self-convictions evolved from common-law confessions to codified criminal procedures, plea bargaining emerged and evolved alongside it. Plea bargaining occurs when a defendant enters a uncontested plea in exchange for a reduced sentence, a plea to a lesser charge, or some other consideration. The fundamental assumption underlying plea bargaining is that prosecutors and defendants risk losing something by going to trial and are guaranteed to gain something by agreeing to a conviction. Prosecutors avoid losing their cases outright by securing a conviction, while defendants mitigate their potential losses by agreeing to a conviction on favourable terms.

Today, plea bargains are both blamed and praised\footnote{.} for the fact that far more criminal cases resolve without a trial than with one.\footnote{Some estimates place the percentage of guilty pleas as high as 97\%. See (the article that says this).} The practice has been heralded as a necessary component of a functioning modern justice system\footnote{See \textit{R v Anthony-Cook}, 2016 SCC 43 at paras 25, 46.} and derided as an efficiency-obsessed mechanism responsible for wrongful convictions and involuntary pleas.\footnote{See Alschuler, Ireland.} Proponents of plea bargaining often note the immense time and resources that modern criminal trials require and the opportunities plea bargaining affords for defendants to minimax their risks and benefits.\footnote{.} On the other hand, critics of plea bargaining argue that this stick-and-carrot-style practice improperly pressures defendants to forego trials\footnote{See An Offer You Can't Refuse.} and increases the likelihood of wrongful convictions.\footnote{See Alschuler.}

Despite widespread criticism and controversy, plea bargaining quickly gained practical traction in early 20\textsuperscript{th} century North American criminal law. As it did so, an obscure common-law plea known as \textit{nolo contendere} experienced a revival in the United States. \textit{Nolo contendere} is a Latin phrase meaning "I do not wish to contest." A defendant entering a \textit{nolo contendere} plea invites the consequences of a criminal conviction without admitting responsibility for the offence. Like guilty pleas, \textit{nolo contendere} are self-convictions. Because of this, where these pleas are available, they may form part of the plea bargaining process. 

\textit{Nolo contendere} pleas provide defendants with certain advantages that guilty and not guilty pleas do not. At common-law, and in most states that allow them, \textit{nolo contendere} pleas are inadmissible in subsequent actions stemming from the same allegations\footnote{This fact that likely contributed to the plea's early popularity in tax and corporate prosecutions.See \hl{that early article discussing \textit{nolo contendere} in tax prosecutions}. This is often true even in jurisdictions that do not otherwise recognize or allow \textit{nolo contendere} pleas. For example, North Dakota and Washington do not allow admitting evidence of \textit{nolo contendere} pleas in subsequent proceedings, despite not allowing local defendants to enter them. Other states, like Arkansas, take a very expansive view of subsequent inadmissibility, despite not otherwise recognizing \textit{nolo contendere} pleas.} Unlike guilty pleas, which a defendant may reasonably enter as a matter of conscience, \textit{nolo contendere} pleas do not involve any form of confession. As a result, these pleas may be more appealing to defendants who would otherwise be inclined to set their matter down for trial while having minimal appeal for defendants that want to self-convict as a matter of conscience. Most states allow them for most offences,\footnote{Although many jurists once thought that defendants should only be allowed to enter \textit{nolo contendere} pleas in trivial cases, in 1926 the US Supreme Court found that these pleas were available for all offences at common law. See \textit{Hudson et al. v United States}, 272 US 451. which led to these pleas gaining popularity and eventually being codified across the United States.} where they may be used as an enticing bargaining chip to convince defendants to disavow responsibility without a trial.

\section{Issues and academic treatment thus far}

The broad question I answer in this thesis is whether Canadian criminal law should expand to include \textit{nolo contendere} pleas. However, \textit{nolo contendere} pleas are ethically entwined in the controversial practice of plea bargaining. Defendants frequently enter them as a result of plea-bargained deals they make with prosecutors, and it is reasonable to suppose that formalizing these pleas will result in more such agreements. Plea bargaining critics argue that these arrangements unfairly induce defendants to plead guilty, result in wrongful convictions, and improperly shift focus away from the criminal law's moral core. If true, these concerns should be considered when asking whether Canadian criminal law should expand plea bargaining's scope.

American academics have extensively discussed the history, use, and characteristics of \textit{nolo contendere} pleas for over a century. However, in Canada, the academic conversation around plea bargaining rarely discusses this plea. This lack of discussion on this subject results from our distinct common law traditions. The \textit{nolo contendere} plea emerged in 15\textsuperscript{th} century England but was last known to have been used there in 1702.\footnote{See \textit{R v Templeman}, 1 Salk 55 (QB 1702).} When English courts last used the \textit{nolo contendere} plea, ``Canada" consisted of Newfoundland and the parts of Rupert's Land explored by the Hudson's Bay Company thus far.\footnote{} By the time \textit{nolo contendere} pleas re-emerged in force in the early 20\textsuperscript{th} century, Canada's criminal law had developed without it, and \textit{nolo contendere}'s American re-emergence did not initially generate much academic or judicial interest in Canada. Until recently, Frederick L. Forsyth's 1997 ``A Plea for Nolo Contendere in the Canadian Criminal Justice System" was a notable but seemingly lone exception to this silence. Since then, court decisions in Ontario have authorized a ``\textit{nolo contendere} procedure" that allows defendants to enter these pleas informally. However, the question of whether these pleas \textit{should} be allowed has yet to be examined. 

\section{Importance of the current research}

As Forsyth's article's title alludes, Canadian criminal law does not \textit{formally} recognize \textit{nolo contendere} pleas. The \textit{Criminal Code of Canada} only allows defendants plead guilty, a not guilty plea, or one of the special ``double jeopardy" pleas listed in section 607.\footnote{Explain these pleas and don't bring it up again.} But the emergence of the \textit{nolo contendere} procedure suggests that it fulfills needs that a naive applications of the statutory framework cannot. However, the procedure may also cause unanticipated problems that the current statutory framework cannot resolve. As a result, a closer analysis is warranted. Because uncontested pleas and plea bargaining are inextricably linked, this thesis considers both.

\section{Research questions, road map, and methodology}

In this thesis, I broadly ask whether Canadian criminal law should explicitly permit \textit{nolo contendere}. I answer this question by exploring the history and examining the nature of uncontested pleas, determining which uncontested pleas are compatible with Canadian criminal law and why, and addressing the legitimate concerns that these pleas and plea bargaining generally raise.

Beginning in Chapter 2, I define uncontested pleas and outline the role they play in the plea bargaining process. From here, I conduct a brief and high-level survey of the history of guilty and \textit{nolo contendere} pleas in England, the United States, and Canada to show how these pleas evolved and how they relate to one another. 

In Chapter 3 I examine whether and to what extent \textit{nolo contendere} pleas are compatible with Canadian criminal law. I identify the relevant statutory provisions in the \textit{Criminal Code}, examine \textit{nolo contendere} against them, and outline the contours of the recently-developed and developing Canadian \textit{nolo contendere} procedure.

Finally, in Chapter 4, I ask whether Canadian criminal law should formalize \textit{nolo contendere} pleas. I begin by considering and responding to common and compelling categories of objections to plea bargaining generally. I go on to examine \textit{nolo contendere} pleas specifically and in light of these categorical objections. Finally, I argue that plea bargaining and \textit{nolo contendere} help facilitate normative goods for defendants and society, and should be both encouraged and carefully managed.
\newpage
\section{Definitions}

\begin{itemize}
    \item \textbf{Admission:} a proposition that a defendant acknowledges has been proven.
    \item \textbf{Allegation:} a proposition that a decision maker in court must decide to be proven or not proven. 
    \item \textbf{Belief:} subjective belief. Belief is susceptible to proof, argument, and a sense of justice, such that a person may be convinced that a proposition is true or false. Beliefs do not need to be true or reasonably proven to obtain.
    \item \textbf{Contested plea:} a plea where the prosecutor must prove some or all of their allegations.
    \item \textbf{Exculpatory plea:} 
    \item \textbf{Fact:} an objectively true proposition. 
    \item \textbf{Inculpatory plea:}
    \item \textbf{Non-culpatory plea:}
    \item \textbf{Plea:} a proposition designed to tell the court what, if anything, the prosecutor must \textit{prove} about the charge. Pleas may also convey information about objetive truth and subjective belief, but do not need to do so in order to be valid. These may be called a plea's \textit{proof-component}, \textit{truth-component}, and \textit{belief-component}.
    \item \textbf{Proof:}
    \item \textbf{Proposition:} a statement about the world that can be evaluated as true or false.
    \item \textbf{Truth:} objective truth. All that is the case.
    \item \textbf{Uncontested plea:} a plea where the prosecutor is not required to prove any of their allegations.
\end{itemize}