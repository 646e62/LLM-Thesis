\chapter{Conclusion}

\setcounter{footnote}{247}

To conclude my argument, I outline several ways that Parliament may implement \textit{nolo contendere} pleas to solve different criminal justice problems or provide the system and its participants with different ways to punish, reform, and rehabilitate offenders. To do so, I use Drechsler's criteria to outline how a newly-legislated \textit{nolo contendere} plea may look or how Parliament might implement it. Unlike the United States, where \textit{nolo contendere} pleas have existed for centuries, Canada had no discernable common law contact with \textit{nolo contendere} until the Ontario Court of Appeal decided \textit{DMG} just over a decade ago. As a result, a formal Canadian \textit{nolo contendere} plea may be designed from the ground up to incorporate the best and discard the worst without worrying about retroactive consistency. Because Canadian criminal law is uniform cross-country, Parliament may evenly implement these pleas nationwide. Regulating and codifying \textit{nolo contendere} would allow Parliament to specify when and how defendants may enter pleas and whether they should be admissible in future legal proceedings.

\section{Applicability}

\textit{Nolo contendere} pleas encourage resolution discussions and plea negotiations and allow otherwise recalcitrant defendants to spare their victims the burden of going to trial. Although I argue that this alone is a sufficient basis for permitting these pleas, I also recognize that Parliament may use the applicability criterion to accomplish specific societal goals and criminal justice objectives.

In the United States, courts long held to Hawkins' citation as their guiding common law authority for how \textit{nolo contendere} pleas were to be entered and accepted. Nowadays, following \textit{Hudson}, where \textit{nolo contendere} pleas are permitted, defendants may generally enter them in response to any offence or offence type. Canadian \textit{nolo contendere} pleas do not have such traditional grounding. As a result, they are neither beholden to Hawkins's rule that limits them to misdemeanour offences nor \textit{Hudson}'s rule extending them to any offence punishable by any penalty. Absent this spectre of \textit{stare decisis}, Parliament is free to allow or preclude any defendant it wishes from entering a \textit{nolo contendere} plea and may do so in several ways: it may stick with ancient tradition and limit these pleas to non-serious cases, customize which offences or offence types it wishes to allow the pleas for or open them up to all.

Under the first approach, \textit{nolo contendere} pleas would only be allowed for summary criminal offences. A strict application of this approach could limit the plea to so-called \textit{pure summary offences}.\footnote{See Criminal Notebook, ``List of Summary Conviction Offences" (January 2021) online: \textless http://criminalnotebook.ca/index.php/List\_of\_Summary\_Conviction\_Offences\textgreater.} In contrast, a looser approach would allow the plea for \textit{hybrid offences} where the prosecutors \textit{proceeded summarily}. Where other features commonly associated with \textit{nolo contendere} pleas are present, such as the lack of a need for a factual foundation or a maximum penalty of a small fine, these restrictions are sensible. However, this limitation's utility disappears when judges must ensure sufficient facts to support a conviction and where the plea may result in the same penalties as a guilty plea. 

A second approach would limit the plea's applicability not by how the Crown prosecutes the charge but rather by the alleged offence or type of offence. Canadian law currently prohibits defendants convicted of certain charges from obtaining specific sentences. For example, Canadian judges may generally allow defendants to serve a jail sentence in the community through a Community Supervision Order, or CSO. However, \textit{Criminal Code} s 742.1(f) lists several offences and offence types that are not eligible for a CSO. Adopting a similar approach would allow Parliament to target specific offences or offence types where a \textit{nolo contendere} plea would be inappropriate and prevent the plea from being entered for those offences.

The final approach would open the pleas up to all offences, per the United States Supreme Court in \textit{Hudson}. Doing so would significantly increase the number of defendants able to self-convict. But more importantly, formalizing a \textit{nolo contendere} plea that defendants may enter for all offences is most likely to successfully supplant the haphazard, informal \textit{nolo contendere} plea procedure. One of the primary goals of formalization must be ensuring that these pleas are entered safely and lead to predictable results. These goals are unachievable as long as the dangerous and unpredictable informal plea persists. In order to truly overtake the informal plea procedure, the formal plea must apply to all offences and offence types, thereby helping to ensure that defendants no longer need to enter these uncontested pleas informally.

\section{Acceptability}

Once it has been settled which offences defendants may enter \textit{nolo contendere} pleas for, it must be determined what preconditions must obtain for a court to accept them. In American states where \textit{nolo contendere} pleas are allowed, most defendants require permission from the court to enter them. Several states also require permission from the prosecutor, while only Virginia allows defendants to enter \textit{nolo contedere} pleas by right. As implemented, the Canadian \textit{nolo contendere} plea procedure requires permission from the prosecutor to enter but does not generally require permission from the court. I argue that this is a problem that Parliament should use formalization to correct.

Allowing Canadian defendants to enter these pleas without judicial discretion or oversight is problematic. Because defendants who enter these pleas do not acknowledge their guilt or culpability, there is a risk that some will opt to convict themselves of charges they are not factually or legally culpable for or in circumstances that a judicial observer would deem unfair. If defendants may enter \textit{nolo contendere} pleas to any criminal charge punishable by any potential punishment, judicial oversight is essential to maintain fairness, accuracy, and confidence in the system. Furthermore, because \textit{nolo contendere} pleas do not necessarily clearly reflect a defendant's beliefs, and because they must invariably result in a conviction, additional prerequisites should be put in place to ensure these pleas pass muster. Requiring prosecutors, defendants, or both to jointly show cause for why the court should accept a defendant's \textit{nolo contendere} plea,\footnote{See e.g. \textit{Criminal Code}, \textit{supra} note 2, s 730(1), where Parliament legislated the statutory requirements that must be met for a court to grant a conditional discharge.} evaluating the plea's merits against a set statutory standard, or simply requiring the prosecutor's written consent may each accomplish this task. 

\section{Procedural effects}

The third criterion considers what effects the plea has on the immediate proceedings before the court. As self-convictions, \textit{nolo contendere} pleas should procedurally result in criminal convictions, just as guilty pleas do, and not in a ``conviction by a trial court" as they currently do. Parliament should in turn regulate them as it would regulate a criminal conviction and require that courts provide procedural rights that meet or exceed those provided to defendants pleading guilty. Once validly entered, appellate courts should regard these pleas as secure. Defendants who wish to withdraw their \textit{nolo contendere} pleas should be held to the same stringent burden that defendants who wish to withdraw guilty pleas must meet.\footnote{See \textit{R v Taillefer}; \textit{R v Duguay}, 2003 SCC 70 (CanLII), [2003] 3 SCR 307.}

Similarly, just as Parliament statutorily requires judges to conduct a plea inquiry for defendants entering guilty pleas, it should implement a similar requirement for defendants entering \textit{nolo contendere} pleas. Because defendants entering \textit{nolo contendere} pleas do not admit culpability, judges should be doubly careful to ensure that they know the legal consequences of entering their pleas and that a conviction will inevitably result. To ensure that judges take this requirement seriously, Parliament should consider hinging the plea's validity on whether the court conducted the inquiry. This requirement would distinguish a formal \textit{nolo contendere} plea from a guilty plea, reinforcing the importance of this inquiry for \textit{nolo contendere} pleas. Furthermore, this requirement would significantly improve the informal \textit{nolo contendere} plea procedure, where no such inquiry is strictly required.

\section{Subsequent effects}

The \textit{nolo contendere} pleas' subsequent inadmissibility distinguishes it from guilty pleas, giving it utility it would otherwise lack. Absent the historical common law foundation for this feature, however, it is sensible to ask whether Parliament should implement subsequent inadmissibility and, if so, how they should do so. Presently, the informal \textit{nolo contendere} plea procedure presents very few beneficial subsequent effects, the most notable of which is a technical right to appeal the plea. \textit{Nolo contendere} pleas are useful for defendants who cannot or will not admit guilt but are nonetheless willing to self-convict. However, I argue that adding additional incentives to the plea, as many jurisdictions do in the United States, will further expand the pleas reach and utility. 

Although some defendants enter \textit{nolo contendere} pleas as a matter of personal integrity, rendering evidence of these pleas inadmissible at subsequent proceedings will likely persuade many others to do so. Specific cases will likely be more susceptible to these persuasive improvements than others. Defendants charged with major fraud may be convinced to resolve if evidence of their plea was inadmissible at a subsequent civil proceeding. In contrast, prosecutors may convince defendants charged with domestic violence offences to resolve if evidence was inadmissible at subsequent family proceedings. Defendants of all varieties who risk deportation if convicted might be persuaded to resolve their matters if evidence of that plea was inadmissible for immigration purposes. In cases where these convictions come with real advantages to victims and society, such as courts requiring defendants to repay money for misappropriation or damages, subjecting them to rehabilitative probation orders or warehousing risky offenders for a time, self-convictions can be generally beneficial to society. Where defendants have been wrongfully arrested and must suffer wrongful pre-trial punishments, excluding these convictions from subsequent proceedings may function as a way to limit the damage these wrongful punishments may cause going forward.

However, there may be situations where it is in the interests of justice to allow a defendant to enter a \textit{nolo contendere} plea but not in the interests of justice to exclude evidence of this plea at future proceedings. A defendant charged with sexual interference may be convinced to enter a \textit{nolo contendere} plea to allow them to maintain their innocence privately. However, the interests of justice may require that courts admit evidence of this plea at subsequent family law proceedings. Similarly, prosecutors may induce a foreign national defendant charged with workplace fraud to enter a \textit{nolo contendere} plea to avoid admitting evidence in a costly \textit{civil} suit. However, the interests of justice may require that a court admit evidence of this plea at a subsequent deportation hearing. 

One way to adjudicate these nuances is to implement an application system that requires defendants to apply to the court to have evidence of their pleas excluded.\footnote{New Jersey, a state that does not allow defendants to enter \textit{nolo contendere} pleas, uses a system like this for defendants wishing to apply to have their convictions deemed inadmissible in subsequent proceedings. See NJ R Evid rule 410.} Similarly, requiring defendants to complete a probationary period successfully may provide a reasonable precondition to excluding evidence at future proceedings and an incentive for pro-social behaviour and rehabilitation.\footnote{Rhode Island allows defendants to enter \textit{nolo contendere} pleas, but only excludes evidence of these pleas at subsequent proceedings where the sentence included a probation order and where the defendant completed the probation period without violations. See RI Gen Laws Ann § 12-18-3(a).} Adopting approaches such as these, either individually or in tandem with one another, would give Parliament granular control over how and when it may implement these pleas, providing both procedural and substantive advantages.

\section{Summary}

Having reviewed \textit{nolo contendere}'s history and historical ties with plea bargaining, legality and informal application in Canada, and numerous ethical issues with both \textit{nolo contendere} pleas and plea negotiations, I conclude that Parliament should formalize these pleas. Both plea bargaining and \textit{nolo contendere} pleas may be misused to cut corners and conduct cynical negotiations. However, prosecutors and defendants may also use both appropriately to increase access to justice, improve the quality of justice accessed, and allow more defendants to more truthfully and fairly resolve more criminal charges without the uncertainty of a trial. 