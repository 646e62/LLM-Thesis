\section{Plea bargaining and the three categories of moral concern}

Unlike guilty pleas, both non-culpatory no-contest pleas like \textit{nolo contendere} and exculpatory no-contest pleas like \textit{Alford} are almost exclusively the product of plea bargaining. For this reason, in order to answer whether these pleas should be further incorporated into Canadian criminal law, it is necessary to also examine the propriety of plea bargaining. If it is the case that plea bargaining is an implicitly suspect enterprise that should be avoided where possible, this weighs against expanding the use and availability of no-contest pleas.\footnote{See e.g. the Bibas article. Although this is an article on plea bargaining, its subject-matter is limited to \textit{nolo contendere} and best-interest pleas.} However, if plea bargaining is an implicitly worthwhile and valuable process, or even an ethically neutral one, this may weigh in favour of expanding no-contest pleas. This section examines the plea bargaining process and its most common criticisms in order to examine this aspect of the broader question.

Plea bargaining occurs when a prosecutor offers a defendant an incentive to self-convict, ostensibly to ensure efficient case resolution on a broader scale. Plea bargaining has been controversial since its inception, but it remains widely used. Plea bargaining's proponents argue that the efficiencies gained through plea bargaining are necessary to maintain the day-to-day operation of the justice system\footnote{See \textit{R v Butt}, 1987 CanLII 5192 (NL SC) at para 48: ``\textbf{The guilty plea.} The court will usually impose a lesser sentence if the accused has pleaded guilty. The rational [sic] being that a criminal court can only function if it can induce the great mass of actually guilty defendants to plead guilty. The price it pays for this co-operation is leniency. In this case the plea of guilty was to a lesser charge than that contained in the original indictment."} while its opponents argue that these efficiencies come at too high a price.\footnote{Other plea bargaining opponents dispute the notion that plea bargaining provides any net efficiencies to the justice system. See \hl{example}.} In this section, I examine three of the most widely raised objections to plea bargaining, as outlined by Michael Young in his obsequiously titled \textit{In Defence of Plea-Bargaining's Possible Morality}:

\begin{itemize}
    \item \textbf{The trial penalty.} Defendants who plea bargain with the prosecution do so to secure a more lenient sentence than they would receive after trial. This creates a system where defendants receive dissimilar sentences for similar crimes, thus exacting a penalty on defendants convicted after taking their matters to trial. The trial penalty concern stems from a perceived violation of the principle of equal treatment under the law.
    \item \textbf{The coercion worry.} 
    \item \textbf{Wrongful convictions.}\footnote{Young refers to this as ``the innocence problem," but the problem described is effectively a wrongful conviction problem. See Young at 257: ``The so-called `innocence problem' reflects the worry that plea-bargaining leads to the conviction of too many wrongfully-accused innocents, either overall or with respect to some identifiable subset of prosecutions."  I have chosen to go with the latter terminology for consistency's sake.} 
\end{itemize}

Young refers to these as the ``three main categories of moral concern."\footnote{See Young at 256. Defendants who enter inculpatory no-contest pleas because they wish to take responsibility for the offence charged may benefit from plea bargaining incidentally but do not raise the ``moral concerns" that this section examines. Rather, these concerns arise when the offered bargains threaten to override the defendant's free will or improperly induce defendants to self-convict.} In this section, I will explain these categories and examine the concerns in detail. 

\subsection{The trial penalty}

The trial penalty criticism charges that because defendants who opt for a trial rather than a plea bargain are likely to be sentenced more harshly if convicted, they suffer a ``trial penalty" for this decision. Where two otherwise equally situated defendants are convicted of the same crime, but one of them accepts an early plea deal in lieu of a trial, the one who accepted the plea deal will likely receive a more lenient sentence than the one who did not.\footnote{} The apparent unequal treatment these two defendants receive is the subject of the trial penalty criticism.\footnote{See Young at 269: 
\begin{quote}
    Few criminal defendants would plead guilty absent a credible chance of reducing their expected sentences. Plea-bargaining thus depends on the possibility of a range of sentencing options, and on differential treatment for plea-bargaining and non-plea-bargaining defendants. This effectively means that not all criminal defendants will be treated alike, and that non-plea-bargaining defendants will be (or, if declining a plea-bargain, can reasonably expect to be) sentenced relatively more harshly. ... [T]he critic thinks that this differential treatment betrays a troubling lack of concern for equality among similarly situated criminal defendants, and is a violation of the ancient and venerable maxim of justice to treat like cases alike.
\end{quote}}

Young examines and rejects several responses to this criticism before proposing his own:

\begin{itemize}
    \item \textbf{Recasting the trial penalty as a plea bargain benefit.} 
    \item \textbf{Reducing penalties in exchange for prosecutorial resources, the common good, or both.} 
    \item \textbf{Re-examining the assumptions underlying equal treatment and the defendant's role in the discrepancy.} 
\end{itemize}

\subsubsection{Recasting the trial penalty as a plea bargain benefit}

The trial penalty criticism charges that defendants who set their allegations down for trial, rather than opting to negotiate a sentence through plea bargaining, are punished for their decision to do so. Plea bargaining proponents may reasonably counter that there is no ``penalty" \textit{per se}, but rather a ``reward" for resolving their charges without the need for a trial. Those who take this position may argue that the trial penalty criticism misunderstands plea bargaining. Where the plea bargaining opponent sees a penalty, the proponent sees the normal sentence that a defendant ought to receive as a penalty for the particular offence. However, as Young points out, shifting the emphasis from the ``penalty" of setting a charge for trial onto the ``benefit" of accepting a plea bargain is insufficient, primarily because the criticism is aimed at the \textit{inequality} that plea bargaining creates. Recasting the inequality in more favourable terms is mere public relations, and fails to address the inequality underlying the concern.

\subsubsection{Reducing penalties in exchange for prosecutorial resources, the common good, or both}

Young next suggests that these supposedly equally situated defendants are not equally situated. Where two such defendants are charged with the same offence and one opts to plea bargain, one may fairly ask whether they are truly equally situated, given that one saves ``public resources on an expensive prosecution" while the other does not. To the extent that saving the state the expense of a trial contributes to the public good, and to the extent that contributing to the public good is morally praiseworthy, defendants who plea bargain are more morally praiseworthy than their counterparts and should be treated as such. 

Young rejects this argument on the basis that prosecutorial resources and punishment are not commensurate commodities. Prosecutorial resources are, in effect, monetary resources, and Young argues that allowing one to be exchanged for another invites unjust results.\footnote{See Young at 271: ``It would be perverse, for example, to think that a rich criminal who donated money to the prosecutor's office — thus advancing the common good by providing resources for prosecutions — would even presumptively deserve a sentencing reduction for that reason alone."} Similarly, where a defendant commits a crime with serendipitous consequences,\footnote{A sexual assault that helped spark a movement that raised society's awareness of the problem more generally, destruction of property generally considered to be offensive or an eyesore, and medical malpractice that ended the life of a genocidal dictator are examples of crimes with arguably serendipitous consequences.} they should not be rewarded for these incidental benefits.\footnote{See \textit{ibid}: ``[S]uppose that through some lucky causal accident the commission of some crime also brings about collateral social effects contributing positively to the common good: we would not think that this fortunate fact properly weighed as a sentencing consideration. So it cannot be the case that, in general, contributing to the public good constitutes a reason for a reduction in sentence.} Absent a more direct correlation between the common good and moral praiseworthiness, Young argues that this is insufficient to truly distinguish the plea bargaining defendant from the defendant who elects for trial.

Young correctly points out that merely saving state resources is insufficient to distinguish one person's moral praiseworthiness over an other's. This example, however, is a hasty generalization and a straw man. While certain financial arrangements between defendants and the prosecutor's office, if made, would be objectively seamy, separating criminals from their money remains a time-honoured and occasionally effective punishment. Furthermore, Young fails to consider other reasons why a defendant who foregoes their right to a trial may be more morally praiseworthy than one who does not. Finally, Young's disregard for defendants who incidentally cause some common good or another ignores the fact that the law ``rewards" (or, at a minimum, refrains from punishing) ``moral luck" all of the time.

Whenever a judge imposes a fine on a defendant, they signal that monetary resources and punishment are commensurate resources. In Canada, Parliament has also recognized that there can and ought to be a monetary value placed on time spent in custody, as well as a custodial value placed on misappropriated funds.\footnote{See \textit{Criminal Code} s 462.37(4), which may be formally expressed as: 
\begin{equation*}
    sentence = \begin{cases}
        \le \text{6mo} \quad  \iff fine \le \text{\$10,000} \\
	[\text{6mo, 12mo}] \quad \iff fine = [\text{\$10,000, \$20,000}]\\
	[\text{12mo, 18mo}] \quad \iff fine = [\text{\$20,000, \$20,000}]\\
	[\text{18mo, 2y}] \quad \iff fine = [\text{\$50,000, \$100,000}]\\
	[\text{2y, 3y}] \quad \iff fine = [\text{\$100,000, \$250,000}]\\
	[\text{3y, 5y}] \quad \iff fine = [\text{\$250,000, \$1,000,000}]\\
	[\text{5y, 10y}] \quad \iff fine > \text{\$1,000,000}\\
    \end{cases}
\end{equation*}} 
Where convicted defendants have spent time in pre-trial custody and are sentenced to a fine, some or all of that fine amount may be noted in default in accordance with the ``time is money" provision outlined in \textit{Criminal Code} ss 734(4) \& (5).\footnote{The calculation provided in \textit{Criminal Code} s 734(5) may be formally expressed as: 
\begin{equation*}
    sentence =\begin{cases}
        x_1 \quad  \iff \frac{fine + costs}{minWage \times 8} < (maxSentence \lor 5/2y) \\
        x_2 \quad \iff (maxSentence \lor 5/2y) < \frac{fine + costs}{minWage \times 8}  \\
    \end{cases}
\end{equation*}}
Additionally, statutory victim fine surcharges,\footnote{See \textit{Criminal Code} s } restitution orders, forfeiture orders, and judge-ordered donations are all means by which the court correlates moral blameworthiness with the very same resource used by prosecutors to bring cases to trial. There is a strong correlation between financial resources and punishment, such that there is nothing perverse about saying that a defendant who has been punished financially is no longer equal to another defendant who has not.

Young's criticism of this view of equal treatment is more severely undermined by his almost exclusive focus on prosecutorial resources. Indeed, if sparing the state the \textit{financial expense} of a trial were a defendant's only \textit{quid pro quo}, it would be reasonable to wonder whether this was enough to render otherwise equal defendants unequal. But this is not the case. Indeed, the financial expense of running a trial (and the attendant man-hours that prosecutors and publicly-funded defence attorneys must put into preparing for trial) are arguably the least important aspects of foregoing a trial. 

Defendants who opt to self-convict rather than take their matters to trial relieve the state of its burden to prove the offence. As a result, witnesses who would have had to have testified are no longer required to do so. Where those witnesses are vulnerable individuals, they are spared the potential trauma of having to revisit the offence. Where those witnesses are unreliable individuals, the state is spared the difficulty of having to elicit evidence from hostile parties. All witnesses are spared the pressure of having to come to court and provide evidence. More importantly, self-convicting defendants guarantee that the state will convict them. As will be discussed in greater detail below, trials are inherently volatile processes, and convictions are never guaranteed, even in cases where the evidence appears to be overwhelmingly strong. A self-convicting defendant not only spares the state the expense of a prosecution, they also spare the state the risk of an expensive \textit{and unsuccessful} prosecution.

Finally, although Young charges that defendants should not receive any consideration for accidental or incidental contributions to the common good, this position largely ignores the way that the law has dealt with the ``moral luck" phenomenon. 



\subsubsection{Re-examining the assumptions underlying equal treatment}

Young takes issue with the naive ``distributional" model of inequality, wherein two identically situated defendants who are charged with identical crimes premised on identical facts should not be sentenced differently if one opts to take their matter to trial while the other opts to self-convict. Instead, he proposes substituting a ``relational" model in its place. Where the distributional model of equality judges equal treatment based on how evenly (or unevenly) punishment is distributed across similarly situated defendants, the relational model asks whether ``someone or some group [being] subordinated or dominated, or in some similar way treated without proper respect or concern."\footnote{See Young at 272.}

This implication presents a burden for the "trial penalty" critic who is committed to saying that the differential sentencing of defendants in a system that allows plea-bargaining always or necessarily reflects the inequitable subordination of the (more harshly sentenced) convicted, non-plea-bargaining defendant. To credibly maintain this position, the "trial penalty" critic must be strongly committed to denying that the defendant who declines to plea-bargain, is convicted, and faces a harsher sentence could be responsible for that sentencing outcome, even under ideal conditions. That is, the "trial penalty" critic must be strongly committed to denying that the sentencing outcome could reflect a choice for which the defendant is properly responsible. If, pace the critic, the defendant could be responsible for his choice to elect trial instead of a plea-bargain, then his choice could be free of a concern for inequality or "penalty." But this necessary critical commitment seems too strong. Without some independent reason for doing so, we should not want to insist in advance that there are no conditions under which a defendant could be responsible for the choice whether or not to plea-bargain and that choice's predictable sentencing consequences.

Assuming that similarly situated defendants have equivalent access to plea bargain offers, Michael (III) Young argues that those who turn them down can and should be held responsible for the consequences of their choice to do so

Michael (III) Young suggests that defendants should be and are in fact aggrieved when the offers they receive are worse than those made to similarly situated defendants

To the extent that no two offences or offenders are ever exactly alike, "similar offences" and "similarly  situated offenders" will always be distinguishable to some degree

The incredible array of subtle distinctions between offences and offenders can either be overlooked or microscopically examined, depending on things like

The offence

The offender's history

Jurisdiction the offence occurred in

The judge deciding the disposition

How these factors are interpreted can be very imprecise

Evidenced by the seemingly endless parade of sentencing ranges, guidelines, cases, and so forth

\subsubsection{Conclusion}

Defendants who opt to self-convict, rather than take their chances at a trial, ought to have this sacrifice recognized

A concept like equal treatment of similarly situated defendants cannot reasonably ground a criticism of plea bargaining

\subsection{The coercion worry}

Three distinct normative approaches commonly emerge in plea bargaining literature:

\begin{enumerate}
    \item coercion as restricting a defendant's rational choice and utility;
    \item coercion as requiring a defendant to make a choice with high stakes; and
    \item coercion as overriding a defendant's will by wrongfully influencing them
\end{enumerate}

A coercive choice can be rationally made
A choice can be rationally made in one's own self-interest but still be the product of coercion.

\begin{quote}
    In many cases, it may seem that accepting an offered plea is the only rational choice left for a defendant. Reflecting on this, a critic may try to understand coercion as consisting in an overwhelming amount of rational pressure in favor of a particular choice, namely, the choice to plea. But this approach also generally fails, on reflection, to provide an adequate account of coercion. Consider the example of a person faced with a choice between continuing a lackluster career and accepting his dream job. We do not think that that person is "coerced" into accepting the dream job offer just because that option has so many more reasons recommending it." But then coercion cannot be generally identified with the bare existence of disproportionately weighted reasons favoring some particular choice. A choice is not coerced simply because there were no better options, or simply because all other options were much worse from the chooser's perspective.
\end{quote}

Money to a mugger as an example of coercion

\begin{quote}
    If a mugger pulls a gun and credibly threatens "your money or your life," the victim does not fail to be coerced because he chooses to do the utility-maximizing, rational thing by handing over his wallet. Coercion and rational choice are not mutually exclusive possibilities, as the mugging example shows. But then one cannot sensibly point to the existence of rational choice in answer to a charge of coerciveness. Some given choice may have been rational. It could still, for all that, also have been coerced.
\end{quote}

Influences are coercive when they interfere with the freedoms a person ought to have

Identifying coercion is a more difficult task than the first two approaches seem to appreciate.
\begin{quote}
    Fundamentally, coercion is not about the amount or quantity of influence on a choice, but rather about the kind of influence on choice. On this view, even subtle influence could be coercive if it were an influence of a particular wrongful kind. As one plausible idea with support in the law and philosophy, an influence is of a wrongful kind if it interferes with the sort of positive freedom we think a person generally ought to have. So, for example, influencers that make a choice something less than an exercise of autonomous agency are suspect. Similarly, we might view as coercive any arrangement or situation that denies a person options that he ought to have, given an independently plausible view of his proper rights and dignity.
\end{quote}

A substantial gap between a plea bargained offer and a post-trial sentence recommendation suggests a high-stakes choice

Where a prosecutor offers a very low sentence on a guilty plea, as opposed to a comparatively very high sentence upon conviction after trial, they may be said to be engaging in hard dealing. To the extent that this tactic is likely to result in the defendant accepting the very low sentence offer, it's suggested that the defendant is coerced into making that choice.

\begin{quote}
    The critic of plea-bargaining typically has a rough-and-ready story that fits with this more plausible general conception of coercion. Especially, critics fear the supposed coercive pressure that results when defendants "are led to believe that they will receive longer sentences if they insist on going to trial and are subsequently found guilty."' The critics worry over the case where prosecutors succeed in inducing guilty pleas by threatening defendants with extra charges and (likely) extra punishment so that the defendants feel pressure to accept whatever plea is offered." In such cases, the critic might say, the defendant is confronted with a choice he should not have to face, and that alone makes the choice coerced.
    
    On this view, coercive pressure exists based solely on a large enough gap between (1) the prosecutor's threat of punishment at trial, and (2) the likely punishment given the defendant's acceptance of the prosecutor's offer and a plea. For convenience-and because the critic likely will not protest-let us label as "hard dealing" any and all cases marked by sufficiently large differences in punishment between threat and offer. (We will let the critic quantify "sufficiently large" however she likes.) Cast in this vocabulary, the critics' claim: "hard dealing" is intrinsically objectionable, constitutes the source of wrongful coercion in plea-bargaining, and, consequently, makes the institution of plea-bargaining morally suspect.
\end{quote}

Not all hard deals are coercive
A prosecutor who shows inordinate leniency in a case may create a sentencing gap sufficient to qualify as "hard dealing". But to the extent that the prosecutor is offering a more favourable choice than the defendant might normally be entitled to expect, it can't be reasonably said that the "hard deal" was coercive.

\begin{quote}
    [W]e sometimes think that there may be a range of sentencing options that are not coercive to the particular criminal defendant. Cases of prosecutorial leniency form the easiest example. Imagine an appropriately charged defendant facing a stiff sentence who is then offered, and accepts, a lenient slap-on-the-wrist plea deal. Whatever we want to say about this case, we presumably will not want to say that the defendant was coerced into accepting the deal. (What could be the source of the illicit coercive pressure in this scenario?) Yet, unless we are constructing the category of "hard dealing" to beg the question-simply as an alternate form of words for the ultimate conclusion that plea-bargaining is coercive-the sentencing gap between the appropriate stiff sentence and the wrist-slap could be large enough to meet any parameters that the critic may set for "hard dealing." If so, then there may be a case of "hard dealing" that is not coercive. But then it follows that "hard dealing" is not necessarily coercive and so it cannot be "hard dealing" as such that constitutes coercive pressure.
\end{quote}




\subsection{Wrongful convictions}

It goes without saying that factually innocent people shouldn't be convicted of crimes they didn't commit

Unfair result

Undermines confidence in the administration of justice

Leaves the factually guilty unpunished and undeterred

Concerns about wrongful convictions are live even when discussing run-of-the-mill plea bargaining

The added dimension of a no-contest plea, however, amplifies this concern

\subsubsection{Quantifying wrongful convictions}

The quantification question asks how many wrongful convictions are too many. Judging by the histrionic bloviations so often attending any discussion of wrongful convictions, even asking this question appears to miss a greater moral point. Wrongful convictions have been called a ``scourge," ``a blight to the criminal justice system,"\footnote{} and ``one of the worst nightmares imaginable."\footnote{See Huff at 15.} Wrongful convictions must be ``prevented or eliminated as far as possible,"\footnote{See Sheehy at 984.} ... 

These concerns are greatly overstated and largely ill-considered. While there is no doubt that wrongful convictions are a serious problem that 

\subsubsection{The moral case \textit{for} wrongful convictions}

@bowersPunishingInnocent2007a argument

The fact that these defendants are typically recidivists doesn't mean they aren't entitled to the same due process as a wrongfully accused defendant without a criminal record

But it does often mean that these two classes of defendants will expect and wish to exercise very different levels of due process

Recidivists more inclined to seek early resolution on the most favourable terms possible

Inexperienced defendants more inclined to want every avenue to acquittal explored

The actual jeopardy that a criminal conviction poses will be different for the recidivist than it does for the unexperienced defendant, so it's reasonable to expect that each would treat that factor differently

Weighing the harm of wrongful convictions against the harm of wrongful incarceration

The rhetoric around wrongful convictions is such that they're often framed as the worst possible outcome in the criminal justice system

But it's fair to question whether this is the case

Specifically, whether wrongful punishment (and specifically incarceration) is worse than a wrongful conviction

If it's accepted that some wrongful punishments are worse than some wrongful convictions, then it follows that wrongful convictions are not always the worst outcome in a criminal case

This opens up the possibility that being wrongfully convicted of a particular offence may be better than being wrongfully punished for it

Would it be worse to be wrongfully convicted of an offence and be absolutely discharged without having spent any time in custody, or be denied bail, spend a few months in pre-trial detention, and ultimately be acquitted?

Or perhaps be granted bail after spending just one month in pre-trial custody, and ultimately be acquitted?

Allowing a defendant in this situation to enter an equivocal no-contest plea would provide them a way to curb the harms of wrongful punishment and other pre-trial hardships

Distinct from a conditional plea, in that the wrongful conviction is accepted without any expectation that the conviction can or will be reversed after a future proceeding

Subjecting a wrongfully accused defendant to a greater punishment while waiting for trial than they would receive if convicted after trial is unjust

Requiring that person to further feign guilt and remorse for something they didn't do is also unjust

As discussed in 4.1.3: The apocryphal truth-seeking function of the trial, the idea that trials are a better truth-seeking mechanism than negotiations between counsel is likely misguided

Risk-adverse defendants or those not looking to clear their names are unlikely to want to wait for a trial if they can otherwise resolve their charges on less punitive terms

The specter of wrongful convictions is a live concern

But seeing plea bargaining and equivocal no-contest pleas as the cause of wrongful convictions misapprehends the matter

The biggest problems associated with wrongful convictions begin gestating long before the defendant is required to enter a plea

By the time a defendant is faced with having to enter a best-interest plea, something has irretrievably failed in the process

At this point, there is nothing morally wrong with allowing the defendant to try to mitigate future harms

Ideally no defendant would find themselves in this position

But where they are, having a means to more quickly get them out of it is better than having a principled stance derived from Blackstone's ratio

\subsubsection{The apocryphal truth-finding function of trials}

Because defendants unwilling or unable to offer a guilty plea must set their matters down for trial, a trial may be considered the ``default mode" for resolving criminal disputes. Because trials take place when parties disagree with one another, they are often branded as a ``search for the truth." Each party has the opportunity to present its best evidence and its best interpretation of the evidence. Witnesses are cross-examined on their evidence, identifying the strengths and weaknesses of their testimony, and a neutral trier hears the facts, listens to both sides argue, and decides between them.

By contrast, non-inculpatory no-contest pleas appear to be designed to obfuscate or misrepresent the truth for the purpose of getting negotiated agreements through the courts. Negotiations between prosecutors and defence lawyers are privileged and inadmissible as evidence. There are no judges or juries to provide neutral adjudication.\footnote{This is not necessarily true in all cases. In many jurisdictions, contested matters may go through case management with a judge in order to discuss pre-trial issues, canvass resolution, and offer advice on how to best proceed. }

Defendants can privately (or publicly) resile their admissions without being contradicted by evidence heard at a public trial

But the function of a common-law criminal trial is not to "search for the truth." Rather, the function of a common-law criminal trial is to allow the trier of fact to make decisions based on curated sets of facts, within the confines of a restrictive sets of rules. There are important differences between these two concepts. 

Seeing a trial as a "search for the truth" implies that truth is the primary function of the trial, which subsequently implies that truth is the value in a trial that all other values are subservient to. A public inquest into a social problem, like the Truth and Reconciliation hearings in ZA, or missing and murdered indigenous women in Canada, would be comparable examples of this sort of "no holds barred" approach to the truth. 

Trials, on the other hand, require the fact finder to consider a less complete and arguably less true version of the evidence, as in cases where evidence has been excluded. In criminal trials, the "truth-seeking function" of the trial is always counterbalanced by the need to ensure fair proceedings. Truth is, in fact, severely curtailed at trial by things like what truth the trier of fact is allowed to hear, excluded evidence, excluded testimony, the results of pre-trial and other evidentiary motions, testimony from witnesses who were not called or did not attend, hearsay rules, opinion evidence rules, and so on. Even once the evidence is admitted, special instructions are often needed to ensure that the evidence is interpreted properly. 

Practically speaking, a great deal of truth that could be presented at trial is often excluded simply for the sake of a coherent presentation

In all cases, but especially complex ones, lawyers looking to present their evidence to a judge or jury will regularly cut out large swaths of the available evidence

A lawyer's job at a common law criminal trial is to advocate for a position

Effectively advocating for a position requires the advocate to account for the frailties of human attention and comprehension

This in turn requires both extensive editing of the truth and careful framing of that truth within the context of a larger argument

Information at trial is highly pre-processed before it reaches the court, and needs to be further processed once it gets there. This ensures both procedural fairness and effective presentation of the truth.

Procedural fairness and presentation are key trial concerns because the truth-seeking mechanism is unreliable

The "finders of fact" at trial are no more implicitly qualified at "seeking truth" than anyone else

In the case of jurors, the "finders of fact" are, in effect, "anyone else"

Experience

The hope and expectation is that the fact finders will come to true and fair conclusions through the combined presentation of true and fair evidence

But this isn't guaranteed

Rules have been put in place following decades of learning about the frailties of evidence

DNA evidence created a new class of cases for courts to review

The theory is that certain convictions could be absolutely confirmed or disconfirmed based on how DNA tests came back

Where wrongful convictions were found, we were able to gain some insights into the types of evidence that led jurors to those conclusions

Other forensic experiments in other fields have confirmed the unreliability of certain types of evidence

Eyewitness evidence

Shortcomings in memory

Etc

More confirmed wrongful convictions appear to have resulted from trials rather than negotiated sentences

But until we have new breakthroughs that can offer insight into how well or poorly forensic evidence reflects the truth, there isn't another independent truth-seeking function at play

Other truth-seeking mechanisms, like resolution discussions and plea bargaining, are likely better suited for reaching outcomes that are both true and fair

Both (sets of) lawyers in a criminal action have intimate familiarity with the facts and nuances of the case. They represent the different interests in the action - or at least, the interests that have standing in the action. Assuming ethical actors with access to relatively complete and accurate information, any resolution worked out between the parties by consent is likely to be close to a true and just result.

This also assumes there's some *quid pro quo* between the parties in their resolution discussions, such that the evidence is already being "tested", in a sense

Where a resolution can't otherwise be reached, trials are a reasonably effective means of adjudicating disputes, ensuring all sides have their best explanation of the evidence heard, and deferring to an ostensibly neutral third party to decide between competing narratives.

But trials are probably better understood as adequate alternatives to dispute resolution, rather than a preferred means of dispute resolution

Negotiations between counsel, both of whom have legal training, are duty-bound to advocate for their positions, have access to the fullest permissible truth of a criminal matter, and the opportunity to spend extensive time and resources investigating the nuances thereof, are a much greater truth-seeking device than a criminal trial.
