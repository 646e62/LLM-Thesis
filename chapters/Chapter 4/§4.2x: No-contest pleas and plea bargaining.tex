\section{No-contest pleas and plea bargaining}

In order to have a meaningful discussion about no-contest pleas, it is necessary to address plea bargaining. Doubtlessly many defendants who self-convict do so for selfless or pro-social reasons, such as a desire to spare a complainant from having to testify at trial or because they are genuinely remorseful for their actions. However, many defendants who are otherwise reluctant to admit responsibility may be willing to do so in exchange for some prosecutorial benefit. This transaction, and the processes that underlie it, are what is known as plea bargaining, which Albert W Alschuler defines as "the exchange of official concessions for a defendant's act of self-conviction," \footnote{@alschulerPleaBargainingIts1979} including favourable sentencing recommendations or the chance to plead to a lesser included offence. 

Plea bargaining presumes that some form of \textit{quid pro quo} exchange exists in most, if not all, criminal prosecutions. Prosecutors control whether or not charges exist, while defendants control whether the court will have to adjudicate the matter at trial. More specifically, both sides face specific pressures imposed by the structure of the justice system. For example, defendants detained or released on conditions may be motivated to resolve their matters efficiently by a desire to end their pre-trial restrictions. In contrast, prosecutors may be motivated to resolve their matters efficiently by the constant threat of a successful delay motion and a judicial stay of proceedings. Each party has leverage over the other, thus paving the way for each party to deal with the other. Because plea bargaining dramatically reduces the demand for the judicial resources needed for trial, provides tangible benefits to defendants, and results in a voluntary criminal conviction, it has become standard practice throughout Canada and the United States. Nevertheless, it remains a controversial practice because plea bargaining pressures defendants to resolve their matters, leads to disparate sentences, and almost certainly results in wrongful convictions.

This thesis will detail the intersections between plea bargaining and no-contest pleas in §§4.2 - 4.5 below. For clarity, I summarize the broad arguments of both positions below and provide an overview of the argument I will develop throughout this thesis.

\subsection{Fairness and accuracy: opposition to plea bargaining}

The academics and jurists who have come out against plea bargaining have voiced their opposition in many ways. However, the contours of their criticisms generally tend to fall within a few definable categories of concern:

\begin{itemize}
\item \textbf{Coerced pleas.} The deals that prosecutors offer defendants, coupled with the explicit and implicit consequences of refusing to accept those deals, amount to coercion and result in involuntary self-convictions.
\item \textbf{Adverse sentencing results.} Hastily offered and accepted plea deals encourage a culture of convenience and lead to an arbitrary and inequitable trial penalty for defendants who opt to contest their charges.
\item \textbf{Wrongful convictions.} Defendants who are innocent of the alleged crimes may opt to self-convict, and plea bargains may help induce them to do so.
\item \textbf{Substantive criminal law.} Although the concerns about unfair and inaccurate proceedings are essential, plea bargaining has also negatively influenced the substantive development of the criminal law in areas like rehabilitation and instilling a sense of responsibility in offenders. 
\end{itemize}

\subsubsection{Coerced pleas}

Since the 16\textsuperscript{th} century, common-law courts have sought to ensure that defendants who enter guilty pleas do so voluntarily.\footnote{Citation.} Criminal defendants who are threatened or feel compelled to enter guilty pleas involuntarily should not be allowed to do so. Similarly, defendants who are induced or bribed into involuntary guilty pleas should not be allowed to do so. Plea bargaining opponents worry that the plea bargaining system is such that defendants are indirectly induced or threatened into pleading guilty. 

Not all inducements and threats, or other oppressive conditions, are sufficient to override a defendant's will. Some degree of \textit{quid pro quo} exists between the state and a criminal defendant in every criminal case. However, most plea bargaining opponents would conclude that the inducements that naturally accompany criminal charges are necessarily oppressive, that the balance of power skews heavily towards the prosecution, and that involuntary self-convictions are inevitable. The legitimate concern is that some defendants are being improperly induced or threatened into giving up their chance to have legitimate factual or procedural issues tried. If true, these inducements and threats diminish their agency and access to justice, sometimes in the guise of increasing both. 

The problem of involuntary pleas is a live consideration for all no-contest pleas but is an acute problem for \textit{nolo contendere} and best-interest pleas specifically. After all, an appellate court may be able to point to a defendant's admission of guilt at sentencing as evidence that their plea was valid. However, it would be more difficult for that same court to draw that clear inference from a defendant's stoic acquiescence when pleading \textit{nolo contendere} or their outright claim of innocence when pleading according to their best interests.

\subsubsection{Adverse sentencing results}

In addition to the problems caused by involuntary guilty pleas, plea bargaining opponents warn that defendants who engage in plea bargaining may be setting themselves up for trouble regarding sentencing. These difficulties can manifest in the process of \textit{being sentenced} and \textit{serving a sentence}. Plea bargaining further creates a ``trial penalty'' for those who opt to take their charges to trial by creating a substantial divide between the sentence sought on early resolution and the sentence sought upon conclusion of calling evidence.\footnote{This concern is related to the concern about involuntary pleas, but distinct from it all the same. Where the involuntary plea concern focuses on the psychological effects that plea bargaining has on defendants who engage in it, the trial penalty concern focuses on the adverse sentencing results that it produces.}

Plea bargaining opponents argue that plea bargaining encourages prosecutors to make precipitous resolution offers, prompting defendants to accept them hastily. Opponents argue that plea bargaining in these circumstances leads to a culture in which sentences are jointly recommended out of convenience rather than a careful and considered analysis of the circumstances of the offence or offender. \footnote{See @irelandBargainingExpedienceOveruse2015} Frequently, these plea bargain offers are made to defendants on a time-sensitive basis, thereby adding extra pressure on defendants who may or may not otherwise be considering self-conviction. Additionally, where offers are made early in the proceedings, the ``facts" that the prosecutors expect to rely on in support of the self-conviction constrained both by a limited amount of evidence and the source of that evidence. By accepting early resolution offers, defendants forfeit the ability to receive more nuanced disclosure or investigate possible defences that may not be apparent at the early stages of disclosure.

Any defendant who opts to retract their self-conviction should expect to face an uphill battle. Even in the face of compelling evidence in a defendant's favour, a defendant who knowingly and voluntarily admits the prosecution's case against them will rarely succeed in convincing the court to allow them to withdraw their plea. Unless the evidence is otherwise statutorily precluded, as is often the case with \textit{nolo contendere} pleas, the defendant who enters pleas to unfavourable or vague facts should expect that those facts may be used against them in subsequent proceedings. Those who self-convict but maintain their innocence, either privately or publicly, may be unable to comply with court-ordered conditions and liable to be criminally charged.\footnote{See the YOA case from Ontario, discussed in Chapter 3.} Defendants who insist on their innocence may be limited in other respects, finding themselves unable to access certain programming while in custody or ineligible for early release on parole. 

Plea bargaining opponents dispute that any benefits that accrue to defendants through the plea bargaining system are adequately offset by the costs in the accuracy of the pleadings and procedural fairness to the defendant. 

\subsubsection{Wrongful convictions}

Wrongful convictions are a seriously adverse phenomenon that common law legal systems have gone to significant lengths to distance themselves from and avoid. Though legal aversion to wrongful convictions is not new, it has also not historically been universal.\footnote{Inversions of Blackstone's ratio appear to have been common amongst Chinese and South Asian communist uprisings, and have even made their way into the American political discourse. See \url{https://en.wikipedia.org/wiki/Blackstone\%27s_ratio\#Viewpoints_in_politics}} But British, American, and Canadian jurists have generally tended to adhere much more closely to the principle of Blackstone's famous ratio that ``it is better that ten guilty persons escape than that one innocent suffer.'' The law's commitment to the presumption of innocence and the burden of proof beyond a reasonable doubt are both employed in the service of trying to ensure that only the guilty are punished, even if that means letting some go unpunished. Wrongful convictions are therefore very much at odds with the objectives and optics of our respective legal systems.

There are reasons to suspect that our legal system's fears of wrongful convictions may be overstated in some ways and misplaced in others.\footnote{See the Bowers article.} But there can be absolutely no doubt that wrongful convictions occur and that the wrongfully convicted are wrongfully punished. To the extent that plea bargaining encourages these pleas to be entered, fosters a culture of case efficiency over case evaluation, and makes it easy for defendants to check themselves through this system, it both encourages wrongful convictions and ensures that they inevitably occur. To the extent that this is in conflict with a deeply-held and intuitively strong tenent of our legal system, plea bargaining opponents argue it should be done away with it.

\subsubsection{The impact on substantive law concerns}

While most of the concerns that plea bargaining opponents have are commonly held, some opponents have raised novel but influential considerations. Amongst these opponents is Stephanos Bibas, since appointed as a judge to the US Court of Appeals for the Third Circuit, whose ``substantive law concerns'' are both unique and compelling. As Bibas casts many of his concerns for plea bargaining generally against the backdrop of \textit{nolo contendere} and \textit{Alford} pleas specifically, his criticisms of plea bargaining are especially pertinent to this thesis. Bibas recognizes that most debates over plea bargaining split along the fault lines of efficiency and autonomy versus accuracy and fairness, but proposes that these \textit{procedural} concerns ought to be sublimated under more pressing, and too often neglected, \textit{substantial} concerns. Rather than allow for these procedural concerns to dominate the discussion of the propriety of plea bargaining, Bibas proposes that the conversation shift instead towards substantive values, such as rehabilitation, education, retribution, and denunciation.

\subsection{Efficiency and autonomy: plea bargaining proponents}

On the other side of the plea bargaining controversy are the proponents who point to the following advantages in defence of plea bargaining. Plea bargaining proponents focus on its uses in the efficient administration of justice, the benefits that defendants who are able to secure a plea bargain receive, and the advantages of the certainty of a conviction without the hardship and uncertainty of a trial:

\begin{itemize}
\item \textbf{Procedural efficiency.} Without plea bargaining, our system of justice would not be able to function as well as it currently does. If plea bargaining is problematic, it is nonetheless a ``necessary evil'' that allows greater access to and effective administration of justice overall.
\item \textbf{Increased autonomy.} Plea bargains increase the number of options that a criminal defendant has at their disposal for resolving their charges. Offers may lead to counter-offers, with a wide range of potential variations for both parties to propose. Giving defendants access to these possibilities increases their ability to determine their outcome, thereby increasing their autonomy.
\item \textbf{Certainty.} Because the outcomes of criminal prosecutions are frequently both uncertain and high-stakes, ensuring a degree of certainty in proceedings is a good enjoyed by all parties to the proceedings. This includes defendants, victims, and witnesses, as well as lawyers, judges, and everyone else tasked with the effective and consistent administration of justice.
\end{itemize}

\subsubsection{Increased procedural efficiencies}

The practice of plea bargaining has benefited greatly from its perceived necessity, especially amongst members of the bar and the judiciary. Although the practice is frequently (but not universally) panned by academics and criminal justice advocates, it is much more comfortably tolerated both by practicing lawyers who engage in plea bargaining, and the judges who hear and accept jointly bargained pleas. 

The Supreme Courts in both Canada and the United States have routinely upheld plea bargaining practices and underscored the important function it performs in our justice system. In \textit{Santobello v New York}, 404 US 257 (1971), the Supreme Court of the United States affirmed that the plea bargaining process is a crucial part of the criminal justice process whose agreements ought to be upheld. In Canada, 25 years later, the Supreme Court of Canada followed suit in \textit{Burlingham},\footnote{\textit{R v Burlingham}, 1995 CanLII 88 (SCC), [1995] 2 SCR 206 (\textit{Burlingham}).} by identifying plea bargaining as ``an integral element of the Canadian criminal process.'' Although \textit{Criminal Code} s 606(1.1) is clear that judges in Canada are not required to uphold plea agreements, the Supreme Court of Canada has made it clear that truely jointly recommended sentences are entitled to a great deal of deference from the judge who hears the recommendation.\footnote{See \textit{R v Anthony-Cook}, 2016 SCC 43, 2016] 2 SCR 204}

Plea bargaining works to reduce the number of cases set down for trial without a viable or likely defence. Defendants who are facing a significant sentence are highly motivated to set their matters down for trial, even where they have no obvious defence to the charges. The possibility of a procedural irregularity, a change in a witness's willingness to cooperate, or a favourable evidentiary ruling can are every reason to set a matter down for trial. Offering that defendant the option of a less significant sentence before trial incentivizes defendants in the reverse direction, testing their resolve and commitment to their trial request. This alone separates much of the wheat from the chaff.

For example, if the Crown's case relies entirely upon the testimony of a potentially uncooperative or reluctant witness, the defendant may wish to set their matter down for a trial in the hopes the witness fulfills that potential. If that same defendant is given an offer to plead guilty to that charge for a reduced sentence, or some other consideration, with a promise to pursue the full sentence without consideration after trial, they may be inclined to take the safest path. This incentive is made much stronger when a defendant is remanded into custody, sets a trial date, and is offered a sentence that will guarantee they do less time overall. The process can be extended to all defendants, including those with legitimate defences, further reducing the number of unnecessary trials. Encouraging defendants to resolve these lightly contested or functionally uncontested matters helps triage cases early, leads to less inefficient last-minute trial resolutions, and allows the courts to continue functioning with the resources they've been allotted. 

\subsubsection{Increased autonomy for defendants}

When a person becomes a criminal defendant, their ability to meaningfully control their future is curtailed in a number of ways. From being placed on release conditions to being detained in custody prior to disposition, the state assumes a great deal of control over the lives of those it prosecutes. The main choice that defendants retain is how their charges will be disposed of. Exercising this discretion, defendants may elect which court their charges will appear in for indictable matters, what charges they will admit and what charges they will require the state to prove, and whether they will testify or call other evidence at a contested hearing. Because even seemingly strong prosecutions can falter when the state is put to its burden, defendants have the implicit power to compel the Crown to make a favourable deal in exchange for early resolution. An increased degree of autonomy is a good worth pursuing, as it stands to reason that defendants who have this added measure of control over their futures are more likely to view the justice system favourably and believe they'd been treated fairly than those without such options. Allowing defendants to use a rare advantage to make decisions that may lessen the severity of their sentence is a reasonable concession to make.

\subsubsection{Certainty in facts and outcome for all}

Trials are inherently unpredictable events. Although not all trials are ultimately decisive of all of the issues in a particular case, at a high level view trials are zero-sum games with clear winners and losers. Where both sides believe they have an arguable case, it is reasonable for each to expect that they will be successful. When one party inevitably loses their case, they encounter an outcome they weren't expecting. At lower levels of detail, it is also clear that both parties face uncertainty with respect to many aspects of their case. A witness may testify in a way that benefits or damages either side of an argument, or may decide to not show up for the trial at all. Opposing counsel may find an old article an expert witness published (and forgot about) where they take a different view from the one they've testified to in the case. Even where the state makes out its case, rulings on objections and evidentiary motions can significantly impact the facts a defendant is convicted of. At every level, trials are rife with uncertainty.

By contrast, the plea bargaining process is geared towards achieving as much certainty as possible in an outcome prior to pleas being entered. Everything from the facts that the prosecutor will read into the record to the range of sentences to be recommended may be discussed and decided. Discussions between parties may last weeks or months, during which time both may review the evidence they have, request additional disclosure, review pertinent evidentiary and sentencing decisions. Where the parties jointly recommend a sentence that was the product of a true plea bargain, judges must show it significant deference. Comprehensively compiled plea deals give all parties more certainty of outcome than trials.

\subsection{Accuracy, collaboration, and ethical efficiency: the approach in this thesis}

Throughout the relevant portions of this thesis, I develop and eventually adopt the following approach to plea bargaining, which I explain in more detail below:

\begin{itemize}
\item \textbf{Accurate bargaining.} Plea bargaining has as much or more potential to achieve accuracy and fairness as a trial by jury. Accuracy and fairness do not necessarily need to be traded off for efficiency and autonomy.
\item \textbf{Win-win collaboration.} For both defendants and prosecutors, certainty in outcome in a criminal case is a good that is worth pursuing and fostering. Neither prosecutors nor defendants should be discouraged from pursuing mutually beneficial ``win-win'' solutions to criminal problems.
\item \textbf{Wrongful punishment.} Wrongful convictions and the coercive effects of plea bargaining ought to be assessed in light of the related phenomenon of wrongful punishment and the coercive effects of the criminal justice system more generally.
\item \textbf{Overlooked substantive benefits.} The substantive benefits of plea bargaining have been overlooked. Plea bargaining encourages an environment of reasonable compromise, cooperation, and responsibility. Simple increases in procedural efficiency can result in positive substantive benefits just by reducing time spent on remand or under release conditions.
\end{itemize}

\subsubsection{Plea bargaining can be easily adapted to better ensure accurate results}

Where both parties to a prosecution are mindful of their obligations and committed to engaging with the other in good faith, without prejudice discussions between prosecutors and defence counsel provide a solid foundation for understanding the matter in a safe and controlled environment. When coupled with non-binding judicially-led case management conferences, these discussions become the ideal forum for explicating triable issues and accurately gauging the respective strength of the other's position. Through early collaboration and frequent testing of those positions, resolution discussions become more pointed and trials, where necessary, become more focused. 

Trials are a time-limited and carefully coordinated presentation of admissible evidence, wherein two diametrically opposed parties try to frame the facts in such a way as to convince a third party of their point of view. Plea bargains, on the other hand, are reached between parties with access to all of the disclosed evidence and an abundance of time to go through it. While these discussions are privileged and not subject to judicial scrutiny, this allows both parties to fairly and openly assess the relative strengths and weaknesses of their positions, and the positions of the other side, while remaining aware that those strengths and weaknesses may be tested in court if negotiations fail. Where possible, plea bargaining helps ensure that those who know the case best have significant input into how it is ultimately decided.

The plea bargaining system is well-suited to obtaining accurate results. In all likelihood, it is more accurate than the trial process. That it obtains these results at the expense of transparency and the public's interest in hearing a case adjudicated on its merits is a price worth paying.

\subsubsection{Where possible, collaborative win-win scenarios should be encouraged in criminal proceedings}

Win-win scenarios are rare in the common law criminal justice system, where stakes are high, disputes are common, and the adversarial nature of the proceedings is frequently on display. Unlike civil trials, where the parties to an action may enter pleadings that provide multiple avenues to victory or defeat, criminal trials are ultimately an all-or-nothing affair. If the defendant is convicted, the prosecutor has won and the defendant has lost. If the defendant is acquitted, the defendant has won and the prosecutor has lost. One side must lose for the other to win.

The plea bargaining process gives prosecutors and defendants an opportunity to both examine their respective positions and determine whether there is a way to resolve the matter without a trial. Where counsel identify contingencies in each other's cases, these may be leveraged and place the party in a better bargaining position. In cases where both parties are satisfied that they each have something to lose by taking the case to trial, and something to gain by resolving it on mutually agreed-upon terms, a ``true plea bargain'' between the parties is reached. 

A plea bargain is mutually advantageous where both the prosecutor and the defendant are satisfied with the agreed-upon terms. In cases where both parties had once thought that they had a reasonable chance of winning outright, the interaction between the parties has been transformed from an all-or-nothing contest to a collaborative resolution. Collaboration is not a guarantee against a subsequent sense of ``buyer's remorse'' from either or both parties. But collaboration does at least allow the possibility of both parties being satisfied with the final resolution, a possibility that is not normally available following a criminal trial.

\subsubsection{Wrongful pre-trial restrictions are analogs of wrongful convictions}

Although wrongful convictions have rightly been decried throughout the common law world, the restrictions that precede those convictions have received comparatively little attention. Where a crime is especially serious, or an offender is especially incorrigible, a judge may order that they be detained in custody until their charges are disposed of. Even in less serious cases, defendants will often be released into the community on conditions. Some conditions, such as no contact or communication conditions, can criminalize interactions with a defendant's friends or family. Others, such as substance use and curfew conditions, can allow the police to demand bodily samples or search a defendant's home without a warrant. 

Where these conditions are especially stringent, or where a defendant has been remanded into custody, pre-trial restrictions may be virtually identical to the sentence they could reasonably expect to receive upon pleading guilty. While restricting plea bargaining may eliminate some of the pressures that induce innocent defendants to plead guilty, it would do little to address the wrongful pre-trial ``punishments'' imposed by their remand or release conditions. Removing a defendant's ability to quickly and meaningfully resolve their case on favourable terms may even increase a defendant's wrongful suffering, under the guise of trying to relieve it.

\subsubsection{Efficient and co-operative proceedings provide defendants substantive rights and goods}

Efficiency is often cast as a good that primarily benefits society as a whole, at the expense of some of a defendant's procedural rights. Because judicial resources are limited, the amount of time each case will have to be presented at trial is inversely correlated to the number of defendants who opt to set their matter down for one. Finding ways to hear every trial within the limits of the resources Parliament has allotted requires all parties to find efficiencies. Some possible efficiencies, such as double-booking trials and limiting the amount of time the parties have to put cases in, have obvious negative implications for criminal defendants. But not efficiencies have to come at the expense of defendants. 

Furthermore, to the extent that these efficiencies help guarantee substantive rights and goods for all of the parties to a prosecution, they help guarantee the substantive rights and goods of criminal defendants as well. Speedy trials limit the time defendants spend in coercive pre-trial custody or on similarly coercive and restrictive release conditions. Defendants applying for interim release on a reverse-onus bail have less cause to show for a brief pre-trial release period than a lengthy one. Effective pre-trial plea bargaining and cooperative case review helps triage matters that are unlikely to resolve from those that are, and allows counsel adequate access to timely trial dates as needed. Providing judges and juries with enough time to hear the issues in a legitimate dispute provides defendants and society with the substantive good of a more fair and fulsome presentation of the case on its merits.