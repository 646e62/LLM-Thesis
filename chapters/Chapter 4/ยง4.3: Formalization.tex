\section{Expanding access to plea bargaining though \textit{nolo contendere} pleas}

If one accepts the premise that plea bargaining is a substantive and normative good or that plea bargaining may be such in certain cases, a presumptive case is made for expanding plea bargaining's scope. For defendants who cannot or will not plead guilty but are nevertheless willing to self-convict, expanding the list of permitted pleas to include \textit{nolo contendere} pleas accomplishes this task. For prosecutors, witnesses, victims, and society generally, the potential gains that may be won at a contested trial will not usually be worth the risk when compared with a confirmed conviction through a no-contest plea.\footnote{To the extent that either the defendant or the state may have the facts of the offence come out more favourably for them at trial, neither party has any implicit advantage over the other.}

The following are key advantages of expanding access to no-contest pleas to defendants who cannot or will not plead guilty, but are otherwise prepared to self-convict:

\begin{itemize}
    \item \textbf{Ensuring certainty in outcomes for all.} Trials are inherently unpredictable, while negotiated plea agreements are routinely unsurprising. Expanding the class of defendants who can access these plea arrangements increases certainty for all involved.
    \item \textbf{Delegating more control over the evidence admitted and proven to prosecutors and defendants.} Defendants who set their matters down for trial relinquish control over the evidence they know inside and out to a third-party decision maker hearing the case for the first time. Plea agreements ensure that the parties who know the case best control over presentation.
    \item \textbf{Increasing agency and self-determination for defendants.} Justice system participants, including defendants, witnesses, and victims, have their agency eroded when they are required to take part in procedures they would all otherwise agree to forego. Allowing more defendants to willingly opt out of trials increases their agency and strengthens confidence in the justice system.
    \item \textbf{Transforming zero-sum prosecutions into win-win outcomes.} Whereas non-criminal litigants often have ample access to dispute resolution outside of a trial court, criminal defendants do not. Increasing access to plea bargaining increases a criminal defendant's access to mutually beneficial win-win outcomes. 
    \item \textbf{Sealing loopholes and tying loose ends.} The Canadian \textit{nolo contendere} plea procedure is the result of a seemingly unintentional interaction between two \textit{Criminal Code} procedures with potentially undesirable side-effects. Formalizing and regulating these non-inculpatory self-convictions may address these issues.
    \item \textbf{Customization through regulation.} Although the \textit{nolo contendere} plea procedure gives defendants some access to non-inculpatory no-contest pleas, the informal plea is severely limited. Formalizing and regulating different types of non-inculpatory no-contest pleas may increase access to these pleas and their utility.
\end{itemize}

\subsection{Increased certainty in the outcome}

\subsubsection{Plea bargained cases generally}

Because prosecutions can be fraught with uncertainty for all parties involved, certainty about how a case will resolve motivates all plea bargains, including those with defendants who intend to plead guilty. Such defendants may wish to plead guilty to some offence charged but may only be prepared to plead to a lesser included offence or less aggravating allegations. Similarly, disputes over a sentence's nature and quantum can be hotly contested, even where defendants are prepared to plead guilty. Defendants and prosecutors who pre-negotiate pleas and sentences greatly increase their certainty in the outcome.\footnote{See \textit{R v Anthony-Cook}, 2016 SCC 43 (CanLII), [2016] 2 SCR 204. While jointly-recommended sentencing proposals are subject to judicial discretion and review, judges are generally precluded from rejecting them. Even in cases where prosecutors and defendants cannot come up with a joint proposal or where the requisite \textit{quid pro quo} for a joint proposal is not met, judges remain similarly constrained by the parties' sentences. See \textit{R v Beardy}, 2014 MBCA 23 at para 6; \textit{R v Grant}, 2016 ONCA 639 at paras 163 - 165}

\subsubsection{\textit{Nolo contendere} cases specifically}

Under the current Canadian legislative scheme, defendants who cannot or will not admit guilt must set their matters down for trial. Prosecutors always bear the burden of proving criminal charges beyond a reasonable doubt, and even defendants without a positive defence or cohesive response to the allegations against them may prevail at trial, thereby risking wrongfully acquitting factually guilty defendants who are otherwise willing to accept conviction. Defendants charged with multiple offences may be allowed to self-convict on some in exchange for the others being dropped. However, when those same defendants must contest all or nothing, they risk being convicted of more than they could have otherwise bargained for.

\subsection{Increased control over the evidence admitted and proven}

\subsubsection{Plea bargained cases generally}

In addition to giving defendants and prosecutors a high degree of certainty in the outcome of a case, negotiated pleas give all parties to a prosecution an increased level of control over the evidence. The prosecutor must prove every disputed fact on a balance of probabilities and must prove every disputed aggravating fact beyond a reasonable doubt.\footnote{See \textit{Criminal Code} s 724(3)(d) \& (e).} This requirement gives defendants a considerable bargaining advantage, as they can spare prosecutors the trouble of proving their case generally and any particular allegation specifically, though both parties must suffer the consequences of having witnesses testify. Defendants lose any consideration they might have had on sentencing for resolving their case without a trial. In turn, prosecutors can agree to exclude certain embarrassing or otherwise prejudicial aspects of an allegation in exchange for concessions on the plea entered or the sentence agreed to.  Rather than leaving these determinations to a judge or jury hearing the admissible portions of a case for the first time, plea bargaining allows the parties most familiar with the underlying facts to navigate these complex decisions together.

\subsubsection{\textit{Nolo contendere} cases specifically}

The prosecutor arguably benefits more from defendants having access to \textit{nolo contendere} pleas. Defendants who plead guilty to an offence may be asked to admit to the truth underlying the allegations. Some of those defendants will only want to admit to allegations they believe are true, in fact, true and may put the prosecution to its burden of proof if there is a disagreement about the allegations. Where defendants may instead refuse to admit guilt, it stands to reason that the prosecutor will have greater liberty with their allegations. Where the defendant neither admits nor contests the allegations, there is less impetus, if any, for the prosecutor to have the defendant admit to them. In exchange for this leeway, the defendant can continue to deny their factual guilt in good conscience, contest the allegations in subsequent proceedings, or take advantage of other comparable concessions.

\subsection{Increased agency and self-determination}

\subsubsection{Plea bargained cases generally}

By having some control over the outcome and the allegations, defendants increase their self-determination. Plea bargaining is one of the primary means defendants have to control a process that is otherwise out of their hands. Similarly, because plea bargains guarantee convictions, victims of crime who work with prosecutors and victim services agencies can prepare impact statements and request protective conditions where appropriate. Increasing the ability for defendants to determine their cases and guaranteeing a conviction for victims of crime gives both parties greater agency. It stands to reason that when justice system participants have more agency in proceedings otherwise outside their control, they will have a higher regard for the justice system.

\subsubsection{\textit{Nolo contendere} cases specifically}

Defendants who self-convict without admitting responsibility may repudiate their charges, either formally in subsequent proceedings where available or privately following a non-inculpatory \textit{nolo contendere} plea. By doing so, defendants can remain authentic to a particular moral or ideological standpoint while accepting the inevitable nonetheless. Pleas are not propositions that can or should be understood for the truth of their contents. However, allowing defendants to enter pleas whose content accurately reflects their beliefs and dispenses with the need for a formal hearing may reasonably lead them to conclude that the proceedings were truthful and fair. 

\subsection{The elusive win-win scenario in criminal law}

\subsubsection{Plea bargained cases generally}

Canada operates on an adversarial system where disputing parties present opposing positions to a neutral decision maker. The adversarial system is zero-sum, as one party must lose in order for the other to prevail.\footnote{This holds true in multi-party proceedings as well. It is also slightly more nuanced than just this.} In a zero-sum scenario, mutually favourable outcomes are precluded. 

In non-criminal cases, such as family law disputes or civil actions, it is common for matters to resolve through alternative means such as mediation, dispute resolution, or other negotiated settlements. Although these trial alternatives do not always produce outcomes that both parties are satisfied with or find bearable, such outcomes are possible. Furthermore, the parties who attempt these trial alternatives may actively and intentionally work towards a mutually beneficial outcome, presumably increasing their likelihood of success.

In most criminal cases, plea bargaining is the only means the parties have to achieve a comparable win-win scenario. Defendants rarely wish to plead guilty and accept full responsibility for their charges from the outset of the proceedings, and even those who do rarely agree with every detail of every initial untested allegation.\footnote{Initial allegations are usually not all that high quality. Prepared by cops and designed to deny bail.} While \textit{Criminal Code} s 717 authorizes prosecutors to employ alternative measures for any offence, these alternatives to prosecutions are only available to defendants who agree they were involved with or participated in the offence charged,\footnote{Cite section.} and are therefore able to plea bargain. Plea bargaining opponents must therefore commit to denying criminal defendants the same ability to bargain for a win-win that any other litigant would be entitled to for substantially less impactful matters. Denying a criminal defendant the means to negotiate their second-degree murder charge down to a manslaughter while actively encouraging family litigants to divide their commemorative spoon collection outside of court is incompatible with any legitimate sense of justice.

On the other side, prosecutors, law enforcement, victims, trial witnesses, and society at large stand to gain a great deal from effective case resolution without a trial. Confidential informants are a prime example of the wins that plea negotiations can achieve. While most of the crimes that courts deal with are ``unsophisticated," allegations involving drug trafficking and organized crime are very often complex and variegated affairs. The intricacies of the drug trade make obtaining sufficient information to find drug traffickers and their suppliers difficult, and obtaining sufficient evidence to criminally implicate those most responsible herculean. Confidential informants often make these tasks possible.

Confidential informants are often invaluable, but the pool of viable ones is generally small and typically composed of criminals. Few are ``ordinary citizens," and even fewer are motivated by civic duty. While some may be willing to give information in exchange for financial compensation or some other extra-legal consideration, informing on others can be dangerous, and significant consideration is frequently expected and typically well-deserved. For co-defendants in a drug or organized crime prosecution, or those criminally charged with unrelated matters, a favourable plea deal may be the only way to secure the needed cooperation. 

If any significant portion of the unsophisticated crimes that preoccupy criminal courts are causally connected to illicit drug addictions, hampering the illicit drug trade is an important societal ``win-win" objective that plea bargaining is often uniquely able to achieve. 

\subsubsection{Non-inculpatory no-contest cases specifically}

For defendants who are unable or unwilling to plead guilty, but willing or able to self-convict, expanding the scope of permissible no-contest pleas expands the range of potential win-win resolutions to include these defendants. Presently, defendants who will not formally admit culpability must proceed to trial. Where both the defendant and the prosecutor would prefer a confirmed conviction on agreed-upon terms, creates a potential ``lose-lose" scenario instead.

Bibas's parochial fantasies notwithstanding, the criminal justice system is not a morality play, and none of its participants are entitled to the wholesome and edifying results he imagines. Prosecutors are not entitled to a perfect conviction, complainants are not entitled to a remorseful defendant, and defendants are not entitled to rehabilitation.\footnote{See Forsyth at 250 - 251: \begin{quote}
    It is understandable that victims of crime and consequently the prosecutors who are dealing with those victims, may prefer to see and hear an absolutely unqualified admission of guilt proffered by the person who has committed the crime against them or their loved ones. The idea of confessing our sins on an unqualified basis is usually engrained in us from our childhood days with the teachings of our parents, our teachers, our religious leaders, or whatever the case may be. The concept of public expiation has been, historically, somewhat fundamental to the concept of our criminal justice system. As laudable as that objective may be, it may be somewhat counterproductive to demand that a person accused of a criminal offence not be allowed to take a position of not contesting the strength of the evidence for the prosecution without making admissions to acts alleged, thereby allowing the prosecution's evidence to be accepted by the court, unassailed and unsullied. As long as the effect of such a plea of ``no contest" is exactly the same from the standpoint of the consequences for sentencing purposes then the public should be gratified that victims of crime are not required to testify in court, court time itself is not needlessly usurped and society can get on with its business.
\end{quote}} The net gains from allowing recalcitrant defendants to self-convict when they are fully prepared to do so far exceeds any perceived losses from not having the defendant acknowledge their moral culpability or truly repent for their misdeeds. A win is a win and should be acknowledged as such.

\subsection{Sealing loopholes and tying up loose ends}

Because Canadian criminal law allows \textit{nolo contendere} procedure, it is reasonable to ask whether a formal plea is necessary. If the \textit{nolo contendere} plea procedure currently authorized was sufficient to its task and otherwise unproblematic, it may not be. But the \textit{nolo contendere} plea procedure is grossly deficient. These deficits are attributable to the lack of legislative oversight and may be addressed by increasing the same.

\subsubsection{Plea voluntariness and comprehension inquiry}

The Canadian \textit{nolo contendere} plea procedure is risky for defendants. As \textit{DMG} and \textit{RP} demonstrate, defendants may engage this procedure without fully realizing its implications. Like guilty pleas, the \textit{nolo contendere} procedure guarantees a conviction. However, because these pleas are formally \textit{not guilty pleas}, and because the codified plea inquiry only applies to formal \textit{guilty pleas}, no plea inquiry is formally required. Although \textit{RP} recommended a plea voluntariness and comprehension inquiry for \textit{nolo contendere} plea procedures, it was silent about what this inquiry should entail. Furthermore, while \textit{RP} binds courts in Ontario, it has neither been explicitly adopted and applied everywhere in Canada, nor approved by the Supreme Court of Canada. Because no such inquiry is generally required for defendants who admit \textit{some} of the allegations against themselves via \textit{Criminal Code} s 655, other courts may reasonably conclude that no such inquiry is required to admit \textit{all} of the allegations against themselves. 

Defendants who use the \textit{nolo contendere} procedure do not plead guilty and do not overtly express any remorse, both of which are mitigating factors the court could have otherwise considered on sentencing. While their American counterparts generally benefit from having their pleas excluded at subsequent proceedings, Canadian defendants do not. Canadian defendants therefore accept more risk on sentencing than those who plead guilty and receive no additional benefits. To the extent that all of the considerations underlying the common law and statutory plea inquiry for defendants pleading guilty apply as much to Canadian \textit{nolo contendere} ``pleas," they should be required. To the extent that defendants who utilize the procedure accept the same risk for less consideration, the plea inquiry should be more comprehensive and thorough than the one required for inculpatory no-contest pleas.

\subsubsection{Judicial discretion}

Judges are not obliged to conduct a plea inquiry for admitted facts because they are generally not allowed to reject them. Although some jurisdictions allow judges to reject such agreements. There is no statutory support for this allowance. Without discretion to reject facts agreed to under \textit{Criminal Code} s 655, judges cannot reject \textit{nolo contendere} pleas. Codifying these pleas and giving judges discretion to reject them would enable the courts to review them for propriety, placing these pleas on equal footing with guilty pleas in this respect. Separate criteria can be considered and used where useful.

\subsubsection{Right to appeal}

Canadian defendants do not have a common law right to appeal their sentence or conviction. The right to do so is created entirely by the \textit{Criminal Code}, and limiting this right to defendants convicted after contested proceedings should be addressed through the \textit{Criminal Code}. As discussed above, criminal defendants may only appeal convictions entered by a trial court.\footnote{See \textit{Criminal Code} s. ... } \textit{Criminal Code} s 673 defines a ``trial court" as ``the court by which an accused was tried." Canadian case law has established and repeatedly reinforced that defendants who plead guilty are not convicted by a trial court, and therefore may not appeal their convictions.\footnote{See \textit{Feagan}.} However, because defendants who self-convict through the \textit{nolo contendere} procedure plead not guilty and initiate the trial process, they may appeal their self-inflicted convictions by right.\footnote{See \textit{Fegan}.} By formalizing these pleas, Parliament may both identify them and ensure defendants who enter them are not entitled to appeal those convictions.

\subsubsection{Protecting the lawyers who assist with these pleas}

Defendants who wish to plead \textit{nolo contendere} may find themselves doing so without a lawyer. As the footnote to the \textit{Besant} case underscores, a lawyer's professional body may not be there to protect them if anything goes awry with the procedure. Other administrative decision-makers in other jurisdictions may reasonably disagree with the stance taken in this lone Ontario decision, but by formally authorizing a non-inculpatory plea, Parliament would legitimize the procedure and empower lawyers to assist their clients with entering these pleas. Doing so will allow defendants greater and more informed access to them.

\subsection{Customization through regulation}

Designing new pleas and plea procedures that better accomplish the end goals of justice than the current statutory regime can.

American \textit{nolo contendere} pleas evolved from the common law and took different forms where transplanted and implemented. Canada has no comparable common law tradition of non-inculpatory no-contest pleas, these pleas do not need to kowtow to eccentric traditions or obscure procedural norms. A formal Canadian \textit{nolo contendere} plea may instead the best discard the worst without having to worry about retroactive consistency. Because Canadian criminal law is uniformly legislated, these pleas may be uniformly implemented nation-wide. 

Regulating and codifying \textit{nolo contendere} would allow Parliament to specify when and how defendants may enter pleas and whether they should be admissible in future legal proceedings.

\subsubsection{Applicability}

Early on, procedures guiding \textit{nolo contendere} pleas derived directly from Hawkins' citation, which strongly implied that \textit{nolo contendere} pleas should be accepted only in misdemeanour cases punishable by a small fine. As the pleas were more widely used, their common law scope expanded accordingly, such that most states that authorize it allow it to be entered for all offence types.

To the extent that \textit{nolo contendere} pleas encourage plea resolutions, I argue they should be permitted, but recognize certain societal goals may be met by restricting their availability. Just as Parliament may restrict judges from imposing certain sentences through mandatory minimums, or may make certain sentences unavailable for certain offences,\footnote{One example of this is found in the \textit{Criminal Code} provisions permitting conditional sentence orders. The list in \textit{Criminal Code} s 742.1 is imperfect, and nothing about it obviously recommends that it be used to restrict non-inculpatory no-contest pleas, but it serves as an example of how these restrictions may be implemented.} Parliament may permit or restrict these pleas according to any criteria of its choosing. For example, restricting \textit{nolo contendere} pleas to summary offences, precluding defendants from entering these pleas to violent offences or offences punishable up to a certain maximum, or refusing \textit{nolo contendere} pleas for offences punishable by a mandatory minimum period of incarceration would allow Parliament to make these pleas available to certain offenders convicted of certain offences while restricting others. Although expanding these pleas to allow all defendants who are inclined to enter them to do so is ideal, allowing any defendant to do so is a step in the right direction.

\subsubsection{Acceptability}

Where \textit{nolo contendere} pleas are allowed, most defendants require permission from both the court and the prosecutor in order to enter them. Some jurisdictions only require permission from the court, while only Virginia gives neither the court nor the prosecutor any discretion over whether defendants may enter a \textit{nolo contendere} plea.\footnote{Cite the provision.} In Canada permission from the prosecutor is required.\footnote{} 

Given that \textit{nolo contendere} pleas are inextricably linked to the plea bargaining process, and that the Canadian \textit{nolo contendere} procedure already requires prosecutorial discretion, a formal Canadian \textit{nolo contendere} plea should include this component. Further, as guilty pleas require judicial authorization, requiring the same for \textit{nolo contendere} pleas is sensible and consistent with this existing prerequisite. However, because \textit{nolo contendere} pleas do not necessarily reflect a defendant's beliefs, and because those propositions must invariably result in a conviction, additional prerequisites ought to be put in place to ensure that these pleas pass muster. Requiring both prosecutors and defendants to jointly show cause why a particular defendant's \textit{nolo contendere} plea is justified, evaluating the propriety of these pleas against a settled standard, and requiring a plea comprehension and voluntariness inquiry conducted by the presiding judge are all safeguards that ought to be adopted if these pleas are formalized.

\subsubsection{Procedural effects}

As self-convictions, \textit{nolo contendere} pleas should result in criminal convictions. Whereas the \textit{nolo contendere} procedure suffers from significant procedural defects, codified \textit{nolo contendere} pleas should be as stringently regulated as guilty pleas, with similarly onerus burdens defendants who wish to withdraw them.

\subsubsection{Subsequent effects}

The fact that the \textit{nolo contendere} pleas are inadmissible in subsequent proceedings distinguishes it from guilty pleas, giving it utility it would otherwise lack. Absent the historical common law foundation for this feature, however, it is sensible to consider whether this feature should be included in non-culpatory no-contest pleas built from the ground up and if so, how this feature should be implemented. 

Increasing a defendant's control over their legal fates is key to plea bargaining system. Although some defendants may enter a \textit{nolo contendere} plea out of integrity, rendering evidence of these pleas inadmissible at subsequent proceedings will likely persuade many others to do so. Certain classes of cases will likely be more susceptible to these persuasive improvements than others. Defendants charged with major fraud may be convinced to resolve if evidence of their plea was inadmissible at a subsequent civil proceeding, while defendants charged with domestic violence offences may be convinced to resolve if evidence was inadmissible at subsequent family proceedings. Defendants of all varieties who risk deportation if convicted may be persuaded to resolve their matters if evidence of that plea was inadmissible for immigration purposes.

However, there may be situations where it is in the interests of justice to allow a defendant to enter a non-inculpatory no-contest plea but not in the interests of justice to exclude evidence of this plea at future proceedings. A defendant charged with sexual interference may be convinced to enter a non-inculpatory no-contest plea to allow them to maintain their innocence privately, but the interests of justice may require that evidence of this plea be admitted at subsequent family law proceedings. Similarly, a defendant charged with fraud may be induced to enter a non-inculpatory no-contest plea to avoid admitting evidence in a costly \textit{civil} suit, but the interests of justice may require that evidence be admitted at a subsequent \textit{deportation} hearing. These cases may be addressed by implementing an application system that requires defendants to apply to the court to have evidence of their pleas excluded.\footnote{New Jersey, a state that does not allow defendants to enter \textit{nolo contendere} pleas, uses a system like this. See NJ R EVID N.J.R.E. 410.} Similarly, requiring defendants to complete a probationary period successfully may provide a reasonable precondition to excluding evidence at future proceedings and an incentive for pro-social behaviour and rehabilitation. Adopting these approaches in tandem would give Parliament granular control over how and when these pleas may be implemented, providing both procedural and substantive advantages.