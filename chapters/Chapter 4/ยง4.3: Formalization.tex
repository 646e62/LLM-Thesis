\section{Formalization}

If plea bargaining is or is capable of being a substantive and normative good, a presumptive case exists for expanding its scope. For defendants who cannot or will not plead guilty but are nevertheless willing to self-convict, expanding the list of permitted pleas to include \textit{nolo contendere} accomplishes this task. Delineating the haphazard, informal plea procedure and expanding \textit{Criminal Code} s 606's scope to include these defendants would entail numerous advantages, including:

\begin{itemize}
    \item \textbf{Ensuring certainty in outcomes for all.} Trials are inherently unpredictable, while negotiated plea agreements are routinely unsurprising. Expanding the class of defendants who can access these plea arrangements increases certainty for all involved.
    \item \textbf{Delegating more control over the evidence admitted and proven to prosecutors and defendants.} Defendants who set their matters down for trial relinquish control over the evidence they know inside and out to a third-party decision maker hearing the case for the first time. Plea agreements ensure that the parties who know the case best have some control over its presentation.
    \item \textbf{Increasing agency and self-determination for defendants.} Justice system participants, including defendants, witnesses, and victims, have their agency eroded when they are required to participate in procedures they would each otherwise agree to forego. Allowing more defendants to opt out of trials increases their self-determination and strengthens their confidence in the justice system.
    \item \textbf{Transforming zero-sum prosecutions into win-win outcomes.} Whereas non-criminal litigants often have ample access to dispute resolution outside of a trial court, criminal defendants do not. Increasing access to plea bargaining increases a criminal defendant's access to mutually beneficial win-win outcomes. 
    \item \textbf{Sealing loopholes and tying loose ends.} The Canadian \textit{nolo contendere} plea procedure is the result of a seemingly unintentional interaction between two \textit{Criminal Code} procedures with potentially undesirable side effects. Formalizing and regulating these non-inculpatory self-convictions addresses these issues.
\end{itemize}

\subsection{Increased certainty in the outcome}

\subsubsection{Plea bargained cases generally}

Because prosecutions can be fraught with uncertainty for all parties involved, certainty about how a case will resolve motivates all plea bargains. This rule holds even for those defendants who intend to plead guilty from the outset, as certainty in a case's outcome is critically important to virtually all defendants. Such defendants may wish to plead guilty to some offence charged but may only agree to plead to a lesser included offence or less aggravating allegations. Similarly, prosecutors and defendants may hotly dispute a sentence's nature and quantum, even where defendants agree to admit all allegations against them. Defendants and prosecutors who pre-negotiate pleas and sentences greatly increase their certainty of the outcome.\footnote{See \textit{R v Anthony-Cook}, \textit{supra} note 8. While jointly-recommended sentencing proposals are subject to judicial discretion and review, judges generally may not reject them. Even in cases where prosecutors and defendants cannot come up with a joint proposal or do not meet the requisite \textit{quid pro quo} for a joint proposal, judges remain similarly constrained by the parties' sentences. See e.g. \textit{R v Beardy}, 2014 MBCA 23 at para 6; \textit{R v Grant}, 2016 ONCA 639 at paras 163 — 165. The importance of certainty in the outcome of a proceeding is underscored by how severely appellate authorities curtail judicial discretion in sentencing defendants who come to the court with true plea bargains.}

\subsubsection{\textit{Nolo contendere} cases specifically}

Under the current Canadian legislative scheme, defendants who cannot or will not admit guilt must set their matters down for trial. Prosecutors always bear the burden of proving criminal charges beyond a reasonable doubt, and even defendants without a positive defence or cohesive response to the allegations against them may prevail at trial. By allowing the matter to go to trial, prosecutors risk a judge or jury wrongfully acquitting guilty defendants who are otherwise willing to accept conviction. Defendants charged with multiple offences may be allowed to self-convict on some in exchange for the prosecutor dropping others. However, when those same defendants must contest all or nothing, they risk being convicted of more than they could have otherwise bargained for. Allowing these defendants to effectively plea bargain further extends the practice's benefits.

\subsection{Increased control over the evidence admitted and proven}

\subsubsection{Plea bargained cases generally}

In addition to giving defendants and prosecutors a high degree of certainty in the outcome of a case, negotiated pleas give all parties to a prosecution an increased level of control over the evidence. By default on sentencing, the prosecutor must prove every disputed fact on a balance of probabilities and prove every disputed aggravating fact beyond a reasonable doubt.\footnote{See \textit{Criminal Code}, \textit{supra} note 2, s 724(3)(d) \& (e).} This requirement gives defendants a considerable bargaining advantage, as they can spare prosecutors the trouble of proving their case generally and any particular allegation specifically. In turn, prosecutors can agree to exclude certain embarrassing or otherwise prejudicial aspects of an allegation in exchange for concessions on the plea entered or the sentence agreed to. Rather than leaving these considerations with a judge or jury hearing the admissible portions of a case for the first time, plea bargaining allows the parties most familiar with the underlying facts to navigate these complex decisions together.

\subsubsection{\textit{Nolo contendere} cases specifically}

Prosecutors arguably benefit the most when defendants are allowed to plea \textit{nolo contendere}, as these pleas give prosecutors significantly more leeway with the allegations they put before the court. Defendants who plead guilty to an offence may balk at having to admit allegations they believe are incomplete, misleading, or false. If defendants disagree with these allegations, they may put the prosecution to its burden of proof at a trial or \textit{Gardiner} hearing. However, when a defendant is allowed to plead \textit{nolo contendere}, it reasonably follows that the prosecutor may put forward more and otherwise controversial details before the court, as defendants who enter those pleas overtly do not admit to the underlying allegations. In exchange for this leeway, the defendant can continue to deny their factual guilt in good conscience, contest the allegations in subsequent proceedings where available, or take advantage of other comparable concessions.

\subsection{Increased agency and self-determination}

\subsubsection{Plea bargained cases generally}

Defendants increase their self-determination by having some control over the outcome and the allegations. Plea bargaining is one of the primary means defendants have to control an immensely impactful process otherwise out of their hands. Similarly, because plea bargains guarantee convictions, victims of crime who work with prosecutors and victim services agencies can prepare impact statements and request protective conditions where appropriate. Increasing a defendant's ability to determine their cases and guaranteeing a conviction for victims of crime gives both parties greater agency. It stands to reason that when justice system participants have more agency in proceedings otherwise outside their control, they will regard the justice system more highly.

\subsubsection{\textit{Nolo contendere} cases specifically}

Defendants who self-convict without admitting responsibility may repudiate their charges, either formally in subsequent proceedings where available or privately following a non-inculpatory \textit{nolo contendere} plea. By doing so, defendants can remain authentic to a particular moral or ideological standpoint while accepting the inevitable. Pleas are not propositions that can or should be understood for the truth of their contents. However, allowing defendants to enter pleas whose truth-content accurately reflects their beliefs and dispenses with the need for a formal hearing may reasonably lead them to conclude that the proceedings were truthful and fair. Defendants who can self-convict without committing themselves to the state's version of events are more empowered than those who must give up plea bargaining and set unwanted trials.

\subsection{The elusive win-win scenario in criminal law}

\subsubsection{Plea bargained cases generally}

Canadian criminal procedure uses an adversarial system where disputing parties present opposing positions to an impartial decision-maker. The adversarial system is zero-sum, as one party must lose for the other to prevail. Zero-sum scenarios tend to preclude mutually favourable outcomes. In non-criminal cases, such as family law disputes or civil actions, it is common for matters to resolve through alternative means such as mediation, dispute resolution, or other negotiated settlements. Although these trial alternatives do not always produce outcomes that both parties are satisfied with, such outcomes are at least \textit{possible}. Furthermore, the parties who attempt these trial alternatives may actively and intentionally work towards a mutually beneficial outcome, presumably increasing their likelihood of successfully reaching a deal that both sides can live with.

In most criminal cases, plea bargaining is the only means for parties to achieve a comparable win-win scenario. Defendants rarely wish to plead guilty and accept full responsibility for their charges from the outset of the proceedings, and even those who do rarely agree with every detail of every initial untested allegation. While the \textit{Criminal Code} authorizes prosecutors to employ alternative measures for any offence, these alternatives to prosecutions are only available to defendants who agree they were involved with or participated in the offence charged\footnote{See \textit{Criminal Code}, \textit{supra} note 2, s 717.} and are therefore able to plea bargain. Plea bargaining opponents must therefore commit to denying criminal defendants the ability to bargain for a win-win that any other litigant would be entitled to for substantially less impactful matters. A system that would deny criminal defendants the means to negotiate their second-degree murder charge down to manslaughter while actively encouraging civil litigants to settle disputes outside of court through mediation and negotiation is incompatible with justice and incorrect in principle.

Conversely, prosecutors, law enforcement, victims, trial witnesses, and society generally benefit from effective case resolution without a trial. Confidential informants are a prime example of the wins that plea negotiations can achieve. While most of the crimes that courts deal with are ``unsophisticated," allegations involving drug trafficking and organized crime are often complex and variegated. The intricacies of the drug trade make obtaining sufficient information to find and convict drug traffickers and their suppliers difficult, and confidential informants often make these tasks possible.

However, the pool of viable informants is generally small and typically composed of criminals. Few are ``ordinary citizens" motivated by civic duty. While some may be willing to give information in exchange for financial compensation or some other extra-legal consideration, informing on others can be dangerous. As a result, confidential informants frequently expect and well-deserve significant consideration for their information. For co-defendants in a drug or organized crime prosecution, or those criminally charged with unrelated matters, a favourable plea deal may be the only way to secure the needed cooperation. Many unsophisticated crimes preoccupying criminal courts are causally connected to illicit drug addictions. As a result, high-level blows to the illicit drug trade are ``win-win" objectives that plea bargaining is often uniquely able to achieve. 

\subsubsection{\textit{Nolo contendere} cases specifically}

For defendants unable or unwilling to plead guilty but willing or able to self-convict, expanding the scope of permissible uncontested pleas expands the range of potential win-win resolutions to include these defendants. Presently, defendants who will not formally admit culpability must proceed to trial. Where both the defendant and the prosecutor would otherwise prefer a confirmed conviction on agreed-upon terms, it creates the potential for a ``lose-lose" scenario instead.

Parochial fantasies notwithstanding, the criminal justice system is not a morality play, and none of its participants are entitled to the wholesome and edifying results that critics like Bibas imagine. Prosecutors are not entitled to a perfect conviction, complainants are not entitled to a remorseful defendant, and defendants are not entitled to rehabilitation.\footnote{See Forsyth, \textit{supra} note 16 at 250 — 251.} The net gains from allowing recalcitrant defendants to self-convict when they are willing to do so far exceeds any perceived losses from not having the defendant acknowledge their moral culpability or genuinely repent for their misdeeds. A win is a win and should be acknowledged as such.

\subsection{Sealing loopholes and tying up loose ends}

Because Canadian criminal law technically accommodates a \textit{nolo contendere} procedure, it is reasonable to ask whether a formal plea is necessary. If the \textit{nolo contendere} plea procedure currently authorized was sufficient to its task and otherwise unproblematic, it may not be. But the \textit{nolo contendere} plea procedure is grossly deficient. These deficits stem from the lack of legislative oversight and may be addressed by increasing the same.

\subsubsection{Plea voluntariness and comprehension inquiry}

The Canadian \textit{nolo contendere} plea procedure is risky for defendants. Defendants may engage in this procedure without fully realizing its implications as \textit{DMG} and \textit{RP} demonstrate. Like guilty pleas, the \textit{nolo contendere} procedure guarantees a conviction. However, because these pleas are formally \textit{not guilty pleas}, and because the codified plea inquiry only applies to formal \textit{guilty pleas}, a plea inquiry is not statutorily required. Although \textit{RP} recommended a plea voluntariness and comprehension inquiry for \textit{nolo contendere} plea procedures, it was silent about what this inquiry should entail. Furthermore, while \textit{RP} binds courts in Ontario, it has neither been explicitly adopted nor applied by most Canadian appellate courts nor approved by the Supreme Court of Canada. Because judges are generally not required to inquire about defendants who admit \textit{some} allegations via \textit{Criminal Code} s 655, other appellate courts may reasonably conclude that no inquiry is required to admit \textit{all} of the allegations. 

Defendants who use the \textit{nolo contendere} procedure do not plead guilty or overtly express any remorse, both of which are mitigating factors the court could have otherwise considered on sentencing. While their American counterparts generally benefit from having their pleas excluded at subsequent proceedings, Canadian defendants do not. As a result, Canadian defendants using the \textit{nolo contendere} procedure accept more risk on sentencing than those who plead guilty and receive no additional benefits. To the extent that all of the considerations underlying the common law and statutory plea inquiry for defendants pleading guilty apply as much to Canadian \textit{nolo contendere} pleas, they should be required. To the extent that defendants who utilize the procedure accept the same risk for less consideration, the plea inquiry should be more comprehensive than the one required for inculpatory no-contest pleas.

\subsubsection{Judicial discretion}

Judges are not obliged to conduct a plea inquiry for admitted facts because they are generally not allowed to reject them. Although some jurisdictions allow judges to do so, there is no statutory support for this allowance. Without discretion to reject facts agreed to under \textit{Criminal Code} s 655, judges cannot reject \textit{nolo contendere} pleas. Codifying these pleas and giving judges discretion to reject them would enable the courts to review them for propriety, placing them on equal footing with guilty pleas in this respect.

\subsubsection{Automatic right to appeal}

Self-convictions entered through the \textit{nolo contendere} procedure qualify as convictions ``by a trial court." Appeals in Canada are limited to trial court convictions and sentencing appeals. Because guilty pleas formally waive a defendant's right to be tried, they are not convictions by a trial court. However, because the informal \textit{nolo contendere} plea procedure is a conviction entered by a trial court, it may be appealed by right.

Canadian defendants do not have a common law right to appeal their sentence or conviction. The right to do so is created entirely by the \textit{Criminal Code}, and limiting this right to defendants convicted after contested proceedings should be addressed through the \textit{Criminal Code}. Per \textit{Criminal Code} s 675, criminal defendants may only appeal convictions entered by a trial court, which \textit{Criminal Code} s 673 helpfully defines as ``the court by which an accused was tried." Canadian case law has reinforced that defendants who plead guilty are not ``convicted by a trial court" and, therefore, may not appeal their convictions. However, because defendants who self-convict through the \textit{nolo contendere} procedure plead not guilty and initiate the trial process, they may appeal their self-inflicted convictions by right.\footnote{See \textit{Fegan}, \textit{supra} note 26.}  By formalizing these pleas, Parliament may identify them and control whether defendants who enter them may continue to do so.

\subsubsection{Protecting the lawyers who assist with these pleas}

Defendants who wish to plead \textit{nolo contendere} may find themselves doing so without a lawyer. As the footnote to the \textit{Besant} case underscores, a lawyer's professional body may not be there to protect them if anything goes awry with the procedure. Moreover, while other administrative decision-makers may disagree with this lone Ontario decision, others may agree. By formally authorizing a \textit{nolo contendere} plea, Parliament would legitimize the procedure and empower lawyers to assist their clients with entering these pleas, allowing defendants greater and more informed access to them.