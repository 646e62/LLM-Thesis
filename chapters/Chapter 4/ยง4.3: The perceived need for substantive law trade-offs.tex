\section{The perceived need for substantive law trade-offs}

A unique and forceful objection to both plea bargaining generally and non-inculpatory no-contest pleas specifically is that plea bargaining undermines and detracts from substantive legal values in exchange for procedural efficiencies. This position, introduced by Stephanos Bibas,\footnote{Now Judge Bibas, having been appointed as a federal judge to the Third Circuit Court of Appeals in 2017.} in his 2002 article ''Harmonizing substantive-criminal-law values and criminal procedure: The case of Alford and nolo contendere pleas," argues that it is a mistake to allow plea-bargained no-contest pleas at the expense of substantive criminal values. He outlines this argument as follows:

\begin{enumerate}
    \item \textbf{Item.}
\end{enumerate}

I respond to his argument as follows:

\begin{enumerate}
    \item \textbf{The false dichotomy between procedural expediency and substantive values.}
    \item \textbf{The limits of substantive values in criminal law.}
    \item \textbf{The substantive values implicit in plea bargains and self-convictions.} 
\end{enumerate}

\subsubsection{Bibas's position}



\subsubsection{Procedural efficiency versus substantive values}

\paragraph{The false dichotomy\\}

Stephanos Bibas and other plea bargaining opponents take the position that plea bargaining, no-contest pleas, and procedural efficiencies generally are fundamentally at loggerheads with substantive criminal values

Define substantive criminal values

Plea bargaining proponents often advocate for the process on the premise that it provides immense returns in procedural efficiency

Some proponents, including the Supreme Court of the United States, the Supreme Court of Canada, and other appellate authorities in Canada, have referred to plea bargaining as an essential feature of our criminal justice system

Plea bargaining is treated more as a necessity than a good

The problem with this approach is that it overlooks the fact that the justice system could quite easily function without plea bargaining

It would require significantly more resources, but nothing in theory would prevent the government from spending enough money to ensure that every person charged with a crime received a trial

Plea bargaining opponents would generally prefer that significantly more resources were devoted to the justice system, so as to try to ensure that significantly more matters go to trial

Some, like Albert W Alschuler, see plea bargains as concessions of efficiency and trials as the truest and most reliable test of guilt

Others, like Stephanos Bibas, see trials as morality plays that both defendants and society at large substantively benefit from putting on

Both Albert W Alschuler and Stephanos Bibas  view a trial by jury as being an inherently valuable way of resolving a criminal allegation, especially when contrasted with a plea bargained disposition, or an equivocal plea, or both

Both those who extol plea bargains because of their efficiencies and those who deride them for the impact they've had on criminal trials appear to accept that there is a dichotomy and a trade-off between substantive criminal values on one hand, and procedural efficiencies on the other

Opponents argue that plea bargaining's efficiencies come at the cost of due process, the presumption of innocence, the ability for criminal defendants to receive due process, and the ability of complainants and the community to confront offenders who disrupt the social order

Proponents argue that the gains achieved through plea bargaining's efficiencies far outweigh the cost of having less (and arguably unnecessary) trials

\paragraph{Diminishing returns on justice}

HYP - There is a point of diminishing returns when it comes to money spent on judicial efficiencies versus the substantive benefits received

The *Jordan* framework sets a ceiling on how long a criminal case can go on before a defendant's right to a speedy trial is presumed to be violated, but doesn't prevent trials from being speedier

However, as it's codified, the *Criminal Code* allows justice to be only so speedy

Thirty day notice periods for things like expert evidence

Ninety day time period for an automatic detention review

But it's much more likely that justice will hit severe points of diminishing returns far before any of these milestones are hit

The cost of getting a matter to trial in six months rather than a year can be extraordinary
Go through the hypothetical

\paragraph{What are the substantive costs when nobody wants a trial?}

Opponents of plea bargaining overlook the fact that both accused persons and complainants often want nothing whatsoever to do with a trial

There are plenty of bad reasons for a defendant to not want to go to trial

Not enough money to afford a lawyer for trial

Coercive release conditions or pretrial detention

Abusive trial penalty

Covering for someone you know is guilty

There are also lots of understandable reasons why defendants who wouldn't otherwise be inclined to plead guilty might want to forego a trial, even where these coercive factors aren't present

Difficulty complying with reasonable release conditions (eg, no contact orders between coaccused family members, spouses)

Desire to move past the allegations and the events leading up to it

Finality with respect to disposition

Probation or custody, where ordered, can start to be served very quickly

Quicker access to rehabilitative resources in custody or through Probation

Earlier opportunity to access a record expungement

Similarly, victims of crime stand to benefit from efficient resolution

Finality in the proceedings with no questions left as to conviction or disposition

Things like restitution or incarceration are determined right away

Not needed for trial and all that entails

Allowing proceedings to drag on unnecessarily prolongs the time between offence and resolution

Where a defendant without a viable defence can be convinced to forego trial, their trial date becomes available to another defendant

May not be much in major centres, but in jurisdictions like Southwest Saskatchewan, the effects are immediately apparent

Although the default position on release is an undertaking without conditions, defendants are frequently released on at least some conditions

These can be difficult to comply with in some cases

In others, a moment of heedlessness can turn otherwise lawful behaviour into an administration of justice offence

The longer a trial is delayed, the more likely a defendant is to breach their release conditions, possibly winding up incarcerated for the remainder of their pre-trial period as a result

Where the trial scheduled is fully unnecessary, these breaches of release orders are completely avoidable

\paragraph{Inefficient processes frustrate substantive criminal values}

Public perception of the administration of justice isn't improved by requiring defendants to opt for time-consuming and unwanted trials

Defendants who feel punished by the pretrial process would seem to be unlikely to respect the outcome of it, even (and sometimes especially) if they're ultimately successful in defending themselves

Defendants who want to sue the Crown for wrongful prosecution, take complainants to court for perjury, etc

Generating unnecessary administration of justice offences creates offences and offenders, which frustrates their subsequent rehabilitation for the substantive offences

Relying on police, Crown attorneys, and judges to simply not prosecute these offences or convict these offenders is a doomed strategy, as can be seen from the explosion of administration of justice offences in recent years

Requiring defendants with legitimate defences and questions for the court to delay their trials simply so that another defendant can call no evidence and let the chips fall where they may impedes every aspect of the criminal justice system

None of which is to say that it's never appropriate to require a defendant to stand trial, rather than enter a no-contest plea

Or that it's never appropriate to reject an equivocal no-contest plea in favour of an unequivocal one

Only to say that never having this option is not a reasonable option

\subsection{Limits of substantive criminal law in the plea procedure}

It may be better for the defendant's soul to fully acknowledge their guilt and take responsibility for their actions, but that outcome isn't a prerequisite for justice being done

Stephanos Bibas takes the view that an offender who enters an unequivocal guilty plea is more likely to be rehabilitated than one who enters an equivocal no-contest plea, but the evidence for this view is shaky

What foundation is there to conclude that a focus on substantive values will in any way impact things like offender rehabilitation?

The empirical evidence that Stephanos Bibas cites in support of this proposition is scant and unconvincing

Absent empirical evidence, is there a good reason to believe that defendants who are required to admit actual guilt are more likely to be rehabilitated than those who don't?

Gauging rehabilitation

If they plead guilty

They will appear to have regressed if they continue to maintain their innocence

They will appear to have made no progress if they have begun to accept responsibility

If they equivocally pleading no-contest

They will appear to have made no progress if they assert, or continue to assert their innocence

They will appear to have progressed if they have begun to accept responsibility

One of the factors used to decide whether to grant a federal inmate early parole is the degree to which they've connected with and taken responsibility for the offence they've been charged with

Inaccurate information about whether and to what extent an offender has rehabilitated cannot properly serve the substantive criminal values authors like Stephanos Bibas have in mind

Screening out "false guilty pleas"

If the concern is ensuring that the rehabilitative powers of guilty pleas are maximized, how does this process screen for insincere guilty pleas?

And if it can't, why should defendants not be allowed to self-convict sincerely, rather than lie about their subjective guilt?

To accomplish the substantive value shift he proposes, Stephanos Bibas proposes that the role of the defence lawyer be rethought to encourage lawyers to challenge and confront their clients who are in fact guilty, but who wish to enter no-contest pleas

The problem of convictions after trial

If the substance of a defendant's plea is intricately tied to their prospects for rehabilitation, what hope is there for the defendant who pleads not guilty?

Should the defence lawyer also adopt this position when it comes to clients who wish to exercise their rights and have their matter go to trial?

Don't trials exact a much higher cost on all parties than a self-conviction?

Without some compelling evidence demonstrating that the plea a defendant enters has a meaningful impact on their rehabilitation, the substantive value shift Stephanos Bibas proposes would be foolish to embrace

Absent evidence to the contrary, there are few reasons to believe that the plea a defendant enters has any impact on their rehabilitation at all

Moreover, there are reasons to believe that both plea bargaining and self-convictions bring their own substantive values along with them

\subsection{Substantive values and self-convictions}

Allowing defendants to refuse to accept moral responsibility for an offence while nonetheless accepting legal responsibility helps to expose the disagreement between the defendant and the state for the morality conflict that it is

Stephanos Bibas argues that allowing defendants to enter equivocal no-contest pleas deprives the state of an opportunity to force the defendant to confront their guilt

But by only allowing defendants to enter unequivocal no-contest pleas, courts inadvertently encourage insincere admissions of guilt

Allowing defendants to take legal responsibility while clinging to their incompatible moral position allows the court the opportunity to confront that incompatible position

Rather than pretending to be contrite, defendants may express their disagreement with the court's moral logic

In turn, the court has both the opportunity to confront any errors it sees with the defendant's logic, refuse to accept the plea where the disagreement reveals viable excuses or defences, and enforce judgment against the defendant with the benefit of knowing their position

There will undoubtedly be defendants who enter equivocal no-contest pleas for the purpose of abdicating responsibility, either in subsequent proceedings (where available) or privately

It's difficult to imagine that these defendants wouldn't enter an insincere unequivocal guilty plea, or at least try to

Such defendants may always privately claim pressure, duress, or conspiracy as reasons for having pleaded guilty, rather than remaining silent and entering a sincere equivocal no-contest plea

The mitigating presumptions that accompany a guilty plea do not accompany a *nolo contendere* plea in the same way; namely expressing remorse and admitting responsibility. Conversely, a defendant who is allowed to protest their innocence at sentencing, only to start to make any admissions of responsibility as they await parole, will show much more improvement than one who is required to enter a specious guilty plea. Defendants who do start out protesting their innocence, only to start to acknowledge more responsibility as time goes on, are making more progress. Allowing nolo contendere and best-interest pleas at sentencing would, in fact, lead to a more accurate reflection of a defendant's rehabilitation than if that same defendant had to enter a guilty plea.