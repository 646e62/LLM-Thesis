\chapter{Should no-contest pleas be further incorporated into Canadian criminal law?}

Thus far this thesis should have defined what no-contest pleas are, which kind are allowed to be entered in Canada, and demonstrated their compatibility with Canadian criminal law notwithstanding

The cognitive dissonance of no-contest pleas

The apparent untruthfulness of no-contest pleas



\subsubsection{The apocryphal truth-seeking function of the trial}

Plea bargaining and its attendant controversies

Wrongful convictions

Coerced confessions

Trial penalties

Perceived trade-off between substantive values and procedural efficiency

A bargained plea deal is commonly seen as

Choosing to not contest the state's case may be just as carefully reasoned a decision as choosing to contest it would be

The procedural pitfalls that accompany informal *nolo contendere* pleas in Canada

Judicial oversight effectively precluded in most jurisdictions

Law Society of Upper Canada strongly recommends against the practice

No formal requirement for a plea inquiry

Unclear how mitigating factors might work

These controversies ought to be balanced and considered against the actual advantages conferred to both prosecutors and defendants

Rare "win-win" scenario in criminal proceedings

Lenient sentences for defendants

State is relieved of its burden

Increased certainty in the proceedings for both parties

Increased control over the factual presentation of the offence and offender

No-contest pleas may assist in this process, covering those cases where defendants would be more willing to plead *nolo contendere* or otherwise assert their innocence while resolving than they would be to enter an unequivocal guilty plea

The potential benefits of regulation

Ensure it's clear that there's judicial discretion to accept or reject - or make it clear that the judge does not have this discretion, if so desired

It's not necessary for judges to always have discretion over these sorts of things in order for justice to be done

Eg no discretion to reject a not guilty plea, expert adjournment request where insufficient notice given, etc

Require plea inquiries for *nolo contendere* pleas

The current wisdom seems to direct that the plea inquiry should be conducted when the *nolo contendere* procedure is used

Arguably, the common law doctrine requiring plea comprehension and voluntariness would still hold for a modified process that nonetheless invites a conviction

But ambiguity on this point led to the trial result that was overturned in @2011onca343

Clear guidelines protect both defendants and their lawyers

Further customization of the pleas

Subsequent admissibility considerations

Preconditions for entering the plea

Formalization precedes normalization

Thus far this thesis should have defined what no-contest pleas are, which kind are allowed to be entered in Canada, and demonstrated their compatibility with Canadian criminal law notwithstanding

Plea bargaining and its attendant controversies

Wrongful convictions

Blackstone's ratio - It is better that X guilty persons escape than that one innocent suffer

Coerced confessions

Trial penalties