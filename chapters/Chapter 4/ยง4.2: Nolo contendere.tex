\section{\textit{Nolo contendere}}

One impediment to allowing defendants to enter non-inculpatory no-contest pleas is the cognitive dissonance experienced when trying to hold the conflicting values required to sustain them. A person who pleads guilty but refuses to admit that they are guilty appears to say one thing while meaning another. This phenomenon creates several problems:

\begin{itemize}
    \item \textbf{The truth problem.} Defendants who self-convict while refusing to admit guilt or while actively maintaining their innocence say one thing while meaning another. Doing so is untruthful, and lawyers who assist clients with such pleas mislead the court. 
    \item \textbf{The fairness problem.} People charged with crimes are presumed innocent until proven guilty. The presumption should still stand when a defendant pleads guilty but protests their innocence or refuses to acknowledge their guilt openly. Where a defendant self-convicts but does not acknowledge their guilt, their motives are suspect. Such pleas thus appear to be products of unfairness. 
    \item \textbf{The substance problem.} \textit{Nolo contendere} pleas are evasive.
\end{itemize}

\subsection{The truth problem}

Convictions naturally follow where defendants plead guilty or are found guilty after trial. However, where the plea's content is ``guilty," but the defendant's actions signify ``not guilty" or ``innocent," it raises an apparent problem of truthfulness. This problem is superficial. \hl{\textbf{Rephrase}: It arises from misunderstanding the relationship between facts, belief, and proof on the one hand and the relationship of these concepts to formal pleas and admissions on the other.}

\subsubsection{Facts, propositions, belief, and proof}

If non-inculpatory no-contest pleas raise a problem of truth, it is incumbent to understand what ``truth" means. ``Truth" and ``fact" are closely related terms that may be used interchangeably in specific contexts. For example, a proposition may be synonymously described as being ``the truth" or ``a fact."\textit{Truth} is a quality that all \textit{facts} possess, such that all facts are \textit{true},\footnote{The decision to primarily use the word ``truth" as an adjective and ``fact" as a noun is intentional, as it is more coherent to speak of multiple facts being true rather than multiple truths being factual.} and there are no untrue facts.

\textit{A fact is a statement about the world\footnote{By ``world," I mean the entire realm of objects that have existed, do exist and will exist, as well as the relationships between those objects.} that is objectively true}, regardless of whether one subjectively understands or believes it as true. Statements like ``AlphaGo beat Lee Sedol in four out of five Go games at the DeepMind Challenge Match," ``the earth orbits the sun," and ``the fridge to my right is white" are all examples of facts in that they are all statements about the world that are objectively true.

Not every statement about the world is a fact. Statements like ``Lee Sedol beat AlphaGo in three out of five Go games at the DeepMind Challenge Match," ``the sun orbits Jupiter," and ``the fridge on my right is orange" are all false statements about the world. All facts are statements about the world, but not all statements about the world are facts. I use the term \textit{proposition} to describe statements about the world that may or may not be objectively true.

Propositions are true or false independent of whether anyone is subjectively aware of or acknowledges the fact. The subjective component of a proposition arises in the form of \textit{beliefs}. A subject believes a proposition when they accept it as true and disbelieves it when they reject it as false. Beliefs may be reasonable or unreasonable, and whether a belief can be called reasonable depends on whether that belief can be proven.

\textit{Proof} refers to grounding a subjective belief in ostensibly objective criteria. A proposition is proven when the subject is satisfied there are justifiable reasons to believe it is true. Obversely, a proposition is disproven once the subject is satisfied that there are justifiable reasons to believe it is false or that there are no justifiable reasons to believe it is true. Where multiple subjects share similar or identical criteria, those criteria may form a common standard of proof. Although proofs are generally founded on objective or external criteria, they exist to satisfy the subject that their belief is reasonable. 

True propositions may be proven false, while false propositions may be proven true. A proposition, once proven, may be disproven. For example, at various stages in history, it was proven that women had fewer teeth than men,\footnote{See Bertrand Russell's article.} that the earth revolved around the sun,\footnote{E.g., the Ptolemaic universe.} and that Thomas Sophonow murdered Barbara Stoppel. These propositions have been disproven; therefore, none of these beliefs may be reasonably held today. Although these propositions were once proven, each is untrue, thus demonstrating that \textit{there is no necessary correlation between proof and fact}.

\subsubsection{Pleas and admissions}

A plea is a response to a criminal allegation that a prosecutor must prove. Pleas are also propositions in that they contain a statement about the world that may or may not be true. However, pleas do not necessarily reflect a defendant's belief. A defendant may believe that they are guilty of an offence, may \textit{be} guilty of an offence, but still plead not guilty. A defendant who pleads not guilty does not necessarily provide the prosecutor or the court with any information about their beliefs or the truth of that proposition. Instead, they notify the prosecutor and the court the offence's essential elements must be proven. Similarly, a defendant who enters a no-contest plea notifies the prosecutor and the court that the offence's essential elements need not be proven but does not necessarily provide information about their subjective beliefs or the objective truth of those beliefs. Even where a defendant believes that the content of their plea is true, their subjective beliefs do not necessarily reflect the objective truth.\footnote{In the timeless words of Judge Maya Gamble, ``Your beliefs do not make something true."}

Admissions also advance propositions that do not need to be proven.\ In Canada, defendants may admit any ``facts alleged [by the prosecutor] for the purpose of dispensing with proof thereof."\footnote{See \textit{Criminal Code} s 655. If \textit{facts} signify only true propositions, the phrase ``facts alleged" is nonsensical, as \textit{allegations} may be true or false. The phrase is best understood as synonymous with a \textit{proposition}, and I will refer to it as such.} Typically, these allegations will support or consist entirely of an element of the offence charged. However, any mutually agreed upon proposition can be admitted under this section.\footnote{Defendants may wish to forego proof of some or all of the facts of their case for several reasons. In some cases, there is no reason to doubt that the prosecution will prove certain offence elements. A prosecutor is likely to be able to prove elements like date, time, jurisdiction, and identification in a domestic violence prosecution without difficulty. Agreeing to these facts saves court time and judicial attention, which can be minimal resources in busy court circuits. Admissions can also be advantageous in situations where the evidence is less certain to be proven. For example, when the prosecution wishes to call a witness whose evidence will inevitably harm the defendant to one degree, the defendant may wish to have the witness excused in exchange for a manageable set of agreed facts. In cases where the only genuine dispute lies with a legal issue, both parties may agree to have all evidence admitted by consent and limit trial time to arguing that issue. } Just as a no-contest plea obviates the need for the prosecutor to prove the offence's elements, admissions remove any need for either side to prove the proposition in question.\footnote{See @2009mbca37 for a discussion of the distinction between formal and informal admissions.}

\subsubsection{Summary}

Pleas and admissions are propositions about proof, not truth or belief. Consequently, the ``truth problem" is a definition problem. By entering a plea or making an admission, defendants advise the prosecutor and the court which propositions will require proof if any. Pleas are not evidence, and defendants neither swear nor affirm that the truth-contents of their pleas are genuine. As a result, guilty defendants may plead not guilty, set a trial, and not have to worry about being liable for perjury. Where a defendant is unwilling or unable to enter a plea, the court must enter a not guilty plea on their behalf, regardless of whether the defendant \textit{believes} they are not guilty and whether the defendant \textit{is} not guilty. Where a defendant wishes to self-convict, the \textit{Criminal Code} requires that they understand that a guilty plea admits the offence's essential elements. It does not require defendants to believe that they are guilty. Where a defendant resiles their self-conviction, Canadian courts focus on whether the defendant entered the plea voluntarily, knowingly, and unequivocally, and not on whether the plea was factually true.\footnote{There are exceptions to this. See \textit{R v Catcheway}, 2018 MBCA 54, where the defendant pleaded guilty to being unlawfully in a dwelling house despite having been incarcerated at the time. The Manitoba Court of Appeal allowed him to withdraw his guilty plea because there were ``valid grounds for doing so." Absent evidence that his pleas were uninformed or involuntary, it erred in doing so. To its credit, the court simultaneously arrived at the correct solution by allowing the appeal against conviction based on a miscarriage of justice, as there was compelling evidence that the defendant was innocent.}

\subsection{The presumption of innocence problem}

The presumption of innocence and the prosecutor's burden of proving an offence beyond a reasonable doubt are hallmarks of the common law justice system. Legal professionals and academics see these concepts as indispensable tools for fighting wrongful convictions.

These legal concepts seem to run contrary to the practice of non-inculpatory self-convictions. Where the prosecutor must prove the charges and the court must consider the defendant innocent until proven otherwise, it seems unjust to convict a defendant does not admit guilt.\footnote{Cite to one or more articles that take this position. Alschuler gets into this briefly, but the author who discusses Alford's ``sworn testimony" at length is the source here.} Where defendants maintain their innocence during or following a \textit{nolo contendere} plea, the presumption intuitively holds. Subsequent protestations of innocence should be treated as a signal to the prosecutor and court that a trial is required.

Answering this argument requires also looking at what courts generally require after a defendant has entered a no-contest plea. Specifically, it is necessary to examine both the \textit{factual foundation} requirement and the residual \textit{judicial discretion} to refuse to accept a self-conviction.

\subsubsection{Factual foundation}

Notwithstanding that defendants can legally admit facts against their interests and invite the court to convict them, there will be situations where it would be unjust for the court to do so. One such situation is where the propositions the prosecution relies upon do not make out the offence charged. Where a defendant formally admits the offence but otherwise maintains their innocence, questions reasonably arise as to whether that condition obtains. To help ensure that defendants do not invite baseless convictions, the \textit{Criminal Code} requires that judges ensure that ``the facts support the charge."\footnote{This requirement is also frequently found in American jurisdictions that allow \textit{nolo contendere} pleas. It is a common law requirement for best-interest pleas like \textit{Alford}, which suggests that the factual foundation may also help ensure that judges have a basis for authorizing a conviction, notwithstanding protestations of innocence to the contrary.}

Where a defendant either refuses to accept responsibility for an offence overtly or actively protests their innocence, a review of the allegations and the supporting evidence lets the court evaluate their apparent equivocations. In cases where the allegations are straightforward, the evidence supporting the allegations is strong, or both, defendants who vacillate after self-convicting may be safely ignored. In cases where the allegations are complex or the evidence supporting the allegations is equivocal, judges may consider these factors when deciding whether to allow the defendant to convict themselves.

\subsubsection{Discretion}

In Canada and the United States, judges have immense discretion when deciding whether to accept or reject a no-contest plea. Among jurisdictions that formally allow \textit{nolo contendere} plea, only one state allows defendants to enter that plea by right. This broad discretion is a final safeguard for a defendant's presumption of innocence.

Even where a court is satisfied that a defendant is inviting a conviction and that there is sufficient evidence to support convicting them, a judge may still refuse to authorize a no-contest plea. Because criminal defendants generally cannot self-convict by right, the presumption of innocence remains until a judge is satisfied that it should no longer hold.

\subsubsection{Summary}

The presumption of innocence problem is a non-problem. Defendants are presumed innocent throughout the criminal process but may waive that presumption and invite a conviction. However, the presumption still stands until a court is satisfied that there is a reasonable basis for allowing that conviction. Like any other person, criminal defendants are susceptible to ``buyer's remorse," minimizing or denying their behaviour, and  misunderstanding polysemic legal concepts.\footnote{The intractable problem of properly understanding abstract legal concepts may be insoluble. See e.g. Frederick Schauer ``Formalism" at 514: ``Some terms, like `liberty' and `equality' are \textit{pervaseively indeterminate}. It is not that such terms have no content whatsoever; it is that \textit{every} application, every concretization, every instantiation requires the addition of supplementary premises to apply the general term to specific cases."} Where a defendant has told the court that the prosecution does not need to prove its case against them, and where there is an adequate allegational matrix to support the prosecution's case, the presumption of innocence no longer stands and should no longer stand.

\subsection{The fairness problem}

Although \textit{nolo contendere} pleas are not implicitly untruthful and do not run afoul of the presumption of innocence, there is still an intuitive sense that they are unfair. Defendants may enter a guilty or not guilty plea and sincerely believe the allegations or because they deserve the logical consequences of their plea. By contrast, a defendant who enters a \textit{nolo contendere} plea does not take responsibility for the offence. This makes it unlikely that a defendant will enter the plea because they believe it is accurate or deserves the consequences.

As discussed above, Canadian criminal law has a lengthy history allowing defendants to self-convict only when they do so knowingly, voluntarily, and unequivocally. These principles are so deeply ingrained in Canadian criminal law that they have already found their way into the informal \textit{nolo contendere} plea procedure.\footnote{Similarly, in most American jurisdictions that allow \textit{nolo contendere} pleas, the plea voluntariness and comprehension inquiry required for inculpatory guilty pleas is often mirrored for non-culpatory \textit{nolo contendere} pleas.} 

Any time a defendant invites a conviction, the court receiving their plea has an obligation to ensure that they are not doing so due to any undue pressure, short-sightedness, or confusion.

Despite these protections, some concerns may reasonably remain. Ensuring that defendants enter pleas knowingly and voluntarily ensures that they know the consequences of entering their pleas and that they decided to do so themselves. However, these requirements can only go so far, and these more subtle coercions may arguably produce unjust results. These problems are traceable to the practice of plea bargaining.