\section{\textit{Nolo Contendere}}

The problems that plea bargaining critics have with the practice amplify with the \textit{nolo contendere} plea.\footnote{Critics like Bibas are a prime example. Bibas is not staunchly opposed to plea bargaining and believes the practice may have legitimate uses. But he is adamantly opposed to \textit{nolo contendere} and \textit{Alford} pleas, believing that these pleas amplify the worst aspects of plea bargaining through their quirks and unusual features. See Bibas, ``Bringing Moral Values," \textit{supra} note 240 at 1432.} Even plea bargaining advocates are wont to disdain \textit{nolo contendere} pleas as ineffective, unreliable, and unjust.\footnote{See Bowers, \textit{supra} note 187 at 1165 — 1170.} The criticisms levelled against \textit{nolo contendere} pleas, in particular, may be tracked using the same problem criteria used to track plea bargains generally:

\begin{itemize}
    \item \textbf{The truth problem.} Defendants who self-convict while refusing to admit guilt or while actively maintaining their innocence say one thing while meaning another. Doing so is untruthful, and lawyers who assist clients with such pleas risk misleading the court by allowing defendants to deceive it openly. 
    \item \textbf{The fairness problem.} People charged with crimes are presumed innocent until proven guilty. The presumption should still stand when defendants protest their innocence or refuse to acknowledge their guilt openly. Where a defendant self-convicts but does not acknowledge their guilt, their motives are suspect. Such pleas thus appear unfair. 
    \item \textbf{The substantive problem.} \textit{Nolo contendere} pleas are evasive. Where guilty and not guilty pleas can signify both the truth or falsehood of the underlying allegations and a defendant's belief about the charges, \textit{nolo contendere} pleas cannot. Allowing defendants access to ambivalent pleas sends the message that actual guilt and innocence are secondary concerns, thus undermining criminal law's moral core. 
    
\end{itemize}

\subsection{The Truth Problem}

One impediment to allowing defendants to enter \textit{nolo contendere} pleas is cognitive dissonance. Cognitive dissonance occurs when a person holds or feels they must simultaneously hold two or more contradictory viewpoints. A person who self-convicts but refuses to admit guilt appears to do just that. Convictions naturally follow when defendants plead guilty or when fact finders have found them guilty after hearing enough evidence to convict them beyond a reasonable doubt. However, where defendants enter guilty pleas but appear to be maintaining that they are not guilty, or even innocent, those pleas raise a \textit{prima facie} problem of truthfulness. Although seemingly contradictory, I argue that this problem is superficial and fundamentally misunderstands formal pleas and admissions. When pleas and admissions are defined and understood more carefully and precisely, these apparent difficulties become unproblematic. I approach this definitional problem as follows:

\begin{itemize}
    \item \textbf{Propositions.} \textit{Propositions} are statements about the world that can be true or false. In criminal legal proceedings, facts generally appear as allegations that may be proven, admitted, disproven, or remain unproven.
    \item \textbf{Facts, belief and proof.} \textit{Facts} are true statements about the world. To be a fact, a proposition must be objectively true. A person \textit{believes} a proposition when they subjectively accept that it is true. A person \textit{proves} a proposition when they convince another to believe it. In this sense, belief and proof are subjective, while facts are objective.
    \item \textbf{Pleas and admissions.} In criminal proceedings, \textit{pleas} are propositions that primarily convey general information about what, if anything, prosecutors will be required to prove at trial. Similarly, \textit{admissions} are propositions that a court must accept as having been proven. Pleas and admissions make statements about the world but are primarily used to advise the court what allegations, if any, the prosecutor must prove.
\end{itemize}

\subsubsection{Propositions and Facts}

I define \textit{propositions} as statements about the world that can be objectively true or false. ``The sun rises in the east and sets in the west," ``Labour Day is his favourite holiday," and ``the earth orbits the moon" are all propositions. The first proposition is true, the last proposition is false, and the middle proposition may be true or false, depending on its referent. In the criminal legal context, propositions commonly appear as \textit{allegations} that may be proven, admitted, or disproven. ``The victim consented to the fight," ``the defendant took reasonable steps to verify the complainant's age," and ``the prosecutor has demonstrated guilt beyond a reasonable doubt" are all propositions that can be objectively true or false, can be proven or disproven, and that depend on the opinion of the person assigning those truth-values. Propositions may perform any number of functions, but I focus on three that are particularly important in the criminal legal context: a proposition's ability to convey \textit{truth}, \textit{belief}, and \textit{proof}.

\subsubsection{Facts, Belief and Proof}

Some propositions are true or false, independent of whether any particular person subjectively knows or understands that is the case. For example, although someone may believe the world is flat or refuse to believe that four is two-thirds of six, the first proposition is false, and the second is true. Although legal professionals make a bad habit of referring to trial allegations as ``facts," I only refer to true propositions as facts. Unlike facts and truth, \textit{beliefs} describe propositions that a person takes to be true, regardless of whether they are, in fact, true. Beliefs are true when they correspond to objective reality and false when they do not. \textit{Proof} refers to the process where subjective beliefs are grounded in independent criteria. 

A proposition is proven when a person has justifiable reasons to believe it is true. Obversely, a proposition is disproven once a person is satisfied that there are justifiable reasons to believe it is false or that there are no justifiable reasons to believe it is true. What one person considers a justifiable reason to believe a proposition may differ from what another person may consider a justifiable reason to believe the same proposition. For example, while some people may be satisfied that a creature called Bigfoot exists based on eyewitness evidence and a handful of blurry photographs, others may demand more proof before coming to that same conclusion. To ensure consistency and predictability, many fields of inquiry, such as law, have established common standards of proof. In law, proofs are generally founded on objective or external criteria to satisfy a judge or jury that their beliefs about a proposition are reasonable.\footnote{See John Wigmore, \textit{The Principles of Judicial Proof} (Boston: Little, Brown, and Company, 1913) at 13 — 14.}

True propositions may be proven false, while false propositions may be proven true. Similarly, a proposition, once proven, may be disproven. For example, at various stages in history, it was considered \textit{proven} that women had fewer teeth than men,\footnote{See Bertrand Russell, \textit{The Basic Writings of Bertrand Russell} (New York: Routledge, 2009) at 66.} that the sun revolved around the  earth, and that Thomas Sophonow murdered Barbara Stoppel.\footnote{See \textit{R v Sophonow (No 2)}, [1985] MJ No 10 (QL), 25 CCC (3d) 415. See also MacFarlane, \textit{supra} note 210 at 427.} These propositions have since been disproven, so none of these beliefs may be reasonably held today. But because these untrue propositions were once proven, it follows that \textit{there is no necessary correlation between proof and fact}. Therefore, to properly understand how pleas and admissions function as proof-conveying devices, these concepts of truth, belief, and proof must be kept distinct from one another.

\subsubsection{Pleas and Admissions}

\textit{Pleas} are answers to criminal allegations that a prosecutor must prove. A defendant who says they are or are not guilty of a crime makes a statement about the world that may be true or false. However, pleas do not necessarily reflect a defendant's belief. A defendant may believe that they are guilty of an offence and may, in fact, \textit{be} guilty of an offence but still plead not guilty. Such defendants do not provide information about their beliefs or whether the plea is factually accurate. Instead, they notify the prosecutor and the court that the offence's essential elements must be proven. Similarly, a defendant who enters a guilty or \textit{nolo contendere} plea notifies the prosecutor and the court that the offence's essential elements need not be proven. However, that defendant does not necessarily provide any additional information about whether they believe they are guilty or if they are, in fact, guilty.

Even where a defendant believes what they are saying through their plea, those subjective beliefs do not necessarily reflect the objective truth.\footnote{In the timeless words of Judge Maya Gamble to defendant Alex Jones, ``Your beliefs do not make something true." See Tom Chatford, ``The fate of Alex Jones is a small battle won in the war against alternative facts" (7 August 2022), online: \textless \url{theguardian.com/commentisfree/2022/aug/07/fate-alex-jones-small-battle-war-against-alternative-fact-sandy-hook}\textgreater.} However, in every case, a plea outlines what must and must not be proven. I argue that conveying this information is a plea's most crucial function.

Defendants who make admissions also do so through propositions that do not need to be proven. In Canada, defendants may admit any ``facts alleged [by the prosecutor] for the purpose of dispensing with proof thereof."\footnote{See \textit{Criminal Code}, \textit{supra} note 2, s 655. If \textit{facts} signify only true propositions, the phrase ``facts alleged" is nonsensical, as \textit{allegations} may be true or false, but facts must be true. I read the phrase ``facts alleged" as referring to a \textit{proposition}, and I refer to it as such throughout.} Although admissions generally cover elements of an offence, any mutually agreed-upon proposition can be admitted under this section. Just as an uncontested plea obviates the need for the prosecutor to prove an offence, admissions remove any need for either side to prove the specific proposition in question.\footnote{See \textit{Korski}, \textit{supra} note 167 at paras 119 — 128 for a discussion of the distinction between formal and informal admissions.} Once a defendant admits an allegation, otherwise admissible evidence may not be called or considered to contradict that admission.\footnote{See \textit{Handy}, \textit{supra} note 168 at para 74.}

Pleas and admissions deal primarily with proof, not truth or belief. Consequently, the \textit{truth problem} is a definition problem. By entering a plea or making an admission, defendants advise the prosecutor and the court which propositions will require proof, if any. Pleas are not evidence, and defendants neither swear nor affirm that the propositional content of their pleas is true when they enter them. As a result, guilty defendants may plead not guilty, set a trial, and not worry about being liable for perjury. Where a defendant is unwilling or unable to enter a plea, the court must enter a not guilty plea on their behalf, regardless of whether the defendant \textit{believes} they are not guilty and whether the defendant \textit{is} not guilty. When a defendant resiles their guilty plea, Canadian courts focus on whether the defendant entered it voluntarily, knowingly, and unequivocally, not on whether the plea was factually true.\footnote{See e.g. \textit{R v Behr}, 1966 CanLII 252 (ON SC). In that case, the Ontario Superior Court upheld a guilty plea despite the defendant's lawyer telling him that his guilty plea was not an admission of guilt, as it was nonetheless satisfied that the plea was voluntary. But there are exceptions to this. See \textit{R v Catcheway}, 2018 MBCA 54, where the defendant pleaded guilty to being unlawfully in a dwelling house despite having been incarcerated at the time. The Manitoba Court of Appeal allowed him to withdraw his guilty plea because there were ``valid grounds for doing so." Absent evidence that his pleas were uninformed or involuntary, it erred in doing so. To its credit, the court simultaneously arrived at the a more defensible solution by allowing the appeal against conviction based on a miscarriage of justice, as there was compelling evidence that the defendant was innocent.} Therefore, I argue that a plea's primary purpose is to signal to the court whether evidence will be called or contested, notwithstanding a defendant's subjective beliefs or whether those beliefs are objectively true.

\subsection{The Fairness Problem}

Although \textit{nolo contendere} pleas are not implicitly untruthful, there is still an intuitive sense that they are unfair. Defendants may enter a guilty plea because they sincerely believe the allegations or wish to take responsibility and suffer a deserved consequence. Pleas that a defendant enters in these or similar circumstances seem fair on their face. By contrast, a defendant who enters a \textit{nolo contendere} plea does not take responsibility for the offence. This fact makes it less likely that defendants will self-convict because they believe they are guilty or deserve the consequences. Pleas that a defendant enters in these or similar circumstances seem unfair.

As discussed above, Canadian criminal law has a lengthy history of allowing defendants to plead guilty only when they do so knowingly, voluntarily, and unequivocally. These principles are so deeply ingrained in Canadian criminal law that they found their way into the informal \textit{nolo contendere} plea procedure. As a result, any time a defendant invites a conviction, the court receiving their plea must ensure that the defendant is not self-convicting due to undue pressure, short-sightedness, or confusion. But despite these protections, some concerns remain. Answering these worries requires looking at both the \textit{factual foundation} requirement and the residual \textit{judicial discretion} to refuse to accept a self-conviction.

\begin{itemize}
    \item \textbf{Factual foundation.} Canadian judges may only allow a defendant to plead guilty if there is a factual foundation for the charges. This judicial oversight helps ensure that defendants do not convict themselves when faced with unfounded or unlawful accusations. Insisting on this requirement for \textit{nolo contendere} pleas helps ensure that defendants are not held liable for unfair or untruthful allegations.
    \item \textbf{Judicial discretion.} Defendants cannot self-convict as a matter of right. As a result, even where Canadian judges are satisfied that allegations are factual, they may still reject a defendant's guilty plea. Giving judges final discretion over whether to accept a plea allows them to fulfill their judicial mandate to treat defendants fairly.
\end{itemize}

\subsubsection{Factual Foundation}

Notwithstanding that defendants can legally admit facts against their interests and invite the court to convict them, there are situations where it would be unjust for the court to do so. To help ensure that defendants do not invite baseless convictions, the \textit{Criminal Code} requires that judges ensure that ``the facts support the charge."\footnote{See \textit{Criminal Code}, \textit{supra} note 2, s 606(1.1)(c).} Although the common law does not require judges to ensure that there is a factual foundation for \textit{nolo contendere} pleas,\footnote{See Pickle, \textit{supra} note 46 at 250 — 251.} many American jurisdictions have codified the requirement into law.\footnote{States that have legislated this requirement include Arizona, Arkansas, California, Colorado, Delaware, Florida, Kansas, Maine, Massachusetts, Michigan, New Hampshire, North Carolina, Ohio, Oklahoma, Rhode Island, Utah, and Wisconsin: cf notes 83 \& 84 above for a list of the enabling acts and sections. Similarly, defendants may only enter best-interest pleas like \textit{Alford} when judges are satisfied that a factual foundation exists.}

Where a defendant either refuses to accept responsibility for an offence overtly or actively protests their innocence, reviewing the allegations and the supporting evidence lets the court evaluate any apparent equivocations. In cases where the allegations are straightforward, the evidence supporting the allegations is strong, or both, defendants who vacillate after self-convicting may be safely ignored. In cases where the allegations are complex, or the evidence supporting the allegations is equivocal, judges may consider these factors when deciding whether to allow the defendant to convict themselves or exercise their discretion and set a trial date. In other cases where the propositions the prosecution relies upon do not make out the offence charged, the factual foundation requirement may help stop these false pleas.

Requiring a factual foundation for both guilty and \textit{nolo contendere} pleas reflects an appropriate concern for fairness. Defendants who enter explicit guilty pleas may reassure the court that they understand and admit the allegations they face. Where defendants cannot or will not plead guilty but wish to self-convict, the court may ensure that the allegations are accurate and that the charges stemming from those allegations are fair by reviewing them before accepting a plea.\footnote{See \textit{North Carolina v Alford}, 400 US 25 (1970) at 32.} But even where the allegations support the charge, judicial oversight allows courts to stop self-convictions they deem unfair.

\subsubsection{Judicial Discretion}

In Canada and the United States, judges have immense discretion when deciding whether to accept or reject uncontested pleas.\footnote{But see § 5.2 below.} This broad discretion is a final safeguard for a defendant's presumption of innocence. Even where a court is satisfied that a defendant is knowingly and willingly inviting a conviction and that there is sufficient evidence to support convicting them, a judge may still refuse to authorize the plea. Because criminal defendants cannot self-convict by right,\footnote{See e.g. \textit{R v Goodheart}, 2005 ABCA 126 at para 2; \textit{R v Turnbull}, 2016 NLCA 25 at para 4; \textit{R v Podolski}, 2018 SKPC 13 at para 1.} the presiding judge is under no obligation to give reasons for doing so. However, the judge should direct this role and their power to preventing pleas that they suspect should not result in convictions. Neither defendants nor prosecutors have eminent domain over the presumption of innocence. Instead, defendants are presumed innocent until a judge is satisfied that they are guilty beyond a reasonable doubt.

Like any other person, criminal defendants are susceptible to ``buyer's remorse," minimizing or denying their behaviour and misunderstanding polysemic legal concepts.\footnote{Correctly understanding abstract legal concepts may be an unsolvable problem. See Frederick Schauer, ``Formalism" (1988) 97:4 Yale LJ 509 at 514.} Where a defendant has told the court that the prosecution does not need to prove its case against them, and where there is an adequate allegational matrix to support the prosecution's case, the presumption of innocence no longer stands. It is difficult to convincingly argue that proceedings are unfair when defendants are both presumed innocent and where judges must ensure that defendants only self-convict on demonstrable facts that support the charge.

\subsection{The Substantive Problem}

As discussed above, critics like Bibas fear that the modern common law criminal justice system has developed a hyper-inflated procedural focus that has eroded its ability to achieve its underlying moral aims.\footnote{See Bibas, ``Harmonizing Substantive Values," \textit{supra} note 21 at 1362.} Although this argument applies to expediently-entered and insincere guilty pleas by extension, Bibas's criticisms primarily focus on the ambivalent \textit{nolo contendere} and defiant \textit{Alford} pleas.\footnote{See \textit{ibid} at 1403.} Bibas's \textit{nolo contendere} and \textit{Alford} plea criticism tracks his broader rebuke of efficiency-driven plea bargaining. He argues that courts and legislatures should scrap these pleas as they encourage defendants to avoid confessions, undermine the law's moral authority by allowing defendants to cling to their dissonant values, and bypass the curative trial process to a universal detriment. I argue that although this concern for substantive criminal law is compelling, it is ultimately misdirected. I answer these substantive criticisms under three general headings:

\begin{itemize}
    \item \textbf{Accepting valuable compromises instead of questionable confessions.} Sincere confessions may be preferable to ambiguous admissions, but that does not mean the latter have no value. \textit{Nolo contendere} pleas help secure otherwise unobtainable convictions, mete out punishments to defendants willing to accept them, and allow more defendants to self-convict with a clear conscience.
    \item \textbf{Exposing ideological divides and explicating educational opportunities.} When defendants are allowed to accept the benefits of plea bargaining while refusing to admit guilt outright, they expose an ideological divide between themselves and the state convicting them. Once this divide is explicit, courts have a unique opportunity to investigate its causes and educate those whose private morality is at odds with public standards.
    \item \textbf{Avoiding the harms unwanted trials cause.} Defendants who cannot or will not plead guilty must set a trial. This requirement exposes defendants and all justice system participants to the potentially harmful effects of trials and opens the proceedings up to wrongful acquittals. Expanding the scope of self-convictions reduces these unwanted harms and limits these unjust results.
\end{itemize}

\subsubsection{Avoiding Confessions Through Ambivalent Pleas}

As I discussed in § 4.1.3 above, Bibas places immense value on confessions. I argued that he overvalues confessions by pointing to the apparent value of obtaining unobtainable convictions, the justice system's inability to ensure that confessions are sincere, and the lack of data quantifying the rehabilitative value of any confession. Rehabilitation will always reasonably favour a receptive offender. But without any logical reason or compelling evidence demonstrating that restricting \textit{nolo contendere} or other ambivalent pleas correlates to an increase in sincere guilty pleas, they should not be limited.

Nevertheless, even if Bibas overvalues confessions, this does not necessarily mean that his concerns about ambivalent \textit{nolo contendere} pleas are misplaced. Sincere confessions are valuable, even if they are not as valuable as Bibas believes, and cynically-entered \textit{nolo contendere} pleas can undermine substantive values, even if they are not the moral landslide he fears. While plea bargains may convince some guilty defendants to admit culpability, pleas like \textit{nolo contendere} arguably allow defendants to shirk responsibility for their offending behaviour. It is reasonable to assume that such defendants are less likely to rehabilitate than those whose consciences drive their confessions.

As a result, critics like Bibas may remain open to plea bargains while condemning non-culpatory self-convictions. This position allows Bibas to occupy a reasonable middle ground where plea bargaining is encouraged as long as defendants enter unambiguous pleas. From this vantage, he argues that the harms caused by plea negotiations come from defendants being allowed to avoid taking responsibility. \textit{Nolo contendere} pleas allow defendants to avoid confessions, thereby avoiding the rehabilitation that confronting hard truths about themselves and their actions would achieve. Bibas specifically highlights cases where defendants are ordinarily reluctant to confess\footnote{Sexual offences generally, sexual offences against relatives or children specifically, and other socially heinous crimes fall into this offence class. See \textit{ibid} at 1393 — 1395.} as prime examples of cases where courts, prosecutors, and defence lawyers \textit{should} compel them to plead guilty rather than encouraging plea deals or permitting them to plead \textit{nolo contendere}.\footnote{Bibas does not discuss the rehabilitative prospects of defendants convicted after trial.} This sharply contrasts with the conventional wisdom that encourages these pleas as reasonable compromises to resolve complex cases.

Bibas recognizes that punishment can play a part in reforming and deterring by educating them and society.\footnote{See Bibas, ``Harmonizing Substantive Values," \textit{supra} note 21 at 1390.} But the fact that \textit{nolo contendere} pleas specifically invite punishment, and may thus play a part in reforming and deterring offenders under this definition, is at odds with Bibas's view that \textit{nolo contendere} pleas undermine society's moral authority. Enabling more defendants to self-convict and safeguarding the means for them to do so provides defendants with more opportunities to receive punishment, thereby increasing the number of defendants who can be reformed, deterred, and educated through corrective discipline. 

It is reasonable to infer that the results will be less optimal than those obtained through sincere guilty pleas. But it is also reasonable to infer that the results will be no less optimal than those obtained through insincere guilty pleas and much more optimal than if a judge or jury acquits the defendant after a trial. Absent a reliable way to ensure sincere guilty pleas, defendants willing to self-convict should be allowed to do so regardless of whether they feel they deserve punishment.

More importantly, these pleas may be truthful in ways that a guilty or not guilty plea would not be. One of the law's intractable difficulties is that it relies on equivocal concepts and assumes that judges and juries use them reasonably. In criminal law, \textit{excuse} and \textit{justification} are two particularly difficult examples, as the justifications people provide for their potentially criminal actions do not always line up with legally valid excuses. For example, what Parliament and the courts define as a reasonable amount of force in an assault or all reasonable steps to determine a sexual partner's age will invariably differ from how individual defendants define those concepts. Justifications that individual defendants invoke to explain their actions may or may not coincide with legally valid excuses and defences. Once confronted by this reality, remorseless defendants who accept convictions accept the law's authority to punish them for their actions. This concession allows defendants to enter truthful pleas and exposes the ideological divide between these defendants and the state's broadly held values.

\subsubsection{Ideological Divides}

Defendants willing to self-convict may hesitate to enter a guilty plea for many reasons. One reason stems from the moral divide between individual defendants who commit offences and the state that prosecutes them. Bibas identifies this phenomenon but hastily treats it as a fundamental problem. I argue that these pleas can helpfully identify defendants whose morality is out of sync with society's broader values, thereby providing prosecutors, judges, community corrections officers, and other actors in the justice system with information that may assist with their rehabilitation.

Where defendants enter ambivalent pleas like \textit{nolo contendere}, Bibas contends that ``society loses the authoritative vindication of its norms and the repudiation of the wrong."\footnote{See \textit{ibid} at 1407.} While it is problematic for the state and its citizens to have competing or oppositional values, ambivalent uncontested pleas do not \textit{create} those differences, but merely \textit{expose} them. Rather than undermine the state's moral authority, as Bibas suggests, these pleas give defendants a forum where they can acknowledge the state's authority, \textit{notwithstanding} their differences. These pleas allow the state to gauge a defendant's rehabilitation prospects using more accurate, albeit potentially less encouraging, information. Forcing a defendant to plead guilty as their only route to self-conviction may offer some empty assurances that the defendant \textit{might come to accept their guilty pleas as true},\footnote{See \textit{ibid} at 1399.} but all but ensures that the court will receive no useful or accurate information about \textit{what the defendant believes to be true}. Instead, the interests of justice are better advanced by showing the public that the rule of law prevails over private opinion and morality. A defendant's prospects for rehabilitation can be better understood when these belief differences are explicated, thus advancing the substantive values of reconciliation, mutual responsibility, and accuracy.

Defendants who enter these pleas notify the court that a divide exists between their moral norms and society's. To the extent that the court ought to pursue substantive criminal values on sentencing, they are better positioned to deal effectively with a defendant if they know this divide exists. Defendants who enter insincere guilty pleas or plead not guilty send conflicting or no signals to the court about their beliefs or the truth. \textit{Nolo contendere} pleas, on the other hand, notify the court that the defendant is willing to accept punishment but unwilling to admit that they are guilty, even though pleading guilty unequivocally would very likely mitigate their sentence. This equivocation can and should prompt the substantively-inclined court to ask why, allowing the court to investigate and try to bridge that divide. 

Permitting defendants who neither feel guilty nor remorseful to accept their legal culpability allows them to use uncontested pleas honestly, convey their beliefs effectively, and identify their rehabilitative prospects to the court more accurately. Once the courts and others tasked with rehabilitating offenders understand what defendants think is true, they can better understand the circumstances of offences and offenders. Accurately identifying defendants who are not ideal rehabilitation candidates can help judges tailor fit and appropriate sentences, help probation officers identify clients who require extra or particular focus and alert law enforcement officers to potential community risks. Forcing such defendants to enter insincere guilty pleas or run trials negates these benefits and invites the other ill effects accompanying these unwanted contested hearings.

\subsubsection{The Harms Unwanted Trials Cause}

In § 4.1.3 above, I argued that Bibas's portrayal of trials as morality plays is at odds with the adversarial justice system and inevitably leads to unjust results. Modifying the criminal justice system in the manner he suggests would curtail the defence lawyer's role as an advocate, discourage resolution discussions between counsel, and inevitably result in wrongful convictions. Here, I argue that Bibas uncritically glorifies contested hearings as unqualified goods and ignores the harms that unwanted trials cause. To the extent that expanding permissible self-convictions to include \textit{nolo contendere} allows more defendants, victims, witnesses, court staff, jurors and other justice system participants to avoid these harms, they should be encouraged.

Bibas correctly points out that pleas like \textit{Alford} and \textit{nolo contendere} are likely to be less satisfying for victims and less conducive for rehabilitating offenders than a truly remorseful guilty plea. Unless defendants willingly admit guilt, Bibas proposes that courts should require them to set a trial. He views this outcome favourable as defendants either must confront their wrongdoing, thus availing themselves of the curative properties of confession and punishment, or be taught a lesson through the evidence presented at their morality play of a trial.\footnote{See \textit{ibid} at 1403 — 1404.} But he ignores the genuine possibility that some defendants who must set trials instead of self-convicting will be acquitted, possibly without having to testify or have their version of events tested in court. Trials can be complex affairs, and criminal charges must be proven beyond a reasonable doubt. Problems with securing witnesses or ensuring that all the necessary evidence is admissible are not uncommon. Even strong prosecutorial cases can collapse unexpectedly, resulting in last-minute plea dealings and fire-sale sentencing discounts. 

As Bibas fails to consider the costs of wrongful acquittals and frantic eleventh-hour resolution discussions, he fails to properly evaluate the benefits of sincere confessions and satisfying convictions. It is reasonable to assume that victims will be less satisfied with a non-commital \textit{nolo contendere} plea than a sincere guilty plea. However, it is equally reasonable to assume that they will be less satisfied with a wrongful acquittal than they would have been with a non-committal conviction. Similarly, a lackadaisical \textit{nolo contendere} plea to a substantive offence with a victim is likely to be more satisfying for victims of that crime than a sincere guilty plea to an administration of justice offence a defendant picked up while awaiting trial.

Wrongfully acquitted defendants who would otherwise accept a conviction may not generate the same outrage that wrongful convictions do, and rightfully so. However, for victims who see their perpetrators escape justice, wrongful acquittals still undermine confidence in the justice system and subvert its substantive goals. Although much less harmful than wrongful convictions, wrongful acquittals are untruthful results and miscarriages of justice that leave the guilty unpunished, absent further civil or criminal court proceedings.\footnote{See MacFarlane, \textit{supra} note 210 at 483.} Meanwhile, factually guilty defendants willing to resolve are required to set their matters down for trial. They may spend more time on release conditions or confined to pre-trial detention than if they had been able to enter \textit{nolo contendere} pleas. It is dubious whether the defendant will be more prepared to sincerely admit guilt or participate in rehabilitative measures after hearing the matter tried than when they would have agreed to self-convict before accumulating these process costs. And likely, the unnecessary expense of the trial and the burdens that the hearing imposed on the victim or any other witnesses will be on the sentencing judge's mind.

Requiring defendants to set trials when none of the parties involved want one ensures that there is one less trial date available for a truly contested matter. Society does not have a legitimate interest in requiring unwilling and otherwise cooperative defendants, victims, and professional justice system participants to engage in actual show trials. Defendants who must have their allegations adjudicated, even where they lack viable defences, are forced to spend unnecessary time in pre-trial detention or constrained by pre-trial release conditions. At the same time, defendants with a viable defence are similarly delayed and imposed upon. Those willing to accept a final judgment to end these restrictions should be permitted to do so. Victims should be allowed to see punishment meted out, restitution made, and the process concluded, even if that sentencing package does not include a believably repentant defendant. 