\section{The advantages of access equivocal no-contest pleas}

Even if it's been demonstrated that equivocal no-contest pleas aren't the scourge they're so often made out to be, this doesn't necessarily mean that they ought to be implemented

The fact that something shouldn't be actively discouraged doesn't imply that it should therefore be actively encouraged

The fact that something isn't an evil may simply mean that it's a neutral thing, rather than a good

But as alluded to in 4.3.3: Substantive values and self-convictions, negotiated plea bargains may be thought of as a good in themselves

To the extent that equivocal no-contest pleas opens up the benefits of plea bargaining to a new class of defendants in a more fulsome and forthright manner, this section argues that they should be allowed and encouraged

As will be outlined further in the sections that follow, allowing more defendants to self-convict on their terms

\paragraph{Increased certainty in the outcome\\}

Plea bargaining generally

Having some certainty in the outcome of a case is an motivator for all plea bargains, even those involving defendants who ultimately intend to plead guilty

Just because a defendant has agreed to plead guilty doesn't mean there aren't concessions that the prosecution or complainant may still want with respect to the sentence, the charges pleaded to, etc

Equivocal no-contest pleas specifically

Where defendants inclined to enter an equivocal no-contest plea are required to have their matter go to trial if they can't or won't plead guilty, the increased certainty in the outcome is  important

\paragraph{Increased control over the facts\\}

Plea bargaining generally

Just as with the outcome, the fact that a defendant agrees to plead guilty doesn't mean that there aren't concessions the prosecution or victims may still want with respect to the facts pleaded to

The prosecution is still required to prove all aggravating sentencing factors beyond a reasonable doubt if any of those are disputed

Even factors that aren't strictly aggravating may be such that

Equivocal no-contest pleas specifically

Prosecutors arguably have more leeway over the facts in equivocal no-contest pleas

The defendant, after all, isn't admitting to or contesting the plea

One might expect that a defendant would be less inclined to put up a fuss about the facts where it's already on record that they aren't admitting or agreeing to them

Proof of aggravating factors should (but wouldn't necessarily) transfer to equivocal no-contest pleas as well, even if just ideologically

\paragraph{Increased agency and self-determination\\}

Plea bargaining generally

Having some measure of control over the outcome and the facts is increased self-determination

This increased self-determination also increases a person's agency in the situation

It stands to reason that defendants with more agency in their proceedings will ultimately have a higher regard for the justice system

Equivocal no-contest pleas specifically

Chance to privately explain/deny the offence

Able to contest subsequent proceedings, where available

Remain authentic to a moral standpoint while accepting the inevitable

\paragraph{Access to otherwise rare win-win scenarios}

Plea bargaining generally

Both parties agree to the outcome of a case

Amicable outcome is the definition of a win-win

All parties and their witnesses are relieved of the pressures that accompany court hearings

Criminal proceedings are otherwise geared towards being very adversarial

Equivocal no-contest pleas specifically

Additional potential "wins" for the defendant entering the plea

Additional potential "wins" for the prosecutor seeking to apply or enforce conditions

Additional avenues for all parties to seek resolution

\subsection{Increased certainty in the outcome of the proceedings}

Not certainty as it relates to the "truth-seeking function of the trial", but more as it relates to a defendant's interest in seeing the facts come out a certain way

A defendant who can't or won't enter a guilty plea, but is nonetheless wiling to agree to be convicted and sentenced as though they had, is formally required to put the state to its burden and have a trial

The defendant doesn't want to have a trial for one of many reasons a defendant may not want to have a trial

Cost of proceeding versus the remoteness of the likelihood of success

Fear or discomfort about being confronted with evidence outweighs the fear or discomfort of being punished for the offence

Desire to not have one or more witnesses testify

Inability to separate moral culpability from a formal guilty plea

Belief that one is morally in the right despite being legally in the wrong

Where the state proceeds on a charge, running a trial obviously has no advantages over a confirmed conviction

Either the defendant or the state may have the facts of the offence come out more favourably for them than they would have on an agreed statement of facts

Neither side has any apparent advantage in this respect over the other

A defendant can theoretically allow the Crown to call its case, contest none of its evidence, and still be acquitted due to some deficit in the Crown's case

Likely an extremely rare event

Nothing that wouldn't be found by a judge hearing submissions on whether the facts supported the charge at a guilty or other no-contest plea hearing

Nothing much to distinguish this procedure, as described, from a guilty plea

Likely only a disadvantage for the defendant, as trial dates tend to be harder to secure than guilty plea dates

This leads to a lose-lose, in that both Crown and defence are required to opt for the uncertain outcome

Allowing this class of defendants to self-convict gives its members and the Crown attorneys prosecuting them greater control over the outcome of the proceedings

\subsection{Increased control over the facts}

Plea bargaining generally

Just as with the outcome, the fact that a defendant agrees to plead guilty doesn't mean that there aren't concessions the prosecution or victims may still want with respect to the facts pleaded to

The prosecution is still required to prove all aggravating sentencing factors beyond a reasonable doubt if any of those are disputed

Even factors that aren't strictly aggravating may be such that

Equivocal no-contest pleas specifically

Prosecutors arguably have more leeway over the facts in equivocal no-contest pleas

The defendant, after all, isn't admitting to or contesting the plea

One might expect that a defendant would be less inclined to put up a fuss about the facts where it's already on record that they aren't admitting or agreeing to them

Proof of aggravating factors should (but wouldn't necessarily) transfer to equivocal no-contest pleas as well, even if just ideologically

Like an increased control over the certainty of the outcome of a case, equivocal no-contest pleas give both prosecutors and defendants an increased degree of control over both which facts are presented and how those facts are framed

Which facts are presented

Both parties have the same advantage when it comes to guilty pleas

Prejudicial or embarrassing facts can be negotiated down or out

Aggravating facts can be negotiated up or in

Where there's a dispute as to any aggravating factor on sentencing, the Crown must prove the aggravating factor beyond a reasonable doubt

Some bargaining power accrues to the defendant as a result

How those facts are framed

Allows the defendant to maintain a position of reduced moral blameworthiness, even if only privately, while acknowledging legal guilt

Discussing the factual presentation of an offence like this sounds Machiavellian, but is arguably the best way to guarantee a truthful outcome

As with certainty of outcome, the uncertainty inherent to the trial process is a concern for how facts about the offence and offender are ultimately received

Witnesses, including defendants and complainants, may unexpectedly testify poorly (or well)

Witnesses may fail to show up at all

Expert evidence might be disregarded and disbelieved

The judge or jury may place unexpected weight on facts both parties considered immaterial

Each of these factors can have a very volatile effect on a case, and almost certainly lead to inaccurate results

Both parties benefit by knowing which facts they want to be admitted, how they want those facts to be interpreted, and how they intend for those facts to be presented

Both parties benefit by having the opportunity to collaborate on the presentation and sentence recommendation

Increased control over these variables limits the reasonably likely outcomes of a case

\subsection{Increased agency and self-determination}

\subsubsection{Correlation between agency/self-determination and self-conviction}

Plea bargaining generally

Having some measure of control over the outcome and the facts is increased self-determination

This increased self-determination also increases a person's agency in the situation

It stands to reason that defendants with more agency in their proceedings will ultimately have a higher regard for the justice system

Equivocal no-contest pleas specifically

Chance to privately explain/deny the offence

Able to contest subsequent proceedings, where available

Remain authentic to a moral standpoint while accepting the inevitable

\subsubsection{Should defendants have this increased agency and ability to self-determine?}

Some may question whether criminal defendants should be entitled to increased agency and self-determination

Criminal proceedings don't always involve clear-cut fact patterns where the clearly guilty offend against the clearly innocent

The line between complainant and defendant can be a very thin one in some cases

Example?

It's therefore unsurprising that such criminal defendants will feel very much aggrieved when called upon to plead, even if they are prepared to admit sufficient facts to establish their guilt

Requiring such defendants to set their matters down for loser trials does very little to increase their sense of agency in the proceedings

Even where defendants aren't borderline victims, but are clearly morally blameworthy aggressors and instigators, there is very little value in curbing their ability to make decisions about their case that will benefit themselves, the state, and the complainants

The fact that a defendant is willing to agree to self-convict and spare themselves, the prosecutor, and any witnesses involved the burden and uncertainty of a trial is a clear win-win-win

This will be expanded on in section,4.4.4: The elusive win-win scenario in criminal law

The fact that it isn't the greatest possible win for prosecutors and complainants isn't sufficient to justify not allowing defendants access to them

"A win is a win" mentality

Prosecutors aren't entitled to a conviction, and complainants aren't entitled to a remorseful defendant

Efficiencies aside, the net gains for everyone involved far exceed any perceived losses at not having the defendant acknowledge their moral culpability or truly repent for their misdeeds

\subsection{The elusive win-win scenario in criminal law}

Plea bargaining generally

Both parties agree to the outcome of a case

Amicable outcome is the definition of a win-win

All parties and their witnesses are relieved of the pressures that accompany court hearings

Criminal proceedings are otherwise geared towards being very adversarial

Equivocal no-contest pleas specifically

Additional potential "wins" for the defendant entering the plea

Additional potential "wins" for the prosecutor seeking to apply or enforce conditions

Additional avenues for all parties to seek resolution

Consider confidential informants (CIs)

Information provided by CIs is often very effective in helping secure convictions

This is especially true in the sorts of criminal operations that cause widespread social harm, including organized crime and drug trafficking

CI information can be very difficult to obtain outside of the plea bargaining context

Most confidential informants are criminals themselves, many of whom may be legally motivated through deals secured with the prosecution for reduced charges or generous recommendations on sentencing

Few are "ordinary citizens", and fewer still are motivated by civic duty

Contrast this win-win with the lose-lose situation that formalities currently require
