\section{The cognitive dissonance of non-inculpatory no-contest pleas}

One impediment to allowing defendants to enter non-inculpatory no-contest pleas is the cognitive dissonance experienced when trying to hold the conflicting values required to sustain them. Having a person plead guilty only to protest their innocence is superficially contradictory, in that they say one thing (or nothing at all) while accepting the consequences of its opposite. In this sense, no-contest pleas that either deny guilt outright or refuse to admit it appear equivocal. This apparent equivocation is problematic for several reasons:

\begin{itemize}
    \item \textbf{The truth problem.} Defendants who self-convict while refusing to admit guilt or while actively maintaining their innocence say one thing while meaning another. Doing so is untruthful, and lawyers who assist clients with such pleas mislead the court. 
    \item \textbf{The presumption of innocence problem.} People charged with crimes are presumed innocent until proven guilty. Where a defendant pleads guilty but protests their innocence, they should be presumed innocent and not be convicted. Similarly, where a defendant refuses to admit to the charges, their innocence should still be presumed, and they should not be convicted.
    \item \textbf{The fairness problem.} Where a defendant invites a self-conviction and either does not admit guilt or actively protests their innocence, it is reasonable to suspect their motives. Because these defendants disclaim their guilt one way or the other, the pleas are not overtly products of good motives, such as a desire to take responsibility for their actions or to ensure that the truth wins out. Such pleas thus appear to be products of unfairness. 
\end{itemize}

\subsection{The truth problem}

Convictions naturally follow where defendants plead guilty or are found guilty after trial. However, where defendants cannot or will not do so but nonetheless insist that the court treat them as though they did, they appear to contradict themselves. Where the plea's content is ``guilty," but the defendant's actions signify ``not guilty" or ``innocent," it raises an apparent problem of truthfulness. As this section will demonstrate, this problem is superficial and arises from a misunderstanding of the relationship between facts, belief, and proof on one hand and the relationship of these concepts to formal pleas and admissions on the other.

\subsubsection{Facts, propositions, belief, and proof}

If non-inculpatory no-contest pleas raise a problem of truth, it is essential to understand what ``truth" means. Throughout this section, ``truth" and ``fact" are closely related terms, such that they may even be used interchangeably in certain contexts (e.g., a proposition may be synonymously described as being ``the truth" or ``a fact"). \textit{Truth} is a quality that all \textit{facts} possess, such that anything that can be accurately described as being a \textit{fact} could also accurately be described as being \textit{true}.\footnote{The decision to primarily use the word ``truth" as an adjective and ``fact" as a noun is intentional, as it is more coherent to speak of multiple facts being true, rather than multiple truths being factual.} Obversely, there are no untrue facts. There are many distinctions that can be made between facts and the objects that they describe. However, for the purpose of this thesis, it will suffice to say that \textit{a fact is a statement about the world\footnote{By ``world," I mean the entire realm of objects that have existed, do exist, and will exist, as well as the relationships between those objects.} that is objectively true}, regardless of whether it is subjectively understood as such. Statements like ``AlphaGo beat Lee Sedol in four out of five Go games at the DeepMind Challenge Match," ``the earth orbits the sun," and ``the fridge to my right is white" are all examples of facts in that they are all statements about the world that are objectively true.

Not every statement about the world is a fact. Statements like ``Lee Sedol beat AlphaGo in three out of five Go games at the DeepMind Challenge Match," ``the sun orbits Jupiter," and ``the fridge on my right is orange" are all statements about the world that have not, do not, and will not obtain. All facts are statements about the world, but not all statements about the world are facts. I use the term \textit{proposition} to describe statements about the world that may or may not be objectively true.

Within this definition scheme, propositions are true or untrue independent of whether anyone is subjectively aware of the fact and regardless of whether anyone subjectively acknowledges the fact. The subjective component of a proposition arises in the form of \textit{beliefs}. For the purpose of this thesis, a belief is a subjective acceptance or rejection of a proposition based on the subject's perception of that proposition's truth. A subject believes a proposition when they accept it and disbelieves it when they reject it. Beliefs may be reasonable or unreasonable. Whether a belief can be called reasonable depends on whether that belief can be proven.

\textit{Proof} refers to the act of grounding a subjective belief in ostensibly objective criteria. A proposition is proven once the subject is satisfied there are justifiable reasons to believe it is true, and disproven once the subject is satisfied that there are either justifiable reasons to believe it is false or that there are no justifiable reasons to believe that it is true. Where multiple subjects share similar or identical criteria, those criteria may form a common standard of proof. Although proofs are generally founded on objective or external criteria, they are a predominantly subjective enterprise in that proofs exist to satisfy the subject that their belief is reasonable. 

Propositions that are true may be proven false, while propositions that are false may be proven true. A proposition, once proven, may be disproven. For example, at various stages in history, it was proven that women had fewer teeth than men,\footnote{See Bertrand Russell's article.} that the earth revolved around the sun,\footnote{E.g., the Ptolemaic universe.} and that Thomas Sophonow murdered Barbara Stoppel. These propositions have been disproven; therefore, none of these beliefs may be reasonably held today. Despite each of these propositions haven once been proven, each is untrue. This demonstrates that there is no necessary correlation between proof and fact.

\subsubsection{Pleas and admissions}

Having defined the terms above, the next step in addressing the truth problem is understanding what a defendant does when they enter pleas. A 

\textit{Criminal Code} s 655 allows defendants to admit any ``facts alleged [by the prosecutor] for the purpose of dispensing with proof thereof."\footnote{See \textit{Criminal Code} s 655. If \textit{facts} signify only true propositions, the phrase ``facts alleged" is nonsensical, as \textit{allegations} may be true or false. The phrase is best understood as synonymous with a \textit{proposition} and will be referred to as such.} 

Typically, these allegations will support or consist entirely of an element of the offence charged, though any proposition that both parties agree to can be admitted under this section. Just as a guilty plea obviates the need for the prosecutor to prove the offence, these formal admissions remove any need for either side to prove the fact.\footnote{See @2009mbca37 for a discussion of the distinction between formal and informal admissions.} The legislation does not require a judicial inquiry into the factual foundation.\footnote{Some jurisdictions have established a common law basis for judges to inquire into the facts admitted and accept or reject them based on the outcome of that inquiry. See @2019bcsc2048 at para 7.} Absent this oversight and any directive to the contrary, \textit{Criminal Code} s 655 does not require a defendant to have committed the essential elements of the offence or to admit that they did commit the essential elements of the offence. Propositions that are admitted and later found to be contradicted by evidence may nevertheless remain proven.\footnote{example?}

Prosecutors and defendants may only use \textit{Criminal Code} s 655 where they both agree that the fact in question is admissible.\footnote{See the old SCC case re formal admissions where the defendant tried to sneak one in without the Crown's approval. See §3.x above.} That prosecutors and defendants use this section strongly suggests that it benefits both.

Defendants may wish to forego proof of some or all of the facts of their case for several reasons. In some cases, there is no reason to doubt that the prosecution will prove certain offence elements. A prosecutor is likely to be able to prove elements like date, time, jurisdiction, and identification in a domestic violence prosecution without difficulty. Agreeing to these facts saves court time and judicial attention, which can be minimal resources in busy court circuits. Admissions can also be advantageous in situations where the evidence is less certain to be proven. For example, when the prosecution wishes to call a witness whose evidence will inevitably harm the defendant to one degree, the defendant may wish to have the witness excused in exchange for a manageable set of agreed facts. In cases where the only genuine dispute lies with a legal issue, both parties may agree to have all evidence admitted by consent and limit trial time to arguing that issue. 

Admissions are a way to signal to the court what the parties do and do not need to demonstrate. Each admission dispenses with the presumption of innocence and burden of proof for the fact admitted. Where the parties admit facts but decide to call evidence, witnesses, including the defendant themselves, may testify in a way that is inconsistent with those facts. However, the facts are nonetheless admitted, thus dispensing the need for proof. No-contest pleas admit \textit{all} of the essential elements needed to prove the offence and dispense with the presumption of innocence across all aspects of the offence charged.

The problem with this argument generally is that the presumption of innocence is the defendant's to waive. Like any other person, criminal defendants are susceptible to ``buyer's remorse," minimization, outright denial, and misunderstanding of legal concepts like ``self-defence." Nothing seemingly incongruous or unjust about a defendant who waives this presumption, only to resile in one form or another later in the proceedings.

\subsubsection{Correlation between facts/propositions/beliefs/proof and admissions/pleas}

Applying these definitions to the criminal plea procedure makes it clear that there is nothing incongruous about 

Both pleas and admissions are entirely within the realm of proof. A person making an admission may or may not believe the content of that admission but may nonetheless make it. 

\begin{itemize}
    \item \textbf{Admissions.} Copy some of that goodness from below.
    \item \textbf{Pleas.} By default, a not guilty plea requires the prosecutor to prove all of the facts against a defendant. By default, a no-contest plea relieves the prosecutor of this burden.
    \item \textbf{Facts and admissions.} 
    \item \textbf{Facts and pleas.}
\end{itemize}

\subsubsection{Duty to not mislead the court}

Lawyers are under a strict duty to not intentionally mislead the court

This includes calling evidence at a trial that the lawyer knows to be misleading

A common example is the lawyer who argues that a third party committed the crime after the defendant privately admitted their guilt

See the SK Code of Professional Conduct at Section 5.1-1 (10):

\begin{quote}
    Admissions made by the accused to a lawyer may impose strict limitations on the conduct of the defence, and the accused should be made aware of this. For example, if the accused clearly admits to the lawyer the factual and mental elements necessary to constitute the offence, the lawyer, if convinced that the admissions are true and voluntary, may properly take objection to the jurisdiction of the court, the form of the indictment or the admissibility or sufficiency of the evidence, but must not suggest that some other person committed the offence or call any evidence that, by reason of the admissions, the lawyer believes to be false. Nor may the lawyer set up an affirmative case inconsistent with such admissions, for example, by calling evidence in support of an alibi intended to show that the accused could not have done or, in fact, has not done the act. Such admissions will also impose a limit on the extent to which the lawyer may attack the evidence for the prosecution. The lawyer is entitled to test the evidence given by each individual witness for the prosecution and argue that the evidence taken as a whole is insufficient to amount to proof that the accused is guilty of the offence charged, but the lawyer should go no further than that.
\end{quote}

Though superficially convincing, and seemingly often used, this analogy falls apart in a few ways

It either overestimates or fails to account for how rarely a lawyer actually has the opportunity to **knowingly** or **intentionally** mislead the court in criminal matters

Many lawyers take steps to avoid learning this information in the first place

Those who don't will likely agree that they rarely have enough knowledge, or enough confidence in the knowledge imparted to them by their client, to knowingly or intentionally tell the court much of anything about the case

Like Stephanos Bibas, it recasts the defence lawyer's role as something other than it currently is

Stephanos Bibas also wanted to see defence lawyers take up a more inquisitional role, but for different reasons

Some investigation and thought is obviously required on a defence lawyer's part before they can enter pleas for their client, especially as the client may not realize they have a defence

But for a client whose clear and unambiguous instructions are to enter a guilty plea, there's only so much investigation and interrogation that a defence lawyer should be doing

There's also some question about whether one can convincingly say that a lawyer who hasn't done their due diligence in this manner is intentionally misleading the court

Unless the lawyer actively adduces facts that aren't true on behalf of their client, assisting them with entering a plea isn't misleading the court

The simple plea of "guilty" isn't sufficient, as will be explored immediately below

But that apparent contradiction is only superficial, the fact of which is borne out when examining the absence of any apparent contradiction with not guilty pleas

\subsubsection{Separating legal facts from legal fictions}

In most respects, the law is a relatively straightforward enterprise

But occasionally, legal problems arise that require abstract solutions that, at first blush, appear obtuse

Legal fictions are a means our law has used, and continues to use, in order to effect these solutions

These fictions allow the various actors in the justice system to make decisions and reach conclusions with the benefit of some assumptions in common

Some examples in criminal law

Reasonable person test

Foundation of many tests within the criminal law, especially as it concerns police and their grounds to detain, search, and arrest suspects without a warrant (or apply to a justice to obtain one)

Presumptions

Presumption of innocence

Deference to the trial judge

Presumptions about judges and juries

Public perception of justice

Key to decisions made about excluding evidence or releasing defendant's on bail

Not guilty plea

As discussed at some length in 4.1.1: The apparent untruthfulness of no-contest pleas

Equal treatment for equally-situated defendants

The practical impossibility of this is discussed further in 4.2.3: The trial penalty

In @bowersPunishingInnocent2007a, Josh Bowers recommends recasting equivocal no-contest pleas as legal fictions that have the actual effect of quickly curtailing wrongful pre-trial incarcerations

The potential effect of allowing factually innocent defendants to plead no-contest to end wrongful incarcerations without being entered on their criminal record (eg the rule in Rhode Island)

Asking the court to proceed as if the accused had committed the offences they're charged with, in the manner alleged by the prosecution, is just another of these fictions

\subsection{The presumption of innocence problem}

The presumption of innocence and the prosecutor's burden of proving an offence beyond a reasonable doubt are hallmarks of the common law justice system. Legal professionals and academics see these concepts as indispensable tools for fighting wrongful convictions. When properly implemented, these concepts place a significant burden on the prosecutor, so only those who are guilty of an offence stand any chance of being convicted. The joint operation of these concepts will inevitably result in some guilty persons walking away without conviction. However, this sacrifice is generally considered worth the added protections afforded to the truly innocent. 

These legal concepts seem to run contrary to the practice of non-inculpatory self-convictions. Where the prosecutor has the burden of proving the charges and the court must consider the defendant innocent until the prosecutor discharges that burden, it appears contradictory and unjust to convict a defendant who maintains their innocence either tacitly or explicitly.\footnote{Cite to one or more articles that take this position. Alschuler gets into this briefly, but the author who discusses Alford's ``sworn testimony" at length is the source here.} In cases where the defendant maintains their innocence, it superficially follows that they continue to be presumed innocent until the prosecutor discharges their burden. 

Although this argument appears to have some legs, it is ultimately baseless, as it overlooks the following key factors:

\begin{itemize}
    \item \textbf{Admissions.} Defendants are entitled to make admissions against their legal interests, which may obviate the presumption of innocence.
    \item \textbf{Factual foundation.} Where non-inculpatory self-convictions are permitted, courts typically must be satisfied that there is a factual foundation for the conviction apart from the plea itself. This stipulation generally mirrors the same requirement for inculpatory self-convictions.\footnote{Although this is true for most jurisdictions, it is not universally so. For example, \hl{X-state} requires a factual foundation for guilty pleas, but not for \textit{nolo contendere} pleas.} 
    \item \textbf{Judicial discretion.} Although Canadian courts must accept not guilty pleas on their face, judges have ultimate discretion over whether to allow a defendant to self-conviction.\footnote{Again, although this is true for most jurisdictions, it is not universally so. For example, \hl{Y-state} insists that courts must accept \textit{nolo contendere} pleas. Similarly, the \textit{nolo contendere} ``plea procedure" used in Canadian courts is arguably beyond judicial discretion, as discussed in \hl{Z-section}.}
\end{itemize}

\subsubsection{Factual foundation}

Notwithstanding that defendants can legally admit facts against their interests, concerns that they might do so inappropriately may reasonably remain. While there is nothing implicitly wrong with a defendant admitting a fact against their interests, there may be problems with defendants admitting baseless facts against their interests. There may also be problems with defendants pleading guilty to charges whose underlying facts do not meet the legal threshold for a conviction. For these reasons, common law courts must usually verify the offence's factual foundation. In Canada, judges must be satisfied that the facts support the charges.

The facts underlying both the \textit{Alford} and \textit{Hector} decisions are instructive in this respect. In \textit{Alford}, the defendant 

\textit{Alford} and \textit{Hector} both show that 

Nothing prevents legislatures from requiring judges to more rigorously examine the factual foundation of non-inculpatory self-convictions than they would be required to do for inculpatory pleas. 

\subsubsection{Discretion}

Judges still have discretion over whether to reject or accept self-convictions.

\subsection{The fairness problem}