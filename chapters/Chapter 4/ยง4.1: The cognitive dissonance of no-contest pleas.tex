\section{The cognitive dissonance of non-inculpatory no-contest pleas}

One impediment to allowing defendants to enter non-inculpatory no-contest pleas is the cognitive dissonance experienced when trying to hold the conflicting values required to sustain them. Having a person plead guilty only to protest their innocence is superficially contradictory, in that they are heard to say one thing (or nothing at all) while accepting the consequences of its opposite. In this sense, no-contest pleas that either deny guilt outright or simply refuse to admit it appear equivocal. This apparent equivocation is problematic for several reasons:

\begin{itemize}
    \item \textbf{The presumption of innocence problem.} People charged with crimes are presumed innocent until proven guilty. Where a defendant pleads guilty but protests their innocence, they should be presumed innocent and not be convicted. Similarly, where a defendant refuses to admit to the charges, their innocence should still be presumed, and they should not be convicted. Admissions against legal interests obviate the presumption of innocence.
    \item \textbf{The truth-seeking problem.} Where a defendant cannot or will not plead guilty, the alternative is a trial. People consider trials a search for the truth where all evidence may be looked at and examined. By contrast, they see negotiated plea deals resulting in equivocal pleas as an obfuscation of the truth. 
    \item \textbf{The fairness problem.} 
\end{itemize}

\subsection{The presumption of innocence and the prosecutorial burden of proof}

The presumption of innocence and the prosecutor's burden of proving an offence beyond a reasonable doubt are hallmarks of the common law justice system. Legal professionals and academics alike see these concepts as indispensable tools for fighting wrongful convictions. It is thought that, when properly implemented, these concepts place so significant a burden on the prosecutor that only those who are clearly guilty of an offence stand any chance of being convicted. It is also thought that the joint operation of these concepts will inevitably result in some guilty persons walking away without being convicted, but that this sacrifice is worth it in the long run.



Where it's assumed that the prosecutor has the burden of proving the charges, and that the defendant must be considered innocent until that's been done, it is apparently contradictory to see a defendant be convicted either without admitting their responsibility, or even while actively denying it

As with pleas, admissions are a special type of court statement

@2009mbca37:

Admissions may be formal or informal

Informal admissions are like declarations, and may be judged to be true or false at a trial by the trier of fact

A subsequent denial of an offence to a pre-sentence report writer is an informal admission

Formal admissions, on the other hand, are like pleas, in that they signal to the prosecutor and the court that certain facts, propositions, and claims will not need to be proven

A guilty plea is a formal admission

Both guilty and **nolo contendere** pleas, equivocal or otherwise, have the effect of admitting the prosecutor's case

The **Criminal Code** specifically states that a defendant "admits the essential elements of the offence" by pleading guilty to it

In other words, a guilty plea relieves the prosecutor of the burden of having to prove the essential elements of the offence

It importantly does not require that a defendant to have in fact committed the essential elements of the offence, or to admit that they did, in fact, commit the essential elements of the offence.

Not all courts agree

See @2019bcsc2048 at para 7

Facts that are admitted and later found to be contradicted by evidence may nevertheless remain proven

Example

Witnesses who contradict admitted facts may have adverse inferences drawn about them, but the facts will remain admitted

Example

As outlined above, defendants have been able to admit adverse facts against themselves since the first **Criminal Code**

Defendants may wish to forego proof of some or all of the facts of their case for any number of reasons

Admitting facts overrides the presumption of innocence not by admitting actual responsibility, but by telling the court directly that the facts admitted should be considered to have been proven

There is nothing inherently contradictory about a system that distinguishes formal admissions from informal ones

The presumption of innocence can be formally waived, regardless of whatever buyer's remorse or subsequent clarifications such defendants end up expressing after the fact

\subsection{Guilty pleas as declarations of the truth}
On the surface, a plea is a declaration of the truth

A guilty person seems to declare that they are such when they plead guilty

A person who both declares their guilty in this manner and, at the same time, protests their innocence, seems to be engaging in nonsensical behaviour

Several court decisions appear to reinforce this understanding

Example

Others are more circumspect in the way they frame guilty pleas

Example

When a defendant pleads guilty to an offence, absent more, the most immediate and reasonable assumption is that they are admitting that they did, in fact, commit that offence

The truth underlying this statement turns out to be more nuanced than that

\subsection{Truth and validity do not necessarily correlate to formal admissions}

A formal admission does not need to be factually accurate or ``true" in order to be valid. 

Similarly, though arguably to a lesser extent, there is a dissonance when a person refuses to admit that they are guilty, only to be found guilty without a trial or evidence.

There is, therefore, the potential for a cognitive dissonance that will need to be addressed for some before equivocal no-contest pleas become a viable topic for discussion.

Trials are not an unqualified search for the truth and may not be the best means of obtaining the factual truth in criminal proceedings.

\subsubsection{Equivocal no-contest pleas}

Similarly, on the surface, equivocal no-contest pleas appear to be duplicitous

The defendant either pleads guilty, or at the very least refrains from declaring that they are not guilty

But despite either declaring guilt or refusing to declare that they are not guilty, and despite being convicted of the offence as a result, the defendant may later state that they're factually innocent of the charge

On the surface, these no-contest pleas appear to be duplicitous

A person appears to be pleading one thing while claiming something else

Or they're withholding information that, if true, ought to be admitted

This apparent conundrum is sometimes framed in terms of the need to ensure that the court isn't misled, and counsel's role in that respect

Lawyers are under a strict duty to not intentionally mislead the court

This includes calling evidence at a trial that the lawyer knows to be misleading

A common example is the lawyer who argues that a third party committed the crime after the defendant privately admitted their guilt

See the SK Code of Professional Conduct at Section 5.1-1 (10)

Admissions made by the accused to a lawyer may impose strict limitations on the conduct of the defence, and the accused should be made aware of this. For example, if the accused clearly admits to the lawyer the factual and mental elements necessary to constitute the offence, the lawyer, if convinced that the admissions are true and voluntary, may properly take objection to the jurisdiction of the court, the form of the indictment or the admissibility or sufficiency of the evidence, but must not suggest that some other person committed the offence or call any evidence that, by reason of the admissions, the lawyer believes to be false. Nor may the lawyer set up an affirmative case inconsistent with such admissions, for example, by calling evidence in support of an alibi intended to show that the accused could not have done or, in fact, has not done the act. Such admissions will also impose a limit on the extent to which the lawyer may attack the evidence for the prosecution. The lawyer is entitled to test the evidence given by each individual witness for the prosecution and argue that the evidence taken as a whole is insufficient to amount to proof that the accused is guilty of the offence charged, but the lawyer should go no further than that.

Though superficially convincing, and seemingly often used, this analogy falls apart in a few ways

It either overestimates, or fails to account for how rarely a lawyer actually has the opportunity to **knowingly** or **intentionally** mislead the court in criminal matters

Many lawyers take steps to avoid learning this information in the first place

Those who don't will likely agree that they rarely have enough knowledge, or enough confidence in the knowledge imparted to them by their client, to knowingly or intentionally tell the court much of anything about the case

Like Stephanos Bibas, it recasts the defence lawyer's role as something other than it currently is

Stephanos Bibas also wanted to see defence lawyers take up a more inquisitional role, but for different reasons

Some investigation and thought is obviously required on a defence lawyer's part before they can enter pleas for their client, especially as the client may not realize they have a defence

But for a client whose clear and unambiguous instructions are to enter a guilty plea, there's only so much investigation and interrogation that a defence lawyer should be doing

There's also some question about whether one can convincingly say that a lawyer who hasn't done their due diligence in this manner is intentionally misleading the court

Unless the lawyer actively adduces facts that aren't true on behalf of their client, assisting them with entering a plea isn't misleading the court

The simple plea of "guilty" isn't sufficient, as will be explored immediately below

But that apparent contradiction is only superficial, the fact of which is borne out when examining the absence of any apparent contradiction with not guilty pleas

\subsubsection{Counterexamples}

\paragraph{The not guilty plea\\}

A defendant who protests their innocence after pleading guilty appears to be contradicting themselves

The same apparent contradiction does not appear to hold true when considering a defendant who declares that they are innocent, pleads such, is subsequently convicted after trial, but continues to declare that they're innocent

Part of the lack of apparent contradiction comes from the fact that a trial has been held

The defendant has been given their chance to prove that they weren't guilty, as it were, and wasn't able to do so

One problem with this approach is that it ignores the factual foundation requirement

Some of the other problems with this approach, such as the actual deficiencies with trials and unexpected efficacies of plea bargains, should be explored more in 4.1.3: The apocryphal truth-seeking function of the trial

But that exact same defendant would not be any more or less guilty if that trial hadn't been held

Nor is that same apparent contradiction present when considering a defendant who pleads not guilty, hoping to succeed on an evidentiary issue, knowing full well that they are in fact guilty of the offence they're charged with

A witness who testifies in court stating that they did not do something that they in fact did is guilty of perjury, and may be charged with such

That same witness, as a defendant, may plead not guilty to something that they did in fact do without being guilty of perjury, or chargeable on that offence

Similarly, a defendant who pleaded guilty to a charge, only to be cleared with new evidence, would not have to worry about being prosecuted for their wrongful conviction

A judge may very well eviscerate a witness who said they didn't do something they did, but would be entirely out of line to do so for a defendant who pleaded not guilty to a crime they in fact committed

One would expect that if a plea were to be taken as evidence, it would need to be made under oath

Similarly, because a not guilty plea is understood as being something other than a declaration with truth content, counsel who allow or even encourage their clients to plead not guilty are not thought of as misleading the court

The way that the courts treat not guilty pleas is a clear signal that pleas are not simple declarations, and shouldn't be evaluated as such

There isn't anything inherently contradictory about a person entering a plea

\subsubsection{Conclusion}

It's necessary to distinguish between the different roles that different types of statements have to play in legal proceedings, as well as the "truths" that they're intended to convey

Just as a not guilty plea is not a declaration that the defendant is not, in fact, guilty, a guilty plea should not be understood as a declaration that the defendant is, in fact, guilty

When a defendant pleads not guilty, a trial is held to determine the question

When a defendant pleads guilty, a factual foundation is presented

A guilty plea is what the statute says it is: an admission of the essential elements of the offence and an invitation for the court to convict

Likewise, a not guilty plea signals to the court that none of the essential elements of the offence are admitted, and invites the prosecution to prove its allegations beyond a reasonable doubt

\subsection{The apocryphal truth-seeking function of the trial}

Defendants who are unwilling or unable to offer a guilty plea may be required to set their matters down for trial

Follows long-standing common law tradition

There are some reasons why they may not be able to do so

Judge will only accept an unequivocal plea

Defendant conflates moral blameworthiness with a guilty plea

Etc

A trial may therefore be said to be the "default mode" for resolving criminal disputes, notwithstanding the fact that relatively few criminal charges end up going to trial

Because trials take place when parties disagree with one another, they are often branded as a "search for the truth"

Each party has the opportunity to present its best evidence and its best interpretation of the evidence

Witnesses are cross-examined on their evidence, identifying the strengths and weaknesses of their testimony

A neutral trier hears the facts, listens to both sides argue, and decides between them

By contrast, equivocal no-contest pleas appear to be designed to obfuscate or misrepresent the truth for the purpose of getting negotiated agreements through the courts

Negotiations between prosecutors and defence lawyers are privileged and inadmissible as evidence

There are no judges or juries to provide neutral adjudication

The offence is usually presented in neutral statements of fact, rather than as a living discovery of the truth through cross-examination and evidence

Defendants can privately (or publicly) resile their admissions without being contradicted by evidence heard at a public trial

But the function of a common-law criminal trial is not to "search for the truth"

Rather, the function of a common-law criminal trial is to allow the trier of fact to make decisions based on curated sets of facts, within the confines of a restrictive sets of rules

There are important differences between these two concepts

Seeing a trial as a "search for the truth" implies that truth is the primary function of the trial, which subsequently implies that truth is the value in a trial that all other values are subservient to

A public inquest into a social problem, like the Truth and Reconciliation hearings in ZA, or missing and murdered indigenous women in Canada, would be comparable examples of this sort of "no holds barred" approach to the truth

Trials, on the other hand, require the fact finder to consider a less complete and arguably less true version of the evidence, as in cases where evidence has been excluded

In criminal trials, the "truth-seeking function" of the trial is always counterbalanced by the need to ensure fair proceedings

Truth is, in fact, severely curtailed at trial by things like

What truth the trier of fact is allowed to hear

Excluded evidence

Excluded testimony

Pre-trial and other evidentiary motions

Testimony from witnesses who weren't called or didn't attend

Why they aren't allowed to hear it

CC 276 and related applications

Prior consistent statements and other rules of evidence

Who they can hear evidence from

Hearsay rules

Opinion evidence rules

How they are to interpret the facts when they hear them

Even once the evidence is admitted, special instructions need to

Practically speaking, a great deal of truth that could be presented at trial is often excluded simply for the sake of a coherent presentation

In all cases, but especially complex ones, lawyers looking to present their evidence to a judge or jury will regularly cut out large swaths of the available evidence

A lawyer's job at a common law criminal trial is to advocate for a position

Effectively advocating for a position requires the advocate to account for the frailties of human attention and comprehension

This in turn requires both extensive editing of the truth and careful framing of that truth within the context of a larger argument

Information at trial is highly pre-processed before it reaches the court, and needs to be further processed once it gets there. This ensures both procedural fairness and effective presentation of the truth.

Procedural fairness and presentation are key trial concerns because the truth-seeking mechanism is unreliable

The "finders of fact" at trial are no more implicitly qualified at "seeking truth" than anyone else

In the case of jurors, the "finders of fact" are, in effect, "anyone else"

Experience

The hope and expectation is that the fact finders will come to true and fair conclusions through the combined presentation of true and fair evidence

But this isn't guaranteed

Rules have been put in place following decades of learning about the frailties of evidence

DNA evidence created a new class of cases for courts to review

The theory is that certain convictions could be absolutely confirmed or disconfirmed based on how DNA tests came back

Where wrongful convictions were found, we were able to gain some insights into the types of evidence that led jurors to those conclusions

Other forensic experiments in other fields have confirmed the unreliability of certain types of evidence

Eyewitness evidence

Shortcomings in memory

Etc

More confirmed wrongful convictions appear to have resulted from trials rather than negotiated sentences

But until we have new breakthroughs that can offer insight into how well or poorly forensic evidence reflects the truth, there isn't another independent truth-seeking function at play

Other truth-seeking mechanisms, like resolution discussions and plea bargaining, are likely better suited for reaching outcomes that are both true and fair

Both (sets of) lawyers in a criminal action have intimate familiarity with the facts and nuances of the case. They represent the different interests in the action - or at least, the interests that have standing in the action. Assuming ethical actors with access to relatively complete and accurate information, any resolution worked out between the parties by consent is likely to be close to a true and just result.

This also assumes there's some *quid pro quo* between the parties in their resolution discussions, such that the evidence is already being "tested", in a sense

Where a resolution can't otherwise be reached, trials are a reasonably effective means of adjudicating disputes, ensuring all sides have their best explanation of the evidence heard, and deferring to an ostensibly neutral third party to decide between competing narratives.

But trials are probably better understood as adequate alternatives to dispute resolution, rather than a preferred means of dispute resolution

Negotiations between counsel, both of whom have legal training, are duty-bound to advocate for their positions, have access to the fullest permissible truth of a criminal matter, and the opportunity to spend extensive time and resources investigating the nuances thereof, are a much greater truth-seeking device than a criminal trial

\subsection{Separating legal facts from legal fictions}

In most respects, the law is a relatively straightforward enterprise

But occasionally, legal problems arise that require abstract solutions that, at first blush, appear obtuse

Legal fictions are a means our law has used, and continues to use, in order to effect these solutions

These fictions allow the various actors in the justice system to make decisions and reach conclusions with the benefit of some assumptions in common

Some examples in criminal law

Reasonable person test

Foundation of many tests within the criminal law, especially as it concerns police and their grounds to detain, search, and arrest suspects without a warrant (or apply to a justice to obtain one)

Presumptions

Presumption of innocence

Deference to the trial judge

Presumptions about judges and juries

Public perception of justice

Key to decisions made about excluding evidence or releasing defendant's on bail

Not guilty plea

As discussed at some length in 4.1.1: The apparent untruthfulness of no-contest pleas

Equal treatment for equally-situated defendants

The practical impossibility of this is discussed further in 4.2.3: The trial penalty

In @bowersPunishingInnocent2007a, Josh Bowers recommends recasting equivocal no-contest pleas as legal fictions that have the actual effect of quickly curtailing wrongful pre-trial incarcerations

The potential effect of allowing factually innocent defendants to plead no-contest to end wrongful incarcerations without being entered on their criminal record (eg the rule in Rhode Island)

Summarize the argument

The apparent untruthfulness of no-contest pleas isn't really a matter of truth, per se

It's a simple matter of understanding that what a defendant communicates through their plea is qualitatively different than what they may communicate through sworn testimony or evidence

It's a matter of understanding the role and function of formal admissions, and how these differ from informal admissions of the truthfulness of alleged facts

Asking the court to proceed as if the accused had committed the offences they're charged with, in the manner alleged by the prosecution, is just another of these fictions

Understanding no-contest pleas in this way is a way to move past the cognitive dissonance of allowing a person to plead guilty when they continue to assert their innocence