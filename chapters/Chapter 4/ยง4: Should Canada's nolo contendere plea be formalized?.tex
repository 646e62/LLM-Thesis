\chapter{Should Canada's \textit{nolo contendere} plea be formalized?}

\setcounter{footnote}{147}

\textit{Nolo contendere} pleas are predominantly the product of plea bargains. Federally, \textit{nolo contendere} pleas require prosecutors to consent before the judge may accept the plea,\footnote{Several states also follow this model. See note 82 above.} as does the \textit{nolo contendere} procedure used in Canada. Where defendants and prosecutors must agree for the court to accept a plea, that plea is likely the result of a plea bargain, as prosecutors must have some incentive to allow a defendant to enter an ambiguous \textit{nolo contendere} plea rather than admit that they are guilty. Because victims, police, and prosecutors reasonably prefer a contrite defendant pleading guilty over a waffling defendant self-convicting, a prosecutorial agreement is only likely where some \textit{quid pro quo} is available. Furthermore, while conscience may motivate some defendants to plead guilty, it is much less likely to compel a defendant to enter the more ambivalent \textit{nolo contendere} plea.

Therefore, to determine whether Parliament should formally incorporate these pleas into Canadian criminal law, it is first necessary to examine the propriety of plea bargaining more generally. If plea bargaining is an implicitly suspect enterprise that lawyers and defendants should avoid where possible, this fact should caution against using uncontested pleas more often or expanding the catalogue of such pleas that a defendant may enter. Plea bargaining's disadvantages may be so significant that further enabling the practice with \textit{nolo contendere} pleas is manifestly irresponsible. However, if plea bargaining is an implicitly worthwhile and valuable process or even an ethically neutral one, it may be worthwhile to consider formalizing \textit{nolo contendere} pleas.

The first part of my analysis investigates plea bargaining to identify and answer these concerns. I consider the problems with truth, fairness, and moral values that plea bargaining's critics argue are inherent to the practice. Although plea bargaining is imperfect and susceptible to abuse, I argue that its benefits far exceed its pitfalls. But even assuming that plea bargaining is either generally salvageable as a practice or commendable as an institution, reasonable concerns may remain about allowing \textit{nolo contendere} pleas to form part of the plea bargaining matrix. \textit{Nolo contendere} pleas bring their own ethical concerns, such that they warrant independent consideration. The second part of my analysis thus examines \textit{nolo contendere} pleas specifically to determine what ethical issues these pleas entail. I conclude that these issues can be addressed, and offer several potential solutions. The final portion of my analysis adopts insights from both preceding portions to determine whether \textit{formalizing} these pleas in Canada may help address the issues raised.