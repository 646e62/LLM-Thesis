\chapter{Should Canada's \textit{nolo contendere} plea be formalized?}

\setcounter{footnote}{147}

\textit{Nolo contendere} pleas are predominantly the product of plea bargains. Federally, \textit{nolo contendere} pleas require prosecutors to consent before the judge may accept the plea,\footnote{Several states also follow this model. See note XXX above.} as does the \textit{nolo contendere} procedure used in Canada. Where defendants and prosecutors must agree for the court to accept a plea, that plea is very likely the result of a plea bargain, as prosecutors must have some incentive to allow a defendant to make an implied confession. Because victims, police, and prosecutors prefer a contrite defendant pleading guilty over a waffling defendant self-convicting, a prosecutorial agreement is only likely where some \textit{quid pro quo} is available. Furthermore, while conscience may motivate some defendants to plead guilty, it is much less likely to compell a defendant to enter the more ambivalent \textit{nolo contendere} plea.

Therefore, to determine whether Parliament should formally incorporate these pleas into Canadian criminal law, it is first necessary to examine the propriety of plea bargaining more generally. Because plea bargaining is a controversial practice, the answer to this question is not straightforward. If plea bargaining is an implicitly suspect enterprise that lawyers and defendants should avoid where possible, this fact should caution against expanding the use and availability of uncontested pleas. Plea bargaining's disadvantages may be so significant that further enabling the practice with \textit{nolo contendere} pleas is manifestly irresposible. 

However, if plea bargaining is an implicitly worthwhile and valuable process or even an ethically neutral one, then it may be worthwhile to consider formalizing \textit{nolo contendere} pleas. Parliament should implement them if they advance ethical goals and do not create moral quandaries. The first part of my analysis investigates plea bargaining generally in order to identify and answer these concerns. But even assuming that plea bargaining is either generally salvageable as a practice or commendable as an institution, reasonable concerns may remain about allowing \textit{nolo contendere} pleas to form part of the plea bargaining matrix. \textit{Nolo contendere} pleas are attended by their own ethical concerns, such that they warrant their own consideration. The second part of my analysis examines \textit{nolo contendere} pleas specifically so as to determine what ethical issues these pleas entail and how those issues may be addressed. The final portion of my analysis adopts insights from both of the preceding portions in order to determine what role, if any, formalizing \textit{nolo contendere} pleas may play in addressing the issues raised.