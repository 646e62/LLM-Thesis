\section{Summary}

Plea bargaining is an imperfect enterprise with the potential to be coopted by game theories, economically-based reasoning, and shrewd negotiations. These attitudes, in turn, undermine the law's substantive moral aims. But the fact that bad actors may coopt plea bargaining this way does not mean that plea bargaining is necessarily cynical. When conducted in good faith, plea negotiations give both parties to a criminal prosecution the chance to candidly assess the evidence, explore and develop potential win-win outcomes, avoid unpredictable contested hearings, and curtail wrongful pre-trial punishments.

Similarly, while \textit{nolo contendere} critics view the plea as a manifestation of plea bargaining's cynicism, those who hold these views ignore the plea's nature and effects. \textit{Nolo contendere} pleas are, first and foremost, a means for defendants to self-convict and voluntarily accept punishment. They admit an offence as proven without admitting that it is true. Where a defendant's morality conflicts with society's broadly-held principles or their beliefs about the evidence conflict with what the prosecutors will likely or inevitably prove at a trial, these pleas allow defendants to self-convict in good conscience. Defendants do not need to set unwanted trials, and prosecutors do not risk wrongful acquittals. In turn, the state can identify opportunities to morally educate dissident defendants and make good on those opportunities when sentencing them.

Both plea bargaining and \textit{nolo contendere} pleas have substantive value and promote fairness. When used with suitable safeguards, \textit{nolo contendere} pleas need not come at the expense of other substantive values. Given the unappealing prospects of insincere guilty pleas, unregulated \textit{nolo contendere} procedures, and wrongful acquittals, formalized \textit{nolo contendere} pleas are a reasonable and appropriate compromise. They should be formalized to the extent Parliament can implement these pleas without undermining criminal law's moral core. In the next and final chapter, I conclude my analysis by suggesting how to accomplish this task.