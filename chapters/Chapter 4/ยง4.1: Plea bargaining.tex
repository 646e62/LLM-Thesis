\section{Plea Bargaining}

Plea bargaining occurs when a prosecutor offers a defendant an additional incentive to self-convict, ostensibly to ensure efficient case resolution on a broader scale. Some defendants who self-convict do so for selfless or pro-social reasons, such as a desire to spare a complainant from having to testify at trial or because they are genuinely remorseful for their actions. Other guilty defendants may have no such inclinations but may be convinced self-convict in exchange for some consideration. Plea bargaining relies on the fact that there is potential for some \textit{quid pro quo} in every criminal prosecution.\footnote{See e.g. Ireland, \textit{supra} note 12 at 277.} Prosecutors have the power to lay new charges or withdraw existing ones, while defendants control whether to contest the charges and force a trial. Prosecutors and defendants each face specific pressures to resolve cases quickly and efficiently. Defendants detained in custody or released on conditions may be motivated to resolve their matters efficiently by a desire to end their pre-trial restrictions. At the same time, prosecutors may be motivated to resolve their matters efficiently by the constant threat of a successful delay motion and a judicial stay of proceedings.\footnote{See \textit{R v Jordan}, 2016 SCC 27, [2016] 1 SCR 631. The ``presumptive ceilings" that \textit{Jordan} imposed on all criminal cases place considerable societal pressure on the courts and prosecutors alike to have criminal matters dealt with as soon as possible, as cases outside these ceilings run the real risk of being summarily dismissed.}

Both the Supreme Courts in Canada and the United States have endorsed plea bargaining and highlighted its essential function in the justice system. In \textit{Santobello},\footnote{\textit{Santobello v New York}, 404 US 257 (1971).} the Supreme Court of the United States affirmed that the plea bargaining process is a crucial part of the criminal justice process whose agreements courts should uphold. In that case, the state charged Santobello with two gambling felonies. Santobello initially pleaded not guilty but later pleaded guilty to a lesser included offence with the prosecutor's promise that he would not recommend a sentence to the judge. There was a delay between the guilty plea and sentencing, and when Santobello appeared to resolve his charges, he and the state had different lawyers representing them. The new prosecutor was unaware of any prior deal and recommended the maximum sentence. Santobello protested, but the presiding judge sentenced him to the maximum nonetheless. On appeal, the majority of the United States Supreme Court ordered a new trial because the deal between the defendant and the prosecutor had not been honoured. The majority and the concurring opinions emphasized plea bargaining's importance and underscored its centrality in American criminal justice.

In Canada, 25 years later, the Supreme Court of Canada followed suit in \textit{Burlingham},\footnote{\textit{R v Burlingham}, [1995] 2 SCR 206, 124 DLR (4th) 7 [\textit{Burlingham}].} identifying plea bargaining as ``an integral element of the Canadian criminal process.''\footnote{See \textit{ibid} at para 23.} Although it had disparaged the practice just a few years earlier as amounting to justice ``purchased at the bargaining table,"\footnote{See \textit{R v Lyons}, [1987] 2 SCR 309 at para 103, 44 DLR (4th) 193.} the Supreme Court of Canada soon after recognized plea bargaining's critical role. In \textit{Burlingham}, the appellant was charged and convicted of two first-degree murders with similar \textit{modi operandi}. Before the trial, the police interviewed Burlingham for four days. After Burlingham had spoken with counsel, the police offered to reduce the charge to second-degree murder in exchange for information. Burlingham repeatedly told the police he would not take any such deal without first speaking with his lawyer but eventually capitulated after the police interrogated him for four days. Burlingham fulfilled his end of the bargain, pointed the police to the crime scene, and told them where he had stashed the murder weapon. Later that day, the police told Burlingham that the prosecutors had not authorized the deal. As a result, Burlingham was free to plead guilty to second-degree murder but would face trial for first-degree if he pleaded not guilty.

The trial judge excluded his confession, having found that the police had violated Burlingham's \textit{Charter} s 10(b) rights by offering him a deal without giving him the chance to speak with counsel. However, the judge allowed some evidence derived from the confession. This evidence included the gun the police found and an inculpatory statement that Burlingham made to his girlfriend the next day. Burlingham was ultimately convicted, and his appeals made their way to the Supreme Court, where the majority and dissenting opinions agreed that a new trial was warranted. In reaching this conclusion, the majority discussed the impact such practices had on plea bargaining. They identified plea bargaining as an integral element of the justice system but cautioned that the process required prosecutors and police to act uprightly to function correctly.\footnote{See \textit{Burlingham}, \textit{supra} note 179 at para 23.} Subsequent Supreme Court of Canada decisions have solidified plea bargaining's central role in the criminal justice system, binding judges to comply with these agreements unless doing so would be manifestly unjust.\footnote{Although \textit{Criminal Code}, \textit{supra} note 2, s 606(1.1) ensures that Canadian judges are not required to uphold plea agreements, the Supreme Court of Canada has made it clear that judges must defer to jointly recommended sentences. See \textit{R v Anthony-Cook}, \textit{supra} note 11 at paras 2, 29.}

Although plea bargaining is ubiquitous and judicially authorized across North America, it remains controversial. Plea bargaining proponents point out that the practice is efficient. They argue that these efficiencies are needed to maintain the day-to-day operation of the justice system.\footnote{See \textit{R v Butt}, 190 APR 227 — 62 Nfld \& PEIR 227 (NL SC (TD)) at para 48.} Plea bargaining's opponents, on the other hand, argue that these efficiencies come at too high a price, if in fact plea bargaining does anything to increase efficiency at all. They argue that plea bargaining creates an unfair environment for defendants that encourages hasty resolutions over true and just results.\footnote{See Ireland, \textit{supra} note 12 at 287; Brockman, \textit{supra} note 14 at 128. Generally, trials require more time and effort out of judges, lawyers, and court staff than negotiated self-convictions do. To the extent that these ``judicial resources" are finite, it reasonably follows that replacing more negotiated self-convictions with contested trials will consume more judicial resources. This necessary inference does not exclude the possibility that some plea bargains may also (or instead) be improperly motivated by other factors, such as trial aversion, pecuniary interest, or professional inertia.} I broadly categorize these objections as follows:

\begin{itemize}
    \item \textbf{Plea bargains are unfair to defendants.} Plea bargains penalize those who plead not guilty by creating a sentencing gap between otherwise equally situated defendants and coercing them through high-stakes offers that are too good to refuse. These deals improperly and unfairly induced defendants to self-convict as a result.
    \item \textbf{Plea bargains result in wrongful convictions and inaccurate pleas.} Even if plea bargains do not improperly induce defendants, they increase the risk that factually innocent defendants will self-convict instead of setting their matters for trial. These self-convictions are inaccurate and unjust.
    \item \textbf{Plea bargains undermine the law's moral core.} Plea bargaining transforms the criminal justice process into an economic system that encourages deal-making and the ``gamification" of punishment while discouraging confessions, remorse, and a sense of responsibility. Encouraging plea bargaining comportments undermines the substantive moral values undergirding the criminal justice system.
\end{itemize}
Among these different ethical positions, several approaches raise interesting new questions and warrant closer examination and engagement. In each subsection, I explore three approaches to help determine whether plea bargaining is a good that criminal justice should pursue, an evil it should avoid, or another option. 

First I consider the \textit{fairness problem} with help from Michael Young's article ``In Defense of Plea-Bargaining's Possible Morality."\footnote{See Michael Young III, ``In Defense of Plea-Bargaining's Possible Morality" (2013) 40:1 Ohio NU L Rev 251.} Young's underlying premise is that plea bargaining produces normative goods, making it possible for plea bargaining to be a moral enterprise. I agree with this position but argue that Young pursues it too conservatively. 

Next, I consider the \textit{truth problem} alongside Joshua Bowers' ``Punishing the Innocent."\footnote{See Josh Bowers, ``Punishing the Innocent" (2008) 156:5 U Pa L Rev 1117.} Bowers forcefully advocates \textit{for} wrongful convictions, arguing that they allow an avenue for wrongfully \textit{punished} defendants to short-circuit their unjust penalties. I adopt many of Bowers' insights concerning the truth problem and attempt to synthesize his ``legal fiction" proposal into the truth-, belief-, and proof-function plea model. 

Finally, I address the \textit{substance problem} by considering Stephanos Bibas' ``Harmonizing substantive-criminal-law values and criminal procedure: The case of Alford and nolo contendere pleas."\footnote{See Bibas, ``Harmonizing Substantive Values," \textit{supra} note 21.} While Bibas and I agree that substantive criminal legal principles should be a critical part of our discussions about plea bargaining, I otherwise disagree with his assessment of how plea bargaining impacts those values. Specifically, while Bibas views plea bargaining and non-culpatory uncontested pleas as ethical compromises made in the name of criminal procedure, I argue that both are ethically viable and fully capable of delivering normative goods to all justice system participants.

\subsection{Fairness}

Plea bargaining opponents worry that the process improperly induces or threatens defendants into self-convicting. Although it is true that the criminal justice system is oppressive and that those caught up in it are uniquely vulnerable to inducements and threats, not all inducements and threats override the will. Unlike threats and inducements in police interviews, where the suspect is most vulnerable and unilaterally disadvantaged, inducements and threats in plea bargaining occur in an environment where both parties have something to gain, something to lose, and ample time to consider their options. Most defendants will have an opportunity to consult with a lawyer about a prosecutor's offer, and many will be represented by one to assist them regardless of whether they accept or reject that offer. Nonetheless, some critics argue that plea bargaining leads to unfair results. Following Michael Young and his article ``In Defense of Plea-Bargaining's Possible Morality," I examine one specific and one general fairness criticism that Young highlights:

\begin{itemize}
    \item \textbf{The trial penalty.} Plea bargaining allows defendants to be sentenced differently depending on whether they set their matters for a trial. Defendants who set trials are typically sentenced more harshly than those who plea bargain. A trial penalty results where these defendants are otherwise identically situated.
    \item \textbf{The coercion worry.} Plea bargaining is coercive by nature. It creates situations for defendants where their only rational choice is to self-convict, and in some situations, amplifies this problem through high-stakes plea deals. Alternatively, coercion may be understood in the context of the relationship between the parties and the fairness the dominant party owes to the other.
\end{itemize}

\subsubsection{The Trial Penalty}

Where courts convict two otherwise equally situated defendants of the same crime, but one of them accepts an early plea deal instead of a trial, the one who accepted the plea deal will likely receive a more lenient sentence than the one who did not.\footnote{To the extent that no two offences or offenders are ever exactly alike, ``similar offences" and ``similarly situated offenders" will always be distinguishable. The incredible array of subtle distinctions between offences and offenders can either be overlooked or microscopically examined, depending on the offence, the offender's history, the court's jurisdiction, and other comparable factors.} The apparently uneven and intuitively unfair treatment these two defendants receive constitutes the trial penalty criticism.\footnote{See Young, \textit{supra} note 186 at 269.} Young considers three responses to this critique:

\begin{itemize}
    \item \textbf{Recasting the \textit{trial penalty} as a \textit{plea bargain benefit}.} The disparity between defendants who plea bargain and defendants who do not is not a ``penalty." Rather, it is a benefit afforded to those who plea bargain.
    \item \textbf{Reducing penalties in exchange for prosecutorial resources, the common good, or both.} Defendants who plead guilty free up resources that the justice system would have otherwise expended on their trials. Doing so is a moral good, and rewarding them with reduced punishments is appropriate.
    \item \textbf{Re-examining the assumptions underlying equal treatment and the defendant's role in the discrepancy.} When looking for coercion, the most crucial consideration is whether equally situated defendants had equal opportunities to plea bargain. Defendants who voluntarily opt not to plea bargain are accountable for the impact this may have on their future bargaining positions.
\end{itemize}
Faced with the trial penalty criticism, plea bargaining proponents who hold to the first view may counter that defendants are not \textit{penalized} for \textit{setting a trial}, but rather \textit{rewarded} for \textit{resolving their charges} without insisting on a trial. Under this view, the trial penalty criticism fundamentally misunderstands plea bargaining. But as Young points out, the trial penalty criticism is aimed at the \textit{inequality} that plea bargaining creates. It is not merely a sophistic definitional problem, such that recasting the inequality in more favourable terms is enough to address the true cause for concern. Regardless of whether the disparity is called a \textit{trial penalty} or a \textit{plea bargaining incentive}, the fundamental concern is that plea bargaining treats equally situated defendants unequally.

Young next considers the possibility that ``equally situated defendants" may not be equally situated at all. Where the state charges two such defendants with the same offence and only one opts to plea bargain, the defendant who resolves their charges saves ``public resources on an expensive prosecution"\footnote{See \textit{ibid} at 270.} while the other does not. To the extent that saving the state the expense of a trial contributes to the public good, and to the extent that contributing to the public good is morally praiseworthy, defendants who plea bargain are more morally praiseworthy \textit{in this respect} than those who insist on a trial.\footnote{This principle only holds where the defendant is factually guilty. Innocent defendants who plead not guilty are not more morally blameworthy than any defendant who self-convicts.} The courts should therefore treat them as such.

Young rejects this argument on the grounds that prosecutorial resources and punishment are incommensurate commodities. Prosecutorial resources are, in effect, monetary resources,\footnote{Prosecuting a case requires police to investigate crimes, lawyers to review the evidence and handle court proceedings, judges to adjudicate trials and sentencing, and in many cases, publicly funded defence counsel. To the extent that public spending supplies these goods, they are public monetary resources.} and Young argues that allowing prosecutors and defendants to exchange one for another invites unjust results:

\begin{quote}
    \begin{singlespace}
    
    [T]here seems to be something fundamentally objectionable about strongly tying the idea of deserved punishment to the idea of contributing to the common good, whether by saving public resources on a prosecution or in any other way. Simply stated, the one should not have anything to do with the other. It would be perverse, for example, to think that a rich criminal who donated money to the prosecutor's office — thus advancing the common good by providing resources for prosecutions — would even presumptively deserve a sentencing reduction for that reason alone.\footnote{See Young, \textit{supra} note 186 at 271.}
    \end{singlespace}
\end{quote}
Absent a more direct correlation between the common good and moral praiseworthiness, Young argues that merely saving the public some expense is not enough to make a person morally praiseworthy. This example, however, is a hasty generalization and a straw man. While some financial arrangements between defendants and the prosecutor's office would be objectively inappropriate, separating criminals from their money remains a time-honoured and occasionally effective punishment. Furthermore, by dismissing this argument as he does, Young fails to consider other, more compelling reasons why a defendant who foregoes their right to a trial may be more morally praiseworthy than one who does not.

Judges who fine defendants signal that money and punishment are commensurate resources. Other aspects of Canadian criminal law evince this equivalency. For example, Parliament has recognized that there can and ought to be a monetary value placed on time spent in custody\footnote{See \textit{Criminal Code}, \textit{supra} note 2, s 734(5).} and a custodial value placed on misappropriated funds.\footnote{See \textit{ibid}, s 462.37(4).} Additionally, statutory victim fine surcharges,\footnote{See \textit{ibid}, s 737(1).} restitution orders,\footnote{See \textit{ibid}, s 738.} forfeiture orders,\footnote{See \textit{ibid}, s 734(1).} and court-ordered donations are all ways the court correlates moral blameworthiness with the very same resource used by the Attorney General to bring cases to trial. There is a strong correlation between financial resources and punishment, such that a defendant who has paid a fine or restitution is no longer equally situated with one who has not.\footnote{See e.g. \textit{R v Saucier}, 2019 ONSC 3611 at para 44.}

But sparing the state the \textit{financial expense} of a trial is not a defendant's only bargaining chip or even the most valuable consideration they can offer. Defendants who opt to self-convict rather than take their matters to trial \textit{relieve the state of its burden to prove the offence}. As a result, witnesses who would have had to have testified are no longer required to do so. Where those witnesses are vulnerable individuals, the defendant spares them the potential trauma of having to revisit the offence. Where those witnesses are unreliable, the defendant spares the state the difficulty of eliciting evidence from hostile or unpredictable parties. Defendants spare all witnesses the pressure of being subpoenaed, coming to court under threat of arrest, and testifying. And perhaps most importantly, \textit{self-convicting defendants guarantee convictions}. Trials are inherently volatile processes, and convictions after contested hearings are never guaranteed, even in cases where the evidence appears overwhelmingly strong. A self-convicting defendant spares the state the risk of a potentially expensive \textit{and unsuccessful} prosecution.

Young criticizes a naive view of inequality that he describes as \textit{distributional}. The distributional view proposes that courts should not sentence two identically situated defendants differently solely because one opts for trial while the other self-convicts. In its place, Young proposes substituting a \textit{relational} view. Where distributional equality judges equal treatment based on how evenly courts distribute punishment across similarly situated defendants, relational equality asks whether ``someone or some group [is being] subordinated or dominated, or in some similar way treated without proper respect or concern."\footnote{See Young, \textit{supra} note 186 at 272.} Here, equal treatment lies in the relationships between parties when compared. Courts treat defendants equally when they have \textit{equal opportunity} to plea bargain. By contrast, where one defendant is given an opportunity to plea bargain, but another similarly-situated defendant is not, courts treat them unequally. But where one defendant opts to accept a deal for a lighter penalty and another does not, \textit{their refusal to do so is the cause of their distributive inequality}. Assuming that similarly-situated defendants have equal access to equivalent plea bargain offers, those who turn them down can and should be held responsible for the consequences of that choice.

Young correctly argues that the trial penalty and its primary underlying concern of equal treatment under the law are fatally flawed. But \textit{contra} Young, defendants who opt to self-convict rather than contest their matters make a morally praiseworthy decision that should be recognized. Finally, where distributive inequalities exist between otherwise similarly-situated defendants who set their charges down for trial and defendants who plea bargain, these differences will only meaningfully impact the question of equal treatment where defendants have unequal access to plea bargaining. Where defendants have equal opportunity to self-convict on equitable terms, they receive equal treatment.

\subsubsection{The Coercion Worry}

Even if the plea bargaining critic is convinced that the trial penalty is a non-issue, they may still argue that plea bargaining is otherwise \textit{coercive}. As an institution, plea bargaining incentivizes criminal defendants to give up their right to trial. When prosecutors offer defendants plea deals that are too good to refuse, it is reasonable to ask whether the defendant can make a genuinely free choice and turn the deal down. Plea bargaining opponents argue that these deals are coercive and inappropriate. Young disagrees and responds that these understandings of coercion are too simplistic. He instead identifies three distinct coercion concepts that commonly emerge in plea bargaining literature, the latter of which he ultimately endorses:

\begin{itemize}
    \item \textbf{Restricting a defendant's rational choice.} Coercion depends on whether a person can make meaningful choices in a given situation. Where plea bargain offers are such that defendants have no rational choice but to accept them, they become coercive.
    \item \textbf{Requiring a defendant to choose with high stakes.} Coercion also depends on the type of choice in question. When plea resolution offers require defendants to make a high-stakes choice involving their liberty, they become coercive.
    \item \textbf{Overriding a defendant's will by wrongfully influencing them.} Prosecutors and defendants each have special roles to play in criminal prosecutions. Because of these roles, some minor inducements may be coercive, while some major inducements may not. Whether they are coercive depends on whether the prosecutor improperly interfered with the defendant's rights or position through a plea bargaining inducement.
\end{itemize}
The first position argues that prosecutors, defence lawyers, courts, and the justice system generally coerce defendants into accepting plea bargains when the \textit{only rational choice} is to accept a plea-bargained offer. For example, an armed robber who demands a wallet at gunpoint coerces their victim into parting with their belongings by denying them any other rational choice.\footnote{See \textit{ibid} at 264.} Similarly, the argument goes that defendants may be coerced into deciding to self-convict when that is the only rational choice. The ``bare restriction on rational, utility-maximizing choice"\footnote{See \textit{ibid} at 262.} is itself coercive. 

But correlating coercion with a lack of rational alternatives to choose from fails to account for situations where a person has no rational choice but to accept a clear advantage or a windfall. A person in financial trouble who gets a lucrative job offer from a former business associate may have no rational choice but to accept the job, even if it meant having to move to a different city or work longer hours. However, it would be unreasonable to claim that their old associate coerced them into deciding to do so. \textit{Restricting rational choices} cannot be the sole measure of coercion, even where one may only make a single rational choice. Instead, the \textit{type of choice} that a person has to make must also be considered.

The next position examines the nature of the choice by shifting focus to \textit{the stakes involved} in making the decision. Whereas the job hunter merely faces a single and specific lost opportunity if they do not accept the new position, a criminal defendant may face an infinite array of life-altering consequences if they fail to make the only available rational decision. The robust version of this criticism argues that high-stakes decisions \textit{necessarily} impose coercive pressures on defendants. In contrast, the weak version of this criticism argues that the pressure from high-stakes decisions, \textit{combined with a lack of any other rational choice}, is coercive. 

Both versions miss the mark. Neither the stakes involved nor the lack of any other rational choice determines the coercion issue. For example, a parent who sees their young child wandering near a busy road will have no other rational choice but to step in and try to stop them. However, despite the high stakes and lack of reasonable alternatives, calling that decision coerced would be inaccurate and unreasonable. Similarly, a defendant caught red-handed committing a moderately serious offence may be offered a highly reduced sentence in exchange for information they can safely and secretly provide about an unrelated but much more serious offence. Despite the high stakes and lack of rational alternatives, this is also not appropriately understood as coercion.

Where a prosecutor offers a meagre sentence on a guilty plea, as opposed to a comparatively very high sentence upon conviction after trial, they may be said to be engaging in hard dealing.\footnote{See \textit{ibid} at 267 — 269.} To the extent that this tactic is likely to result in the defendant accepting the very low sentence offer, critics may suggest that the prosecutor coerces the defendant into making that choice. But not all such offers are coercive. In cases where both rehabilitation and deterrence are equally important sentencing factors, a wide range of sentences may be reasonable. In such cases, there may be nothing implicitly coercive about a large sentencing gap. In other cases, an excessively lenient prosecutor may create a sentencing gap sufficient to qualify as \textit{hard dealing}, but where the prosecutor offers a more favourable choice than the defendant might generally be entitled to, such a deal is more reasonably described as \textit{generous}, not \textit{coercive}. Whether a prosecutorial position is coercive or not will always depend upon myriad factors, including the strength of the state's case, the index offence's severity, where the offer lands in the juridsiction's sentecing range for the index offence, whether the defendant is adequately represented by counsel or sufficiently capable of representing themselves, and the conditions attached to the offer, if any. A wide sentencing gap is simply one of these factors to consider when evaluating an offer for coercion.

The last position Young considers and ultimately endorses evaluates coercion on whether an agent interfered with rights or freedoms to which a person was entitled. Under this view, coercion requires more than mere circumstances. It also requires an agent in a position of power to act improperly. Young's preferred view of coercion requires understanding the relational dynamic between the parties involved. However, he does not elaborate further on how a prosecutor might use their position to induce criminal defendants coercively. His position is nonetheless salvageable and defensible. It accounts for deficiencies in the other two views and lays a constructive foundation for properly framing coercion in plea bargaining.

In the above examples, the parent running into traffic to save their child does not exemplify coercion. Assuming nobody is intentionally putting the child in danger, and assuming the parent's motivation for running into traffic is not motivated by a diligent desire to comply with \textit{Criminal Code} s 215, no agent is forcing their decision. The issue is more nuanced than this in the example of the criminal defendant receiving a bargain-basement plea deal. Assuming the prosecutor acts with appropriate motives, does not rely on excessively tight deadlines, and does not attach abusive caveats or conditions to their offers, the defendant is not coerced by an inordinately generous deal. However, the fact remains that prosecutors have a great deal of power compared to criminal defendants. This power imbalance is susceptible to coercion if left unchecked and unconsidered, as the \textit{Burlingham} decision exemplified.\footnote{See \textit{Burlingham}, \textit{supra} note 179.}

In Canada, prosecutors are responsible for authorizing the criminal charges that private citizens, the police and other peace officers submit and have the power to withdraw or stay proceedings.\footnote{Although rare, private citizens may initiate private prosecutions where the police have not done so, and for whatever reason either cannot or will not proceed with charges: see \textit{Criminal Code}, \textit{supra} note 2, s 507.1. Where private citizens can establish sufficient cause for a prosecution, the matter is referred to either the provincial or federal public prosecutor with jurisdiction over the offence. Once referred, the prosecutor may continue or discontinue the prosecution. See e.g. Ontario Court of Justice, ``Guide For Applying For a Private Prosecution," online (pdf): \textless ontariocourts.ca/ocj/files/guides/guide-private-prosecution-EN.pdf\textgreater; Deskbook, \textit{supra} note 117, ``5.9 Private Prosecutions" (1 March 2014), online: \textless ppsc-sppc.gc.ca/eng/pub/fpsd-sfpg/fps-sfp/tpd/p5/ch09.html\textgreater.} Where there is a reasonable prospect of securing a conviction, and it is in the public's interest, prosecutors may pursue charges. Likewise, prosecutions should discontinue if one or both of those conditions do not obtain. When criminal defendants are charged with an offence, the police, prosecutors, or both may decide to apply to detain them pending the outcome of their cases. Where courts agree, they may remand defendants into custody pending the outcome of their case. In other cases, the police, prosecutors, or courts may release defendants on conditions pending the outcome of their case. Although these conditions must be the least restrictive form of release possible,\footnote{See \textit{R v Zora}, 2020 SCC 14 at paras 6 — 7.} they may nonetheless be very stringent for defendants with lengthy criminal records or charged with serious offences.

Because defendants decide whether to set their criminal charges down for trial, there is \textit{some} ostensible balance to the power between prosecutors and defendants. Notwithstanding that fact, prosecutors are generally at a clear advantage. Unlike defendants, they are not personally liable and face no comparable jeopardy if the case does not go their way. They have no pre-trial conditions or detention orders to comply with and are under comparatively less pressure than defendants to resolve their cases quickly. Within this context, it is easy for high-stakes decisions with no rational alternatives to become coercive. This power dynamic should be explicit when evaluating individual interactions between prosecutors and defendants for coercion concerns. Where prosecutors and police do, in fact, ``act honourably and forthrightly"\footnote{See \textit{Burlingham}, \textit{supra} note 179 at para 25.} in these dealings, plea bargaining's coercive potential is curtailed. Where prosecutors and police do not, the entire system suffers, with or without plea bargaining.

\subsection{Truth}

Courts should not convict factually innocent people of crimes they did not commit. Wrongful convictions are unfair and rightfully undermine the public's confidence in the justice system. They leave the factually guilty unpunished and undeterred, and where the crimes involve victims, those victims do not receive justice. Where plea bargaining is allowed, some factually innocent defendants will be inclined to self-convict in exchange for a favourable resolution to their case, wrongfully convicting themselves while leaving the factually guilty unpunished. Such results are not truthful because they inaccurately assess responsibility. 

Some plea bargaining opponents believe that plea bargains meaningfully or even dramatically increase the rate of wrongful convictions. \textit{Assuming this is true for the argument's sake}, plea bargaining proponents must demonstrate that plea bargaining somehow offsets this adverse effect. In this section, I undertake this task by looking at how wrongful convictions may be quantified and how their inevitability may be qualified. I next examine trials as a truth-finding process and note their limitations. Finally, I consider Josh Bowers' argument in ``Punishing the Innocent" that endorses wrongful convictions as ``legal fictions" and advocate for my position:

\begin{itemize}
    \item \textbf{Quantifying wrongful convictions and qualifying their inevitability.} Wrongful convictions are inevitable. The true issue is not whether society can tolerate a dispositive forensic \textit{procedure that causes wrongful convictions} but whether society can tolerate \textit{the rate at which it does so}.
    \item \textbf{The apocryphal truth-finding function of trials.} Trials involve many compromises between prosecutors, defendants, and the decision-makers hearing their cases. Many of these compromises limit what may or may not be adduced as evidence, constraining the trial's utility as a truth-finding process. To properly judge plea bargaining's efficacy and accuracy, the relative merits and demerits of contested trials must be considered.
    \item \textbf{The moral case for wrongful convictions.} The process costs for many innocent defendants who set trials are often much higher than for similarly situated innocent defendants who self-convict. Wrongful convictions can help end wrongful pre-trial punishments and should be encouraged.
    
\end{itemize}

\subsubsection{Quantifying Wrongful Convictions and Qualifying Their Inevitability}

It is reasonable to assume that a system that allows plea bargaining will permit more wrongful convictions than an identical system would without it.\footnote{See e.g. Ireland, \textit{supra} note 12 at 293 — 296.} Wrongful convictions do not, of course, result exclusively from negotiated plea agreements. Many documented wrongful convictions occurred following a trial, notwithstanding the commonly-held notion that trials reliably arrive at truthful conclusions.\footnote{See Bruce MacFarlane, ``Convicting the Innocent: A Triple Failure of the Justice System," 2006 31 Man LJ 403. at 408 — 431, 444 - 445. MacFarlane lists eight ``principal causes" of wrongful convictions, as well as five additional ``less prevalent" causes. None of the causes he lists include plea bargaining as an institution.} But to the extent that negotiated plea agreements are responsible for \textit{at least some} wrongful convictions, this reasonable assumption necessarily follows. Therefore, it is imperative to consider whether plea bargaining is defensible. Judging by the commentary surrounding wrongful convictions, just asking this question appears to miss a greater ethical point. Wrongful convictions have been called a ``scourge,"\footnote{See H. Mitchell Caldwell, ``Coercive Plea Bargaining: The Unrecognized Scourge of the Justice System" (2011) 61:1 Cath U L Rev 63.} a ``disaster," \footnote{See Stephen J. Schulhofer, ``Plea Bargaining as Disaster" (1992) 101:8 Yale LJ 1979.} ``a blight to the criminal justice system,"\footnote{See Anoushka Dey, ``Wrongful Conviction: A Blight to the Criminal Justice System" (2021) 24 Supremo Amicus [314].} and ``one of the worst nightmares imaginable."\footnote{See C Ronald Huff, ``Wrongful Conviction: Causes and Public Policy Issues" (2003) 18:1 Crim Just 15. at 15.} Even strong plea bargaining proponents recognize that wrongful convictions are a ``failure" of our justice system.\footnote{See Bowers, \textit{supra} note 187 at 1119.} If true, it would seem to follow that plea bargaining should be curtailed to the extent that it encourages wrongful convictions.\footnote{See Bibas, ``Harmonizing Substantive Values," \textit{supra} note 21 at 1381.} But although wrongful convictions are a problem to avoid where possible, I argue that plea bargaining critics who connect the practice to wrongful convictions misapprehend the link between them.

The mere fact that plea bargaining causes \textit{some additional} wrongful convictions is not enough to render it indefensible, regardless of how seriously one views wrongful convictions. For example, a plea bargaining system that only resulted in one more wrongful conviction in every 1,000,000,000 criminal cases than an identical one that did not allow plea bargaining would likely be \textit{tolerable} for most. Conversely, another system that resulted in one more wrongful conviction in every five cases with plea bargaining than without would likely be \textit{intolerable} for most. The mere fact that using a procedure results in \textit{some number} of additional wrongful convictions is not enough to render that procedure indefensible. To make this claim, both the \textit{actual} increase in wrongful convictions and the \textit{proportion} of wrongful convictions that can reasonably be said to result from plea bargaining must be accounted for.\footnote{See e.g. MacFarlane, \textit{supra} note 210 at 476.}

Additionally, the \textit{type of offence} underlying a wrongful conviction bears upon whether that injustice is tolerable in the grander scheme. A defendant who is wrongfully convicted of first-degree murder and sentenced to life in prison is tragic, while a defendant who is wrongfully convicted of a curfew breach and sentenced to a three-month conditional discharge is largely just inconvenienced. Both scenarios are unjust, but while critics may fairly categorize the former scenario as a nightmare, categorizing the other as such is a much more problematic pitch to deliver. There are qualitative differences between different types of wrongful convictions.

Wrongful convictions are an inevitable part of the Canadian justice system, a reality that underscores the fact that they are not inherently intolerable.\footnote{See esp \textit{ibid} at 405, 433. See also Lynne Weathered, ``Does Australia Need a Specific Institution to Correct Wrongful Convictions?" (2007) 40:2 Aust \& NZ J Criminology 179 at 195; Robert J. Norris et al., ``Thirty Years of Innocence: Wrongful Convictions and Exonerations in the United States, 1989-2018" (2020) 1:1 Wrongful Conv L Rev 2 at 3; Marvin Zalman \& Matthew Larson, ``Elephants in the Station House: Serial Crimes, Wrongful Convictions, and Expanding Wrongful Conviction Analysis to Include Police Investigation" (2015) 79:3 Alb L Rev 941 at 1031; Arye Rattner, ``Convicted but Innocent: Wrongful Conviction and the Criminal Justice System" (1988) 12:3 Law \& Hum Behav 283 at 291. But see Dianne L. Martin, ``Lessons about Justice from the Laboratory of Wrongful Convictions: Tunnel Vision, the Construction of Guilt and Informer Evidence" (2002) 70:4 UMKC L Rev 847 at 848.} Parliament premised the Canadian justice system on the foundation of proof beyond a reasonable doubt. Judges must remind juries at every criminal trial that proof beyond a reasonable doubt is not the same as proof to a moral certainty.\footnote{See esp the byzantine formula from \textit{Lifchus}, \textit{supra} note 90 at para 36. Despite deftly dodging any functional definition of what ``beyond a reasonable doubt" actually means, the \textit{Lifchus} liturgy remains a staple of Canadian jury instructions on the subject.} Trials are time-limited showcases where the judge or jury hears each party's best interpretation of the evidence admitted. The criminal conviction appeal system operates on the understanding that judges and juries sometimes make mistakes. Similarly, the summary conviction appeal courts and the Supreme Court of Canada both operate on the understanding that appellate courts do the same. 

Requiring judges and juries to be \textit{sure} that a defendant was guilty may drastically reduce the number of wrongful convictions and the number of all convictions alongside it. Blackstone's frequently-cited ratio reminds us that it is better for ten guilty people to walk free than for the system to punish one innocent person wrongly.\footnote{See e.g. Marvin Zalman, ``The Anti-Blackstonians" (2018) 48:4 Seton Hall L Rev 1319 at 1321; Fritz Allhoff, ``Wrongful Convictions, Wrongful Acquittals, and Blackstone's Ratio" (2018) 43 Austl J Leg Phil 39 at 44; Jan W. de Keijser, Evianne G. M. de Lange \& Johan A. van Wilsem, ``Wrongful Convictions and the Blackstone Ratio: An Empirical Analysis of Public Attitudes" (2014) 16:1 Punishment \& Soc'y 32 at 34.} But it tells us nothing about whether it is better for a thousand, a hundred, or even eleven people to walk free than for our system to punish one innocent person wrongly. Even this presumption-of-innocence shibboleth must be qualified, as the criminal justice system must tolerate \textit{some} wrongful convictions to function at all. 

Where a forensic procedure causes wrongful convictions, it should be evaluated on how many it causes and at what rate, not merely on the fact that it does so. Young argues a similar point when he suggests comparing wrongful convictions resulting from plea bargained deals to wrongful convictions resulting from jury trials to determine whether there is a meaningful difference between the two. Although Young neither supplies the data nor does the math, he argues that plea bargaining can only be meaningfully said to result in \textit{too many wrongful convictions} if it results in \textit{more wrongful convictions} than a contested trial in front of a jury.\footnote{Because jury trials are often cast as the gold standard for substantive and procedural rights, Young uses this as a benchmark for assessing plea bargaining's efficacy. See Young, \textit{supra} note 186 at 258.} I do not definitively answer how many wrongful convictions plea bargains cause, nor do I undertake to answer that question through empirical data. Instead, I argue that the trial process is implicitly suspect as a truth-revealing mechanism and that plea bargaining's critics too often overlook the truth-revealing processes that plea negotiations so often entail. 

\subsubsection{The Apocryphal Truth-finding Function of Trials}

While both plea bargaining opponents and proponents generally view criminal jury trials as the gold standard for delivering due process and ensuring that the truth wins out, trials are, in fact, a series of intricate compromises between prosecutors, defendants, and decision-makers.\footnote{See e.g. MacFarlane, \textit{supra} note 210 at 435.} In trials, truth is only one of many critical competing interests that must be balanced against many other considerations. Trials are an adequate failsafe for when good faith discussions and negotiations between prosecutors and defendants are unsuccessful or otherwise impossible. However, they should not be idealized as quests for the truth, as ``the judge at a criminal trial is not attempting to resolve the broad factual question of what happened."\footnote{See \textit{R v Quintin}, 2015 SKQB 16 at para 41, citing \textit{R v Mah}, 2002 NSCA 99, re \textit{W(D)}, \textit{supra} note 89.}

The idealized understanding of trials sees them as an opportunity for each side of a dispute to present its best evidence and interpretation of the evidence. Adequately prepared counsel cross-examine witnesses on their testimony, ably identifying the strengths and weaknesses of their evidence. A neutral and dispassionate trier hears the facts, listens to both sides argue, and decides between them. By contrast, \textit{nolo contendere} pleas appear untrustworthy because they appear to obfuscate or misrepresent the truth to push hastily negotiated agreements through the courts. Negotiations between prosecutors and defence lawyers are privileged, such that there are no judges or juries to provide neutral adjudication\footnote{This is not necessarily true in all cases. In many jurisdictions, contested matters may go through case management with a judge to discuss pre-trial issues, canvass resolution, and offer advice on how to best proceed.} and the public may never know what considerations went into a negotiated plea deal.\footnote{See Debra Parkes, ``Plea Deals Shrouded in Mystery," Winnipeg Free Press (22 November 2013) online: \textless \url{winnipegfreepress.com/opinion/analysis/plea-dealsshrouded-in-mystery-232964851.html}\textgreater.}

Notwithstanding its image problem, plea negotiations between prosecutors, defendants, and lawyers are vital to the criminal justice system's truth-seeking function. Both parties study the same allegations with substantially similar information, find an interpretation of those allegations congruent with their position, and form arguments from those findings to help negotiate admissions, concessions, and other agreements. Both parties can examine their cases in detail, speak with witnesses, forecast likely results should the matter go to trial, and try to reach a mutually acceptable compromise. In some cases, defendants will self-convict, while the prosecutors may discontinue charges in others. Where successful, plea negotiations reach a resolution that both parties are satisfied with, or at the very least are both prepared to tolerate. Even where plea negotiations are unsuccessful, lawyers who actively review the evidence and negotiate in good faith may clarify and streamline trial issues in advance amongst themselves rather than in front of a judge or jury. Plea bargaining is an integral part of the dispute resolution process.

Unlike resolution discussions, trials require the fact finder to consider a less complete and accurate version of the evidence than the prosecution and defence have during negotiations. At trial, fairness, efficiency, relevance, a panoply of legal privileges and the prejudicial effect that factual evidence might have all offset the truth, as do the trial strategies, probability considerations, and the game theories that go into deciding what evidence to adduce and how to adduce it. Judges may exclude unfairly gathered evidence, juries will not hear testimony from uncalled or missing witnesses, hearsay rules will stop witnesses from repeating what they heard, and opinion evidence rules will prevent most witnesses from telling the court what they think, \textit{all regardless of whether the underlying evidence is true or reliable}.\footnote{See e.g. \textit{R v Taylor}, 2014 SCC 50, [2014] 2 SCR 495 at para 38. Although the DNA evidence seized in that case was wholly reliable, the Supreme Court of Canada excluded it, having found that allowing it in would bring the administration of justice into disrepute.} The evidence ultimately admitted may be subject to special instructions directing judges and juries on how to interpret it correctly.\footnote{See James Fitzjames Stephen, \textit{The Indian Evidence Act (of 1872): With an Introduction on the Principles of Judicial Evidence} (Indianapolis: Alpha Editions, 2019) at 33.} Where cases are heard by a jury, the decision maker does not give any reasons for their decision, and may legally reach otherwise illegal results through jury nullification.\footnote{See \textit{R v Latimer}, 2001 SCC 1 at paras 57 — 71, [2001] 1 SCR 3.}

Judges and juries possess no special analytical powers to decide cases, and have no way to guarantee that the witnesses who attend court will be reliable, credible, or both. As a result, lawyers must manage their witnesses, curate their questions, and account for human frailties and weaknesses to effectively present their case theory, not to ensure a balanced presentation of the truth. For this reason, I argue that the primary function of a trial is not to ensure that \textit{the truth is heard} but rather to ensure that the judge or jury can make a \textit{fair decision} in a case. Where the parties cannot otherwise resolve, trials are a reasonably effective means of adjudicating disputes, ensuring all sides have their best explanation of the evidence heard and deferring to an ostensibly neutral third party to decide between competing narratives. However, trials are not primarily a search for the truth. Instead, they are an adequate truth-finding substitution when other methods, like bargaining, have failed. Negotiations between counsel, both of whom have legal training, are duty-bound to advocate for their positions, have access to the most information, and have the opportunity to spend extensive time and resources investigating the nuances thereof, are a much greater truth-seeking device than a criminal trial.

\subsubsection{The Moral Case For Wrongful Convictions}

In his article ``Punishing the Innocent," Josh Bowers acknowledges that wrongful convictions are a ``failure" of the justice system, and that self-convictions cause at least some.\footnote{See Bowers, \textit{supra} note 187 at 1119.} But rather than blame voluntary wrongful convictions on the uncontested pleas these defendants enter, Bowers traces the source of this failure further back to the wrongful arrest, charge, pre-trial process costs and prosecution of the allegation. Doing so allows him to argue that voluntary wrongful convictions prompted by a favourable plea deal may be a \textit{categorical good} for correcting the harms caused by wrongful incarceration and other oppressive pre-trial penalties and that these self-convictions should be made more readily available to such defendants. Bowers acknowledges that innocent defendants get caught up in the criminal justice system but argues that most people misunderstand what sort of defendants find themselves wrongfully charged with a crime, what kinds of crimes they are wrongfully charged with, and the degree to which due process will resolve their wrongful punishments.\footnote{See \textit{ibid} at 1121.} By addressing these misperceptions, Bowers makes an ethical case for actively encouraging a particular class of innocent defendants to plead guilty.

High-profile wrongful conviction cases are a part of the public's imagination, so it becomes easy to imagine that our system could wrongfully convict anyone. But Bowers argues that most innocent criminal defendants are recidivists charged with minor offences and that systemic biases, pressures, and tendencies place them at a higher risk of being wrongfully arrested.\footnote{See \textit{ibid} at 1125f. These inferences follow if recidivists are \textit{wrongfully} arrested no less frequently than first-time defendants and are \textit{generally} arrested at a much higher rate than first-time offenders. See also MacFarlane, \textit{supra} note 210 at 413.} Once arrested, recidivist defendants are more likely than first-time defendants to be charged than released without charges, less likely to have even weak charges dismissed and more likely to be denied bail and remanded into custody. Should they wish to testify at trial, they will open themselves up to their prior criminal convictions, thus potentially eroding their credibility in front of the judge or jury. 

Innocent defendants are more likely to be recidivists than first-time ``offenders" because defendants, in general, are more likely to be recidivists than first-timers. In the same way, an innocent defendant is more likely to be charged with a minor offence because defendants generally are more likely to be charged with minor offences. Other pressures, such as the increased frequency of minor incidents over major ones and the comparative lack of police time and investigative resources spent on such complaints, make it more likely that an innocent defendant will be wrongfully accused of a minor crime than a major one. Because these offences are more likely to be minor than major, prosecutorial plea deals are similarly more likely to be favourable than punitive.

Finally, Bowers draws attention to the overwhelmingly negative impact the pre-trial process has on defendants. Although Canadian courts deny that pre-trial detention is a punishment,\footnote{See \textit{R v Morales}, [1992] 3 SCR 711, [1992] ACS no 98.} Bowers correctly characterizes the pre-trial criminal process as such. The pre-trial process can be lengthy and tedious for defendants released on conditions and detained in custody. It may require multiple court appearances, arranging for and meeting with lawyers when represented by one, and researching and preparing a defence when self-represented, all with the threat of an additional punishment looming at the end of it all. For defendants remanded or denied bail, the pre-trial detention period serves as a starker method of pre-trial punishment.\footnote{To the extent that recidivists raise more concerns that they will re-offend or fail to appear for court, they are more likely to be detained pre-trial. Where defendants face more time in pre-trial detention than they would likely receive as a sentence after an unsuccessful trial, they become much more likely to be motivated to resolve their charges. See Bowers, \textit{supra} note 187 at 1137 — 1139.} Where these process costs outweigh the penalty that the prosecutor is willing to settle for, innocent and guilty defendants may reasonably wish to stop their punishment by self-convicting as soon as possible and moving forward unencumbered. As Bowers aptly summarizes, ``plea bargaining works best for innocent defendants for whom the process is the punishment."\footnote{See \textit{ibid} at 1158.}

Underlying Bowers' position is the notion that truth and accuracy must sometimes come at a criminal defendant's expense and the argument that innocent defendants should not be expected to bear these burdens unnecessarily. While some factually innocent defendants want to pay that price, others will not. Whether they wish to do so will depend on the relative value of clearing their names versus the expense of the punishment they must endure while doing so. Recidivists are generally less concerned about a criminal conviction than first-time defendants, while all defendants are less concerned about convictions for minor infractions than for major offences. Bowers concludes that innocent defendants charged with minor offences benefit most from having a plea bargaining stop-gap available to them and suffer most when forced to take these cases to trial. Defendants charged with serious offences are least likely to receive lowball plea offers, such that a plea deal's incentives quite appropriately ``matter least where due process and trial rights matter most."\footnote{See \textit{ibid} at 1153.} 

It is unjust to subject a wrongfully accused defendant to a greater pre-trial punishment than they would receive if convicted after a trial. Allowing self-convictions when there is no other legal recourse to end wrongful punishment provides defendants with a functional stop-gap solution. It is a functional resolution for those who do not expect extensive due process, are highly motivated to resolve and are likely to receive generous offers that they otherwise would not receive after a trial. 

None of these observations or concessions diminish the fact that wrongful convictions are miscarriages of justice. Even where the defendant is a recidivist who receives a favourable deal, a wrongful conviction can severely disrupt their lives and undermine the public's confidence in the administration of justice. But a narrow focus on wrongful \textit{convictions} risks overlooking the fact that they are the aggregate results of wrongful \textit{accusations}, \textit{arrests}, \textit{prosecutions}, \textit{pre-trial restrictions}, and occasional \textit{pre-trial detentions} that precede them. Not all wrongful accusations end in wrongful convictions, but all wrongful convictions necessarily begin with wrongful accusations and arrests. Similarly, not all wrongful prosecutions end in wrongful convictions, but all wrongful convictions necessarily entail a wrongful prosecution. Furthermore, and as discussed above, wrongful convictions are not exclusively or even primarily the result of self-convictions, with many documented wrongful convictions occurring after a trial. By citing wrongful convictions as a reason to be wary of plea bargaining or certain forms of self-conviction, plea bargaining opponents overlook these more fundamental and foundational miscarriages of justice.

This oversight is especially problematic in cases where the effects of the proceedings leading up to a wrongful conviction are functionally identical to the wrongful conviction itself. A wrongfully-charged defendant who is arrested for an offence and denied bail but ultimately acquitted at trial is in functionally the same position as another wrongfully-charged defendant who is arrested for that same offence, denied bail, but ultimately convicted on that trial date, either through self-conviction or through the trial process itself. The only meaningful distinction between these two hypothetical defendants is the conviction itself. Where that conviction does not involve further punishment or restrictions, it is arguably a negligible factor to be considered when evaluating the situation for miscarriages of justice.

For example, the police find two functionally identical defendants inside a vehicle and wrongfully arrest them for stealing it a week earlier. Both claim to have not known the vehicle was stolen and tell the police that an acquaintance who lent them the vehicle is the most likely suspect, but are nonetheless remanded into custody. Both apply for bail but are denied, as they live outside of the jurisdiction normally, cannot access a cash deposit, have limited criminal records comprised of minor property and administration of justice offences, and cannot otherwise satisfy the bail judge that they will attend the trial. Neither is prepared to admit that they stole the vehicle, so a trial date is set.

A week before trial, the prosecutor learns from an out-of-province colleague that the mutual acquaintance confessed to several vehicle thefts with similar facts to this one in their home jurisdiction several months ago. The prosecutor discloses this fact to the defendants but still believes that the recent possession doctrine\footnote{See \textit{R v Terrence}, [1983] 1 SCR 357, 4 CCC (3d) 193.} gives the state a reasonable prospect of a conviction as parties to the offence. Because there is now an alternate suspect, and both have spent a considerable amount of time in custody, the prosecutor offers them the option to plead guilty to a lesser-included offence of possessing property obtained by crime in exchange for a small fine noted in default. Should they opt for trial, the prosecutor indicates they'll proceed on the charged offence and recommend a time-served custodial sentence followed by probation. Defendant \#1 accepts this deal on the trial date by agreeing they should have known the vehicle was stolen and is released after the judge authorizes the sentence. Defendant \#2 refuses the deal, runs the trial on an alternate suspect defence, and is acquitted.

Because neither defendant was subjected to court-imposed restrictions past their trial date, the only meaningful distinction between them is that one was formally convicted and received an addition to their record while the other was not and did not. But in this scenario, the conviction itself is the factor that arguably contributes the least to the overall miscarriage of justice. Rather, the pre-trial detention in a different jurisdiction, coupled with the attendant opportunity loss, separation from one's family or social group, public repudiation, lack of viable bail programs, investigative inertia, low and vague prosecutorial thresholds for proceeding and similar social and legal deficiencies contribute the lion's share to the injustice each faced. Framed negatively, the defendant who was acquitted after risking a trial is functionally no less the victim of a miscarriage of justice than the one who attenuated their potential jeopardy by self-convicting and accepting a nominal formal punishment. Both defendants would have benefitted from more thorough inter-jurisdictional investigation and information-sharing, ongoing plea negotiations throughout the pre-trial period, and the ability to self-convict without admitting the offence.

Not all voluntary wrongful convictions are the same, and many factors can mitigate or aggravate the degree to which they might reasonably amount to miscarriages of justice. Questions about the offence's severity, whether the defendant had a criminal record before the wrongful conviction, whether the defendant received a criminal entry on their record or was discharged, the amount of time spent in custody, if any, and the circumstances of their pre-trial restrictions or detention are all factors that can make a wrongful conviction more or less unjust. By comparison, whether a defendant formally self-convicts as a result of a plea bargain or is convicted by a judge or jury is arguably inconsequential and assists very little in addressing the factors that make wrongful convictions miscarriages of justice. Instead, focusing on the composite stages preceding a wrongful conviction is likely a more profitable avenue to address the end problem. 

Requiring arrest warrants as a rule, rather than an exception, or requiring the police to establish something akin to a \textit{prima facie} case before arresting a defendant without one may help reign in wrongful accusations and arrests. Requiring prosecutors to have a clear and demonstrable likelihood of conviction, rather than a merely reasonable prospect of conviction, standardizing and expanding inter-jurisdictional information sharing, and funding legal aid programs to allow defendants to meaningfully investigate their cases and develop positive defences may curtail more wrongful prosecutions. Similarly, mandating and enforcing continuous internal peer review throughout the prosecution or requiring judicially-supervised case management between counsel in more criminal cases may do the same. Establishing adequate bail programs and accommodations in provinces that lack them and expanding the range of community support and supervision throughout the country may make it possible for more defendants to be safely and securely released into the community pending trial and thus may curtail the perceived need to self-convict to end wrongful pre-trial detention. 

Plea bargaining opponents who point to wrongful convictions as a reason to repudiate the practice miss the point that simply preventing or limiting a defendant's ability to bargain for their freedom accomplishes none of these objectives. Plea bargaining is not a proximate cause of most wrongful convictions, and eliminating or curtailing it does little to address the wrongful accusations, arrests, prosecutions, pre-trial conditions and pre-trial detentions that underlie and aggravate voluntary wrongful convictions. Where a wrongful self-conviction is capable of redressing some or all of these injustices in a way that is expedient, mutually beneficial, or both, I argue they should be encouraged. Realistically, the justice system's task is to minimize the miscarriages of justice that culminate in wrongful convictions, not to rally around fantastical calls to eliminate them altogether.\footnote{See MacFarlane, \textit{supra} note 210 at 405.} 

Rather than embrace a seemingly duplicitous and entirely unnecessary ``legal fiction," I argue that the best way to allow wrongfully punished defendants is for Parliament to formally codify a non-culpatory uncontested plea. A non-culpatory uncontested plea like \textit{nolo contendere} allows defendants to self-convict without having to admit the allegations against them. Under American common law, they are inadmissible in subsequent proceedings. To the extent that \textit{nolo contendere} pleas are available in this format, they are the ideal legal instrument for enabling an innocent defendant to end their wrongful punishment. They allow defendants to waive formal proof requirements while claiming nothing about what the defendant believes and may be inadmissible in subsequent proceedings. Although critics view these pleas as dishonest, defendants stuck in a wrongful punishment holding pattern may enter them honestly to draw their unjust circumstances to an end.

\subsection{Substance}

Although allowing defendants to end wrongful incarcerations with as little fallout as possible is laudable, some plea bargaining critics argue that this individual good comes at too great a collective price. In his article ``Harmonizing substantive-criminal-law values and criminal procedure: The case of Alford and nolo contendere pleas," Stephanos Bibas argues that \textit{nolo contendere} and \textit{Alford} pleas detract from substantive legal values by taking the focus off denouncing crimes and reforming criminals and shifting the focus onto economic bargaining for sentence mitigation. Substantive criminal values are the moral underpinnings of the criminal justice system. Denouncing and deterring criminal behaviour, reintegrating offenders into society, and giving them a personal sense of accountability are substantive criminal values, as are the other sentencing principles codified in \textit{Criminal Code} sections 718 — 718.2. Per Bibas, procedural values include choice, autonomy, accuracy, and efficiency.

It is important to note that Bibas is not opposed to plea bargaining \textit{per se}. His exact concerns are with \textit{nolo contendere} and \textit{Alford} pleas, and he recognizes that plea bargaining has some value as a means to ``induce guilty defendants to confess and start repenting."\footnote{See Bibas, ``Harmonizing Substantive Values," \textit{supra} note 21 at 1400. A former academic, Bibas was appointed by former US President Donald Trump to the United States Court of Appeals for the Third Circuit in 2017.} But most of the substantive problems he raises with \textit{nolo contendere} and \textit{Alford} pleas specifically apply to plea bargaining generally, insofar as plea bargaining prizes procedural values like efficiency and certainty over substantive values like denunciation and deterrence.\footnote{See \textit{ibid} at 1367.} Bibas addresses these concerns under three broad categories:

\begin{itemize}
    \item \textbf{Justifications for punishment and the law's moral norms.} Punishing defendants is the ethical business of the criminal justice system. The criminal justice system centres on the belief that the courts should denounce offences and offenders should be educated and reformed. Plea bargaining disregards these norms in the name of procedural efficiencies and economic autonomy.
    \item \textbf{Reluctance to confess and the value of confession.} Plea bargaining defendants may be motivated by a reluctance to confess their misdeeds. Confession is good for defendants, victims, and society and should therefore be encouraged. Conversely, insofar as plea bargaining allows defendants to avoid confession, they should be discouraged.
    \item \textbf{The substantive value of trials.} Trials are not merely a way to determine a defendant's guilt or innocence. Instead, they are ``morality plays" that denounce unlawful conduct and place society's values on display. On the other hand, shadowy plea-bargained deals try to keep this process out of the public's eye.
\end{itemize}
These substantive harms Bibas associates with \textit{nolo contendere} and \textit{Alford} pleas are inductively applicable to plea bargaining generally. Merely disallowing \textit{nolo contendere} and \textit{Alford} pleas is insufficient to reverse the erosion of substantive legal values, halt it, or even meaningfully slow it down. In Canada, where plea bargains are encouraged but \textit{nolo contendere} and \textit{Alford} pleas are formally unrecognized, Bibas's critiques of \textit{nolo contendere} and \textit{Alford} pleas are nonetheless applicable to plea bargaining. However, Bibas's praises for plea bargaining are confined to its ability to coerce guilty defendants into confessing their crimes.\footnote{See also Stephanos Bibas, ``Bringing Moral Values into a Flawed Plea-Bargaining System" (2003) 88:5 Cornell L Rev 1425 [Bibas, ``Bringing Moral Values"].} This limited view both fails to touch upon the real \textit{substantive benefits} that plea bargains entail and ignores the \textit{substantive harms} that trials cause. To the extent that plea bargaining encourages and develops different substantive law benefits, the substantive law objection is less an objection in principle and more an objection rooted in a preference for competing substantive values.

\subsubsection{Justifications for Punishment and the Law's Moral Norms}

Bibas argues that criminal law is not simply a means for meting out punishments to counteract crimes. It is also a way to achieve retributive, denunciatory, and rehabilitative ends. Bibas proposes a ``theory of punishment as moral education"\footnote{See Bibas, ``Harmonizing Substantive Values," \textit{supra} note 21 at 1390.} where he argues that properly constructed criminal sanctions can develop an offender's atrophied conscience and foster a sense of responsibility in them. By doing so, the justice system increases the offender's chance to rehabilitate and decreases their risk to society. Values like retribution and rehabilitation are considered alongside harm reduction principles, such that victims are vindicated by seeing their perpetrators take responsibility for their actions and receive their due punishment. More generally, the courts deter all members of society from committing such crimes by reminding them that they will be similarly corrected if they choose to similarly offend. 

Plea bargaining, on the other hand, gamifies the justice system by prompting its various participants to focus on cost-benefit analyses rather than on the substantive values Bibas champions. Plea bargaining critics argue that the practice encourages defendants, prosecutors, and courts to focus on cost efficiencies rather than the fittest and most appropriate sentence.\footnote{See Alschuler, \textit{supra} note 4; Bibas, ``Bringing Moral Values," \textit{supra} note 240.} Values like retribution and rehabilitation take a back seat to defendants and prosecutors minimizing and maximizing sentences, respectively, and courts encourage resolutions that keep process costs at a premium. The focus shifts to individual autonomy and procedural efficiency rather than on society and the system's underlying values.

Bibas correctly notes that plea bargaining's proponents advocate for the process because it dramatically boosts procedural efficiency and astutely notes that these efficiencies come at the cost of substantive legal values. He rightly categorizes these efficiencies as \textit{procedural} benefits but incorrectly casts them as coming at the expense of substantive values generally and necessarily. Although procedural efficiencies have a substantive cost, Bibas fails to consider the \textit{substantive benefits} that procedural efficiencies concomitantly bring with them. Instead, I argue that procedural efficiencies \textit{qua} efficiencies are normative goods necessary for the law to realize its underlying substantive aims. Plea bargaining opponents generally believe that defendants who receive a trial receive superior due process compared to defendants who negotiate an early plea deal. This belief's problems begin with the misperception that trials are effective truth-finding devices.

A trial must be held in cases where negotiations have failed or are otherwise impossible. The time and expense required to schedule and run trials are prohibitive and scale poorly alongside due process. While a single judge in a single courtroom can accommodate dozens of resolved cases in a day, that same judge is only likely able to hear one or two very brief trials in the same amount of time. Trials with more witnesses or complicated legal issues may require additional days and weeks for a judge or jury to hear the entire case. Any significant increase in the number of trials set, as would be reasonably expected in a system with plea bargaining removed, results in a significant backlog and increases the likelihood of errors at first instance. Relieving this backlog and protecting against these errors would require increasing ``court resources," such as judges, lawyers, court staff, correctional staff, facilities and logistics. Failing to relieve this backlog would severely interfere with a defendant's procedural right to be tried in a reasonable time and with the substantive values of having a trial heard while recollections are fresh and the community still feels the offence's impact. Conversely, the reverse is true when more unnecessary trials stop and only the issues that need adjudication receive judicial attention.

The second branch of this criticism argues that the cost-benefit analysis inherent to plea bargaining necessarily prioritizes deal-making at the expense of a thoughtful analysis of the fittest and most appropriate sentence.\footnote{See Bibas, ``Harmonizing Substantive Values," \textit{supra} note 21 at 1404.} This criticism too quickly ignores deal-making's role \textit{as a form of thoughtful sentencing analysis}. Plea negotiations allow prosecutors and defendants to discuss the most mitigating and aggravating aspects of the case, weigh these values against one another, and determine whether a middle ground can adequately account for both. Even where an agreement is not ultimately reached, the process narrows the range of possible issues where trials are required or the range of appropriate sentencing recommendations where a defendant still wishes to self-convict.

Plea bargaining is a streamlining process where the parties with the most access to the best information review, discuss and analyze the evidence amongst themselves. Hastily negotiated plea deals are a hazard to be avoided, not an inevitable consequence of this process. Where effective, plea bargaining helps ensure accurate and fair pleas. Accuracy, fairness, and efficient functions that lower process costs for all parties involved give the system the time and resources it needs to address substantive justice concerns. Although these values differ from the confessional principles that Bibas prizes, they are no less substantive or essential.

\subsubsection{Reluctance to Confess and the Value of Confession}

Bibas next discusses the substantive value of defendants publicly confessing their crimes. He draws upon examples of defendants whose personal epiphanies in the face of their guilt brought them to confess their guilty honestly,\footnote{See e.g. \textit{ibid} at 1364. It is unknown whether this particular confession assisted the defendant's rehabilitation.} contrasting them with defendants who avoid these self-confrontations. Although this criticism primarily targets ambivalent \textit{nolo contendere} and defiant \textit{Alford} pleas, as defendants who enter such pleas do not attempt to confess at all, it applies with some force to plea bargaining more generally. 

To the extent that plea bargaining may induce defendants to enter insincere guilty pleas just as readily as it may induce them to enter \textit{nolo contendere} pleas, the relative value of the guilty plea as a \textit{confession} is minimal. For a confession to have any substantive value, it must be sincere. An insincere confession does little to rehabilitate a defendant, make society more secure, or provide victims with any satisfaction. Bibas discusses such confessions against the presupposition that police and prosecutors should pursue them over certain convictions.\footnote{See \textit{ibid} at 1399 — 1400.} I disagree with this presumption. 

I agree with Bibas when he argues that a sincere guilty plea is preferable to a non-committal \textit{nolo contendere} plea or a false guilty plea. Defendants who genuinely see the error of their ways, commit to turning their lives around, and have the means and support to do so are prime candidates for rehabilitation and poster children for the system's ability to return just results. But for many reasons, such offenders are not the norm, and the justice system has no method to produce more of them reliably. Instead, justice system participants must grapple with a tide of sub-optimal rehabilitation candidates, many of whom are likely to be more ambivalent about their culpability and less remorseful for the consequences of their actions. Given this reality, those advocating for confessions as rehabilitative tools should consider the relative value of a sincere confession that may never materialize versus a disingenuous acceptance of responsibility. 

To support his position, Bibas relies on a series of informal surveys and limited-scale studies as evidence that he believes supports his approach to moral criminal punishment. These include anecdotal accounts from judges and lawyers he knows and a study of recidivism rates among eight offenders in one state who entered \textit{Alford} pleas.\footnote{See \textit{ibid} at 1395 — 1398.} While these sources are not without value, they are neither clear nor convincing. Unlike the \textit{value of confessions}, which are difficult to quantify without hard empirical data and diligent study, the \textit{value of self-convictions} are readily inferrable. When defendants self-convict, they excuse all witnesses from the trial, admit all the elements of the offences alleged against them, and submit to the state for punishment. Victims can confront the one who wronged them without the added pressure of having to help convict them, perceived or otherwise. 

Without dependable data explaining the efficacy of confessions and a way to reliably increase the percentage of sincere confessions amongst self-convicting defendants, sacrificing certain conviction and resolution to a case in exchange for a merely possible and similarly remorseless conviction after a trial is a dubious prosecutorial position. Requiring vulnerable or unwilling witnesses to testify at such trials and forcing defendants willing to resolve to continue on punitive pre-trial conditions further problematizes this already dicey proposition. Although it would be preferable for defendants to repent sincerely, this is never a guaranteed outcome in any case, nor is it necessary for justice to be done. Absent evidence that defendants who plea bargain are less likely to be rehabilitated than defendants convicted after unwanted trials, plea bargains should be encouraged.

\subsubsection{The Substantive Value of Trials}

Bibas sees the jury trial as the law's ideal solution for defendants who cannot or will not plead guilty or for factually innocent defendants.\footnote{See \textit{ibid} at 1400.} He envisions trials as ``morality plays" that ``dramatically present the conflicting moral values of a community in a way that could not be done by logical formalization."\footnote{See \textit{ibid} at 1401.} Through them, the court confronts defendants with ``solemn pronouncements of guilt" and condemns them in the hopes of ``break[ing] through the defendant's denial mechanisms, driving home in undeniable detail the wrongfulness of the crime."\footnote{See \textit{ibid}.} Allowing defendants to enter \textit{nolo contendere} or insincere guilty pleas sends the message that justice does not require defendants to take a clear stance and should therefore be discouraged. Instead, Bibas believes defence lawyers should challenge their clients when they deny they are guilty and encourage them to consider the long-term advantages of sincerely pleading guilty.\footnote{See \textit{ibid} at 1404f.}

Although reimagining defence counsel's role in this manner would undoubtedly encourage more confessions, this approach is immensely problematic. Beyond the fact that it would require reconceiving the entire adversarial trial system to execute, this proposition ignores that criminal defendants hire lawyers because they wish to contest their charges or negotiate for the most favourable outcome.\footnote{See Bowers, \textit{supra} note 187 at 1122 — 1123.} They do not hire lawyers because confession is good for the soul. Expecting defence lawyers to roleplay as grand inquisitors creates an unnecessary and unhelpful ``prisoner's dilemma." In this dilemma, all defence lawyers are encouraged to try to convince defendants to confess, knowing that the first one to tell the defendant that they have a case to fight ultimately walks away with a client. Enforcing this expectation would be a herculean effort without a fully publicly funded system for criminal defence attorneys.

Even if this confessional expectation were enforceable, there is good reason to doubt that it would be a wise idea. Bibas is alive to the problem of wrongful convictions,\footnote{See Bibas, ``Harmonizing Substantive Values," \textit{supra} note 21 at 1382 — 1386.} but does not consider how this proposal will certainly and dramatically increase them. Though imperfect, the adversarial system ensures that innocent criminal defendants have an advocate to advance their position against the state's allegations. By transforming that advocate into another person whose job is to convince them to confess, even recalcitrant innocent defendants may come to believe there is little point in contesting the allegations against them.\footnote{It is interesting to consider the similarities between this proposal and the events that occurred at trial in \textit{Alford}. Alford's trial lawyer actively and successfully convinced their client to plead guilty after reviewing the seemingly overwhelming evidence against him, giving rise to the plea procedure Bibas believes so profoundly flawed.} Bibas does not account for how the ``morality play" model of justice will handle the swath of wrongful convictions that are sure to result. Nor does he address how society will go about reconciling the wrongfully convicted after presuming them guilty, gaslighting them into thinking that they \textit{were} guilty, and publicly denouncing them for their unrepentance. 

Bibas correctly draws attention to the impact of efficiency-driven plea bargains on substantive criminal values. The principles he defends are worth defending and should not be uncritically put to the side in exchange for procedural efficiencies and fairness. But the line that he draws between substantive and procedural values is less clear than he alludes, such that the values he classifies as procedural impact the values he classifies as substantive, and vice versa. The substantive confessional shift Bibas suggests is a disaster but is generally directed at \textit{nolo contendere} and \textit{Alford} pleas specifically, rather than plea bargaining generally. After considering both the truth and fairness problems in the next section, I conclude by examining Bibas's critiques of \textit{nolo contendere} pleas specifically to determine whether they should be formalized in Canada.