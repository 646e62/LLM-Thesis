\section{Plea bargaining}

Some defendants who self-convict do so for selfless or pro-social reasons, such as a desire to spare a complainant from having to testify at trial or because they are genuinely remorseful for their actions. However, some defendants who are reluctant to admit responsibility may be willing to do so for some consideration. Plea bargaining occurs when a prosecutor offers a defendant an incentive to self-convict, ostensibly to ensure efficient case resolution on a broader scale. 

Plea bargaining relies on the fact that some \textit{quid pro quo} exchange exists in every prosecution. Prosecutors lay new charges and may withdraw existing ones, while defendants control whether to contest the charges and force a trial. Both sides face specific pressures that may result in overlapping goals. For example, both prosecutors and defendants are motivated to resolve cases quickly and efficiently. Defendants detained or released on conditions may be motivated to resolve their matters efficiently by a desire to end their pre-trial restrictions. In contrast, prosecutors may be motivated to resolve their matters efficiently by the constant threat of a successful delay motion and a judicial stay of proceedings.

Although plea bargaining is a widely used practice, it remains controversial. Plea bargaining's proponents argue that plea bargaining's efficiencies are needed to maintain the day-to-day operation of the justice system\footnote{See \textit{R v Butt}, 1987 CanLII 5192 (NL SC) at para 48.} while its opponents argue that these efficiencies come at too high a price.\footnote{Other plea bargaining opponents dispute the notion that plea bargaining provides any net efficiencies to the justice system. See \hl{example}.} Although the objections to plea bargaining are legion, they may be generally summarized as falling under one of three categories:

\begin{itemize}
    \item \textbf{Fairness.} Plea bargains penalize defendants who plead not guilty by creating a sentencing gap between otherwise equally situated defendants, and coerce them through high-stakes offers that are too good to refuse. These plea bargaining facets improperly and unfairly induce self-convictions.
    \item \textbf{Truth.} Even if plea bargains do not improperly induce defendants, they increase the risk that factually innocent defendants will self-convict. This leads to inaccurate results and wrongful convictions.
    \item \textbf{Substance.} Plea bargaining transforms the criminal justice process into an economic system that encourages deal-making and the ``gamification" of punishment. Encouraging this comportment undermines the substantive moral values undergirding the criminal justice system.
\end{itemize}

The Supreme Courts in Canada and the United States have endorsed plea bargaining because of its important function in our justice system. In \textit{Santobello v New York}, 404 US 257 (1971), the Supreme Court of the United States affirmed that the plea bargaining process is a crucial part of the criminal justice process whose agreements should be upheld. In Canada, 25 years later, the Supreme Court of Canada followed suit in \textit{Burlingham},\footnote{\textit{R v Burlingham}, 1995 CanLII 88 (SCC), [1995] 2 SCR 206 (\textit{Burlingham}).} identifying plea bargaining as ``an integral element of the Canadian criminal process.'' Although \textit{Criminal Code} s 606(1.1) ensures that Canadian judges are not required to uphold plea agreements, the Supreme Court of Canada has made it clear that judges must defer to truly jointly recommended sentences.\footnote{See \textit{R v Anthony-Cook}, 2016 SCC 43, 2016] 2 SCR 204}

While plea bargaining advocates point to the procedural efficiencies and economic benefits that plea bargaining generally brings to criminal defendants and society, plea bargaining opponents stress that these efficiencies have a cost.\footnote{} Opponents generally see plea bargained deals as compromises bought at the expense of a defendant's right to adequate due process and argue that the efficiencies gained through plea bargaining are not worth the price paid.

Although many ethical positions on plea bargaining are similar, a few approaches raise interesting new questions and warrant closer treatment. In each subsection, I engage with an author with such an approach to determine whether plea bargaining is a good that should be pursued, an evil that should be avoided, or another option in between. In \S 4.2.1, I consider the \textit{fairness problem} against the insights Michael Young offers in his article ``In Defense of Plea-Bargaining's Possible Morality."\footnote{} While I agree with Young's underlying premise, I argue that he does not go far enough with his execution. Next, in \S 4.2.2, I consider the \textit{truth problem} alongside Joshua Bowers' ``Punishing the Innocent."\footnote{} I adopt many of Bowers' insights concerning the truth problem but find that he fails to address the problem directly. I attempt to do so. Finally, in \S 4.2.3, I address the \textit{substance problem} through Stephanos Bibas' ``Harmonizing substantive-criminal-law values and criminal procedure: The case of Alford and nolo contendere pleas."\footnote{} While I agree with Bibas that substantive criminal legal principles should be a critical part of our discussions about plea bargaining, I otherwise disagree with his assessment of how plea bargaining impacts those values.

\subsection{Fairness}

Plea bargaining opponents worry that the plea bargaining system is such that defendants are indirectly induced or threatened into pleading guilty. Although not all inducements, threats, or other oppressive conditions, are sufficient to override a defendant's will, convincing them to plead guilty where they may be acquitted and to accept resolutions that seem too good to refuse. The inducements that naturally accompany criminal charges appear oppressive in that the balance of power skews heavily towards the prosecution and that involuntary self-convictions appear inevitable. 

\subsubsection{The trial penalty}

The trial penalty criticism charges that defendants suffer a trial penalty when they opt for a trial rather than a plea bargain and are likely to be sentenced more harshly if convicted. Where two otherwise equally situated defendants are convicted of the same crime, but one of them accepts an early plea deal in lieu of a trial, the one who accepted the plea deal will likely receive a more lenient sentence than the one who did not.\footnote{To the extent that no two offences or offenders are ever exactly alike, "similar offences" and "similarly situated offenders" will always be distinguishable to some degree. The incredible array of subtle distinctions between offences and offenders can either be overlooked or microscopically examined, depending on things like the offence, the offender's history, the jurisdiction where the offence occurred, and so forth.} The apparently uneven and intuitively unfair treatment these two defendants receive is the subject of the trial penalty criticism.\footnote{See Young at 269: 
\begin{quote}
    Few criminal defendants would plead guilty absent a credible chance of reducing their expected sentences. Plea-bargaining thus depends on the possibility of a range of sentencing options, and on differential treatment for plea-bargaining and non-plea-bargaining defendants. This effectively means that not all criminal defendants will be treated alike, and that non-plea-bargaining defendants will be (or, if declining a plea-bargain, can reasonably expect to be) sentenced relatively more harshly. ... [T]he critic thinks that this differential treatment betrays a troubling lack of concern for equality among similarly situated criminal defendants, and is a violation of the ancient and venerable maxim of justice to treat like cases alike.
\end{quote}}

Young considers three responses to this trial penalty critique:

\begin{itemize}
    \item \textbf{Recasting the trial penalty as a plea bargain benefit.} The disparity between defendants who plea bargain and defendants who do not should not be cast as a ``penalty." Rather, it should be understood as a benefit afforded to those who plea bargain.
    \item \textbf{Reducing penalties in exchange for prosecutorial resources, the common good, or both.} Defendants who plead guilty free up resources that would have otherwise been spent on their trials. Doing so is a moral good, and it is appropriate to reward defendants with reduced punishments for freeing up those resources.
    \item \textbf{Re-examining the assumptions underlying equal treatment and the defendant's role in the discrepancy.} When determining whether a defendant has been coerced, the most important consideration is whether equally situated defendants are given equal opportunities to plea bargain. Defendants who voluntarily opt not to plea bargain are accountable for the impact this may have on future bargaining positions.
\end{itemize}

Faced with the trial penalty criticism, plea bargaining proponents may reasonably counter that there is no ``penalty" \textit{per se}, but rather a ``reward" for resolving their charges without needing a trial. The trial penalty criticism therefore misunderstands plea bargaining. As Young points out, this response is insufficient as the criticism is aimed at the \textit{inequality} that plea bargaining creates. Recasting the inequality in more favourable terms is merely public relations and fails to address the inequality underlying the concern.

Young next considers that these supposedly equally situated defendants may not be equally situated. Where two such defendants are charged with the same offence and one opts to plea bargain, one defendant saves ``public resources on an expensive prosecution" while the other does not. To the extent that saving the state the expense of a trial contributes to the public good and to the extent that contributing to the public good is morally praiseworthy, defendants who plea bargain are therefore more morally praiseworthy than their counterparts and should be treated as such. 

Young rejects this argument because prosecutorial resources and punishment are not commensurate commodities. Prosecutorial resources are, in effect, monetary resources,\footnote{See Young} and Young argues that allowing one to be exchanged for another invites unjust results.\footnote{See Young at 271: ``It would be perverse, for example, to think that a rich criminal who donated money to the prosecutor's office — thus advancing the common good by providing resources for prosecutions — would even presumptively deserve a sentencing reduction for that reason alone."} Similarly, where a defendant commits a crime with serendipitous consequences,\footnote{A sexual assault that helped spark a movement that raised society's awareness of the problem more generally, destruction of property generally considered to be offensive or an eyesore, and medical malpractice that ended the life of a genocidal dictator are examples of crimes with arguably serendipitous consequences.} they should not be rewarded for these incidental benefits.\footnote{See \textit{ibid}: ``[S]uppose that through some lucky causal accident the commission of some crime also brings about collateral social effects contributing positively to the common good: we would not think that this fortunate fact properly weighed as a sentencing consideration. So it cannot be the case that, in general, contributing to the public good constitutes a reason for a reduction in sentence." But see \textless \url{https://plato.stanford.edu/entries/moral-luck/}\textgreater} Absent a more direct correlation between the common good and moral praiseworthiness, merely saving state resources is insufficient to distinguish one person's moral praiseworthiness over an other's. This example, however, is a hasty generalization and a straw man. While certain financial arrangements between defendants and the prosecutor's office would be objectively seamy, separating criminals from their money remains a time-honoured and occasionally effective punishment. Furthermore, Young fails to consider other reasons why a defendant who foregoes their right to a trial may be more morally praiseworthy than one who does not.

When a judge imposes a fine on a defendant, they signal that money and punishment are commensurate resources. Parliament has recognized that there can and ought to be a monetary value placed on time spent in custody and a custodial value placed on misappropriated funds. Where convicted defendants have spent time in pre-trial custody and are sentenced to a fine, some or all of that fine amount may be noted in default under the ``time is money" provision outlined in \textit{Criminal Code} ss 734(4) \& (5). Additionally, statutory victim fine surcharges,\footnote{See \textit{Criminal Code} s } restitution orders, forfeiture orders, and judge-ordered donations are all ways the court correlates moral blameworthiness with the very same resource used by the Attorney General to bring cases to trial. There is a strong correlation between financial resources and punishment, such that a defendant who has been punished financially is no longer equal to another defendant who has not.

Young's criticism is more severely undermined by his focus on prosecutorial resources. Sparing the state the \textit{financial expense} of a trial is not a defendant's only \textit{quid pro quo} or even the most valuable consideration they can offer. Defendants who opt to self-convict rather than take their matters to trial relieve the state of its burden to prove the offence. As a result, witnesses who would have had to have testified are no longer required to do so. Where those witnesses are vulnerable individuals, they are spared the potential trauma of having to revisit the offence. Where those witnesses are unreliable individuals, the state is spared the difficulty of having to elicit evidence from hostile parties. All witnesses are spared the pressure of having to come to court and provide evidence. More importantly, self-convicting defendants guarantee that the state will convict them. Trials are inherently volatile processes, and convictions are never guaranteed, even in cases where the evidence appears overwhelmingly strong. A self-convicting defendant not only spares the state the expense of prosecution but also the risk of an expensive \textit{and potentially unsuccessful} prosecution.

Finally, Young takes issue with the naive ``distributional" model of inequality that argues that two identically situated defendants should not be sentenced differently solely because one opts for trial while the other self-convicts. Instead, he proposes substituting a ``relational" model in its place. Where the distributional model of equality judges equal treatment based on how evenly punishment is distributed across similarly situated defendants, the relational model asks whether ``someone or some group [being] subordinated or dominated, or in some similar way treated without proper respect or concern."\footnote{See Young at 272.} Under Young's relational model, equal treatment lies in the relationships between parties when compared. Defendants are equally treated when they have \textit{equal opportunity} to engage in plea bargaining. By contrast, where one defendant is given an opportunity to plea bargain but another similarly-situated defendant is not, they receive unequal treatment under the relational model. Where one defendant opts to accept a deal for a lighter penalty and another does not, \textit{their refusal to do so is the cause of their distributive inequality}. Assuming that similarly-situated defendants have equivalent access to equivalent plea bargain offers, those who turn them down can and should be held responsible for the consequences of that choice.

Young correctly argues that the ``trial penalty" and its primary underlying concern of equal treatment under the law are not valid criticisms of the plea bargaining procedure. \textit{Contra} Young, defendants who opt to self-convict rather than contest their matters make a morally praiseworthy decision that should be recognized. Finally, where distributive inequalities exist between otherwise similarly-situated defendants who set their charges down for trial and defendants who plea bargain, these differences will only meaningfully impact the question of equal treatment where defendants have unequal access to plea bargaining. 

\subsubsection{The coercion worry}

By giving defendants offers that are simply too good to refuse, the plea bargaining opponent argues that prosecutors coerce defendants into accepting self-convictions through inappropriate inducements. Young identifies three distinct definitions of coercion that commonly emerge in plea bargaining literature, the latter of which he ultimately endorses:

\begin{enumerate}
    \item \textbf{Restricting a defendant's rational choice.} Plea resolution offers become coercive when defendants have no rational choice but to accept them.
    \item \textbf{Requiring a defendant to make a choice with high stakes.} When a plea resolution offer requires a defendant to make a high-stakes choice involving their liberty, they are coerced into minimizing their risks.
    \item \textbf{Overriding a defendant's will by wrongfully influencing them.} Prosecutors and defendants each have special roles in criminal prosecutions. Because of these roles, some minor inducements may be coercive while some major inducements may not. The issue depends on whether the prosecutor improperly interfered with the defendant's rights or position through a plea bargaining inducement.
\end{enumerate}

The first position argues that criminal defendants are coerced into accepting plea bargains when the only rational choice they may make is to accept a plea-bargained offer. Just as an armed robber who demands a wallet at gunpoint coerces their victim into parting with their belongings by denying them any other rational choice.\footnote{See Young at 264.} Defendants may be effectively coerced into making that decision when the only rational choice is to self-convict. The ``bare restriction on rational, utility-maximizing choice"\footnote{See Young at 262.} is itself coercive. 

But identifying coercion with a lack of rational alternatives to choose from fails to account for situations where a person is left with no rational choice but to accept a clear advantage or a windfall. A person who purchases a winning lottery ticket may have no rational choice but to go to the lottery commission and claim their prize but would not be reasonably said to have been coerced into making that decision. \textit{Restricting choices}, even to the point where there is no other rational choice to be made, cannot be the sole measure of coercion.

The next coercion criticism shifts focus to \textit{the stakes involved} in making the decision. Whereas a lotto winner faces little to no penalty for failing to collect their ticket (other than having their lives go on exactly as they did before), a criminal defendant may be looking at life-altering consequences if they fail to make the only rational decision they have been left with. The strong version of this criticism argues \textit{high-stakes decisions necessarily} impose coercive pressures on defendants. The weaker version of this criticism argues that the pressure from high-stakes decisions, \textit{combined with a lack of any other rational choice}, is coercive. However, both versions similarly fail to hit the mark. For example, a parent who sees their young child wandering near a busy road will have no other rational choice but to step in and try to stop them, but despite the high stakes and lack of reasonable alternatives, it would be inaccurate and unreasonable to call that decision coerced. 

Where a prosecutor offers a very low sentence on a guilty plea, as opposed to a comparatively very high sentence upon conviction after trial, they may be said to be engaging in hard dealing. To the extent that this tactic is likely to result in the defendant accepting the very low sentence offer, it's suggested that the defendant is coerced into making that choice. But not all such are coercive. A prosecutor who shows inordinate leniency in a case may create a sentencing gap sufficient to qualify as ''hard dealing."\footnote{} But to the extent that the prosecutor offers a more favourable choice than the defendant might normally be entitled to expect, it can't be reasonably said that the deal was coercive.

The last position that Young considers and endorses is one that primarily judges coercion on whether an agent acted improperly to interfere with rights or freedoms to which a person was entitled. Under this view, coercion requires more than mere circumstances to be made out. It requires an agent to act improperly. Although Young favours this position, he does not develop it very far, save to say that it is the quality of the influence that must be evaluated for coercion, and not the quantity.\footnote{Young at 266.} Young's preferred view of coercion requires understanding the relational dynamic between the parties involved, but he does not elaborate any further on how the position of a prosecutor specifically may be used to induce criminal defendants coercively. His position is nonetheless defensible. It accounts for deficiencies in the other two positions and lays a reasonable foundation for properly framing discussions on coercion in plea bargaining.

In Canada, prosecutors are responsible for laying that the police and other peace officers submit to them, and have the power to withdraw or stay proceedings on them. Where there is a reasonable prospect of securing a conviction and it is in the public's interest, prosecutors may formally pursue charges. Prosecutions should discontinue where one or both of those conditions do not obtain. On the other hand, defendants have no control over whether charges are laid or pursued. However, defendants have the sole power to decide whether their charges proceed to a contested trial, where the prosecutor must prove them beyond a reasonable doubt.

When criminal defendants are charged with an offence, the police, prosecutors, or both may decide that they should be detained pending the outcome of their cases. Where courts agree, defendants may be remanded into custody pending the outcome of their case. In other cases, defendants may be released on conditions pending the outcome of their case. These conditions can be very strict for defendants with lengthy criminal records or charged with serious offences.

Although there is some balance to the power between prosecutors and defendants, prosecutors are generally at a clear advantage. Unlike defendants, they are not personally liable and face no jeopardy if the case does not go their way. They have no pre-trial conditions or detention orders to comply with and are under minimal comparative pressure to resolve their cases quickly. Within this context, it is easy for high-stakes decisions with no rational alternatives to become coercive. This power dynamic should be explicit whenever individual interactions between prosecutors and defendants are evaluated for coercion concerns.

\subsection{Truth}

It goes without saying that factually innocent people should not be convicted of crimes they did not commit. Wrongful convictions are unfair and undermine the public's confidence in the justice system. Where the crimes involve victims, wrongful convictions leave the factually guilty unpunished and undeterred. Although wrongful convictions are a problem in jurisdictions like Canada, where \textit{nolo contendere} pleas are formally prohibited, it is reasonable to fear that allowing defendants to be convicted of crimes without them admitting guilt or having guilt proven will result in additional wrongful convictions. Furthermore, where plea bargaining is allowed, some factually innocent defendants will be inclined to self-convict in exchange for a favourable resolution to their case. Because these self-convicting defendants are factually innocent, they are wrongfully convicted while the factually guilty parties are left unpunished. Such results are not truthful in that they inaccurately assess responsibility.

Wrongful convictions occur when the justice system fails to produce truthful results. Some plea bargaining opponents believe that plea bargains increase the rate of wrongful convictions. Assuming this is true for the argument's sake, plea bargaining proponents must demonstrate that plea bargaining somehow offsets this negative effect. In this section, I undertake this task by looking at how wrongful convictions may be quantified and how their inevitability may be qualified. I next examine trials as a truth-finding process and note their limitations. Finally, I consider Josh Bowers' argument in ``Punishing the Innocent" that endorses wrongful convictions as ``legal fiction," and advocate for my own position.

\begin{itemize}
    \item \textbf{Quantifying wrongful convictions and qualifying their inevitability.} Wrongful convictions are inevitable. The issue is not whether society can tolerate a dispositive forensic procedure that causes wrongful convictions but rather whether society can tolerate the rate at which it does so.
    \item \textbf{The case against trials.} Trials involve many compromises between prosecutors, defendants, and the decision-makers hearing their cases. Many of these compromises limit what may or may not be adduced as evidence, constraining the trial's utility as a truth-finding process.
    \item \textbf{Moral case for wrongful convictions.} The process costs for many innocent defendants who set trials are often much higher than for similarly situated innocent defendants who self-convict. Wrongful convictions can help end wrongful pre-trial punishments and should be encouraged.
    
\end{itemize}

\subsubsection{Quantifying wrongful convictions and qualifying their inevitability}

Assuming that a system that allows plea bargaining will result in more wrongful convictions than an identical system without it, is plea bargaining defensible? Simply asking this question appears to miss a greater moral point. Wrongful convictions have been called a ``scourge," ``a blight to the criminal justice system,"\footnote{} and ``one of the worst nightmares imaginable."\footnote{See Huff at 15.} Even strong plea bargaining proponents recognize that wrongful convictions are a ``failure" of our justice system.\footnote{See Bowers at 1119.} Plea bargaining opponents argue that wrongful convictions must therefore be ``prevented or eliminated as far as possible"\footnote{See Sheehy at 984.} to maintain confidence in the justice system.\footnote{See Bibas at 1381.} Although wrongful convictions are a problem that should be avoided where possible, these concerns are greatly overstated and largely ill-considered.

The mere fact that plea bargaining causes additional wrongful convictions is not enough to render it indefensible, regardless of how seriously one views wrongful convictions. For example, a system that allowed plea bargaining only resulted in one more wrongful conviction in every 1,000,000 cases than an identical one that did not allow plea bargaining would likely be tolerable for most. Conversely, a system that resulted in one more wrongful conviction in every five cases would likely be intolerable for most. The mere fact that using a procedure results in \textit{some number} of additional wrongful convictions is not enough to render that procedure indefensible. 

This is underscored by the fact that wrongful convictions are an inevitable part of the Canadian justice system, as it is built on the foundation of proof beyond a reasonable doubt. Although wrongful convictions are inevitable in a system that allows plea bargaining, this is tempered by the fact that wrongful convictions are inevitable in our system of justice more generally. Judges must remind juries at every criminal trial that proof beyond a reasonable doubt is not the same as proof to a certainty. Trials are time-limited enterprises where the judge or jury hears each party's best interpretation of the evidence admitted. The criminal conviction appeal system is premised on the fact that judges and juries sometimes make mistakes, while the summary conviction appeal courts and the Supreme Court of Canada are premised on the fact that appellate courts do the same. Requiring judges and juries to be \textit{certain} that a defendant was guilty may drastically reduce the number of wrongful convictions but would likely drastically reduce the number of all convictions alongside it. After all, while Blackstone's cherished ratio reminds us that it is better for ten guilty people to walk free than for one innocent to be wrongfully punished, it tells us nothing about whether it is better for a thousand guilty people to walk free, or a hundred.\footnote{Or even eleven.}

Young argues a similar point when he posits that wrongful convictions resulting from plea bargained deals should be compared to wrongful convictions resulting from jury trials to determine whether there is a meaningful difference between the two. Although Young neither supplies the data nor does the math, he nonetheless points out that plea bargaining can only be meaningfully said to result in too many wrongful convictions if it results in more wrongful convictions than the acceptable alternative.\footnote{See Young at 258.} 

Where a forensic procedure causes wrongful convictions, it should be evaluated on how many it causes and at what rate, not merely on the fact that it does so. I cannot definitively answer whether plea bargained self-convictions cause more wrongful convictions than trials in Canada or elsewhere. Judges and juries make mistakes and are not expected to perform their duties perfectly.\footnote{} Wrongful convictions inevitably result. As I discuss in the next section, the trial's truth problem runs deeper than the standard of proof.

\subsubsection{The apocryphal truth-finding function of trials}

While both plea bargaining opponents and proponents generally view criminal jury trials as the gold standard for delivering due process and ensuring that the truth wins out, trials are better understood as a series of intricate compromises between prosecutors, defendants, and decision-makers. In trials, truth is only one of many critical competing interests that must be balanced against others. Trials are an adequate failsafe for when good faith discussions and negotiations between prosecutors and defendants are unsuccessful or otherwise impossible, but should not be idealized as quests for the truth.

Trials are idealized as an effective ``search for the truth." Where the parties cannot agree, each has the opportunity to present its best evidence and its best interpretation of the evidence. Witnesses are cross-examined on their evidence, identifying the strengths and weaknesses of their testimony, and a neutral trier hears the facts, listens to both sides argue, and decides between them. By contrast, \textit{nolo contendere} pleas appear to be untrustworthy, in that they appear to obfuscate or misrepresent the truth to push hastily negotiated agreements through the courts. Negotiations between prosecutors and defence lawyers are privileged and inadmissible as evidence. There are no judges or juries to provide neutral adjudication,\footnote{This is not necessarily true in all cases. In many jurisdictions, contested matters may go through case management with a judge to discuss pre-trial issues, canvass resolution, and offer advice on how to best proceed.} and the public may never know what considerations went into a negotiated plea deal. 

Notwithstanding this image problem, plea negotiations between prosecutors, defendants, and their lawyers are a key part of the criminal justice system's truth-seeking function. Both parties study the same set of allegations with substantially similar information, find an interpretation of those allegations that is aligned with their interests, and use those positions to argue for admissions, concessions, and so forth. Both parties can examine their cases in detail, speak with the parties involved, forecast likely results should the matter go to a contested hearing, and try to reach a mutually acceptable compromise. In some cases, defendants will self-convict while prosecution may be discontinued in other cases. Where successful, plea negotiations reach a resolution that both parties are satisfied with. Where unsuccessful, the matter is set for trial.

Unlike plea negotiations, trials require the fact finder to consider a less complete and less true version of the evidence than the prosecution and defence have during negotiations. At trial, the truth is offset by fairness, efficiency, relevance, and the prejudicial effect that evidence might have, as well as trial strategies, probability considerations, and the game theories that go into deciding what evidence will or will not be adduced. Unfairly gathered evidence may be excluded, testimony from uncalled or missing witnesses will not be heard, hearsay rules will stop witnesses from repeating what they heard, and opinion evidence rules will prevent most witnesses from telling the court what they think, \textit{regardless of whether it is true}. The evidence that is admitted may be subject to special instructions directing judges and juries how to interpret it properly.\footnote{Testimony from co-defendants, or evidence about a complainant's past sexual history, if admitted, are guaranteed to be subject to detailed instructions warning the fact finder how this evidence can and cannot be used.} Judges and juries possess no special powers to decide cases. As a result, lawyers must manage their witnesses, curate their questions, and account for human frailties and weaknesses to most effectively give the decision maker their case theory, not to ensure a balanced presentation of the truth. The primary function of a trial is not to ensure that the truth is heard, but rather to ensure that the judge or jury can make a \textit{fair decision} in a case. 

Where a resolution cannot otherwise be reached, trials are a reasonably effective means of adjudicating disputes, ensuring all sides have their best explanation of the evidence heard, and deferring to an ostensibly neutral third party to decide between competing narratives. However, trials are not primarily a search for the truth, and should not be idealized as such. Rather, they are an adequate truth-finding substitution when other methods, like bargaining, have failed. Negotiations between counsel, both of whom have legal training, are duty-bound to advocate for their positions, have access to the fullest permissible truth of a criminal matter, and the opportunity to spend extensive time and resources investigating the nuances thereof, are a much greater truth-seeking device than a criminal trial.

\subsubsection{The moral case \textit{for} wrongful convictions}

In his article ``Punishing the Innocent", Josh Bowers acknowledges that wrongful convictions are a ``failure" of the justice system.\footnote{Bowers at 1119.} Rather than laying all the blame for wrongful convictions on the uncontested pleas these defendants enter, however, Bowers traces the source of this failure further back to the wrongful arrest, charge, pre-trial process costs and prosecution of the allegation. Doing so allows him to argue that wrongful convictions may be a ``categorical good" for the wrongfully incarcerated that should be made more readily available to such defendants.

Outline Bowers' position.

If most wrongfully convicted defendants are recidivists with limited due process expectations, this does not justify their wrongful convictions. This proposition merely goes to explain why these wrongful convictions result. First-time defendants both have a great deal to risk in even low-stakes criminal prosecutions and are much less likely to be subject to stringent pre-trial conditions or detained. They thus have more incentive to contest their charges, and less incentive to resolve them. Recidivists have less to risk from a criminal conviction than first-time defendants, and are more likely to be subject to pre-trial detention or stringent release conditions. They thus have less incentive to contest their charges and have more incentive to resolve them.

Bowers emphasizes that pre-trial detention and release conditions should be viewed as a form of punishment. Innocent defendants who have few incentives to contest their charges may be very amenable to wrongfully self-convict if they are offered a chance to mitigate this punishment meaningfully. Unfortunately, these conditions are oftentimes necessary to ensure that criminal matters proceed to trial. Defendants who fail to show up for court dates or who commit additional offences while out on release conditions, or who have done so in the past may be fairly labelled a flight or public safety risk and detained. No-contact conditions between domestic violence parties may be imposed to protect the complainant but may be frequently breached with impunity by either or both parties. Although imposing restrictive conditions on a defendant will reasonably increase their motivation to resolve, these conditions are nonetheless necessary to ensure the criminal justice system functions. To the extent that recidivists raise more concerns that they will re-offend or fail to appear for court, they are more likely to be detained pre-trial and therefore more likely to be motivated to resolve their charges. 

Although all wrongful convictions are evidence that the justice system has obtained inaccurate results, some wrongful convictions are worse than others. Being wrongfully convicted of a minor offence with little to no possibility of jail is much less serious than being convicted of a major offence with a mandatory minimum. Similarly, a lengthy pre-trial detention period is generally more punitive than a shorter one, while less-stringent release conditions are less punitive than more-stringent ones. Therefore, some pre-trial punishments are also worse than others. 

It follows that in some circumstances, being wrongfully convicted of some offences may be better than being wrongfully punished for them. For example, a defendant wrongfully convicted of a minor assault charged and given an absolute discharge is arguably better positioned than one denied bail, spends a few months in pre-trial detention, and ultimately be acquitted of the same offence. Risk-averse defendants or those not looking to clear their names are unlikely to want to wait for a trial if they can otherwise resolve their charges on less punitive terms. 

Subjecting a wrongfully accused defendant to a greater pre-trial punishment they would receive if convicted after a trial is unjust. Allowing them to self-convict when they have no other legal recourse to put an end to their wrongful punishment both recognizes that the problem of wrongfully punishsed and incarcerated defendants exist and that a self-conviction of convenience is a functional stopgap solution for those who do not expect extensive due process, are highly motivated to resolve, and are likely to receive generous offers to do. Although wrongful convictions of this sort are ``untruthful," it is less important to secure a ``truthful" plea in the circumstances than it is to put an end to the defendant's wrongful punishment.

\subsection{Substance}

In his article ''Harmonizing substantive-criminal-law values and criminal procedure: The case of Alford and nolo contendere pleas," Stephanos Bibas argues that plea bargained \textit{nolo contendere} and \textit{Alford} pleas detract from substantive legal values in exchange for procedural efficiencies and that it is therefore a mistake to allow them to prevail at the expense of substantive criminal values. Substantive criminal values are the moral underpinnings of the criminal justice system. Denouncing and deterring criminal behaviour, reintegrating offenders into society, and giving them a personal sense of accountability are substantive criminal values, as are the other sentencing principles codified in \textit{Criminal Code} ss 718 — 718.2.

While Bibas shares the wrongful conviction concern, he equally worries that focusing on procedural efficiencies and ``economic" plea deal maximization undermines society's chance to educate offenders while denouncing their conduct. Even if \textit{nolo contendere} pleas never resulted in wrongful convictions, Bibas argues that they would continue to undermine substantive legal principles and therefore remain problematic. He expresses three broad concerns in support of this position:

\begin{enumerate}
    \item \textbf{Justifications for punishment and the law's moral norms.} Punishing defendants is the moral business of the criminal justice system. The criminal justice system is founded on the belief that offences should be denounced and offenders should be educated and reformed. Plea bargaining and \textit{nolo contendere} pleas disregard these norms in the name of procedural efficiencies and individual choice.
    \item \textbf{Reluctance to confess and the value of confession.} Plea bargains may be motivated by a reluctance to confess one's misdeeds. Confession is good for defendants, victims, and society as a whole, and should be encouraged. To the extent that plea bargaining and \textit{nolo contendere} pleas allow defendants to avoid confession, they should be discouraged.
    \item \textbf{The substantive value of trials and the harm of guilty-but-not-guilty pleas.} Trials are not merely a way to determine a defendant's guilt or innocence. Rather, they are ``morality plays" that denounce unlawful conduct and place society's values on display. Plea bargained deals keep these discussions from the public, while non-committal pleas like \textit{nolo contendere} leave society with the impression that justice does not demand finality.
\end{enumerate}

Bibas highlights the substantive harms that \textit{nolo contendere} pleas can cause and the substantive benefits of trials. However, he fails to touch upon the real substantive benefits \textit{nolo contendere} entails while ignoring the substantive harms that trials cause. To the extent that both allowing and permitting \textit{nolo contendere} pleas entail different substantive law benefits, the substantive law objection is less an objection in principle and more an objection rooted in a preference for competing substantive values. These values include giving priority to legitimate disputes and discouraging specious ones, ... 

\subsubsection{Justifications for punishment and the law's moral norms}

For Bibas, criminal law is not simply a means for meting out punishments to counteract crimes but also a way to achieve retributive, denunciatory, and rehabilitative ends. Bibas's ``theory of punishment as moral education"\footnote{See Bibas at 1390.} argues that properly constructed criminal sanctions can develop an offender's sense of guilt and foster a sense of responsibility, thereby increasing the offender's chance to rehabilitate and decreasing the risk they pose to society. Values like retribution and rehabilitation are considered alongside harm reduction principles, such that victims are vindicated by seeing their perpetrators take responsibility for their actions and receive their due punishment, and society is generally deterred from committing such crimes for fear that they are similarly corrected. In contrast to this, pleas like \textit{nolo contendere} allow defendants to avoid taking responsibility for their offences and thwart these substantive values.

Bibas highlights an important problem, but his response is uncompelling. As argued above, criminal pleas are primarily matters of proof, not fact or belief. There is little reason to doubt that a defendant who sincerely pleads guilty is generally a better candidate for rehabilitation than one who enters a stolid \textit{nolo contender plea}. 

There is little to suggest that a defendant who is required to set a trial is a better candidate for rehabilitation than one who self-convicts via \textit{nolo contendere}. Rehabilitation will always reasonably favour a receptive participant, but Bibas has not argued or pointed to compelling evidence that preventing ambivalent pleas or restricting plea bargaining are correlated to an increase in such defendants.\footnote{Bibas points to some evidence, but it is not compelling.} Unless restricting these pleas increases the substantive good of contrite defendants, it would be hasty to conclude that they should be restricted for this reason.

Plea bargaining proponents advocate for the process because it provides immense returns in procedural efficiency. Some proponents, including the Supreme Court of the United States, the Supreme Court of Canada have referred to plea bargaining as an essential feature of our criminal justice system.\footnote{Cite this.} Bibas identifies these as procedural legal concerns but incorrectly distinguishes them from substantive legal concerns. Procedural efficiencies are normative goods that bring their own substantive legal benefits with them when implemented. 

Plea bargaining opponents generally believe that defendants who receive a trial receive superior due process compared to defendants who negotiate an early plea deal. The problems with this belief begin in the misperception that trials are effective truth-finding devices, and are compounded by the plea bargaining opponent's failure to see that negotiated pleas are substantive goods that ought to be pursued. 

The time and expense required to schedule and run a trial are prohibitive and do not scale well alongside due process. While a single judge in a single courtroom can accommodate dozens of resolved cases in a day, that same judge is only likely able to hear one or two very brief trials in the same amount of time. Trials with more witnesses or complicated legal issues may require additional days and weeks for the case to be fully heard. Any significant increase in the number of trials being set, such as would be reasonably expected in a system where plea bargaining was removed, would necessarily result in a backlog. Relieving this backlog would require increasing ``court resources," such as judges, lawyers, court staff, correctional staff, facilities and logistics. Failing to relieve this backlog interferes with a defendant's procedural right to be tried in a reasonable time and with the substantive values of having a trial heard while recollections are fresh and the community still feels the offence's impact. The reverse is true when more unnecessary trials are stopped. 

\subsubsection{Reluctance to confess and the value of confession}

Bibas next discusses the substantive value of having defendants publicly confess their crimes. He draws upon examples of defendants whose personal epiphanies in the face of their guilt brought them to confess their guilty honestly,\footnote{See Bibas at 1364. It is unknown whether this particular confession assisted the defendant's rehabilitation.} contrasting them with defendants who avoid these self-confrontations. Bibas specifically highlights those classes of cases where defendants are normally reluctant to confess, such as sexual crimes, as prime examples of cases where defendants \textit{should} be compelled to plead guilty and not permitted to enter \textit{nolo contendere} pleas.\footnote{Bibas does not mention the rehabilitative prospects of defendants who are convicted after trial.} His position presumes that sincere confessions should be pursued over certain convictions and that \textit{nolo contendere} do not bring their own substantive values with them. I disagree with both of those conclusions.

Bibas correctly notes that a sincere guilty plea is preferable to a non-committal \textit{nolo contendere} plea. He does not, however, ask whether it is better to have more non-committal self-convictions and fewer trials or to have fewer non-committal self-convictions and more trials. Although it would seem evident that plea bargaining advocates would be enthusiastic about setting these matters down for trial and allowing the morality play its due course, this approach is capricious. It fails to appreciate that trials are unpredictable affairs where witnesses fail to show up, the evidence does not come out as expected or is excluded altogether, judges and juries have difficulty with certain facts or concepts, and the law and evidence can be misunderstood and misapplied. Assuming that a factually guilty defendant who is convicted after a trial is unlikely to be more remorseful than a factually guilty defendant who pleads \textit{nolo contendere}, risking certain conviction and resolution to a case in exchange for a merely possible and similarly remorseless conviction after a trial is dubious. 

Finally, while Bibas correctly notes that \textit{nolo contendere} pleas identify ideological and moral differences between defendants and the state, he hastily concludes that this is a problem. He neglects to identify the opportunity these pleas allow for defendants to acknowledge the authority of the state, notwithstanding their private differences, and the opportunity for the state to gauge rehabilitation through more accurate information. Although it would be preferable for defendants to sincerely repent of their ways, accede to the state's wisdom, and go and sin no more, this is never a guaranteed outcome in any case. More importantly, that outcome is not necessary for justice to be done. Rather, the interests of justice are better advanced by showing the public that justice prevails over private opinion and morality. A defendant's prospects for rehabilitation can be better understood when this factual interpretation and belief differences are explicated, advancing the substantive values of reconciliation, mutual responsibility, and accuracy.\footnote{For example, three similar defendants are rightfully convicted of the same offence and sentenced to a year of probation. After the year of probation, each appears to be similarly rehabilitated. However, one defendant was convicted after a trial where they protested their innocence, another pleaded \textit{nolo contendere} and privately protested their innocence, and the other sincerely pleaded guilty. The defendants who protested their innocence but were nonetheless guilty gained more of a sense of responsibility for their offence than the one who pleaded guilty and were thus arguably more rehabilitated.} Absent evidence that defendants who plead \textit{nolo contendere} are less likely to be rehabilitated than defendants who set unwanted trials or enter insincere guilty pleas, they should be encouraged.\footnote{Throughout the article, Bibas relies on informal surveys and a handful of limited-scale studies as evidence that he believes support his approach to moral criminal punishment. See Bibas at 1395 - 1398. These include anecdotal accounts from judges and lawyers Bibas knows and a study of recidivism rates among eight offenders in one state who entered \textit{Alford} pleas. Without belabouring the point, I do not agree that these studies have the same impact and bearing that Bibas does.} 

\subsubsection{The substantive value of trials and the harm of guilty-but-not-guilty pleas}

For Bibas, the jury trial is the law's solution for defendants who cannot or will not plead guilty or for defendants who are factually innocent.\footnote{See Bibas at 1400.} Bibas envisions trials as ``morality plays" that ``dramatically present the conflicting moral values of a community in a way that could not be done by logical formalization."\footnote{See Bibas at 1401.} Through them, the court confronts defendants with ``solemn pronouncements of guilt" and condemns them, in the hopes of ``break[ing] through the defendant's denial mechanisms, driving home in undeniable detail the wrongfulness of the crime."\footnote{See Bibas at 1401.} Allowing defendants to enter \textit{nolo contendere} pleas sends the message that justice does not require defendants to take a clear stance and should therefore be discouraged.\footnote{See section XXX above, where I argue that pleas are not statements of fact or belief but are rather statements about formal proof. Bibas' objection fundamentally misunderstands this point by conflating pleas into fact/belief statements rather than proof statements.} Instead, Bibas believes defence lawyers should challenge their clients when they deny that they are guilty and encourage them to consider the long-term advantages of entering sincere guilty pleas.\footnote{See Bibas at 1404f. It is interesting to note the similarities between what Bibas recommends and what Alford's lawyer encouraged him to do in the \textit{Alford} decision. Suffice it to say, I believe that the risks involved in a system where all parties see it as their role to convince a defendant to plead guilty far outweigh any benefits.}

Requiring defendants to set trials when none of the parties involved want one ensures that there is one less trial date available for a truly contested matter. Society does not have a legitimate interest in requiring unwilling and otherwise cooperative defendants, victims, and professional justice system participants to engage in these literal show trials. Defendants who must have their allegations adjudicated, even where they lack viable defences, must spend unnecessary time in pre-trial detention or pre-trial release conditions. At the same time, defendants with a viable defence are similarly delayed and must spend unnecessary time in pre-trial detention or on restrictive release conditions. Defendants who are willing to accept a final judgment in exchange for ending these restrictions should be permitted to do so, and victims should be allowed to see punishment meted out, restitution made, and the process concluded. 