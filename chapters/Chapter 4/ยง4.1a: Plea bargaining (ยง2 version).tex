\section{No-contest pleas and plea bargaining}

Some defendants who self-convict do so for selfless or pro-social reasons, such as a desire to spare a complainant from having to testify at trial or because they are genuinely remorseful for their actions. However, some defendants who are reluctant to admit responsibility may be willing to do so for some consideration.

Plea bargaining relies on the fact that some \textit{quid pro quo} exchange exists in every prosecution. Prosecutors lay new charges and may withdraw existing ones, while defendants control whether to contest the charges and force a trial. 

Both sides face specific pressures that may result in overlapping goals. For example, both prosecutors and defendants are motivated to resolve cases quickly and efficiently. Defendants detained or released on conditions may be motivated to resolve their matters efficiently by a desire to end their pre-trial restrictions. In contrast, prosecutors may be motivated to resolve their matters efficiently by the constant threat of a successful delay motion and a judicial stay of proceedings.


\subsection{Fairness and accuracy: opposition to plea bargaining}

\begin{itemize}
\item \textbf{Coerced pleas.} The deals that prosecutors offer defendants, coupled with the explicit and implicit consequences of refusing to accept those deals, amount to coercion and result in involuntary self-convictions.
\end{itemize}

\subsubsection{Coerced pleas}

Plea bargaining opponents worry that the plea bargaining system is such that defendants are indirectly induced or threatened into pleading guilty. Although not all inducements and threats, or other oppressive conditions, are sufficient to override a defendant's will. The inducements that naturally accompany criminal charges appear oppressive in that the balance of power skews heavily towards the prosecution, and that involuntary self-convictions appear inevitable. 

\subsubsection{Adverse sentencing results}

Sentencing difficulties can arise when \textit{being sentenced} and \textit{serving a sentence}. 

By accepting early resolution offers, defendants forfeit the ability to receive more nuanced disclosure or investigate possible defences that may not be apparent at the early stages of disclosure.

Those who self-convict but maintain their innocence, either privately or publicly, may be unable to comply with court-ordered conditions and liable to be criminally charged.\footnote{See the YOA case from Ontario, discussed in Chapter 3.} Defendants who insist on their innocence may be limited in other respects, finding themselves unable to access certain programming while in custody or ineligible for early release on parole. 

\subsubsection{Wrongful convictions}

Wrongful convictions are a problem that our modern legal system has been designed to avoid.

\subsection{Efficiency and autonomy: plea bargaining proponents}

Plea bargaining proponents focus on its uses in the efficient administration of justice, the benefits that defendants who are able to secure a plea bargain receive, and the advantages of the certainty of a conviction without the hardship and uncertainty of a trial:

\begin{itemize}
\item \textbf{Certainty.} Certainty affects everyone tasked with the effective and consistent administration of justice.
\end{itemize}

\subsubsection{Increased procedural efficiencies}

The practice of plea bargaining has benefited greatly from its perceived necessity, especially amongst members of the bar and the judiciary.

The Supreme Courts in Canada and the United States have endorsed plea bargaining because of its important function in our justice system. In \textit{Santobello v New York}, 404 US 257 (1971), the Supreme Court of the United States affirmed that the plea bargaining process is a crucial part of the criminal justice process whose agreements should be upheld. In Canada, 25 years later, the Supreme Court of Canada followed suit in \textit{Burlingham},\footnote{\textit{R v Burlingham}, 1995 CanLII 88 (SCC), [1995] 2 SCR 206 (\textit{Burlingham}).} identifying plea bargaining as ``an integral element of the Canadian criminal process.'' Although \textit{Criminal Code} s 606(1.1) ensures that Canadian judges are not required to uphold plea agreements, the Supreme Court of Canada has made it clear that judges must defer to truly jointly recommended sentences.\footnote{See \textit{R v Anthony-Cook}, 2016 SCC 43, 2016] 2 SCR 204}

Plea bargaining works to reduce trials without a viable or likely defence. Defendants who are facing a significant sentence are highly motivated to set their matters down for trial, even where they have no obvious defence to the charges. The possibility of a procedural irregularity, a change in a witness's willingness to cooperate, or a favourable evidentiary ruling are reasons to set a matter down for trial. Offering that defendant the option of a less significant sentence before trial reduces that incentive.

Encouraging defendants to resolve these lightly and functionally uncontested matters helps triage cases early, leads to less inefficient last-minute trial resolutions, and allows the courts to continue functioning with the resources they've been allotted. 

\subsubsection{Increased autonomy for defendants}

When a person becomes a criminal defendant, their ability to meaningfully control their future is curtailed. From being placed on release conditions to being detained in custody before disposition, the state assumes a great deal of control over the lives of those it prosecutes. The main choice defendants have is how to dispose of their charges. Exercising this discretion, defendants may elect which court their charges will appear in for indictable matters, what charges they will admit and what charges they will require the state to prove, and whether they will testify or call other evidence at a contested hearing. Because even seemingly strong prosecutions can falter when the state is put to its burden, defendants have some power to bargain for a favourable deal in exchange for resolution. An increased degree of autonomy is a good worth pursuing, as it stands to reason that defendants with this added measure of control over their futures are more likely to view the justice system favourably and believe they'd been treated fairly than those without such options.

\subsubsection{Certainty in facts and outcome for all}

At a high-level, trials are fairly described as zero-sum games with clear winners and losers. Where both sides believe they have an arguable case, it is reasonable for each to expect that they will be successful. When one party inevitably loses their case, they encounter an outcome they weren't expecting. Both parties face uncertainty concerning many aspects of their case at lower levels of abstraction. A witness may testify in a way that benefits or damages either side of an argument or may decide not to show up for the trial at all. Opposing counsel may find an old article an expert witness published (and forgot about) where they take a different view from the one they've testified to in the case. Even where the state makes out its case, rulings on objections and evidentiary motions can significantly impact which facts the decision maker accepts as proven. At every level, trials are rife with uncertainty.

By contrast, the plea bargaining process is geared towards achieving as much certainty as possible in an outcome before pleas being entered. Everything from the allegations that the prosecutor will read into the record to the range of sentences to be recommended may be discussed and decided. Discussions between parties may last weeks or months, during which time both may review the evidence they have, request additional disclosure, review pertinent evidentiary and sentencing decisions. Where the parties jointly recommend a sentence that was the product of a true plea bargain, judges must show it significant deference. Comprehensively compiled plea deals give all parties more certainty of outcome than trials.

\subsection{Accuracy, collaboration, and ethical efficiency: the approach in this thesis}

\begin{itemize}
\item \textbf{Accurate bargaining.} Plea bargaining achieves accuracy and fairness, neither of which needs to be traded off for efficiency and autonomy.
\item \textbf{Win-win collaboration.} For both defendants and prosecutors, certainty in the outcome of a criminal case is a good that is worth pursuing and fostering. Both should be encouraged to pursue mutually beneficial ``win-win'' solutions to criminal problems.
\item \textbf{Wrongful punishment.} Wrongful convictions should be assessed against wrongful punishment and the coercive effects of the criminal justice system.
\item \textbf{Overlooked substantive benefits.} Plea bargaining encourages an environment of reasonable compromise, cooperation, and responsibility.
\end{itemize}

\subsubsection{Plea bargaining can be easily adapted to better ensure accurate results}

Good faith and open discussions between prosecutors and defence counsel create an ideal and controlled environment for understanding a criminal allegation that explicates triable issues and accurately gauges the respective strength of the other's position. By collaborating early and frequently testing the other's positions, resolution discussions become more pointed and trials, where necessary, become more focused. 

Trials are a time-limited and carefully coordinated presentation of admissible evidence, wherein two diametrically opposed parties try to frame the evidence to convince a third party. Plea bargains, on the other hand, are reached between parties with access to the disclosed evidence ample time to go through it. Defendants have far fewer disclosure obligations, which gives them an advantage throughout this process. While critics characterize these privileged discussions as ``backroom deals," these protections allow both parties to openly assess their positions. During plea bargaining, those who know the case best help control how it resolves.

\subsubsection{Where possible, collaborative win-win scenarios should be encouraged in criminal proceedings}

Plea bargaining gives prosecutors and defendants ways to examine their respective positions and determine if they can without a trial. Contingencies in each other's cases may be leveraged to place the party in a better bargaining position. Where both parties each have something to lose by taking the case to trial, and something to gain by resolving it on mutually agreed-upon terms, a ``true plea bargain'' between the parties is reached. 

\subsubsection{Wrongful pre-trial restrictions are analogs of wrongful convictions}

Where these conditions are especially stringent, or where a defendant has been remanded into custody, pre-trial restrictions may be virtually identical to the sentence they could reasonably expect to receive upon pleading guilty. While restricting plea bargaining may eliminate some of the pressures that induce innocent defendants to plead guilty, it would do little to address the wrongful pre-trial ``punishments'' imposed by their remand or release conditions. Removing a defendant's ability to quickly and meaningfully resolve their case on favourable terms may even increase a defendant's wrongful suffering, under the guise of trying to relieve it.

\subsubsection{Efficient and co-operative proceedings provide defendants substantive rights and goods}

Efficiency is often cast as a good that primarily benefits society as a whole, at the expense of some of a defendant's procedural rights. But not all efficiencies are detrimental to defendants. 

Speedy trials limit the time defendants spend in coercive pre-trial custody or on similarly coercive and restrictive release conditions. Defendants applying for interim release on a reverse-onus bail have less cause to show for a brief pre-trial release period than a lengthy one. Effective pre-trial plea bargaining and cooperative case review helps triage matters that are unlikely to resolve from those that are, and allows counsel adequate access to timely trial dates as needed. Providing judges and juries with enough time to hear the issues in a legitimate dispute provides defendants and society with the substantive good of a more fair and fulsome presentation of the case on its merits.