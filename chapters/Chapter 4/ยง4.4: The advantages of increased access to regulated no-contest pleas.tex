\section{The advantages of expanding access to no-contest pleas}

If one accepts the premise that plea bargaining is a substantive and normative good or that plea bargaining may be such in certain cases, a presumptive case is made for expanding plea bargaining's scope. For defendants who cannot or will not plead guilty but are nevertheless willing to self-convict, expanding the list of permitted pleas to include both non-culpatory and exculpatory no-contest pleas accomplishes this task. For prosecutors, witnesses, victims, and society generally, the potential gains that may be won at a contested trial will not usually be worth the risk when compared with a confirmed conviction through a no-contest plea.\footnote{To the extent that either the defendant or the state may have the facts of the offence come out more favourably for them at trial, neither party has any implicit advantage over the other.}

The following are key advantages of expanding access to no-contest pleas:

\begin{itemize}
    \item \textbf{Ensuring certainty in outcomes for all.}
    \item \textbf{Delegating more control over the evidence admitted and proven to prosecutors and defendants.}
    \item \textbf{Increasing agency and self-determination for defendants.}
    \item \textbf{Transforming zero-sum prosecutions into win-win outcomes.}
    \item \textbf{Solidifying procedural protections for defendants and their lawyers.}
    \item \textbf{Rectifying the loopholes and loose ends created by informal pleas.}
\end{itemize}

\subsection{Increased certainty in the outcome}

\subsubsection{Plea bargained cases generally}

Because prosecutions can be fraught with uncertainty for all parties involved, having a measure of certainty about how a case will resolve motivates all plea bargains. This fact tracks even in cases with defendants who intend to plead guilty. A defendant may wish to plead guilty to some offence charged but may only be prepared to plead to a lesser included offence. Similarly, disputes over a sentence's nature and quantum can be hotly contested, even where defendants are otherwise prepared to plead guilty at their first court appearance. While these questions may have very different answers in a contested hearing, parties who agree to these matters in advance are likely to have their expectations met.\footnote{See \textit{R v Anthony-Cook}, 2016 SCC 43 (CanLII), [2016] 2 SCR 204. While jointly-recommended sentencing proposals are subject to judicial discretion and review, judges are generally precluded from rejecting them. Even in cases where prosecutors and defendants cannot come up with a joint proposal or where the requisite \textit{quid pro quo} for a joint proposal is not met, judges remain similarly constrained by the parties' sentences. See \textit{R v Beardy}, 2014 MBCA 23 at para 6; \textit{R v Grant}, 2016 ONCA 639 at paras 163 - 165} Defendants and prosecutors who pre-negotiate pleas and sentences greatly increase their certainty in the outcome.

\subsubsection{Non-inculpatory no-contest cases specifically}

Under the current Canadian legislative scheme, defendants who cannot or will not admit that they are factually responsible for the offence they are charged with must set their matters down for trial. As discussed above, trials are risky enterprises fraught with dangers for defendants charged with the underlying crimes and the prosecutors carrying their cases. Prosecutors always bear the burden of proving criminal charges beyond a reasonable doubt, and even defendants without a positive defence or cohesive response to the allegations against them may prevail at trial. Forcing defendants to contest charges they are otherwise prepared to admit as proven risks wrongful acquittals. Defendants charged with multiple offences may be offered the opportunity to self-convict on some in exchange for the others being dropped. However, when those same defendants must contest all of those offences to trial, they risk being convicted of more than they could have otherwise bargained for.

\subsection{Increased control over the evidence admitted and proven}

\subsubsection{Plea bargained cases generally}

In addition to giving defendants and prosecutors a high degree of certainty in the outcome of a case, negotiated no-contest pleas give all parties to a prosecution an increased level of control over the evidence admitted. The prosecutor must prove every disputed fact on a balance of probabilities and must prove every disputed aggravating fact beyond a reasonable doubt.\footnote{See \textit{Criminal Code} s 724(3)(d) \& (e).} This requirement gives defendants a considerable bargaining advantage, such that they can spare prosecutors the trouble of proving their case generally and any particular allegation specifically in exchange for some consideration on sentencing. In turn, prosecutors can agree to exclude certain embarrassing or otherwise prejudicial aspects of an allegation in exchange for some concession on the plea entered or the sentence agreed to. Rather than leaving these determinations to a judge or jury hearing the admissible portions of a case for the first time, plea bargaining allows the parties most familiar with the underlying facts to navigate these complex decisions together. As with certainty of outcome, the uncertainty inherent to the trial process causes concern for how facts about the offence and offender are ultimately received.

\subsubsection{Non-inculpatory no-contest cases specifically}

Just as prosecutors and defendants forced to go to trial lose control over which offences they are convicted of, they also lose control over how the allegations are ultimately received. Prosecutors must risk an acquittal, while defendants must risk the possibility that the evidence comes out unfavourably for them. Both parties must suffer the consequences of having witnesses testify, while defendants lose any consideration they might have had on sentencing for resolving their case without a trial.

In this respect, it is arguably the prosecutor who benefits more from defendants having access to non-culpatory no-contest pleas. Defendants who plead guilty to an offence may reasonably expect that they are being asked to admit to the truth underlying the allegations. A subset of those defendants will want to ensure that the allegations they are admitting to are, in fact, true and may put the prosecution to its burden of proof where the allegations fall short of what they believe to be the case. Where defendants are not asked to enter inculpatory no-contest pleas and may instead either refuse to admit guilt or actively protest their innocence, it stands to reason that the prosecutor will have greater leeway with what they allege. Where the defendant neither admits nor contests the allegations, there is less impetus, if any, for the prosecutor to have the defendant admit to them. In exchange for this leeway, the defendant can continue to deny their factual guilt in good conscience, contest the allegations in subsequent proceedings, or take advantage of other comparable concessions.

\subsection{Increased agency and self-determination}

\subsubsection{Plea bargained cases generally}

Having some control over the outcome and the facts increases self-determination. Plea bargaining is one of the primary avenues defendants have to exert control over a process that is otherwise entirely out of their hands. Similarly, because plea bargains guarantee convictions, victims of crime who work with prosecutors and victim services agencies can let the court know how the offence impacted them and request protective conditions in appropriate cases. By increasing the ability for defendants to determine their cases and guaranteeing a conviction for victims of crime, both parties are given greater agency over the situation as a whole. It stands to reason that when defendants and other justice system participants have more agency in proceedings otherwise outside their control, each will ultimately have a higher regard for the justice system.

\subsubsection{Non-inculpatory no-contest cases specifically}

Defendants who are allowed to self-convict without admitting responsibility are given a unique chance to repudiate their charges, either formally through a best-interest exculpatory plea or privately following a non-inculpatory \textit{nolo contendere} plea. Where allowed, defendants who enter non-inculpatory no-contest pleas may also repudiate their charges at any subsequent civil proceedings. In every instance, defendants allowed to enter these pleas can remain authentic to a particular moral or ideological standpoint while accepting the inevitable nonetheless. As discussed above, pleas are not propositions that can or should be understood for the truth of their contents. However, the subtle distinctions between propositions expressing a statement of proof and propositions expressing a statement of belief are unlikely to resonate with defendants called to plead to their criminal charges. Allowing defendants to enter pleas whose content accurately reflects their beliefs and dispenses with the need for a formal hearing may reasonably lead them to conclude that the proceedings were truthful and fair. 

\subsection{The elusive win-win scenario in criminal law}

\subsubsection{Plea bargained cases generally}

Common law countries like Canada operate on an adversarial system of justice, where disputing parties present opposing positions to a neutral decision maker who ultimately decides which side will prevail. The adversarial system is zero-sum, as one or more parties must lose in order for the other party to win. In a zero-sum scenario like this, mutually favourable outcomes are effectively precluded. In non-criminal cases, such as family law disputes or civil actions, it is common for matters to resolve through alternative means such as mediation, dispute resolution, or other negotiated settlements. Although these trial alternatives do not always produce outcomes that both parties are satisfied with or find bearable, such outcomes are possible. Furthermore, the parties who attempt these trial alternatives may comport themselves in such a way that they actively work towards a mutually beneficial outcome and intend to achieve one, thereby increasing their likelihood of successfully doing so.

In most criminal cases, plea bargaining is the only means the parties have to achieve a comparable win-win scenario. Defendants rarely wish to plead guilty and accept full responsibility for their charges from the outset of the proceedings, and even those who do rarely fully agree with every detail of every bloviation found in the initial police reports that prosecutors rely upon. While \textit{Criminal Code} s 717 authorizes prosecutors to employ alternative measures for any offence, these alternatives to prosecutions are only available to defendants who agree they were involved with or participated in the offence charged, and are therefore the potential subject-matter of a plea bargain. Plea bargaining opponents must therefore commit to denying criminal defendants the same procedural right to bargain for a win-win that any other litigant would be entitled to for substantially less impactful matters. Denying a criminal defendant the ability to negotiate their second-degree murder charge down to a manslaughter while actively encouraging family litigants to divide their commemorative spoon collection outside of court is an affront to any legitimate sense of justice.

On the other side of the win-win equation, prosecutors, law enforcement, victims, trial witnesses, and society at large stand to gain a great deal from effective case resolution outside of a trial, and from the deals that one side is able to make with another. Confidential informants are a prime example of the wins that these parties can achieve through plea negotiations. While most of the crimes that criminal courts deal with on a day-to-day basis may be fairly called ``unsophisticated," allegations involving drug trafficking and organized crime are very often complex and variegated affairs. Due to the complexity of the drug trade, obtaining sufficient information to find drug traffickers and their suppliers can be difficult, while obtaining sufficient evidence to criminally implicate those most responsible can be herculean. Confidential informants can make these tasks manageable in some cases and remotely possible in others.

Although confidential informants can be invaluable, the pool of viable confidential informants is generally small and typically composed of criminals. Few confidential informants are ``ordinary citizens," and even fewer are motivated by civic duty. While some may be willing to give information in exchange for financial compensation or some other extra-legal consideration, informing on others can be dangerous, and significant consideration is often expected and deserved. For individuals who may be co-defendants in a drug or organized crime prosecution, or for those who may be concomitantly criminally charged with unrelated matters, offering a favourable plea deal may be the only way to secure the needed cooperation. To the extent that some significant portion of the unsophisticated crimes that preoccupy criminal courts are motivated by illicit drug addictions, hampering the illicit drug trade is an important societal ``win-win" objective that plea bargaining is often uniquely able to achieve. 

\subsubsection{Non-inculpatory no-contest cases specifically}

For defendants who are unable or unwilling to plead guilty, but would otherwise self-convict without the need for a trial, expanding the scope of permissible no-contest pleas also expands the range of potential win-win resolutions to include these defendants. Under the present scheme, defendants who will not formally admit culpability must proceed to trial. Where both the defendant and the prosecutor would prefer a confirmed conviction on partially or fully agreed-upon terms, precluding them from doing so runs the risk of creating a ``lose-lose" scenario instead. Some may reasonably point out that defendants are not entitled to enter guilty pleas, or to have those pleas accepted, and reasonably argue that self-convictions should be reserved for those 

Bibas's parochial fantasies notwithstanding, the criminal justice system is not a morality play, and none of its participants are entitled to the wholesome and edifying results he imagines. Prosecutors are not entitled to a perfect conviction, complainants are not entitled to a remorseful defendant, and defendants are not entitled to find their salvation at the end of a sentence.\footnote{See Forsyth at 250 - 251: \begin{quote}
    It is understandable that victims of crime and consequently the prosecutors who are dealing with those victims, may prefer to see and hear an absolutely unqualified admission of guilt proffered by the person who has committed the crime against them or their loved ones. The idea of confessing our sins on an unqualified basis is usually engrained in us from our childhood days with the teachings of our parents, our teachers, our religious leaders, or whatever the case may be. The concept of public expiation has been, historically, somewhat fundamental to the concept of our criminal justice system. As laudable as that objective may be, it may be somewhat counterproductive to demand that a person accused of a criminal offence not be allowed to take a position of not contesting the strength of the evidence for the prosecution without making admissions to acts alleged, thereby allowing the prosecution's evidence to be accepted by the court, unassailed and unsullied. As long as the effect of such a plea of ``no contest" is exactly the same from the standpoint of the consequences for sentencing purposes then the public should be gratified that victims of crime are not required to testify in court, court time itself is not needlessly usurped and society can get on with its business.
\end{quote}} The net gains from allowing recalcitrant defendants the ability to self-convict when they are fully prepared to do so far exceed any perceived losses at not having the defendant acknowledge their moral culpability or truly repent for their misdeeds. Put more simply, a win is a win and should be acknowledged as such.

\subsection{Sealing loopholes and tying up loose ends}

Because Canadian criminal law currently provides for an informal \textit{nolo contendere} procedure, it is reasonable to ask whether a new formal plea is necessary. If the \textit{nolo contendere} plea procedure currently authorized is sufficient to its task and otherwise unproblematic, there may be no need to modify the current regime. However, the \textit{nolo contendere} plea procedure is grossly deficient for a number of reasons, each of which may be rectified through regulation:

\begin{itemize}
    \item \textbf{No judicial discretion to reject non-culpatory no-contest pleas.} Informal \textit{nolo contendere} pleas in Canada are formal not guilty pleas that are subsequently modified to ensure a conviction. The \textit{Criminal Code} gives defendants the ability to plead not guilty by right and to admit any fact against them without prior judicial authorization or subsequent judicial interference. By implication, Canadian judges have no discretion to refuse these pleas, \textit{contra} virtually every other jurisdiction that formally allows them.
    \item \textbf{Access to conviction appeals by right.} 
    \item \textbf{No plea comprehension and voluntariness inquiry.}
    \item \textbf{Limited assistance from counsel.}
\end{itemize}

\subsubsection{Judicial discretion}



Judges have no discretion to reject a not guilty plea

The discretion that judges have to reject an agreed statement of facts appears to vary by jurisdiction

Appellate decisions in most Canadian jurisdictions give judges little to no discretion to reject an agreed statement of facts

Only the British Columbia Court of Appeal appears to have explicitly outlined a process that allows judges to reject them. Ontario has arguably done the same by requiring judges to conduct a plea comprehension and voluntariness inquiry for the \textit{nolo contendere} plea procedure using 

As a result, judges in jurisdictions that don't have this discretion may be unable to reject an informal *nolo contendere* plea

Clear legislation making the plea formally available would address this issue

Giving judges explicit discretion over whether to accept equivocal no-contest pleas would give further protections to defendants who enter them

Judges retain ultimate discretion over whether to accept a guilty plea for a good reason

It is important and often useful to have judges sit as the final check and balance for a recommendation from counsel

Counsel who have intimate familiarity with the facts and nuances of a case may fail to see the forest for the trees at times

Having that sober last look helps guard against false guilty pleas and inappropriate sentencing recommendations

Yet another safeguard for defendants who currently enter these pleas without them

\subsubsection{Right to appeal}

Appeals against criminal convictions in Canada may be made against conviction or sentence.\footnote{CC s 675, CC s 813}

A guilty plea is not a conviction for the purpose of an appeal, but defendants who self-convict through the informal \textit{nolo contendere} procedure sidestep that distinction. This may lead to these pleas overtaking guilty pleas for some offences, such as those with broad sentencing ranges, like sexual assault or manslaughter. 

On the other hand, allowing a defendant to invite the court to convict them while preserving their right to appeal may have a utility of its own. It may, for example, allow defendants to balance the future potential for buyer's remorse against foregoing the mitigating effects of accepting responsibility for the offence. 

\subsubsection{Plea voluntariness and comprehension inquiry}

No statutory requirement for one outside of a guilty plea formally entered under CC 606

Limited/no common law history of equivocal no-contest pleas, which amounts to limited/no authority for requiring plea comprehension and voluntariness

Comprehension and voluntariness are not (normally) required for a not guilty plea

Authorities that see the informal *nolo contendere* process as the functional equivalent of a guilty plea exist, but also ignore the real differences between the two

Appellate courts outside of Ontario, including the Supreme Court of Canada, may disagree that the informal *nolo contendere* process is sufficiently akin to a guilty plea to require a plea comprehension inquiry

In a case where an agreed statement of facts is filed with a defendant's signature, an appellate court may find that plea comprehension and voluntariness is made out without the need for any further inquiry

No inquiry is required where a defendant agrees to facts at a normal trial, for example

Because the current statutory scheme only technically applies to guilty pleas entered under CC 606, it is technically not required when entering a plea through the informal *nolo contendere* procedure

Although the authorities authorizing informal *nolo contendere* procedures in Ontario have generally required that some form of plea comprehension and voluntariness inquiry be done, nothing guarantees that other courts in other jurisdictions will do the same

There may be a social interest in having a more comprehensive plea inquiry for defendants who enter equivocal no-contest pleas

For example, the plea voluntariness and comprehension inquiry could be mandatory for equivocal pleas, despite effectively being optional for unequivocal guilty pleas

The common law regarding plea inquiries arguably still applies, and has been so applied

But the common law on plea voluntariness is imprecise generally, and it is not entirely clear to what extent a voluntariness inquiry needs to be done when a defendant makes admissions via CC 655
 
\subsubsection{Protecting the lawyers who assist with these pleas}

Ineffective assistance of counsel allegations and complaints to the Law Society

Warning shots from the Law Society of Upper Canada

Explicitly authorizing a process by law sends a clear signal that it can and should be used

An informal procedure, on the other hand, runs the risk of being viewed as illegitimate

Improved ability to resolve criminal matters

Potential for increased efficiencies

Additional options to provide to clients

\subsection{Customization through regulation}

Non-inculpatory no-contest pleas are qualitatively different from their inculpatory counterparts, in that defendants do not admit their factual guilt while nonetheless accepting the consequences of a conviction. Because they are qualitatively different, it follows that the effects and consequences of the plea may also be different. For example, and as outlined at length above, American jurisdictions that permit \textit{nolo contendere} pleas typically allow defendants to actively contest their guilt in some or all subsequent proceedings that may be initiated against them. Acknowledging this qualitative difference and designing new pleas around it may introduce a host of new pleas and plea procedures that better accomplish the end goals of justice than the current statutory regime can.

American \textit{nolo contendere} pleas evolved from a common law source, morphing into different common law traditions in the separate states where the plea was transplanted and implemented. As the pleas were codified, this divergent common law history was statutorily solidified. However, because Canada has no comparable common law tradition of non-inculpatory no-contest pleas, these pleas may be constructed from the ground up without any need to kowtow to the eccentricities that \textit{stare decisis} occasionally produces, taking the best ideas and discarding the worst without having to worry about retroactive inconsistency. This lack of common law baggage further allows Parliament to legislate exculpatory no-contest pleas in much the same way. Furthermore, because Canadian criminal law is uniformly legislated, these pleas may be uniformly implemented nation-wide. Regulating and codifying non-inculpatory no-contest pleas can provide many benefits that range from specifying when and under what conditions these pleas can be entered to placing conditions on their applicability in future legal proceedings. These benefits may be expressed using the four categories used to classify \textit{nolo contendere} earlier in this thesis.

\subsubsection{Applicability}

One of the early debates amongst the courts that accepted \textit{nolo contendere} pleas was the question of what offences defendants could enter these pleas to. Much of the common law wisdom around \textit{nolo contendere} pleas derived directly from Hawkins' citation, which strongly implied that \textit{nolo contendere} pleas should be accepted only in misdemeanour cases that were punishable by a small fine. As the pleas were more widely used and more widely seen as useful, their common law scope expanded accordingly. While some states continue to restrict the plea's applicability, most states that authorize it allow it to be entered for all offence types.

To the extent that non-inculpatory no-contest pleas encourage mutually beneficial plea resolutions, I argue that they should be permitted, but recognize that certain societal goals may be ostensibly met by restricting their availability. Just as Parliament may reasonably restrict judges from imposing certain sentences through mandatory minimums, or may make certain sentences unavailable for certain offences,\footnote{One example of this is found in the \textit{Criminal Code} provisions permitting conditional sentence orders. The list in \textit{Criminal Code} s 742.1 is imperfect, and nothing about it obviously recommends that it be used to restrict non-inculpatory no-contest pleas, but it serves as an example of how these restrictions may be implemented.} clearly codifying non-inculpatory no-contest pleas into law would allow Parliament to permit or restrict these pleas according to any offence based criteria of its choosing. Restricting non-inculpatory no-contest pleas to offences where the prosecutors proceed summarily, precluding defendants from entering these pleas to violent offences or offences punishable up to a certain maximum, or disallowing non-inculpatory no-contest pleas for offences punishable by a mandatory minimum period of incarceration would allow Parliament to make these pleas available to certain offenders convicted of certain offences while restricting others. Although expanding these pleas to allow all defendants who are inclined to enter them to do so is ideal, allowing any defendant to do so is a step in the right direction.

\subsubsection{Acceptability}

The acceptability criterion considers what conditions must obtain in order for a plea to be accepted. In most jurisdictions where non-inculpatory no-contest pleas are allowed, permission from both the court and the prosecutor is required. Some jurisdictions only require permission from the court, while only Virginia gives neither any discretion over whether defendants may enter a \textit{nolo contendere} plea. Best-interest pleas entered per \textit{Alford} require the consent of the court. In Canada, where non-inculpatory no-contest pleas may be entered informally through \textit{Criminal Code} s 655, permission from the prosecutor is required.\footnote{} 

Given that non-culpatory \textit{nolo contendere} pleas and exculpatory best-interest pleas are inextricably linked to the plea bargaining process, prosecutorial permission is a natural requirement. Further, given that fully inculpatory no-contest pleas require judicial authorization, requiring the same for non-inculpatory no-contest pleas is sensible and consistent with this existing prerequisite. However, because defendants who enter non-inculpatory no-contest pleas offer propositions of proof that do not necessarily reflect their beliefs, and because those propositions must invariably result in a conviction, I argue that additional prerequisites ought to be put in place to ensure that these pleas pass muster. Requiring both the prosecution and defendants to show cause why a non-inculpatory no-contest plea is justified, evaluating the propriety of these pleas against the question of whether the administration of justice would be brought into disrepute by accepting them, and insisting that a plea comprehension and voluntariness inquiry must be conducted by the presiding judge as a condition precedent to the plea's validity are all 

Statutory burdens on both the prosecution and defence to show cause why the plea is justified

Eg public desire for efficient administration of justice

\subsubsection{Procedural effects}

As self-convictions, non-inculpatory no-contest pleas ought to result in a criminal conviction. 

\subsubsection{Subsequent effects}

Legal scholars have identified the fact that the non-culpatory \textit{nolo contendere} plea does not bar defendants from actively contesting their matters in subsequent proceedings as the key factor that distinguishes these pleas from their inculpatory counterparts. Although this claim is debatable, it is difficult to doubt that the inadmissibility of \textit{nolo contendere} pleas in subsequent civil or criminal proceedings 

For example, preventing subsequent use of a plea when an offender successfully completes a period of community supervision may serve as an excellent incentive for pro-social behaviour and compliance with probation orders

Like a discharge, but one that could accompany an otherwise official custodial sentence

Inadmissibility in subsequent proceedings

Can lead to increasingly granular customization of the effects of a guilty plea

Civil

Criminal

Immigration

Family

The application system used in New Jersey could be considered as well, wherein some (or all) of these subsequent inadmissibilities are only available upon an application being made and granted

No common law tradition implying that inadmissibility should be limited to civil proceedings

Ancillary orders
