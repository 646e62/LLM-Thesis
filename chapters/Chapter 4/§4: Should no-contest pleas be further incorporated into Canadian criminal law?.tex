\chapter{Should Canada's \textit{nolo contendere} plea be formalized?}

Unlike guilty pleas, \textit{nolo contendere} pleas are almost exclusively the product of plea bargains. To the extent that victims, law enforcement officials, and (consequently) prosecutors all favour a contrite defendant who admits responsibility for an offence over an ambivalent or defiant one, outright guilty pleas are preferred to ambivalent \textit{nolo contendere} pleas. Additionally, most jurisdictions that allow \textit{nolo contendere} pleas require prosecutors to consent as a condition precedent to a judge accepting the plea. This includes the \textit{nolo contendere} procedure used in Canada. To the extent that there is some natural resistance to defendants entering \textit{nolo contendere} pleas, and to the extent that prosecutors must consent to the plea being entered, it is generally only entered as the result of a plea bargain.

Therefore, to determine whether these pleas should be further incorporated into Canadian criminal law, it is necessary to also examine the propriety of plea bargaining more generally. Plea bargaining is a controversial practice, such that the answer to this question is not straightforward. If it is the case that plea bargaining is an implicitly suspect enterprise that should be avoided where possible, this weighs against expanding the use and availability of no-contest pleas. Plea bargaining's disadvantages may be so great that further enabling the practice with \textit{nolo contendere} pleas is manifestly irresposible.\footnote{See e.g. the Bibas article. Although this is an article on plea bargaining, its subject-matter is limited to \textit{nolo contendere} and best-interest pleas.} However, if plea bargaining is an implicitly worthwhile and valuable process, or even an ethically neutral one, then it may be worthwhile to examine \textit{nolo contendere} pleas further. If \textit{nolo contendere} pleas further plea bargaining's ethical goals and do not create moral quandries of their own, then they ought to be explored and implemented.

