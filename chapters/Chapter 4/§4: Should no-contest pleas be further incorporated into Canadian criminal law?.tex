\chapter{Should Canada's \textit{nolo contendere} plea be formalized?}

Unlike guilty pleas, \textit{nolo contendere} pleas are almost exclusively the product of plea bargains. Many jurisdictions that allow \textit{nolo contendere} pleas require prosecutors to consent before the judge may accept the plea.\footnote{List the states.} This includes the \textit{nolo contendere} procedure used in Canada. Where defendants and prosecutors must come to an agreement in order for the court to accept a plea, that plea is very likely the result of a plea bargain. Furthermore, to the extent that victims, police, and prosecutors prefer a contrite defendant pleading guilty over a waffling defendant self-convicting, prosecutorial agreement is only likely where some \textit{quid pro quo} is available. \textit{Nolo contendere} pleas depend on plea bargaining.

Therefore, to determine whether these pleas should be further incorporated into Canadian criminal law, it is necessary to examine the propriety of plea bargaining more generally. Because plea bargaining is a controversial practice, the answer to this question is not straightforward. If it is the case that plea bargaining is an implicitly suspect enterprise that should be avoided where possible, this weighs against expanding the use and availability of no-contest pleas. Plea bargaining's disadvantages may be so great that further enabling the practice with \textit{nolo contendere} pleas is manifestly irresposible.\footnote{See e.g. the Bibas article. Although this is an article on plea bargaining, its subject-matter is limited to \textit{nolo contendere} and best-interest pleas.} However, if plea bargaining is an implicitly worthwhile and valuable process, or even an ethically neutral one, then it may be worthwhile to consider formalizing \textit{nolo contendere} pleas. If they further ethical goals and do not create moral quandaries of their own, then they ought to be implemented.

