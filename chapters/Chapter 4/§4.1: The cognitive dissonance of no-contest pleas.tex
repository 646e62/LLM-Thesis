\section{The cognitive dissonance of non-inculpatory no-contest pleas}

One impediment to allowing defendants to enter non-inculpatory no-contest pleas is the cognitive dissonance experienced when trying to hold the conflicting values required to sustain them. Having a person plead guilty only to protest their innocence is superficially contradictory, in that they say one thing (or nothing at all) while accepting the consequences of its opposite. This phenomenon creates several problems:

\begin{itemize}
    \item \textbf{The truth problem.} Defendants who self-convict while refusing to admit guilt or while actively maintaining their innocence say one thing while meaning another. Doing so is untruthful, and lawyers who assist clients with such pleas mislead the court. 
    \item \textbf{The presumption of innocence problem.} People charged with crimes are presumed innocent until proven guilty. The presumption should still stand when a defendant pleads guilty but protests their innocence or refuses to acknowledge their guilt openly.
    \item \textbf{The fairness problem.} Where a defendant self-convicts but does not acknowledge their guilt, it is reasonable to suspect their motives. Because these defendants disclaim their guilt one way or the other, the pleas are not overtly products of good motives, such as a desire to take responsibility for their actions or to ensure that the truth wins out. Such pleas thus appear to be products of unfairness. 
\end{itemize}

\subsection{The truth problem}

Convictions naturally follow where defendants plead guilty or are found guilty after trial. However, where defendants cannot or will not do so but insist that the court treat them as though they did, they appear to contradict themselves. Where the plea's content is ``guilty," but the defendant's actions signify ``not guilty" or ``innocent," it raises an apparent problem of truthfulness. As this section will demonstrate, this problem is superficial. It arises from misunderstanding the relationship between facts, belief, and proof on the one hand and the relationship of these concepts to formal pleas and admissions on the other.

\subsubsection{Facts, propositions, belief, and proof}

Suppose non-inculpatory no-contest pleas raise a problem of truth. In that case, it is essential to understand what ``truth" means. Throughout this section, ``truth" and ``fact" are closely related terms that may be used interchangeably in specific contexts. For example, a proposition may be synonymously described as being ``the truth" or ``a fact."\textit{Truth} is a quality that all \textit{facts} possess, such that anything accurately described as a \textit{fact} could also accurately be described as being \textit{true}.\footnote{The decision to primarily use the word ``truth" as an adjective and ``fact" as a noun is intentional, as it is more coherent to speak of multiple facts being true rather than multiple truths being factual.} Obversely, there are no untrue facts. One can make many distinctions between facts and the objects that they describe. However, for this thesis, it will suffice to say that \textit{a fact is a statement about the world\footnote{By ``world," I mean the entire realm of objects that have existed, do exist and will exist, as well as the relationships between those objects.} that is objectively true}, regardless of whether one subjectively understands it as such. Statements like ``AlphaGo beat Lee Sedol in four out of five Go games at the DeepMind Challenge Match," ``the earth orbits the sun," and ``the fridge to my right is white" are all examples of facts in that they are all statements about the world that are objectively true.

Not every statement about the world is a fact. Statements like ``Lee Sedol beat AlphaGo in three out of five Go games at the DeepMind Challenge Match," ``the sun orbits Jupiter," and ``the fridge on my right is orange" are all statements about the world that have not, do not, and will not obtain. All facts are statements about the world, but not all statements about the world are facts. I use the term \textit{proposition} to describe statements about the world that may or may not be objectively true.

Within this definition scheme, propositions are true or untrue independent of whether anyone is subjectively aware of or acknowledges the fact. The subjective component of a proposition arises in the form of \textit{beliefs}. For this thesis, a belief is a subjective acceptance or rejection of a proposition based on the subject's perception of that proposition's truth. A subject believes a proposition when they accept it and disbelieves it when they reject it. Beliefs may be reasonable or unreasonable, and whether a belief can be called reasonable depends on whether that belief can be proven.

\textit{Proof} refers to the act of grounding a subjective belief in ostensibly objective criteria. A proposition is proven when the subject is satisfied there are justifiable reasons to believe it is true. Obversely, a proposition is disproven once the subject is satisfied that there are justifiable reasons to believe it is false or that there are no justifiable reasons to believe it is true. Where multiple subjects share similar or identical criteria, those criteria may form a common standard of proof. Although proofs are generally founded on objective or external criteria, they are a predominantly subjective enterprise in that proofs exist to satisfy the subject that their belief is reasonable. 

True propositions may be proven false, while false propositions may be proven true. A proposition, once proven, may be disproven. For example, at various stages in history, it was proven that women had fewer teeth than men,\footnote{See Bertrand Russell's article.} that the earth revolved around the sun,\footnote{E.g., the Ptolemaic universe.} and that Thomas Sophonow murdered Barbara Stoppel. These propositions have been disproven; therefore, none of these beliefs may be reasonably held today. Although these propositions were once proven, each is untrue, thus demonstrating that \textit{there is no necessary correlation between proof and fact}.

\subsubsection{Pleas and admissions}

Having defined the terms above, the next step in addressing the ``truth problem" is understanding what a defendant does when they enter pleas, which in turn requires understanding the role and function of \textit{admissions} in criminal law. 

A plea is a response to a criminal allegation. Criminal allegations are propositions that a prosecutor must prove by default. Pleas are also propositions in that they contain a statement about the world that may or may not be true. However, pleas do not necessarily reflect a defendant's belief. A defendant may believe that they are guilty of an offence, may \textit{be} guilty of an offence, but still plead not guilty by right or by not entering a plea. A defendant who pleads not guilty does not necessarily provide the prosecutor or the court with any information about their beliefs or the truth of that proposition. Instead, they notify the prosecutor that they must prove the offence's essential elements and the court that these elements must be proven. Similarly, a defendant who enters a no-contest plea notifies the prosecutor and the court that the offence's essential elements need not be proven but does not necessarily provide information about their subjective beliefs or the objective truth of those beliefs. Even where a defendant believes that the content of their plea is true, their subjective beliefs do not necessarily reflect the objective truth.\footnote{In the timeless words of Judge Maya Gamble, ``Your beliefs do not make something true."}

An admission is related to a no-contest plea in that it advances a proposition that does not need to be proven.\ In Canada, \textit{Criminal Code} s 655 allows defendants to admit any ``facts alleged [by the prosecutor] for the purpose of dispensing with proof thereof."\footnote{See \textit{Criminal Code} s 655. If \textit{facts} signify only true propositions, the phrase ``facts alleged" is nonsensical, as \textit{allegations} may be true or false. The phrase is best understood as synonymous with a \textit{proposition}, and I will refer to it as such.} Typically, these allegations will support or consist entirely of an element of the offence charged. However, any mutually agreed upon proposition can be admitted under this section.\footnote{Defendants may wish to forego proof of some or all of the facts of their case for several reasons. In some cases, there is no reason to doubt that the prosecution will prove certain offence elements. A prosecutor is likely to be able to prove elements like date, time, jurisdiction, and identification in a domestic violence prosecution without difficulty. Agreeing to these facts saves court time and judicial attention, which can be minimal resources in busy court circuits. Admissions can also be advantageous in situations where the evidence is less certain to be proven. For example, when the prosecution wishes to call a witness whose evidence will inevitably harm the defendant to one degree, the defendant may wish to have the witness excused in exchange for a manageable set of agreed facts. In cases where the only genuine dispute lies with a legal issue, both parties may agree to have all evidence admitted by consent and limit trial time to arguing that issue. } Just as a no-contest plea obviates the need for the prosecutor to prove the offence's elements, admissions remove any need for either side to prove the proposition in question.\footnote{See @2009mbca37 for a discussion of the distinction between formal and informal admissions.} Although parties to criminal proceedings most frequently use admissions in trials, they may also be used in sentencing hearings where the parties do not agree about the allegation's details.\footnote{E.g., a defendant charged with sexual assault may admit the essential elements of the offence, but dispute some or all of the underlying allegations.}

\subsubsection{Summary}

Pleas and admissions are both matters of proof, not truth or belief. Consequently, the ``truth problem" is a definition problem. By entering a plea or making an admission, defendants advise the prosecutor and the court which propositions will require proof if any. Pleas are not evidence, and defendants neither swear nor affirm that their pleas are genuine. Where a defendant is unwilling or unable to enter a plea, the court must enter a not guilty plea on their behalf, regardless of whether the defendant \textit{believes} they are not guilty and whether the defendant \textit{is} not guilty. A defendant who resiles their not guilty plea will not be liable for perjury any more than a defendant who wishes to withdraw their guilty plea. Where a defendant wishes to self-convict, the \textit{Criminal Code} requires that they understand that a guilty plea admits the offence's essential elements. It does not require defendants to believe that they are guilty. Where a defendant resiles their self-conviction, Canadian courts have focused exclusively on whether the defendant entered the plea voluntarily, knowingly, and unequivocally, and not on whether the plea was factually true.\footnote{There are exceptions to this. See \textit{R v Catcheway}, 2018 MBCA 54, where the defendant pleaded guilty to being unlawfully in a dwelling house despite having been incarcerated at the time. The Manitoba Court of Appeal allowed him to withdraw his guilty plea because there were ``valid grounds for doing so." Absent evidence that his pleas were uninformed or involuntary, it erred in doing so. To its credit, the court simultaneously arrived at the correct solution by allowing the appeal against conviction based on a miscarriage of justice, as there was compelling evidence that the defendant was innocent.}

\subsection{The presumption of innocence problem}

The presumption of innocence and the prosecutor's burden of proving an offence beyond a reasonable doubt are hallmarks of the common law justice system. Legal professionals and academics see these concepts as indispensable tools for fighting wrongful convictions. When properly implemented, these concepts place a significant burden on the prosecutor, so only those who are guilty of an offence stand any chance of being convicted. The joint operation of these concepts will inevitably result in some guilty persons walking away without conviction. However, this sacrifice is generally considered worth the added protections afforded to the truly innocent. 

These legal concepts seem to run contrary to the practice of non-inculpatory self-convictions. Where the prosecutor has the burden of proving the charges and the court must consider the defendant innocent until the prosecutor discharges that burden, it appears contradictory and unjust to convict a defendant who maintains their innocence either tacitly or explicitly.\footnote{Cite to one or more articles that take this position. Alschuler gets into this briefly, but the author who discusses Alford's ``sworn testimony" at length is the source here.} Where defendants refuse to contest the evidence through a non-culpatory no-contest plea like \textit{nolo contendere}, they have arguably waived their right to be presumed innocent. However, in cases where the defendant maintains their innocence, it seems they continue to be presumed innocent until the prosecutor discharges their burden. Although the no-contest plea itself is not evidence, subsequent protestations of innocence arguably should, under this argument, be treated as a signal to the prosecutor and court that a trial is required.

Answering this argument requires looking at what courts generally require after a defendant has entered a no-contest plea. Specifically, it is necessary to examine both the \textit{factual foundation} requirement and the residual \textit{judicial discretion} to refuse to accept a self-conviction.

\subsubsection{Factual foundation}

Once a defendant admits the essential elements of the offence charged, they have waived their right to be presumed innocent of that offence. Notwithstanding that defendants can legally admit facts against their interests and invite the court to convict them, there will be situations where it would be unjust for the court to do so. One such situation is where the propositions the prosecution relies upon do not make out the offence charged. Where a defendant formally admits the elements of the charged offence but otherwise maintains that they are innocent, legitimate questions arise as to whether that condition obtains. To help ensure that defendants do not invite baseless convictions, the \textit{Criminal Code} requires that judges ensure that ``the facts support the charge." 

The ostensible purpose of this section is to ensure that defendants do not convict themselves of baseless charges.\footnote{For example, a defendant allegedly crashes their car into the side of a corner shop late at night, fleeing the scene shortly after that. No other people or vehicles are involved. The police charge them with failing to stop after an accident under \textit{Criminal Code} s 320.16(1). The prosecutors offer to resolve the charge for a small fine and no driving prohibition on an early guilty plea, which the defendant accepts. However, because the accident did not involve another person or a conveyance, the charge is not made out, and a judge should reject that plea after hearing the allegations.} This requirement is also frequently found in American jurisdictions that allow \textit{nolo contendere} pleas. It is a common law requirement for best-interest pleas like \textit{Alford}, which suggests that the factual foundation may also help ensure that judges have a basis for authorizing a conviction, notwithstanding protestations of innocence to the contrary. Both the \textit{Alford} and \textit{Hector} decisions are instructive in this respect.

In \textit{Alford}, the defendant allegedly murdered another individual at a bar following a disagreement. Several witnesses saw the defendant retrieve his gun after the argument, heard him say that he intended to kill the victim, and heard him say that he had killed the victim. The defendant had a lengthy prior violent criminal history that included a murder conviction, and the witnesses he said would affirm his innocence gave inculpatory statements instead. Although the defendant took the stand and stated that he was innocent, he nonetheless pleaded guilty to a lesser included offence. By entering this plea, the defendant ensured he would not receive the death sentence he would have received if convicted of first-degree murder after trial. The court accepted that plea based on the strength of the allegations it heard.\footnote{See \textit{Alford} at 32 - 33: \begin{quote}
    If Alford's statements were to be credited as sincere assertions of his innocence, there obviously existed a factual and legal dispute between him and the State. Without more, it might be argued that the conviction entered on his guilty plea was invalid, since his assertion of innocence negatived any admission of guilt, which, as we observed last Term in Brady, is normally "[c]entral to the plea and the foundation for entering judgment against the defendant . . . ." 397 U. S., at 748.
    
    In addition to Alford's statement, however, the court had heard an account of the events on the night of the murder, including information from Alford's acquaintances that he had departed from his home with his gun stating his intention to kill and that he had later declared that he had carried out his intention. Nor had Alford wavered in his desire to have the trial court determine his guilt without a jury trial. Although denying the charge against him, he nevertheless preferred the dispute between him and the State to be settled by the judge in the context of a guilty plea proceeding rather than by a formal trial. Thereupon, with the State's telling evidence and Alford's denial before it, the trial court proceeded to convict and sentence Alford for second-degree murder.
\end{quote}}

In \textit{Hector}, the defendant allegedly murdered three people. The police found the gun used in the murders in the defendant's truck and bullets that matched the ones found in the victims. The defendant's brother agreed to testify that the defendant described one of the murders in detail. The defendant refused to allow his lawyer to hire an investigator to investigate possible defences. Although the defendant's wife agreed to provide an alibi, her evidence on that point was highly suspect and the subject of criminal obstruction charges against her. The defendant pleaded guilty to the murders, and in exchange, the Crown agreed not to pursue a dangerous offender application against him.

Where a defendant either refuses to accept responsibility for an offence overtly or actively protests their innocence, a review of the allegations and the supporting evidence lets the court determine what weight, if any, to give to those apparent equivocations. In cases where the allegations are straightforward, the evidence supporting the allegations is strong, or both, defendants who vacillate after self-convicting may be safely ignored. In cases where the allegations are complex or the evidence supporting the allegations is equivocal, judges may consider these factors when deciding whether to allow the defendant to convict themselves. Put otherwise, the factual foundation requirement is a reasonable means to rebut the presumption of innocence when defendants do not adequately do so through their plea.

\subsubsection{Discretion}

In Canada and the United States, judges have immense discretion when deciding whether to accept or reject a no-contest plea. In jurisdictions that formally allow \textit{nolo contendere} plea, only one state allows defendants to enter that plea by right, and all others require judicial authorization. Similarly, in states that authorize best-interest pleas via the \textit{Alford} decision, judges are under no obligation to accept them. This broad discretion is a final safeguard for a defendant's presumption of innocence.

Even where a court is satisfied that a defendant is inviting a conviction and that there is sufficient evidence to support convicting them, a judge may still refuse to authorize a no-contest plea. Where there are compelling reasons for a judge to suspect that a defendant is factually innocent, or where the prosecution relies on evidence that would likely be inadmissible at trial, a judge may exercise their discretion and refuse to allow the defendant to self-convict. Until a court is satisfied that a no-contest plea should be accepted, it generally will not be. Because criminal defendants generally cannot self-convict by right, the presumption of innocence remains until a judge is satisfied that it should no longer hold.

\subsubsection{Summary}

The presumption of innocence problem is a non-problem. Defendants are presumed innocent throughout the criminal process but may waive that presumption and invite a conviction. However, the presumption still stands until a court is satisfied that there is a reasonable basis for allowing that conviction. A refusal to admit guilt and active protestations of innocence may be sufficient to keep the presumption intact but need not be in all cases. Like any other person, criminal defendants are susceptible to ``buyer's remorse," minimizing or denying their behaviour, and  misunderstanding polysemic legal concepts.\footnote{The intractable problem of properly understanding abstract legal concepts may be insoluble. See e.g. Frederick Schauer ``Formalism" at 514: ``Some terms, like `liberty' and `equality' are \textit{pervaseively indeterminate}. It is not that such terms have no content whatsoever; it is that \textit{every} application, every concretization, every instantiation requires the addition of supplementary premises to apply the general term to specific cases."} Where a defendant has told the court that the prosecution does not need to prove its case against them, and where there is an adequate propositional matrix to support the prosecution's case independent of a defendant's plea, the presumption of innocence no longer stands.

\subsection{The fairness problem}

Although non-inculpatory no-contest pleas are not implicitly untruthful and do not run afoul of the presumption of innocence, there is still an intuitive sense that they are unfair. Defendants may enter an inculpatory no-contest or not guilty plea because they sincerely believe those propositions are accurate, and they believe they deserve the logical consequences of their plea. By contrast, a defendant who enters a non-inculpatory no-contest plea does not take responsibility for the offence or actively protests their innocence. This makes it unlikely that a defendant will enter the plea because they believe it is accurate or deserves the consequences. This begrudging aspect of non-inculpatory no-contest pleas grounds the fairness problem. 

\subsubsection{Self-convictions must be voluntary, informed, and unequivocal}

As discussed at length above, the common law that Canadian criminal law stems from has a lengthy history of precluding courts from allowing defendants to self-convict unless a judge is satisfied that they are doing so knowingly and voluntarily. Canadian courts also require that defendants unequivocally intend to enter the formal plea that is ultimately recorded. These principles are so deeply ingrained in Canadian criminal law that they have found their way into the informal \textit{nolo contendere} plea procedure emerging across the country.\footnote{} Similarly, in most American jurisdictions that allow \textit{nolo contendere} pleas, the plea voluntariness and comprehension inquiry required for inculpatory guilty pleas is often mirrored for non-culpatory \textit{nolo contendere} pleas.\footnote{Exculpatory best-interest pleas entered per \textit{Alford} are formally guilty pleas and therefore susceptible to whatever plea inquiry is required for their inculpatory counterparts.} Any time a defendant invites a conviction, the court receiving their plea has an obligation to ensure that they are not doing so due to any undue pressure, short-sightedness, or confusion. By doing so, courts further ensure that defendants who self-convict are treated fairly.

\subsubsection{Summary}

The fact that defendants must enter no-contest pleas knowingly and voluntarily mitigates their apparent unfairness. Despite these protections, some concerns may reasonably remain. Ensuring that defendants enter pleas knowingly and voluntarily ensures that they know the consequences of entering their pleas and that they decided to do so themselves. However, these requirements can only go so far. Defendants facing a lengthy prison term if convicted after trial may knowingly and voluntarily plead guilty if it is possible to resolve for a less severe sanction, even if they are factually innocent of the charge. Although courts do not consider these influences strong enough to override a defendant's voluntariness, these more subtle coercions may arguably produce unjust results. These problems are traceable to the practice of plea bargaining.