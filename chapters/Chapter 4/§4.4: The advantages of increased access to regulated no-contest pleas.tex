\section{The advantages of expanding access to no-contest pleas}

If one accepts the premise that plea bargaining is a substantive and normative good or that plea bargaining may be such in certain cases, a presumptive case is made for expanding plea bargaining's scope. For defendants who cannot or will not plead guilty but are nevertheless willing to self-convict, expanding the list of permitted pleas to include both non-culpatory and exculpatory no-contest pleas accomplishes this task. For prosecutors, witnesses, victims, and society generally, the potential gains that may be won at a contested trial will not usually be worth the risk when compared with a confirmed conviction through a no-contest plea.\footnote{To the extent that either the defendant or the state may have the facts of the offence come out more favourably for them at trial, neither party has any implicit advantage over the other.}

The following are key advantages of expanding access to no-contest pleas:

\begin{itemize}
    \item \textbf{Ensuring certainty in outcomes for all.}
    \item \textbf{Delegating more control over the evidence admitted and proven to prosecutors and defendants.}
    \item \textbf{Increasing agency and self-determination for defendants.}
    \item \textbf{Transforming zero-sum prosecutions into win-win outcomes.}
    \item \textbf{Solidifying procedural protections for defendants and their lawyers.}
    \item \textbf{Rectifying the loopholes and loose ends created by informal pleas.}
\end{itemize}

\subsection{Increased certainty in the outcome}

\subsubsection{Plea bargained cases generally}

Because prosecutions can be fraught with uncertainty for all parties involved, having a measure of certainty about how a case will resolve motivates all plea bargains. This fact tracks even in cases with defendants who intend to plead guilty. A defendant may wish to plead guilty to some offence charged but may only be prepared to plead to a lesser included offence. Similarly, disputes over a sentence's nature and quantum can be hotly contested, even where defendants are otherwise prepared to plead guilty at their first court appearance. While these questions may have very different answers in a contested hearing, parties who agree to these matters in advance are likely to have their expectations met.\footnote{See \textit{R v Anthony-Cook}, 2016 SCC 43 (CanLII), [2016] 2 SCR 204. While jointly-recommended sentencing proposals are subject to judicial discretion and review, judges are generally precluded from rejecting them. Even in cases where prosecutors and defendants cannot come up with a joint proposal or where the requisite \textit{quid pro quo} for a joint proposal is not met, judges remain similarly constrained by the parties' sentences. See \textit{R v Beardy}, 2014 MBCA 23 at para 6; \textit{R v Grant}, 2016 ONCA 639 at paras 163 - 165} Defendants and prosecutors who pre-negotiate pleas and sentences greatly increase their certainty in the outcome.

\subsubsection{Non-inculpatory no-contest cases specifically}

Under the current Canadian legislative scheme, defendants who cannot or will not admit that they are factually responsible for the offence they are charged with must set their matters down for trial. As discussed above, trials are risky enterprises fraught with dangers for defendants charged with the underlying crimes and the prosecutors carrying their cases. Prosecutors always bear the burden of proving criminal charges beyond a reasonable doubt, and even defendants without a positive defence or cohesive response to the allegations against them may prevail at trial. Forcing defendants to contest charges they are otherwise prepared to admit as proven risks wrongful acquittals. Defendants charged with multiple offences may be offered the opportunity to self-convict on some in exchange for the others being dropped. However, when those same defendants must contest all of those offences to trial, they risk being convicted of more than they could have otherwise bargained for.

\subsection{Increased control over the evidence admitted and proven}

\subsubsection{Plea bargained cases generally}



As with the outcome, the fact that a defendant agrees to plead guilty doesn't mean that there aren't concessions the prosecution or victims may still want concerning the facts pleaded to

The prosecution is still required to prove all aggravating sentencing factors beyond a reasonable doubt if any of those are disputed

Even factors that aren't strictly aggravating may be such that one or the other party wishes to have them admitted 

\subsubsection{Non-inculpatory no-contest cases specifically}

Prosecutors arguably have more leeway over the facts in equivocal no-contest pleas

The defendant, after all, isn't admitting to or contesting the plea

One might expect that a defendant would be less inclined to put up a fuss about the facts where it's already on record that they aren't admitting or agreeing to them

Proof of aggravating factors should (but wouldn't necessarily) transfer to equivocal no-contest pleas as well, even if just ideologically

Like an increased control over the certainty of the outcome of a case, equivocal no-contest pleas give both prosecutors and defendants an increased degree of control over both which facts are presented and how those facts are framed

Which facts are presented

Both parties have the same advantage when it comes to guilty pleas

Prejudicial or embarrassing facts can be negotiated down or out

Aggravating facts can be negotiated up or in

Where there's a dispute as to any aggravating factor on sentencing, the Crown must prove the aggravating factor beyond a reasonable doubt

Some bargaining power accrues to the defendant as a result

How those facts are framed

Allows the defendant to maintain a position of reduced moral blameworthiness, even if only privately, while acknowledging legal guilt

Discussing the factual presentation of an offence like this sounds Machiavellian, but is arguably the best way to guarantee a truthful outcome

As with certainty of outcome, the uncertainty inherent to the trial process is a concern for how facts about the offence and offender are ultimately received

Witnesses, including defendants and complainants, may unexpectedly testify poorly (or well)

Witnesses may fail to show up at all

Expert evidence might be disregarded and disbelieved

The judge or jury may place unexpected weight on facts both parties considered immaterial

Each of these factors can have a very volatile effect on a case, and almost certainly lead to inaccurate results

Both parties benefit by knowing which facts they want to be admitted, how they want those facts to be interpreted, and how they intend for those facts to be presented

Both parties benefit by having the opportunity to collaborate on the presentation and sentence recommendation

Increased control over these variables limits the reasonably likely outcomes of a case

\subsection{Increased agency and self-determination}

\subsubsection{Plea bargained cases generally}

Having some measure of control over the outcome and the facts is increased self-determination.

This increased self-determination also increases a person's agency in the situation

It stands to reason that defendants with more agency in their proceedings will ultimately have a higher regard for the justice system.

\subsubsection{Non-inculpatory no-contest cases specifically}

Chance to privately explain/deny the offence

Able to contest subsequent proceedings, where available

Remain authentic to a moral standpoint while accepting the inevitable

Some may question whether criminal defendants should be entitled to increased agency and self-determination

Criminal proceedings don't always involve clear-cut fact patterns where the clearly guilty offend against the clearly innocent

The line between complainant and defendant can be a very thin one in some cases

It's therefore unsurprising that such criminal defendants will feel very much aggrieved when called upon to plead, even if they are prepared to admit sufficient facts to establish their guilt

Requiring such defendants to set their matters down for loser trials does very little to increase their sense of agency in the proceedings

Even where defendants aren't borderline victims, but are clearly morally blameworthy aggressors and instigators, there is no value in curbing their ability to make decisions about their case that will benefit themselves, the state, and the complainants.

The fact that a defendant is willing to agree to self-convict and spare themselves, the prosecutor, and any witnesses involved the burden and uncertainty of a trial is a clear win-win-win

This will be expanded on in section,4.4.4: The elusive win-win scenario in criminal law

The fact that it isn't the greatest possible win for prosecutors and complainants isn't sufficient to justify not allowing defendants access to them

"A win is a win" mentality

Prosecutors aren't entitled to an ideal conviction, and complainants aren't entitled to a remorseful defendant. 

Efficiencies aside, the net gains for everyone involved far exceed any perceived losses at not having the defendant acknowledge their moral culpability or truly repent for their misdeeds

\subsection{The elusive win-win scenario in criminal law}

\subsubsection{Plea bargained cases generally}

Both parties agree to the outcome of a case

Amicable outcome is the definition of a win-win

All parties and their witnesses are relieved of the pressures that accompany court hearings

Criminal proceedings are otherwise geared towards being very adversarial

\subsubsection{Non-inculpatory no-contest cases specifically}

Additional potential "wins" for the defendant entering the plea

Additional potential "wins" for the prosecutor seeking to apply or enforce conditions

Additional avenues for all parties to seek resolution

Consider confidential informants (CIs)

Information provided by CIs is often very effective in helping secure convictions

This is especially true in the sorts of criminal operations that cause widespread social harm, including organized crime and drug trafficking

CI information can be very difficult to obtain outside of the plea bargaining context

Most confidential informants are criminals themselves, many of whom may be legally motivated through deals secured with the prosecution for reduced charges or generous recommendations on sentencing

Few are "ordinary citizens", and fewer still are motivated by civic duty

Contrast this win-win with the lose-lose situation that formalities currently require

\subsection{Customization through regulation}

Surveying the American jurisdictions that allow *nolo contendere* pleas reveals the breadth of specification and customization that can be done to the plea once it's formally incorporated into law

Regulating *nolo contendere* pleas can provide many benefits that range from specifying when and under what conditions these pleas can be entered to placing conditions on their applicability in future legal proceedings



\subsubsection{Applicability}

Summary and indictable proceedings restrictions

Offence class restrictions

Types of offences (property, sexual, violent, etc)

Classes based on mandatory minimums

\subsubsection{Acceptability}

Statutory burdens on both the prosecution and defence to show cause why the plea is justified

Eg public desire for efficient administration of justice

\subsubsection{Procedural effects}

Specialized plea inquiry for equivocal no-contest pleas

\subsubsection{Subsequent effects}

For example, preventing subsequent use of a plea when an offender successfully completes a period of community supervision may serve as an excellent incentive for pro-social behaviour and compliance with probation orders

Like a discharge, but one that could accompany an otherwise official custodial sentence

Inadmissibility in subsequent proceedings

Can lead to increasingly granular customization of the effects of a guilty plea

Civil

Criminal

Immigration

Family

The application system used in New Jersey could be considered as well, wherein some (or all) of these subsequent inadmissibilities are only available upon an application being made and granted

No common law tradition implying that inadmissibility should be limited to civil proceedings

Ancillary orders

\subsection{Sealing loopholes and resolving loose ends}

\begin{itemize}
    \item \textbf{No judicial discretion to reject non-culpatory no-contest pleas.}
    \item \textbf{Access to conviction appeals by right.}
    \item \textbf{No plea comprehension and voluntariness inquiry.}
\end{itemize}

\subsubsection{Judicial discretion}

Judges have no discretion to reject a not guilty plea

The discretion that judges have to reject an agreed statement of facts appears to vary by jurisdiction

Appellate decisions in most Canadian jurisdictions give judges little to no discretion to reject an agreed statement of facts

Only the British Columbia Court of Appeal appears to have explicitly outlined a process that allows judges to reject them. Ontario has arguably done the same by requiring judges to conduct a plea comprehension and voluntariness inquiry for the \textit{nolo contendere} plea procedure using 

As a result, judges in jurisdictions that don't have this discretion may be unable to reject an informal *nolo contendere* plea

Clear legislation making the plea formally available would address this issue

Giving judges explicit discretion over whether to accept equivocal no-contest pleas would give further protections to defendants who enter them

Judges retain ultimate discretion over whether to accept a guilty plea for a good reason

It is important and often useful to have judges sit as the final check and balance for a recommendation from counsel

Counsel who have intimate familiarity with the facts and nuances of a case may fail to see the forest for the trees at times

Having that sober last look helps guard against false guilty pleas and inappropriate sentencing recommendations

Yet another safeguard for defendants who currently enter these pleas without them

\subsubsection{Right to appeal}

Appeals against criminal convictions in Canada may be made against conviction or sentence.\footnote{CC s 675, CC s 813}

A guilty plea is not a conviction for the purpose of an appeal, but defendants who self-convict through the informal \textit{nolo contendere} procedure sidestep that distinction. This may lead to these pleas overtaking guilty pleas for some offences, such as those with broad sentencing ranges, like sexual assault or manslaughter. 

On the other hand, allowing a defendant to invite the court to convict them while preserving their right to appeal may have a utility of its own. It may, for example, allow defendants to balance the future potential for buyer's remorse against foregoing the mitigating effects of accepting responsibility for the offence. 

\subsubsection{Plea voluntariness and comprehension inquiry}

No statutory requirement for one outside of a guilty plea formally entered under CC 606

Limited/no common law history of equivocal no-contest pleas, which amounts to limited/no authority for requiring plea comprehension and voluntariness

Comprehension and voluntariness are not (normally) required for a not guilty plea

Authorities that see the informal *nolo contendere* process as the functional equivalent of a guilty plea exist, but also ignore the real differences between the two

Appellate courts outside of Ontario, including the Supreme Court of Canada, may disagree that the informal *nolo contendere* process is sufficiently akin to a guilty plea to require a plea comprehension inquiry

In a case where an agreed statement of facts is filed with a defendant's signature, an appellate court may find that plea comprehension and voluntariness is made out without the need for any further inquiry

No inquiry is required where a defendant agrees to facts at a normal trial, for example

Because the current statutory scheme only technically applies to guilty pleas entered under CC 606, it is technically not required when entering a plea through the informal *nolo contendere* procedure

Although the authorities authorizing informal *nolo contendere* procedures in Ontario have generally required that some form of plea comprehension and voluntariness inquiry be done, nothing guarantees that other courts in other jurisdictions will do the same

There may be a social interest in having a more comprehensive plea inquiry for defendants who enter equivocal no-contest pleas

For example, the plea voluntariness and comprehension inquiry could be mandatory for equivocal pleas, despite effectively being optional for unequivocal guilty pleas

Informal *nolo contendere* pleas are laden with procedural pitfalls that are ripe for exploitation by counsel and courts so inclined

Because they aren't guilty pleas, there's no statutory requirement that the judge conduct any particular plea inquiry with the defendant

The common law regarding plea inquiries arguably still applies, and has been so applied

But the common law on plea voluntariness is imprecise generally, and it is not entirely clear to what extent a voluntariness inquiry needs to be done when a defendant makes admissions via CC 655

The criteria set out in CC 606(1.1) may be instructive, but do not apply to not guilty pleas

\subsection{Protecting the lawyers who assist with these pleas}

\subsubsection{Authorization through regulation}

Ineffective assistance of counsel allegations and complaints to the Law Society

Warning shots from the Law Society of Upper Canada

Explicitly authorizing a process by law sends a clear signal that it can and should be used

An informal procedure, on the other hand, runs the risk of being viewed as illegitimate

\subsubsection{Improved ability to resolve criminal matters}

Potential for increased efficiencies

Additional options to provide to clients