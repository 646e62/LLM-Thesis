\section{The benefits of regulation}

Surveying the American jurisdictions that allow *nolo contendere* pleas reveals the breadth of specification and customization that can be done to the plea once it's formally incorporated into law

Regulating *nolo contendere* pleas can provide many benefits that range from specifying when and under what conditions these pleas can be entered to placing conditions on their applicability in future legal proceedings

For example, preventing subsequent use of a plea when an offender successfully completes a period of community supervision may serve as an excellent incentive for pro-social behaviour and compliance with probation orders

Like a discharge, but one that could accompany an otherwise official custodial sentence

Informal *nolo contendere* pleas are laden with procedural pitfalls that are ripe for exploitation by counsel and courts so inclined

Because they aren't guilty pleas, there's no statutory requirement that the judge conduct any particular plea inquiry with the defendant

The common law regarding plea inquiries arguably still applies, and has been so applied

But the common law on plea voluntariness is imprecise generally, and it is not entirely clear to what extent a voluntariness inquiry needs to be done when a defendant makes admissions via CC 655

The criteria set out in CC 606(1.1) may be instructive, but do not apply to not guilty pleas

The mitigating presumptions that accompany a guilty plea do not accompany a *nolo contendere* plea in the same way; namely

Expression of remorse

Admission of responsibility

Conversely, a defendant who is allowed to protest their innocence at sentencing, only to start to make any admissions of responsibility as they await parole, will show much more improvement than one who is required to enter a specious guilty plea

In fact, defendants who do start out protesting their innocence, only to start to acknowledge more responsibility as time goes on, are making more progress

Allowing nolo contendere and best-interest pleas at sentencing would, in fact, lead to a more accurate reflection of a defendant's rehabilitation than if that same defendant had to enter a guilty plea

\subsection{Loopholes and loose ends}

\subsubsection{Judicial discretion}

Judges have no discretion to reject a not guilty plea

The discretion that judges have to reject an agreed statement of facts appears to vary by jurisdiction

Appellate decisions in most Canadian jurisdictions give judges little to no discretion to reject an agreed statement of facts

Only the British Columbia Court of Appeal appears to have outlined a process that allows judges to reject them

As a result, judges in jurisdictions that don't have this discretion may be unable to reject an informal *nolo contendere* plea

Clear legislation making the plea formally available would address this issue

\subsubsection{Right to appeal}

Appeals against criminal convictions in Canada may be made against conviction or sentence.\footnote{CC s 675, CC s 813}

A guilty plea is not a conviction for the purpose of an appeal

Defendants who opt for self-conviction in this fashion can sidestep that distinction

This may lead to informal *nolo contendere* pleas overtaking guilty pleas with respect to some offences

Specifically, offences with broad sentencing ranges, like sexual assault or manslaughter

Allows defendants to balance the future potential for buyers remorse against the mitigating effects of unequivocally accepting responsibility for the offence via a guilty plea

On the other hand, allowing a defendant to invite the court to convict them while preserving their right to appeal may have utility of its own

See 4.5.3: Customizing \textit{nolo contendere} and best-interest pleas for some proposals

\subsubsection{Plea voluntariness and comprehension inquiry}

No statutory requirement for one outside of a guilty plea formally entered under CC 606

Limited/no common law history of equivocal no-contest pleas, which amounts to limited/no authority for requiring plea comprehension and voluntariness

Comprehension and voluntariness are not (normally) required for a not guilty plea

Authorities that see the informal *nolo contendere* process as the functional equivalent of a guilty plea exist, but also ignore the real differences between the two

Appellate courts outside of Ontario, including the Supreme Court of Canada, may disagree that the informal *nolo contendere* process is sufficiently akin to a guilty plea to require a plea comprehension inquiry

In a case where an agreed statement of facts is filed with a defendant's signature, an appellate court may find that plea comprehension and voluntariness is made out without the need for any further inquiry

No inquiry is required where a defendant agrees to facts at a normal trial, for example

\subsubsection{Conclusion}

Each of these loopholes could probably be closed by deeming informal *nolo contendere* convictions to be guilty pleas within the meaning of CC 606

\subsection{Procedural protections for defendants and their lawyers}

\subsubsection{Defendants}
\paragraph{Plea comprehension and voluntariness inquiry\\}

Because the current statutory scheme only technically applies to guilty pleas entered under CC 606, it is technically not required when entering a plea through the informal *nolo contendere* procedure

Although the authorities authorizing informal *nolo contendere* procedures in Ontario have generally required that some form of plea comprehension and voluntariness inquiry be done, nothing guarantees that other courts in other jurisdictions will do the same

There may be a social interest in having a more comprehensive plea inquiry for defendants who enter equivocal no-contest pleas

For example, the plea voluntariness and comprehension inquiry could be mandatory for equivocal pleas, despite effectively being optional for unequivocal guilty pleas

\paragraph{Factual foundation requirement\\}

As with plea comprehension and voluntariness, a factual foundation is a technical requirement of CC 606, but not required for a formal not guilty plea

The Ontario Court of Appeal in both *Hector* and *RP* noted the strong factual foundation underlying the best-interest and informal *nolo contendere* pleas in their respective cases

Again, making this mandatory, rather than optional (as with unequivocal guilty pleas) could protect truly innocent defendants from entering these pleas inconsiderately

A hearing process like the one used with conditional sentence order breaches may be useful

\paragraph{Judicial discretion\\}

Giving judges explicit discretion over whether to accept equivocal no-contest pleas would give further protections to defendants who enter them

Judges retain ultimate discretion over whether to accept a guilty plea for a good reason

It is important and often useful to have judges sit as the final check and balance for a recommendation from counsel

Counsel who have intimate familiarity with the facts and nuances of a case may fail to see the forest for the trees at times

Having that sober last look helps guard against false guilty pleas and inappropriate sentencing recommendations

Yet another safeguard for defendants who currently enter these pleas without them

\subsubsection{Their lawyers}
\paragraph{Righteousness through regulation\\}

Ineffective assistance of counsel allegations and complaints to the Law Society

Warning shots from the Law Society of Upper Canada

Explicitly authorizing a process by law sends a clear signal that it can and should be used

An informal procedure, on the other hand, runs the risk of being viewed as illegitimate

\paragraph{Improved ability to resolve criminal matters\\}

Potential for increased efficiencies

Additional options to provide to clients

\subsection{Customizing \textit{nolo contendere} and best-interest pleas}

Applicability

Summary and indictable proceedings restrictions

Offence class restrictions

Types of offences (property, sexual, violent, etc)

Classes based on mandatory minimums

Acceptability

Statutory burdens on both the prosecution and defence to show cause why the plea is justified

Eg public desire for efficient administration of justice

Procedural effects

Specialized plea inquiry for equivocal no-contest pleas

Subsequent effects

Inadmissibility in subsequent proceedings

Can lead to increasingly granular customization of the effects of a guilty plea

Civil

Criminal

Immigration

Family

The application system used in New Jersey could be considered as well, wherein some (or all) of these subsequent inadmissibilities are only available upon an application being made and granted

No common law tradition implying that inadmissibility should be limited to civil proceedings

Ancillary orders