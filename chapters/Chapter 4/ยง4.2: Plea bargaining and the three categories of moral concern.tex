\section{Plea bargaining and the three categories of moral concern}



See @1987CanLII5192 at para 48:
\begin{quote}
    **The guilty plea.** The court will usually impose a lesser sentence if the accused has pleaded guilty. The rational [sic] being that a criminal court can only function if it can induce the great mass of actually guilty defendants to plead guilty. The price it pays for this co-operation is leniency. In this case the plea of guilty was to a lesser charge than that contained in the original indictment.
\end{quote}

@youngDefensePleaBargainingPossible2013

Identifies and runs through the three categories of moral concern

Defendants who enter unequivocal guilty pleas without incentives needn't be counted, as there is no plea bargain at play

There will of course be defendants who receive a plea bargain without necessarily looking for one, but there's likely nothing controversial about this

The "moral concerns" arise when bargains are offered that threaten to override the defendant's free will, or otherwise improperly induce them to resolve rather than set their matters down for trial

Opponents of plea bargaining believe they can demonstrate that plea bargaining isn't necessary to run the justice system efficiently. That could be. But there may be other benefits to plea bargaining besides the efficiencies

Further, these thinkers argue that if it can be demonstrated that the justice system has previously been capable of operating without the administrative efficiencies won through plea bargaining, the belief that plea bargaining is necessary will be undermined.

Efficiencies can be of different types

It could be that, without plea bargaining, we have sufficient resources to give every potentially contestable matter a two hour trial. Or a one hour trial. Or however long.

Proponents of plea bargaining see guilty pleas as an age-old institution, while opponents of plea bargaining tend to view them as a more recent procedural compromise made for expedient justice.

I argue that the former position relies on the historical presence of guilty pleas to little actual effect. Guilty pleas and the institution of plea bargaining

One line of thought, common in reported cases and among legal professionals, is that guilty pleas have been a fixture of the common law since time immemorial.

Reported cases of such confessions were rare until the late 17th century. The common wisdom among academics and similarly-inclined jurists is that guilty pleas were generally frowned upon in English and American criminal law at this time and remained so until the mid-19th century. This shift in attitude is thought to coincide with a dramatic shift in judicial and legal attitudes towards procedural rights and fairness.

Trials by jury were thought to be the "safest test of justice"

They may very well have been at the time

But there are reasons to believe that it no longer is

There is reason to doubt this common wisdom. Albert W Alschuler bases his claim that confessions and guilty pleas were rarely used on the fact that they only infrequently appear in published decisions.

Notwithstanding the impression that our system has always insisted on viewing defendants as innocent until proven guilty, some of the most basic manifestations of this presumption aren't always obvious.

Entering a not guilty plea for a defendant who refuses to plead is a most basic manifestation of the presumption of innocence

But this feature didn't start to emerge in English law until 1827

A mature, comprehensive version of the doctrine is a relatively recent innovation

Albert W Alschuler acknowledges that a plea of "confession" was recorded and available "from the earliest days of the common law" (@alschulerPleaBargainingIts1979 at 7), but argues that it was infrequently used

Albert W Alschuler supports the argument that guilty pleas and their earlier analogues were rarely used by pointing to their infrequent appearances in medieval reported cases. This premise isn't convincing.

The fact that a phenomenon doesn't occur frequently in a group of reported cases does not imply that it doesn't occur frequently

Reported cases, by their nature, tend to deal with novel and contentious points of law, or high profile and public interest cases, or both

Multiple sources, including Albert W Alschuler, estimate that 90 - 97\%+ of all criminally charged defendants will enter a guilty plea, rather than opt for trial

Statistics in Canada, where available, corroborate the claim that almost all criminal charges resolve without the need for a trial, with only 3 - 5\% of cases proceeding to trial

2021 ONCJ Offence-based criminal offences](https://www.ontariocourts.ca/ocj/files/stats/crim/2021/2021-Offence-Based-Criminal.xlsx)

The statistics don't, however, necessarily corroborate the implication that 95 - 97\% of cases therefore resolve in a guilty plea

2021 ONCJ Offence-based criminal offences](https://www.ontariocourts.ca/ocj/files/stats/crim/2021/2021-Offence-Based-Criminal.xlsx)

3.1\% of all cases received went to trial

But only {{calc:(67210 / 212512 * 100)}}\% were resolved via a guilty plea

A full {{calc:(105838 / 212512 * 100)}}\% were withdrawn or stayed prior to trial

It's unlikely that a sample of reported decisions in North America would reflect this reality

\subsection{The spectre of wrongful convictions}

It goes without saying that factually innocent people shouldn't be convicted of crimes they didn't commit

Unfair result

Undermines confidence in the administration of justice

Leaves the factually guilty unpunished and undeterred

Concerns about wrongful convictions are live even when discussing run-of-the-mill plea bargaining

The added dimension of a no-contest plea, however, amplifies this concern

It's reasonable to assume that

A person unwilling to admit that they're guilty is more likely to be innocent than a person who admits their guilt; and

A person who actively protests their innocence is more likely to be innocent than a person who admits their guilt

The fact that equivocal no-contest pleas must typically be accompanied by a factual foundation is often overlooked when considering the problem of wrongful convictions, equivocal no-contest pleas, and plea bargaining more generally

Where a defendant cannot or will not unequivocally admit their guilt, or where they protest their innocence, nothing prevents Parliament from requiring the parties to undergo a comprehensive factual review

See 4.5.3: Customizing \textit{nolo contendere} and best-interest pleas for suggestions

Where both parties agree that the relevant legal facts have or will be proven beyond a reasonable doubt, a defendant's inability or unwillingness to formally plead guilty is less important

See @400us25 at 32 - 33:
\begin{quote}
    If Alford's statements were to be credited as sincere assertions of his innocence, there obviously existed a factual and legal dispute between him and the State. Without more, it might be argued that the conviction entered on his guilty plea was invalid, since his assertion of innocence negatived any admission of guilt, which, as we observed last Term in Brady, is normally "[c]entral to the plea and the foundation for entering judgment against the defendant . . . ." 397 U. S., at 748.
    
    In addition to Alford's statement, however, the court had heard an account of the events on the night of the murder, including information from Alford's acquaintances that he had departed from his home with his gun stating his intention to kill and that he had later declared that he had carried out his intention. Nor had Alford wavered in his desire to have the trial court determine his guilt without a jury trial. Although denying the charge against him, he nevertheless preferred the dispute between him and the State to be settled by the judge in the context of a guilty plea proceeding rather than by a formal trial. Thereupon, with the State's telling evidence and Alford's denial before it, the trial court proceeded to convict and sentence Alford for second-degree murder.
\end{quote}

It may be better for the defendant's soul to fully acknowledge their guilt and take responsibility for their actions, but that outcome isn't a prerequisite for justice being done

The moral case for wrongful convictions

@bowersPunishingInnocent2007a argument

The fact that these defendants are typically recidivists doesn't mean they aren't entitled to the same due process as a wrongfully accused defendant without a criminal record

But it does often mean that these two classes of defendants will expect and wish to exercise very different levels of due process

Recidivists more inclined to seek early resolution on the most favourable terms possible

Inexperienced defendants more inclined to want every avenue to acquittal explored

The actual jeopardy that a criminal conviction poses will be different for the recidivist than it does for the unexperienced defendant, so it's reasonable to expect that each would treat that factor differently

Weighing the harm of wrongful convictions against the harm of wrongful incarceration

The rhetoric around wrongful convictions is such that they're often framed as the worst possible outcome in the criminal justice system

But it's fair to question whether this is the case

Specifically, whether wrongful punishment (and specifically incarceration) is worse than a wrongful conviction

If it's accepted that some wrongful punishments are worse than some wrongful convictions, then it follows that wrongful convictions are not always the worst outcome in a criminal case

This opens up the possibility that being wrongfully convicted of a particular offence may be better than being wrongfully punished for it

Would it be worse to be wrongfully convicted of an offence and be absolutely discharged without having spent any time in custody, or be denied bail, spend a few months in pre-trial detention, and ultimately be acquitted?

Or perhaps be granted bail after spending just one month in pre-trial custody, and ultimately be acquitted?

Allowing a defendant in this situation to enter an equivocal no-contest plea would provide them a way to curb the harms of wrongful punishment and other pre-trial hardships

Distinct from a conditional plea, in that the wrongful conviction is accepted without any expectation that the conviction can or will be reversed after a future proceeding

Subjecting a wrongfully accused defendant to a greater punishment while waiting for trial than they would receive if convicted after trial is unjust

Requiring that person to further feign guilt and remorse for something they didn't do is also unjust

As discussed in 4.1.3: The apocryphal truth-seeking function of the trial, the idea that trials are a better truth-seeking mechanism than negotiations between counsel is likely misguided

Risk-adverse defendants or those not looking to clear their names are unlikely to want to wait for a trial if they can otherwise resolve their charges on less punitive terms

The specter of wrongful convictions is a live concern

But seeing plea bargaining and equivocal no-contest pleas as the cause of wrongful convictions misapprehends the matter

The biggest problems associated with wrongful convictions begin gestating long before the defendant is required to enter a plea

By the time a defendant is faced with having to enter a best-interest plea, something has irretrievably failed in the process

At this point, there is nothing morally wrong with allowing the defendant to try to mitigate future harms

Ideally no defendant would find themselves in this position

But where they are, having a means to more quickly get them out of it is better than having a principled stance derived from Blackstone's ratio

\subsection{The coercion worry}

Three distinct normative approaches commonly emerge in plea bargaining literature:

\begin{enumerate}
    \item coercion as restricting a defendant's rational choice and utility;
    \item coercion as requiring a defendant to make a choice with high stakes; and
    \item coercion as overriding a defendant's will by wrongfully influencing them
\end{enumerate}

A coercive choice can be rationally made
A choice can be rationally made in one's own self-interest but still be the product of coercion.

\begin{quote}
    In many cases, it may seem that accepting an offered plea is the only rational choice left for a defendant. Reflecting on this, a critic may try to understand coercion as consisting in an overwhelming amount of rational pressure in favor of a particular choice, namely, the choice to plea. But this approach also generally fails, on reflection, to provide an adequate account of coercion. Consider the example of a person faced with a choice between continuing a lackluster career and accepting his dream job. We do not think that that person is "coerced" into accepting the dream job offer just because that option has so many more reasons recommending it." But then coercion cannot be generally identified with the bare existence of disproportionately weighted reasons favoring some particular choice. A choice is not coerced simply because there were no better options, or simply because all other options were much worse from the chooser's perspective.
\end{quote}

Money to a mugger as an example of coercion

\begin{quote}
    If a mugger pulls a gun and credibly threatens "your money or your life," the victim does not fail to be coerced because he chooses to do the utility-maximizing, rational thing by handing over his wallet. Coercion and rational choice are not mutually exclusive possibilities, as the mugging example shows. But then one cannot sensibly point to the existence of rational choice in answer to a charge of coerciveness. Some given choice may have been rational. It could still, for all that, also have been coerced.
\end{quote}

Influences are coercive when they interfere with the freedoms a person ought to have

Identifying coercion is a more difficult task than the first two approaches seem to appreciate.
\begin{quote}
    Fundamentally, coercion is not about the amount or quantity of influence on a choice, but rather about the kind of influence on choice. On this view, even subtle influence could be coercive if it were an influence of a particular wrongful kind. As one plausible idea with support in the law and philosophy, an influence is of a wrongful kind if it interferes with the sort of positive freedom we think a person generally ought to have. So, for example, influencers that make a choice something less than an exercise of autonomous agency are suspect. Similarly, we might view as coercive any arrangement or situation that denies a person options that he ought to have, given an independently plausible view of his proper rights and dignity.
\end{quote}

A substantial gap between a plea bargained offer and a post-trial sentence recommendation suggests a high-stakes choice

Where a prosecutor offers a very low sentence on a guilty plea, as opposed to a comparatively very high sentence upon conviction after trial, they may be said to be engaging in hard dealing. To the extent that this tactic is likely to result in the defendant accepting the very low sentence offer, it's suggested that the defendant is coerced into making that choice.

\begin{quote}
    The critic of plea-bargaining typically has a rough-and-ready story that fits with this more plausible general conception of coercion. Especially, critics fear the supposed coercive pressure that results when defendants "are led to believe that they will receive longer sentences if they insist on going to trial and are subsequently found guilty."' The critics worry over the case where prosecutors succeed in inducing guilty pleas by threatening defendants with extra charges and (likely) extra punishment so that the defendants feel pressure to accept whatever plea is offered." In such cases, the critic might say, the defendant is confronted with a choice he should not have to face, and that alone makes the choice coerced.
    
    On this view, coercive pressure exists based solely on a large enough gap between (1) the prosecutor's threat of punishment at trial, and (2) the likely punishment given the defendant's acceptance of the prosecutor's offer and a plea. For convenience-and because the critic likely will not protest-let us label as "hard dealing" any and all cases marked by sufficiently large differences in punishment between threat and offer. (We will let the critic quantify "sufficiently large" however she likes.) Cast in this vocabulary, the critics' claim: "hard dealing" is intrinsically objectionable, constitutes the source of wrongful coercion in plea-bargaining, and, consequently, makes the institution of plea-bargaining morally suspect.
\end{quote}

Not all hard deals are coercive
A prosecutor who shows inordinate leniency in a case may create a sentencing gap sufficient to qualify as "hard dealing". But to the extent that the prosecutor is offering a more favourable choice than the defendant might normally be entitled to expect, it can't be reasonably said that the "hard deal" was coercive.

\begin{quote}
    [W]e sometimes think that there may be a range of sentencing options that are not coercive to the particular criminal defendant. Cases of prosecutorial leniency form the easiest example. Imagine an appropriately charged defendant facing a stiff sentence who is then offered, and accepts, a lenient slap-on-the-wrist plea deal. Whatever we want to say about this case, we presumably will not want to say that the defendant was coerced into accepting the deal. (What could be the source of the illicit coercive pressure in this scenario?) Yet, unless we are constructing the category of "hard dealing" to beg the question-simply as an alternate form of words for the ultimate conclusion that plea-bargaining is coercive-the sentencing gap between the appropriate stiff sentence and the wrist-slap could be large enough to meet any parameters that the critic may set for "hard dealing." If so, then there may be a case of "hard dealing" that is not coercive. But then it follows that "hard dealing" is not necessarily coercive and so it cannot be "hard dealing" as such that constitutes coercive pressure.
\end{quote}

\subsection{The trial penalty}

\subsubsection{What is the trial penalty?}
The trial penalty criticism charges that, because defendants who opt for a trial rather than a plea bargain are likely to be sentenced more harshly if convicted, they can be reasonably said to suffer a "trial penalty" for this decision
\begin{quote}
    Few criminal defendants would plead guilty absent a credible chance of reducing their expected sentences. Plea-bargaining thus depends on the possibility of a range of sentencing options, and on differential treatment for plea-bargaining and non-plea-bargaining defendants. This effectively means that not all criminal defendants will be treated alike, and that non-plea-bargaining defendants will be (or, if declining a plea-bargain, can reasonably expect to be) sentenced relatively more harshly. Distinct from the coercion worry, the critic thinks that this differential treatment betrays a troubling lack of concern for equality among similarly situated criminal defendants, and is a violation of the ancient and venerable maxim of justice to treat like cases alike. The critic charges that plea-bargaining exacts a "trial penalty."
\end{quote}

\subsubsection{Trial penalty vs plea bargain}

Rather than seeing an increased sentence following a trial as a "trial penalty", one can simply view the sentence negotiated without a trial as a "plea bargain"

But Michael (III) Young argues that doing so fails to answer the broader criticism of the trial penalty:

\begin{quote}
    Attempting to answer this criticism, plea-bargaining's economic defenders might suggest that the sentencing difference between plea-bargaining and non-plea-bargaining defendants reflects a mere benefit for electing to plea-bargain, as opposed to a penalty for electing trial. Such a response, if offered, would really fail to answer the critic: the critical judgment that the harsher, non-plea-bargained sentence is an objectionable "penalty" just depends (the critic will say) on the plausible demand of fairness or equality to treat like cases alike and the sense that, in the case of plea-bargaining, this moral demand is being ignored. In fact, the critics' use of the word "penalty" is a label that could be wholly avoided in expressing the claim that plea-bargaining involves unfairness in sentencing; the point is meant to go deeper than the label.
\end{quote}

The trial penalty criticism is effectively a criticism of the "unequal" treatment an offender will receive if they opt to set their matter down for trial
Engaging with this critique should therefore focus on whether a sentence discount prior to trial constitutes unequal treatment

\subsubsection{Mitigation for contribution to the common good}

One may reasonably argue that a defendant who opts for a plea bargain spares the state the burden of proving its case

In this *quid pro quo* scenario, the defendant makes concessions in exchange for certain benefits, rendering them dissimilar to those defendants who don't do so:

\begin{quote}
    [A] defender of plea-bargaining might instead suggest that plea-bargaining defendants typically deserve a downward departure in sentencing by reason of contributing to the public good in saving public resources on an expensive prosecution. This response suggests that the two cases (plea-bargaining and non-plea-bargaining) are relevantly dissimilar: one case presents a defendant morally deserving a downward sentencing departure and the other does not. And surely, the defense goes, moral desert is a consideration in the balance of justice-and criminal sentencing-if anything is.
\end{quote}

Michael (III) Young argues that prosecutorial resources and prison time are incommensurate commodities that should not have anything to do with one another

\begin{quote}
    This defense raises some puzzles. Do we think we have an in-principle way of measuring the appropriate amount of downward sentencing departure given a plea-bargaining defendant? Are the quantities in question-prosecutorial resources and prison time-even commensurable? Would we say, for example, that prosecutorial effort at trial on the order of hundreds (or even thousands) of hours could be worth many years off of a sentence for a felony conviction?
\end{quote}

Michael (III) Young  argues that intermingling the two can lead to perverse results

\begin{quote}
    Perhaps, though, it is premature to assume that these questions could not be answered. Still, there seems to be something fundamentally objectionable about strongly tying the idea of deserved punishment to the idea of contributing to the common good, whether by saving public resources on a prosecution or in any other way. Simply stated, the one should not have anything to do with the other. It would be perverse, for example, to think that a rich criminal who donated money to the prosecutor's office-thus advancing the common good by providing resources for prosecutions-would even presumptively deserve a sentencing reduction for that reason alone. Or, suppose that through some lucky causal accident the commission of some crime also brings about collateral social effects contributing positively to the common good: we would not think that this fortunate fact properly weighed as a sentencing consideration. So it cannot be the case that, in general, contributing to the public good constitutes a reason for a reduction in sentence. We should reject the idea that the plea-bargaining criminal deserves a lesser sentence merely for having contributed to the public good by saving public prosecutorial resources.
\end{quote}

I don't know that I share Michael (III) Young's intuitions about there being something fundamentally objectionable about tying the idea of deserved punishment to the idea of contributing to the common good

Not only can we place a value on time spent in custody, we have and often do

I'm also not sure that I agree that the results he points to are really all that perverse

Criminal law "rewards" moral luck all of the time, and it stands to reason that it would (and should) do so where a crime committed causes less harm than it otherwise would have

It also stands to reason that things like contributions to the common good would be encouraged, to the extent that CC 718 includes the objective of providing reparations for harm and promoting a sense of responsibility in the offender

Beyond the **common good**, a defendant deciding to plea bargain also contributes to the **specific good** of those involved in their case

In particular, complainants who are spared from having to testify

Also complainants who want some certainty in the outcome, orders for compensation for loss or damage sooner than later, etc

\subsubsection{Placing a value on time in custody}

Michael (III) Young notes the difficulty in comparing time spent in custody with time saved on prosecuting a file, and observes that these may not be commensurable comparisons

\begin{quote}
    This defense raises some puzzles. Do we think we have an in-principle way of measuring the appropriate amount of downward sentencing departure given a plea-bargaining defendant? Are the quantities in question-prosecutorial resources and prison time-even commensurable? Would we say, for example, that prosecutorial effort at trial on the order of hundreds (or even thousands) of hours could be worth many years off of a sentence for a felony conviction?
\end{quote}

In Canada, the Criminal Code actually provides a calculation that effectively allows us to infer a price on time spent in custody:

\begin{quote}
    **CC 734: Power of court to impose fine**
    
    **(4)** Where an offender is fined under this section, a term of imprisonment, determined in accordance with subsection (5) [fines], shall be deemed to be imposed in default of payment of the fine.
    
    **Determination of term**
    
    **(5)** The term of imprisonment referred to in subsection (4) [fine – imprisonment in default of payment] is the lesser of
    
        **(a)** the number of days that corresponds to a fraction, rounded down to the nearest whole number, of which
        
            **(i)** the numerator is the unpaid amount of the fine plus the costs and charges of committing and conveying the defaulter to prison, calculated in accordance with regulations made under subsection (7) [power of province to make regulations re fines], and
            
            **(ii)** the denominator is equal to eight times the provincial minimum hourly wage, at the time of default, in the province in which the fine was imposed, and
        
        **(b)** the maximum term of imprisonment that the court could itself impose on conviction or, if the punishment for the offence does not include a term of imprisonment, five years in the case of an indictable offence or two years less a day in the case of a summary conviction offence.

\end{quote}

May also be expressed as
\begin{equation*}
sentence =\begin{cases}
x_1 \quad  \iff \frac{fine + costs}{minWage \times 8} < (maxSentence \lor 5/2y) \\
x_2 \quad \iff (maxSentence \lor 5/2y) < \frac{fine + costs}{minWage \times 8}  \\
\end{cases}
\end{equation*}

Similarly, for defendants who default on a repayment ordered for laundering the proceeds of crime, the Criminal Code assigns a monetary value to time spent in custody

**CC 462.37: Order of forfeiture of property**

**(4): Imprisonment in default of payment of fine**

Where a court orders an offender to pay a fine pursuant to subsection (3), the court shall

(a) impose, in default of payment of that fine, a term of imprisonment

(i) not exceeding six months, where the amount of the fine does not exceed ten thousand dollars,

(ii) of not less than six months and not exceeding twelve months, where the amount of the fine exceeds ten thousand dollars but does not exceed twenty thousand dollars,

(iii) of not less than twelve months and not exceeding eighteen months, where the amount of the fine exceeds twenty thousand dollars but does not exceed fifty thousand dollars,

(iv) of not less than eighteen months and not exceeding two years, where the amount of the fine exceeds fifty thousand dollars but does not exceed one hundred thousand dollars,

(v) of not less than two years and not exceeding three years, where the amount of the fine exceeds one hundred thousand dollars but does not exceed two hundred and fifty thousand dollars,

(vi) of not less than three years and not exceeding five years, where the amount of the fine exceeds two hundred and fifty thousand dollars but does not exceed one million dollars, or

(vii) of not less than five years and not exceeding ten years, where the amount of the fine exceeds one million dollars; and

(b) direct that the term of imprisonment imposed pursuant to paragraph (a) be served consecutively to any other term of imprisonment imposed on the offender or that the offender is then serving.

May also be expressed as:

\begin{equation*}
		  sentence = \begin{cases}
		             \le \text{6mo} \quad  \iff fine \le \text{\$10,000} \\
		            [\text{6mo, 12mo}] \quad \iff fine = [\text{\$10,000, \$20,000}]\\
		            [\text{12mo, 18mo}] \quad \iff fine = [\text{\$20,000, \$20,000}]\\
		            [\text{18mo, 2y}] \quad \iff fine = [\text{\$50,000, \$100,000}]\\
		            [\text{2y, 3y}] \quad \iff fine = [\text{\$100,000, \$250,000}]\\
		            [\text{3y, 5y}] \quad \iff fine = [\text{\$250,000, \$1,000,000}]\\
		            [\text{5y, 10y}] \quad \iff fine > \text{\$1,000,000}\\
		       \end{cases}
		  \end{equation*}

Assuming that prosecutorial effort at trial can also be quantified, even just by calculating their wage divided by their time spent on the file (and ignoring all of the other expenses associated with a prosecution), these are statutorily commensurate standards of measure

It would seem much more difficult to try to quantify how much custody is needed to accomplish the substantive goals of sentencing
\begin{itemize}
    \item Denounciation
    \item Deterrence
    \item Promoting a sense of responsibility
    \item Restoring the community
    \item Rehabilitation
\end{itemize}

\subsubsection{Rethinking equal treatment in sentencing}

Although I disagree with Michael (III) Young regarding the feasibility and appropriateness of quantifying deserved punishment in terms of prosecutorial resources, I agree that it is appropriate to reconsider what constitutes "equal treatment" between defendants in sentencing.

Michael (III) Young takes issue with what can be called a naive view of inequality, wherein two identically situated defendants who are charged with identical crimes premised on identical facts should not be sentenced differently if one opts to take their matter to trial, while the other opts to self-convict

\begin{quote}
    One implication of the proper equality-as-non-subordination view is that, in general, a person is not treated unfairly by differences for which he himself is responsible. In such a case, there is no person or institution to accuse of doing the dominating except one's own self; but such self-accusation is an absurdity.
\end{quote}

This implication presents a burden for the "trial penalty" critic who is committed to saying that the differential sentencing of defendants in a system that allows plea-bargaining always or necessarily reflects the inequitable subordination of the (more harshly sentenced) convicted, nonplea-bargaining defendant. To credibly maintain this position, the "trial penalty" critic must be strongly committed to denying that the defendant who declines to plea-bargain, is convicted, and faces a harsher sentence could be responsible for that sentencing outcome, even under ideal conditions. That is, the "trial penalty" critic must be strongly committed to denying that the sentencing outcome could reflect a choice for which the defendant is properly responsible. If, pace the critic, the defendant could be responsible for his choice to elect trial instead of a plea-bargain, then his choice could be free of a concern for inequality or "penalty." But this necessary critical commitment seems too strong. Without some independent reason for doing so, we should not want to insist in advance that there are no conditions under which a defendant could be responsible for the choice whether or not to plea-bargain and that choice's predictable sentencing consequences.

Assuming that similarly situated defendants have equivalent access to plea bargain offers, Michael (III) Young argues that those who turn them down can and should be held responsible for the consequences of their choice to do so

\begin{quote}
    Moreover, it seems that we can imagine that a defendant might be responsible for his choices about plea-bargaining. The following hypothetical suggests, in a slightly more vivid way, that a defendant in fact might be responsible for such a choice and its consequences: (1) suppose that, apart from the concern for equality, we have no reason to view a system that presents defendants with a choice between bargain and trial as necessarily unjust (and, after all, the critic cannot just assume plea-bargaining's injustice in an independent argument meant to show its special injustice); (2) suppose that the choice whether or not to plea-bargain is free-uncoerced-and that its consequences are reasonably predictable; (3) suppose that similarly situated defendants are given similar offers; 142 and (4) suppose that natural justice is not offended by any sentence that will be imposed on either plea-bargaining or convicted non-plea-bargaining defendants, so that no sentence for any given defendant will strike us as either too harsh or too lenient given the demands of justice as understood apart from whatever it is that the positive law prescribes or allows. Under conditions such as these, I think it is tempting to say that the defendant who rejects a plea-bargain, is convicted, and receives a stiffer punishment cannot justly complain that he was subordinated or dominated by a system that failed to respect the value of equality.
\end{quote}

Michael (III) Young suggests that defendants should be and are in fact aggrieved when the offers they receive are worse than those made to similarly situated defendants

It's true that this is the case when it occurs

But the problem with this is that it can be very difficult to decide who is a similarly situated offender when it comes time to sentence them

\subsubsection{Difficulty assessing equality between similarly situated defendants}

As Michael (III) Young discusses, the trial penalty criticism is grounded in the legal fiction that equally situated defendants charged with substantively the same offence should be handed the same punishment

The difficulty in putting this fiction into practice is in determining what constitutes both an equally situated offender and a substantively similar offence

To the extent that no two offences or offenders are ever exactly alike, "similar offences" and "similarly  situated offenders" will always be distinguishable to some degree

The incredible array of subtle distinctions between offences and offenders can either be overlooked or microscopically examined, depending on things like

The offence

The offender's history

Jurisdiction the offence occurred in

The judge deciding the disposition

How these factors are interpreted can be very imprecise

Evidenced by the seemingly endless parade of sentencing ranges, guidelines, cases, and so forth

\subsubsection{Conclusion}

Defendants who opt to self-convict, rather than take their chances at a trial, ought to have this sacrifice recognized

A concept like equal treatment of similarly situated defendants cannot reasonably ground a criticism of plea bargaining

\subsection{The apocryphal truth-seeking function of the trial}

Because defendants who are unwilling or unable to offer a guilty plea must set their matters down for trial, a trial may be considered the ``default mode" for resolving criminal disputes. Because trials take place when parties disagree with one another, they are often branded as a ``search for the truth." Each party has the opportunity to present its best evidence and its best interpretation of the evidence. Witnesses are cross-examined on their evidence, identifying the strengths and weaknesses of their testimony, and a neutral trier hears the facts, listens to both sides argue, and decides between them.

By contrast, non-inculpatory no-contest pleas appear to be designed to obfuscate or misrepresent the truth for the purpose of getting negotiated agreements through the courts. Negotiations between prosecutors and defence lawyers are privileged and inadmissible as evidence. There are no judges or juries to provide neutral adjudication.\footnote{This is not necessarily true in all cases. In many jurisdictions, contested matters may go through case management with a judge in order to discuss pre-trial issues, canvass resolution, and offer advice on how to best proceed. }

Defendants can privately (or publicly) resile their admissions without being contradicted by evidence heard at a public trial

But the function of a common-law criminal trial is not to "search for the truth." Rather, the function of a common-law criminal trial is to allow the trier of fact to make decisions based on curated sets of facts, within the confines of a restrictive sets of rules. There are important differences between these two concepts. 

Seeing a trial as a "search for the truth" implies that truth is the primary function of the trial, which subsequently implies that truth is the value in a trial that all other values are subservient to. A public inquest into a social problem, like the Truth and Reconciliation hearings in ZA, or missing and murdered indigenous women in Canada, would be comparable examples of this sort of "no holds barred" approach to the truth. 

Trials, on the other hand, require the fact finder to consider a less complete and arguably less true version of the evidence, as in cases where evidence has been excluded. In criminal trials, the "truth-seeking function" of the trial is always counterbalanced by the need to ensure fair proceedings. Truth is, in fact, severely curtailed at trial by things like what truth the trier of fact is allowed to hear, excluded evidence, excluded testimony, the results of pre-trial and other evidentiary motions, testimony from witnesses who were not called or did not attend, hearsay rules, opinion evidence rules, and so on. Even once the evidence is admitted, special instructions are often needed to ensure that the evidence is interpreted properly. 

Practically speaking, a great deal of truth that could be presented at trial is often excluded simply for the sake of a coherent presentation

In all cases, but especially complex ones, lawyers looking to present their evidence to a judge or jury will regularly cut out large swaths of the available evidence

A lawyer's job at a common law criminal trial is to advocate for a position

Effectively advocating for a position requires the advocate to account for the frailties of human attention and comprehension

This in turn requires both extensive editing of the truth and careful framing of that truth within the context of a larger argument

Information at trial is highly pre-processed before it reaches the court, and needs to be further processed once it gets there. This ensures both procedural fairness and effective presentation of the truth.

Procedural fairness and presentation are key trial concerns because the truth-seeking mechanism is unreliable

The "finders of fact" at trial are no more implicitly qualified at "seeking truth" than anyone else

In the case of jurors, the "finders of fact" are, in effect, "anyone else"

Experience

The hope and expectation is that the fact finders will come to true and fair conclusions through the combined presentation of true and fair evidence

But this isn't guaranteed

Rules have been put in place following decades of learning about the frailties of evidence

DNA evidence created a new class of cases for courts to review

The theory is that certain convictions could be absolutely confirmed or disconfirmed based on how DNA tests came back

Where wrongful convictions were found, we were able to gain some insights into the types of evidence that led jurors to those conclusions

Other forensic experiments in other fields have confirmed the unreliability of certain types of evidence

Eyewitness evidence

Shortcomings in memory

Etc

More confirmed wrongful convictions appear to have resulted from trials rather than negotiated sentences

But until we have new breakthroughs that can offer insight into how well or poorly forensic evidence reflects the truth, there isn't another independent truth-seeking function at play

Other truth-seeking mechanisms, like resolution discussions and plea bargaining, are likely better suited for reaching outcomes that are both true and fair

Both (sets of) lawyers in a criminal action have intimate familiarity with the facts and nuances of the case. They represent the different interests in the action - or at least, the interests that have standing in the action. Assuming ethical actors with access to relatively complete and accurate information, any resolution worked out between the parties by consent is likely to be close to a true and just result.

This also assumes there's some *quid pro quo* between the parties in their resolution discussions, such that the evidence is already being "tested", in a sense

Where a resolution can't otherwise be reached, trials are a reasonably effective means of adjudicating disputes, ensuring all sides have their best explanation of the evidence heard, and deferring to an ostensibly neutral third party to decide between competing narratives.

But trials are probably better understood as adequate alternatives to dispute resolution, rather than a preferred means of dispute resolution

Negotiations between counsel, both of whom have legal training, are duty-bound to advocate for their positions, have access to the fullest permissible truth of a criminal matter, and the opportunity to spend extensive time and resources investigating the nuances thereof, are a much greater truth-seeking device than a criminal trial.