\section{Types of no-contest pleas}

Any plea that admits the state's case against a defendant withour requiring the state to prove its allegations qualifies as a no-contest plea. What distinguishes these pleas from each other are the admissions they ostensibly ask defendants to make: 

\begin{itemize}

    \item \textbf{Inculpatory no-contest pleas.} A defendant who enters a sincere guilty plea formally admits that the allegations are proven and that they are responsible for the offences alleged inculpates themselves. This no-contest plea is \textit{inculpatory} because the defendant accepts that they are factually guilty.
    
    \item \textbf{Non-culpatory no-contest pleas.} A defendant who enters a \textit{nolo contendere} plea accepts the consequences of the crime alleged but neither admits nor denies their actual involvement in the offence. This no-contest plea is \textit{non-culpatory} because the defendant who enters it neither acknowledges nor repudiates responsibility.

    \item \textbf{Exculpatory no-contest pleas.} A defendant who enters a best-interest plea accepts both the consequences of a conviction for the alleged crime and that the offence would have been proven beyond a reasonable doubt if taken to trial but formally maintains their innocence. These pleas are exculpatory to the extent that the defendants who enter them actively protest their innocence. \textit{Exculpatory no-contest} pleas like \textit{Alford} introduce a number of interesting questions related to those asked and answered in this thesis, but will not be considered in detail.\footnote{Although these ``pleas" have both emerged and been authorized in places like America and England, Canadian courts have not explicitly considered them as such. While some Canadian decisions come out very strongly against the notion of a defendant pleading guilty while protesting their innocence, the proper procedure for withdrawing a guilty plea does not even consider the factual innocence problem. No Canadian decisions have come out overtly in favour of best-interest pleas as \textit{Alford} did. However, appellate decisions like \textit{R v Hector} have employed similar reasoning.} 
    
\end{itemize}

\subsection{Guilty pleas}

Since the first \textit{Criminal Code of Canada, 1892}, criminal defendants could opt to self-conviction rather than take their charges to trial.\footnote{See \textit{Criminal Code 1892} s 657.} This early provision did not limit a defendant's ability to enter a guilty plea and did not give judges discretion to accept or reject it. Over time, Canadian common law limited a defendant's ability to plead guilty and stipulated that judges may only accept knowing, voluntary, and unequivocal guilty pleas. In Canada, the \textit{Criminal Code} requires that defendants enter their guilty pleas knowingly and voluntarily, while Canadian common law requires that they enter their guilty pleas unequivocally.\footnote{This common law requirement was eventually codified as \textit{Criminal Code} s 606(1.1) in 2002.}

In Canada, the most frequently used no-contest plea is a simple guilty plea.\footnote{In Canada and the United States, courts deal with criminal allegations through guilty pleas far more frequently than through trials. See e.g. https://www.ontariocourts.ca/ocj/files/stats/crim/2021/2021-Offence-Based-Criminal.xlsx} Courts typically accept guilty pleas as a mitigating sentencing factor. While the general contours of a guilty plea are similar across jurisdictions, the specifics will often vary. For example, courts recognize that defendants must plead guilty knowingly and voluntarily. However, the specific conditions that must be satisfied to meet these standards vary.\footnote{See §2.3.2 below for a more detailed discussion of these pleas and their characteristics.} 

\subsection{\textit{Nolo contendere} pleas}

Finally, \textit{nolo contendere} pleas admit the prosecution's case without admitting factual guilt or the state's procedural compliance. In the United States, \textit{nolo contendere} pleas are \textit{formally} recognized and accepted in most states. In Canada, an \textit{informal} \textit{nolo contendere} ``plea procedure" has recently been authorized by appellate courts in several jurisdictions. 

By entering a \textit{nolo contendere} plea, a defendant signals to the court that they are not contesting the allegations against them. In this sense, a \textit{nolo contendere} plea is comparable to a guilty plea, as the defendant invites the court to convict them, but differs in that it does not formally admit the truth of those charges. By contrast, a defendant entering a formal \textit{nolo contendere} plea does not overtly admit the truth of those charges. While formal and informal \textit{nolo contendere} share many key traits, some differences distinguish them from or liken them to other pleas. Generally, these differences arise in the grey areas between the interacting legislative provisions that make informal pleas possible.