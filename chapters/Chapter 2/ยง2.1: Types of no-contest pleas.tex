\section{Types of pleas}

Any plea that admits the state's case against a defendant without requiring it to prove the allegations qualifies as an \textit{uncontested} plea. Conversely, any plea that does not admit the state's case against a defendant and requires proof of some or all of the allegations qualifies as a \textit{contested} plea. Both contested and uncontested pleas may be \textit{inculpatory}, \textit{exculpatory}, or \textit{non-culpatory}. ``Non-culpatory" may also be read as ``not-inculpatory \textit{and} not-exculpatory." Where an inculpatory plea admits responsibility and an exculpatory plea denies responsibility, non-culpatory pleas do neither. Uncontested pleas may be contrasted with contested pleas. By entering a contested plea, defendants advise that the prosecutor must prove some or all of their allegations. Conversely, by entering an uncontested plea, defendants advise that the prosecutor does not have to prove any of the charges against them. 

\subsection{Contested pleas}

Pleas that require the prosecutor to prove some or all of their charges are \textit{contested} pleas. A defendant who admits responsibility for the offence charged but disagrees with some or all of the prosecutor's specific allegations enters an \textit{inculpatory contested} plea. Where a defendant actively contends that they are not guilty of the charged offence and sets the matter for a hearing, they enter an \textit{exculpatory contested} plea. Finally, defendants who neither admit nor deny their guilt or innocence but set their charges for a hearing enter \textit{non-culpatory contested} pleas.

\begin{itemize}
    \item \textbf{Inculpatory contested: \textit{Gardiner} hearing.} A defendant entering a contested plea may wish to admit guilt but not agree to all of the prosecutor's allegations. These defendants may enter guilty pleas to the formal charge and require the prosecutor to prove any contested allegations and aggravating factors.\footnote{See \textit{Criminal Code} s 724(3).} These procedures are often referred to as \textit{Gardiner} hearings,\footnote{See} named after the Supreme Court of Canada decision in \textit{Gardiner}.\footnote{\textit{R v Gardiner}, 1982 CanLII 30 (SCC), [1982] 2 SCR 368 (\textit{Gardiner})}
    
    \item \textbf{Inculpatory contested: Conditional plea.} A defendant may want to argue that some of the evidence in their case should be ruled inadmissible, but would be prepared to plead guilty if the court disagreed. Conditional pleas are a means for defendants to do so. These pleas are codified in several American states. They are formally unavailable in Canada, but may be constructively entered through specially-designed plea agreements.\footnote{See \textit{R v Fegan}, 1993 CanLII 8607 (ON CA) (\textit{Fegan}).}
    
    \item \textbf{Exculpatory \& non-culpatory contested: Not guilty plea.} Where defendants in Canada maintain their innocence or refuse to plead one way or the other, judges must enter not guilty pleas on their behalf and set their charges down for trial.\footnote{See \textit{Criminal Code} s 606(2).}
    
\end{itemize}

\subsection{Uncontested pleas}

Defendants self-convict without requiring the prosecutor to prove any part of their case enter \textit{uncontested pleas}.\footnote{This definition only applies to the allegations that the prosecutor relies on in support of the charge. It does not extend to sentencing submissions. A defendant who concedes all of the prosecutor's allegations but disagrees with their sentencing recommendation nonetheless still enters an uncontested plea.} A defendant who admits responsibility for an offence and does not require the prosecutor to prove their allegations enters an \textit{inculpatory uncontested} plea. Where a defendant self-convicts but nonetheless maintains their innocence, they enter an \textit{exculpatory uncontested} plea. Finally, defendants who self-convict without admitting or denying responsibility enter \textit{non-culpatory uncontested} pleas. 

\begin{itemize}

    \item \textbf{Inculpatory uncontested: Guilty plea.} A defendant who enters a sincere guilty plea formally admits that the allegations are proven and that they are responsible for the offences alleged inculpates themselves. This uncontested plea is \textit{inculpatory} because the defendant accepts that they are factually guilty.
    
    \item \textbf{Non-culpatory uncontested: \textit{Nolo contendere} plea.} A defendant who enters a \textit{nolo contendere} plea accepts the consequences of the crime alleged but neither admits nor denies their actual involvement in the offence. This uncontested plea is \textit{non-culpatory} because the defendant who enters it neither acknowledges nor repudiates responsibility.

    \item \textbf{Exculpatory uncontested: best-interest/\textit{Alford} pleas.} A defendant who enters a best-interest plea accepts both the consequences of a conviction for the alleged crime and that the offence would have been proven beyond a reasonable doubt if taken to trial but formally maintains their innocence. These pleas are exculpatory to the extent that the defendants who enter them actively protest their innocence. \textit{Exculpatory uncontested} pleas like \textit{Alford} introduce a number of interesting questions related to those asked and answered in this thesis, but will not be considered in detail.\footnote{Although these ``pleas" have both emerged and been authorized in places like America and England, Canadian courts have not explicitly considered them as such. While some Canadian decisions come out very strongly against the notion of a defendant pleading guilty while protesting their innocence, the proper procedure for withdrawing a guilty plea does not even consider the factual innocence problem. No Canadian decisions have come out overtly in favour of best-interest pleas as \textit{Alford} did. However, appellate decisions like \textit{R v Hector} have employed similar reasoning.} 
    
\end{itemize}