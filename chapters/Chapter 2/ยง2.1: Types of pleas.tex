\section{Types of pleas}

In criminal cases, a plea is a type of proposition whose primary purpose is to tell the court what evidence a defendant will require the prosecutor to prove. On this primary metric, pleas may be uncontested or contested. Any plea that admits the state's case without requiring it to prove allegations against a defendant is an \textit{uncontested plea}. Conversely, any plea that does not admit the state's case against a defendant and requires proof of some or all of the allegations is a \textit{contested plea}. Throughout this thesis, I refer to the plea's contested/uncontested aspect as its \textit{proof-content}. In Canada, a \textit{guilty plea} is an example of an uncontested plea. Defendants in Canada who enter this plea advise the court that they admit all of the offence's essential elements as proven and that a trial is unnecessary. Conversely, a \textit{not guilty plea} is an example of a Canadian contested plea. Defendants who enter this plea advise the court that all of the offence's essential elements must be proven unless otherwise agreed and that a trial is necessary.

Although the plea's primary purpose is to convey its proof-content, pleas also contain other propositional values. In normal parlance, when a person says they are either guilty or not guilty of something, they are understood to be conveying some subjective meaning about what they \textit{believe to be true}, some objective meaning about what \textit{is, in fact, true}, or both. I refer to these residual aspects as a plea's \textit{belief-content} and \textit{truth-content}, respectively. Broadly categorized, a criminal plea's belief/truth-content may be \textit{inculpatory}, \textit{exculpatory}, or \textit{non-culpatory}, which may also be read as \textit{not-inculpatory and not-exculpatory}.\footnote{Hawkins distinguished guilty and \textit{nolo contendere} pleas by describing one as an ``express confession" and the other as an ``implied confession," respectively. See Willaim Hawkins, \textit{A Treatise of the Pleas of the Crown: Or, A System of the Principle Matters Relating to that Subject, Digested Under Proper Heads}, vol 2, 8th ed by John Curwood (London, UK: Sweet, 1824) at 466.} While inculpatory pleas admit responsibility and exculpatory pleas deny responsibility, non-culpatory pleas do neither. Both contested and uncontested pleas may be inculpatory, exculpatory, or non-culpatory.
    
\subsection{Contested pleas}

Pleas that require the prosecutor to prove some or all elements of their allegations are \textit{contested}. A defendant who admits responsibility for the offence charged but disagrees with some or all of the prosecutor's specific allegations enters an \textit{inculpatory contested} plea. Where a defendant actively denies guilt and sets the matter for a hearing, they enter an \textit{exculpatory contested} plea. Finally, defendants who neither admit nor deny their guilt or innocence but set their charges for a hearing enter \textit{non-culpatory contested} pleas.

\begin{itemize}
    \item \textbf{Inculpatory contested: \textit{Gardiner} hearings.} A defendant entering a contested plea may wish to admit guilt but not agree to some or all of the prosecutor's specific allegations. For example, a defendant is charged with assaulting their domestic partner. The defendant may admit the assault, which is the offence, but deny that the complainant is their domestic partner, a statutorily aggravating factor.\footnote{See \textit{Criminal Code}, \textit{supra} note 2, s 718.2(a)(ii).} This defendant and others similarly situated may enter guilty pleas to the formal charge and require the prosecutor to prove any contested allegations and aggravating factors.\footnote{See \textit{ibid}, s 724(3).} These procedures are often referred to as \textit{Gardiner} hearings,\footnote{See Criminal Notebook, ``Sentencing Evidence" (January 2020) online: \textless \url{http://criminalnotebook.ca/index.php/Sentencing_Evidence}\textgreater} named after the Supreme Court of Canada decision in \textit{Gardiner}.\footnote{\textit{R v Gardiner}, 1982 CanLII 30 (SCC), [1982] 2 SCR 368, 140 DLR (3d) 612 [\textit{Gardiner}].}
    
    \item \textbf{Inculpatory contested: Conditional pleas.} A defendant may want to argue that some evidence in their case should be ruled inadmissible but would plead guilty if the court disagreed. Conditional pleas are a means for defendants to do so. Several American states have codified these pleas, and although they are formally unavailable in Canada, defendants may constructively enter them through specially-designed plea agreements.\footnote{See \textit{R v Fegan}, (1993) 13 OR (3d) 88, 80 CCC (3d) 356 [\textit{Fegan}].}
    
    \item \textbf{Exculpatory \& non-culpatory contested: Not guilty pleas.} Where defendants in Canada maintain their innocence or refuse to plead one way or the other, the court must enter not guilty pleas and set their charges down for trial.\footnote{See \textit{Criminal Code}, \textit{supra} note 2, s 606(2).} 
    
\end{itemize}

\subsection{Uncontested pleas}

Defendants who self-convict without requiring the prosecutor to prove any part of their case enter \textit{uncontested pleas}.\footnote{This definition only applies to the allegations that the prosecutor relies on to support the charge. It does not extend to sentencing submissions. A defendant who concedes all of the prosecutor's allegations but disagrees with their sentencing recommendation still enters an uncontested plea.} A defendant who admits responsibility for an offence and does not require the prosecutor to prove their allegations enters an \textit{inculpatory uncontested} plea. Where a defendant self-convicts but maintains their innocence, they enter an \textit{exculpatory uncontested} plea. Finally, defendants who self-convict without admitting or denying responsibility enter \textit{non-culpatory uncontested} pleas. 

\begin{itemize}

    \item \textbf{Inculpatory uncontested: Guilty plea.} A defendant who enters a guilty plea formally admits that the allegations are proven and that they are responsible for the offences alleged inculpates themselves. This uncontested plea is \textit{inculpatory} because the defendant subjectively conveys that they are objectively guilty of the offence.

    \item \textbf{Exculpatory uncontested: best-interest/\textit{Alford} pleas.} A defendant who enters a best-interest plea accepts the consequences of a conviction and acknowledges that the prosecutor would have proven the offence if taken to trial but formally maintains their innocence. These pleas are \textit{exculpatory} as the defendants who enter them actively protest their innocence despite inviting the court to convict them.
    
    \item \textbf{Non-culpatory uncontested: \textit{Nolo contendere} plea.} A defendant who enters a \textit{nolo contendere} plea accepts the consequences of the crime alleged but neither admits nor denies their actual involvement in the offence. This uncontested plea is \textit{non-culpatory} because the defendant who enters it neither admits subjective belief nor discusses objective truth, and thus neither acknowledges nor repudiates responsibility.
    
\end{itemize}

The overarching question in this thesis is whether Canadian criminal law should formally adopt \textit{nolo contendere} pleas. As a result, the main focus throughout my thesis will be on non-culpatory uncontested pleas. But to properly understand \textit{nolo contendere} pleas, it is helpful first to consider the role, function, and conceptual basis of the more familiar guilty plea. By understanding the common law and statutory substructure undergirding the most common Canadian uncontested plea it will become apparent how \textit{nolo contendere} pleas fit in this framework.