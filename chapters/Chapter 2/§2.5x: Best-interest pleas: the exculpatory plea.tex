\section{Best-interest pleas}

A best-interest plea is one in which a defendant admits the prosecution’s case against them, either by way of a guilty plea or an agreed statement of facts or both, despite maintaining their innocence, or having a claim that their procedural rights were violated, or both. Best interest pleas are presumably entered for the purpose of securing an advantageous plea deal in the face of overwhelming evidence of guilt, such that it is in a defendant’s best interests to resolve the charge or charges rather than contesting them at trial. This presumption stems from the notion that a defendant who either believed they were innocent or had a legitimate claim of a procedural irregularity wouldn't invite a conviction unless there was something that made it in their best interests to do so. The conditions for a best-interest plea to emerge are met when a defendant is faced with severe consequences if convicted, little reasonable prospect for an acquittal, and the prosecutor is willing to resolve for a substantially mitigated sentence. The conditions for a best-interest plea may emerge where some but not all of these factors are present, or where they emerge in greater or lesser degrees. Unlike guilty, not guilty and, in most jurisdictions \textit{nolo contendere} pleas, best-interest pleas are better understood as variants of pleas, rather than pleas themselves.

What distinguishes best-interest pleas from ordinary guilty pleas is the ability for a defendant to follow their guilty plea with a protestation of their innocence, a declaration that the state has failed to comply with its procedural or ethical obligations, or both. Although such pleas are functionally indistinguishable from guilty pleas in procedure or result,\footnote{Cite sources supporting that \textit{Alford} pleas have the same legal consequences as guilty pleas.} best-interest pleas open the prospect of plea resolution up to defendants who cannot or will not show contrition for the offences they have been charged with.

\subsection{Best-interest pleas in the United States}

In the United States, best-interest pleas are common, having been recognized by the United States Supreme Court as a valid response to a criminal charge. In 1970, the United States Supreme Court heard \textit{North Carolina v Alford}.\footnote{\textit{North Carolina v Alford}, 400 US 25 (\textit{Alford}).} Although other courts had heard similar cases before and since,\footnote{California recognizes a form of the best interest plea known as the West plea, in which a court accepts a nolo contendere plea without having the defendant admit a factual basis for it. It is understood as being synonymous with an Alford plea, despite the differences in history and type between the two. See \textit{People v West} (1970) 3 Cal.3d 595, 91 Cal.Rptr. 385, 477 P.2d 409 (West). Other examples include the \textit{Norgaard} plea in Minnesota \textit{Cite} and the decision emerging out of Ohio in \textit{State v Diehl}, 2017-Ohio-7708, 96 N.E.3d 1271 (Ohio Ct. App. 7th Dist. Harrison County 2017. In other states, decisions predating \textit{Alford} led courts to maintain their disproval subsequent to \textit{Alford}. See \textit{Harshman v State}, 232 Ind 618, 115 NE2d 501 at 3: ``As we view it, a plea of guilty tendered by one who in the same breath protests his innocence, or declares he actually does not know whether or not he is guilty, is no plea at all. Certainly it is not a sufficient plea upon which to base a judgment of conviction. No plea of guilty should be accepted when it appears to be doubtful whether it is being intelligently and understandingly made, or when it appears that, for any reason, the plea is wholly inconsistent with the realities of the situation. We may add parenthetically that so far as the record before us discloses, no evidence whatever pointing to appellant's guilt was adduced, either before, during or after the entry of the plea.''} the \textit{Alford} decision is by far the most common. \textit{Alford} pleas are accepted both federally and in 47 states, and have become the \textit{de facto} best-interest plea in the United States.

\subsubsection{\textit{North Carolina v Alford}, 400 US 25}

After the widespread adoption of \textit{nolo contendere} pleas in the early20th century, and following the codification that carried on thereafter, the courts began to explore the limits of these pleas and how they interact with constitutional rights and freedoms. Viewed one way, \textit{Alford} was an adaptation of both the guilty plea and the \textit{nolo contendere} plea, wherein the scope of both were expanded to create a distinct type of plea procedure.

In 1963, Henry Alford, a black man living in North Carolina, was accused of having murdered a man named Nathaniel Young. On the night that Young died from a shotgun blast, several witnesses said they saw Alford and Young get into a fight. After the fight, Alford was seen leaving the bar with his gun and heard saying that he was going to kill Young. Later that night, Young was shot, and Alford was heard confessing to the murder. Alford had a criminal record, including a previous conviction for murder. His attorney believed that the evidence against Alford was overwhelming, and recommended he accept a plea deal that would take a death sentence off the table. Alford agreed to the plea deal but protested his innocence when making the plea. Alford’s guilty plea was accepted, but he appealed and sought to withdraw his plea on the basis that it wasn’t voluntarily made.

The majority in \textit{Alford} recognized the similarities between \textit{nolo contendere} pleas, and the circumstances that Alford had pleaded guilty under. Given that 

\footnote{Implicit in the nolo contendere cases is a recognition that the Constitution does not bar imposition of a prison sentence upon an accused who is unwilling expressly to admit his guilt but who, faced with grim alternatives, is willing to waive his trial and accept the sentence. ... An individual accused of crime may voluntarily, knowingly, and understandingly consent to the imposition of a prison sentence even if he is unwilling or unable to admit his participation in the acts constituting the crime.} 


Some jurisdictions have taken For example, Delaware has codified a plea of guilty without admission\footnote{See DE R SUPER CT RCRP Rule 11(b).} that functions as its name implies. Like defendants who plead guilty, but unlike defendants who plead \textit{nolo contendere}, defendants who plead guilty without admission may have evidence of that plea admitted in subsequent proceedings.\footnote{Delaware Trial Handbook § 17:6. Admissibility of a Criminal Plea or Judgment in a Subsequent Civil Suit.}

Although most American jurisdictions allow **Alford** pleas to be entered, few provide a statutory framework identifying the plea as such, or outlining the legislative contours of something similar.

Delaware is an apparent exception, having legislated for a plea of "guilty without admission" alongside its plea of *\textit{nolo contendere}*:
\begin{quote}
    **(b) Nolo Contendere; Guilty Without Admission.** A defendant may plead nolo contendere or guilty without admitting the essential facts constituting the offense charged only with the consent of the court. Such a plea shall be accepted by the court only after due consideration of the views of the parties and the interest of the public in the effective administration of justice.
\end{quote}

\subsection{Best-interest pleas in England and Wales}

Much like in the United States, England and Wales do not see a defendant's protestation of innocence as a bar to them pleading guilty. Support for this practice is typically grounded in the influential \textit{R v Herbert} decision.

\subsubsection{R v Herbert, (1992), 94 Cr App R 230}

Herbert was jointly charged with his wife and two others of drug offences, and the matter was set down for trial. At trial, Crown counsel and counsel for Herbert's wife worked out a deal where the Crown would drop Herbert's wife's charges in exchange for Herbert's guilty plea. Herbert agreed to the deal, but maintained his innocence. Herbert's counsel advised him to continue his trial if he was, in fact, not guilty, and that he would likely receive a jail term if he opted to plead guilty. Herbert nonetheless maintained that he wanted to change his plea, was convicted, and sentenced to five and a half years of custody. He appealed his conviction.

The issues for the Court of Appeal to decide were whether the prosecutor placed improper pressure on the defendant to plead, and whether the defendant's lawyer failed to provide adequate representation. The Court of Appeal found that neither Crown counsel nor the defendant's trial attorney had acted improperly. Citing the provisions from the Bar Association's Code of Conduct noted above, the court upheld the plea as an appropriate exercise of the defendant's free choice, and dismissed Herbert's appeal

Since this decision, it has been understood that protestations of innocence do not bar a defendant from pleading guilty in England and Wales. Similarly, plea bargains involving reduced or removed charges for co-defendants are not always inappropriate for the prosecution to offer.

\subsection{Best-interest pleas in Canada}

In Canada, the legal status of best-interest pleas is uncertain. Although they are not clearly barred by the \textit{Criminal Code}, there are no decisions like \textit{Alford} or \textit{Herbert} in Canadian case law that explicitly authorize or encourage such pleas. Although several appellate decisions explicitly excoriate the practice of allowing defendants to plead guilty when they protest their innocence, these decisions may be readily contrasted with others that have no such compunctions. The larger debate between these two positions in Canada and the nuances therein will be examined in detail in §3.4 below.