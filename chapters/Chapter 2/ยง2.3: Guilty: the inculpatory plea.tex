\section{Guilty: the inculpatory plea}

\subsection{The history of the guilty plea}

The modern guilty plea evolved from a longstanding practice where courts allowed defendants to confess their crimes without requiring the state to prove them.\footnote{Before the 11\textsuperscript{th} century Norman conquest, the Anglo-Saxons of England established a procedure for confessions to criminal charges. See Alschuler, Plea Bargaining and Its History at 7. See also H. ADAMS, H. LODGE, E. YouNG \& J. LAUGHLIN, ESSAYS IN ANGLO-SAXON LAW at 285, \url{https://babel.hathitrust.org/cgi/pt?id=mdp.39015017688113\&view=1up\&seq=314.}} Confessions were rudimentary and fraught with risks to those who entered them. Compared to confessions confessions, the modern guilty plea increased procedural protections for those who entered them and finality to the proceedings by deeming the allegations proven.\footnote{See \textit{R v Baird}, 1908 CarswellSask 23 at para 2.}

There is some debate between jurists, scholars, and legal professionals as to how frequently the courts used these procedures and customs and how well the judiciary historically received them. Their disagreements tend to form along the fault lines of plea bargaining. Proponents of plea bargaining see guilty pleas as an age-old and venerable institution,\footnote{See example.} while opponents of plea bargaining view them as recent procedural compromises made in the name of expedient justice.\footnote{See Alschuler, Plea Bargaining, at 2.} 

There is evidence that judges were initially reluctant to accept guilty pleas.\footnote{Cite an example.} It was only toward the end of the 19\textsuperscript{th} century that courts began regularly reporting cases involving guilty pleas.\footnote{See Alschuler at 7 - 8.} By the 1920s, criminal justice surveys in America revealed that many more defendants preferred to plead guilty, and the guilty plea was fixed in American and Canadian criminal proceedings.\footnote{See Alschuler at 26. The percentages of guilty pleas Alschuler reports range between 76 - 90\% in several major American cities at the time.}

\subsection{Plea voluntariness and comprehension}

English common law has long required that defendants enter guilty pleas knowingly and voluntarily, but how courts understand and apply this requirement differs from jurisdiction to jurisdiction. Factors the courts consider commonly include:

\begin{itemize}
\item \textbf{Voluntariness and knowledge.} Defendants entering pleas must know the nature and consequences of their plea and enter them voluntarily. As early as the 16\textsuperscript{th} century, English courts required that defendants enter guilty pleas voluntarily,\footnote{See Alschuler at 12.} a practice that Alschuler attributes to the co-evolution of the plea alongside the common law confessions rule.\footnote{See Alschuler at 12 - 15.} As with confessions, guilty pleas allow defendants to self-convict. However, unlike confessions, guilty pleas are presumed valid and admissible.
\item \textbf{Factual basis.} In addition to ensuring that defendants enter guilty pleas knowingly and voluntarily, judges must also ensure that there is some factual basis for the plea. Courts must be satisfied that there is enough evidence to sustain a conviction.\footnote{This requirement is similar to the requirement in many (though not all) American states that defendants enter \textit{nolo contendere} pleas with a factual foundation. Similarly, all \textit{Alford} pleas must be supported by a factual foundation.} A sentencing judge's obligation to ensure a factual foundation for the plea ensures that the allegations are reviewed.\footnote{Judges are not expected or allowed to operate as independent fact-finders. See \textit{R v Woroniuk}, 2019 MBCA 77, where the Manitoba Court of Appeal reversed a ``flagrant error in law" committed by sentencing judge Brian Corrin PCJ who undertook to investigate the factors laid out by counsel at sentencing. As a result, the scope of the judge's inquiry into the charge's factual foundation is limited, but the inquiry is nevertheless required.}
\item \textbf{Collateral consequences.} ``Collateral consequences" are consequences resulting from convictions that the court does not impose. These consequences can be legal or extra-legal. Legal collateral consequences result from intersections between different legislative schemes, where a decision made under one legislative scheme has consequences under another.\footnote{These consequences may include extended driving restrictions imposed by the provincial licensor (rather than the court), restrictions on purchasing certain medications or other substances, and cancelled foreign work permits and permanent resident designations, among others. Many American jurisdictions have codified requirements that courts alert defendants of the inevitable collateral consequences of pleading guilty to an offence. Requirements that courts tell defendants about collateral gun ownership and immigration consequences are standard. See e.g. the National Inventory of Collateral Consequences of Conviction, online: \textless https://niccc.nationalreentryresourcecenter.org/ \textgreater} Extra-legal collateral consequences refer to the non-legal consequences that flow from the defendant's offending behaviour.\footnote{These may include breakdowns in social or familial relationships, opportunity losses, or physical injuries, among many others. See \textit{R v Suter}, 2018 SCC 34.}
\end{itemize}

\subsection{Guilty pleas in Canada}

Before 1892, reported decisions dealing with guilty pleas were rare in Canada,\footnote{I was only able to locate two such decisions while researching this thesis. See \textit{R v Morrison}, 1879 CarswellNB 35 and \textit{R v Morin}, 1890 CarswellQue 17.} but following codification, reported cases dealing with guilty pleas abounded and Canadian guilty plea procedures adapted in turn.

\subsubsection{Voluntariness and knowledge}

In the early days of Canadian criminal law, much remained unsettled about guilty pleas, their nature, and what consequences stemmed from entering one. As the 20\textsuperscript{th} century unfolded, Canadian courts began insisting that defendants may only enter guilty pleas knowingly and voluntarily and to refuse guilty pleas from defendants who were induced or threatened to self-convict. In \textit{R v Baird},\footnote{See \textit{R v Baird}, 1908 CarswellSask 23.} the Saskatchewan District Court heard an appeal from a landowner convicted of allowing a fire to spread. Baird, the landowner, pleaded guilty but argued on appeal that he had entered his guilty plea under oppression. The reviewing court found no evidence that the justice who took Baird's plea had acted oppressively or had induced him to plead guilty and dismissed his appeal. 

Canadian courts similarly required defendants entering guilty pleas to know the legal consequences of doing so. Before codification, courts operated on the principle that a person pleading guilty must be presumed to know the law.\footnote{See \textit{Morrison} at paras 9, 13; \textit{Morin} at para 145.} However, developments in the following decades moved towards ensuring that defendants better knew the implications of their plea before entering it. In \textit{R v Tom},\footnote{\textit{R v Tom}, 1928 CarswellNS 10.} the defendant pleaded guilty to possessing narcotics. On appeal, Tom argued that he had pleaded guilty but had not understood the consequences, and that the prosecutor induced him to self-convict by promising he would receive a small fine if he did so. A majority of the Nova Scotia Supreme Court allowed the appeal. The dissent agreed that there seemed to be a miscarriage of justice but disagreed that they had jurisdiction to order a remedy.

Early on, questions such as whether a guilty plea waived a right to trial\footnote{See \textit{R v Davidson}, 1992 CanLII 2570: ``At common law a plea of guilty was simply an admission of the facts stated in the information"; see also \textit{R v Gillis}, 1914 CarswellYukon 6, where the court treats the subject as a viable but generally settled question of law: `` Counsel for the appellant claims the right to a trial de novo, and that the respondent should be called upon to prove, and that it was incumbent upon him to prove, the appellant guilty of the offence charged. I am of opinion, following the cases of Rex v. Brook, 7 Can. Crim. Cas. 216, and Harrop v. Bayley, 6 E. \& B. 218, that under his plea of ``guilty" the appellant is estopped from calling upon the respondent to produce evidence to establish that he is guilty of the offence with which he is charged; and, so far as the facts relating to his guilt or innocence are concerned, he is not a person who thinks himself aggrieved within the meaning of sec. 749 of the Criminal Code of Canada."} were still live issues when Parliament codified Canadian criminal law. As these issues made their way through the justice system, courts settled on predictable rules and uniform procedures that made it possible for defendants to know the consequences of waiving their right to trial much better.

\subsubsection{Factual basis}

Canadian criminal courts have long required a factual foundation as a prerequisite for accepting a guilty plea. \textit{R v Herbert}, 1903 CarswellOnt 829 provides an early example of this requirement. In that case, the Ontario Supreme Court reviewed a case where the defendant, Herbert, pleaded guilty to a murder nobody suspected he had committed. After pleading guilty, Herbert implicated a co-defendant, who pleaded not guilty and was acquitted. On review, the court found that there was ``no theory that can be suggested upon which [the co-defendant] could be innocent and [Herbert] guilty." Absent any possible factual foundation for the plea, the court allowed the defendant to withdraw it.

\subsubsection{Collateral consequences}

Canadian courts recognize legal and extra-legal collateral consequences as valid sentencing considerations. However, Canadian courts are not explicitly required to canvass these consequences with defendants before they self-convict. Problems with this approach frequently arise, and often in the context of immigration proceedings. An emblematic example of this problem can be found in \textit{Wong}, where the defendant pleaded guilty to a cocaine trafficking charge but did not know that doing so would affect his Canadian residency. He appealed to withdraw his guilty plea. 

The majority and the dissent agreed that defendants must know the ``legally relevant collateral consequences" of pleading guilty for the plea to be valid, but split on which legal test they should apply. The majority held that a defendant must be able to show subjective prejudice to withdraw an uninformed guilty plea of this variety. The dissent suggested accomplishing this through a modified objective test. Although both agreed that Wong had entered his guilty plea without sufficient information, he did not produce evidence of subjective prejudice. The majority dismissed his appeal while the dissent would have allowed it.

Canadian courts also recognize extra-legal collateral consequences as a valid sentencing consideration, as seen in the \textit{Suter} decision.\footnote{See \textit{R v Suter}, 2018 SCC 34 (CanLII), [2018] 2 SCR 496.} Suter drove onto a restaurant patio and killed a two-year-old child. The police suspected he was driving impaired and demanded a breath sample. Having spoken with a lawyer who told him not to provide a sample, Suter refused to do so when required. Before he dealt with his charges, a group of vigilantes attacked Suter, cutting off his thumb, and his wife was attacked in a separate incident. He ultimately pleaded guilty to refusing to provide a sample in a case causing death. There was no evidence led suggesting that Suter was impaired while driving.

At sentencing, the judge relied on the fact that Suter was the victim of a violent attack and sentenced Suter to four months of custody. Both he and the Crown appealed. The Alberta Court of Appeal ruled that the sentencing judge erred for several reasons, including relying on vigilante violence that did not ``emanate from state misconduct", and re-sentenced Suter to 26 months of custody.

The Supreme Court of Canada found that the Alberta Court of Appeal incorrectly dismissed Suter's attack and injuries as irrelevant. The majority instead ruled that collateral consequences may include ``any consequence arising from the commission of an offence, the conviction for an offence, or the sentence imposed for an offence, that impacts the offender." In Suter's case, this included the violent and retributive attack he suffered for his role in the offence. The majority allowed the appeal and varied the sentence to time served. The lone dissenting judge would have upheld the original four-month custodial sentence.

\subsubsection{Unequivocal}

Unlike other common law jurisdictions, Canadian common law requires that defendants who enter guilty pleas do so \textit{unequivocally}.\footnote{Prior to \textit{Wong}, courts usually cited this requirement to an influential dissent in \textit{Adgey v R}, 1973 CanLII 37 (SCC), [1975] 2 SCR 426: ``[A]n appellate Court should interfere to set aside a conviction made upon a plea of guilty if it be the case that (1) the accused did not understand the nature of the charge before pleading; or (2) the accused did not unequivocally plead guilty to the charge as properly understood; or (3) the accused, on the facts offered in support of the charge, could not in law have been convicted of the offence charged. The majority in \textit{Wong} affirmed this approach, while the minority dissented on the issue \hl{discussed above}. Although this requirement has been functionally ubiquitous in appellate decisions since \textit{Adgey}, the term ``unequivocal'' is not uniformly understood and was not included in the \textit{Criminal Code} when Parliament codified the plea inquiry in 2002. How this term is explicitly understood impacts whether Canadian law permits exculpatory no-contest pleas.} An unequivocal plea is one that the defendant intended to enter. In cases where a reviewing or first-instance court is left wondering whether a defendant intended to plead guilty, the plea is equivocal and should not be accepted or upheld. Canadian courts have interpreted this requirement in two main ways. The first applies the ``unequivocal'' principle to the \textit{plea's content}. Under this approach, the defendant pleads unequivocally when they admit they were responsible for the offence.\footnote{See e.g. } The second applies the ``unequivocal'' principle to the \textit{defendant's subjective intent} to bring about the consequences of their plea. Under this approach, the defendant pleads unequivocally when they unreservedly acknowledge their intention to self-convict.\footnote{See e.g. Although Canadian case law supports both approaches, I prefer the latter approach throughout this thesis for several reasons. As will be discussed in greater detail below, the propositional content of a plea is an admission of proof, not belief or fact. A defendant who pleads guilty admits that the allegations against them are proven, while a defendant who pleads not guilty advises that the allegations against them must be proven. The former approach reads the unequivocal requirement as a demand that defendants acknowledge the factual accuracy of the allegations. In doing so, it fundamentally misunderstands the nature of a plea.} 

