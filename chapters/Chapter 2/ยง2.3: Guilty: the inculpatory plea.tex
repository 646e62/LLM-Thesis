\section{Guilty: the inculpatory plea}

\subsection{The history of the guilty plea}

The modern guilty plea evolved from a longstanding practice where courts allowed defendants to confess their crimes without requiring the state to prove them. Before the 11\textsuperscript{th} century Norman conquest, the Anglo-Saxons of England established a procedure for confessions to criminal charges.\footnote{See Alschuler, Plea Bargaining and Its History at 7. See also H. ADAMS, H. LODGE, E. YouNG \& J. LAUGHLIN, ESSAYS IN ANGLO-SAXON LAW at 285, \url{https://babel.hathitrust.org/cgi/pt?id=mdp.39015017688113\&view=1up\&seq=314.}} Confessions were rudimentary and fraught with risks to those who entered them. Nevertheless, they served as the precursor to the modern-day guilty plea, which both provides increased procedural protections for those who enter them, and provides a finality to the proceedings by estopping defendants from later contesting the allegations.\footnote{See \textit{R v Baird}, 1908 CarswellSask 23 at para 2.}

Although common law jurisdictions have long allowed self-convictions, there is some debate between jurists, scholars, and legal professionals as to how frequently the courts used these procedures and customs and how well the judiciary historically received them. These disagreements tend to form along the lines of one's view of the practice of plea bargaining. Proponents of plea bargaining see guilty pleas as an age-old institution,\footnote{See example.} while opponents of plea bargaining tend to view them as a more recent procedural compromise made in the name of expedient justice.\footnote{See Alschuler, Plea Bargaining, at 2.} Both should agree that evidence of a modern equivalent of a guilty plea has been largely absent for much of common law history. Fewer reports of common law confessions exist compared to reports of jury trials,\footnote{Alschuler takes the view that guilty pleas were relatively rare until the 19\textsuperscript{th} century. He relies on the fact that relatively few reported decisions cover defendants pleading guilty before this time. There are problems with this approach. Notably, the percentage of reported criminal decisions recording guilty pleas nowadays does not align with the percentage of criminal cases that resolve in a guilty plea. Compare ONCJ stats for 2021 and the 2021 PC survey results. Thus it is hasty to conclude that the low number of reported guilty pleas before the 19\textsuperscript{th} century correlates with a low number of guilty pleas defendants entered. Alschuler's observation further undermines his position that courts had already begun requiring that all guilty pleas entered be voluntary, a relatively sophisticated development for a supposedly underutilized plea procedure.} and there is evidence to suggest that some courts historically ran trials as frequently as contemporary courtrooms hear guilty pleas.\footnote{Find and cite the article that mentions courts where ran dozens of trials daily. Probably Bibas.} 

As common law confessions gave way to the nascent beginnings of a modern guilty plea, there is evidence that judges were reluctant to accept them.\footnote{Cite an example.} It was only toward the end of the 19\textsuperscript{th} century when courts began regularly reporting cases involving guilty pleas.\footnote{See Alschuler at 7 - 8.} By the 1920s, criminal justice surveys in America revealed that many more defendants opted for self-conviction over a trial.\footnote{See Alschuler at 26. The percentages of guilty pleas Alschuler reports range between 76 - 90\% in several major American cities at the time.} These surveys prompted subsequent investigations into the role that plea bargaining played in these statistics. Many were initially surprised by these outcomes, and plea bargaining began with more critics than advocates.\footnote{Cite this as well.} Nevertheless, it continues to be practiced, developed, regulated and refined to the present day.

\subsection{Plea voluntariness and comprehension}

Common law jurisdictions typically require that defendants enter guilty pleas knowingly and voluntarily. However, the courts' understanding and applying the idea of a ``knowing and voluntary'' guilty plea differs from jurisdiction to jurisdiction. Other jurisdictions require courts to canvass different factors, but some that commonly arise include:

\begin{itemize}
\item \textbf{Voluntariness.} Defendants entering pleas must do so voluntarily and not as the result of undue coercion, threats, or inducements.
\item \textbf{Knowledge.} The defendant must understand the nature and consequences of their plea.
\item \textbf{Factual basis.} Courts must be satisfied that there is enough evidence to sustain a conviction.\footnote{This requirement is similar to the requirement in many (though not all) American states that defendants enter \textit{nolo contendere} pleas with a factual foundation and the rule that all \textit{Alford} pleas must be supported by a factual foundation.}
\item \textbf{Collateral consequences.} A criminal conviction may result in automatic penalties that affect other aspects of a person's life, such as their ability to remain in the country, operate a vehicle, or purchase a firearm.
\end{itemize}

Not every jurisdiction requires courts to canvass each of these factors with defendants before sentencing. Where courts canvass these factors, local legislation, court rules, and customs may require courts to consider additional and different sub-factors or examine the factors in light of particular and different circumstances. 

\subsubsection{Voluntariness and knowledge}

Assuming that judges found guilty pleas suspect for much of the plea's existence, the common law requirement that defendants enter guilty pleas voluntarily, subjectively understood, and based on a factual foundation makes perfect contextual sense. As early as the 16\textsuperscript{th} century, English courts required that defendants enter guilty pleas voluntarily,\footnote{See Alschuler at 12.} a practice that Alschuler attributes to the co-evolution of the plea alongside the common law confessions rule.\footnote{See Alschuler at 12 - 15.} A defendant's right to be presumed innocent until proven guilty beyond a reasonable doubt is the common law's primary tool to prevent state overreach and abuses. To preserve these fundamental rights, defendants must understand that confessing may compromise those rights and increase the likelihood of a subsequent criminal conviction. Where the prosecutors wish to use a defendant's confession, Canadian courts require the state to prove that the confession was voluntary. 

As with confessions, guilty pleas allow defendants to self-convict. However, unlike confessions, courts presume that defendants validly entered their guilty pleas. Where a defendant wishes to argue otherwise, the burden of proof is on them to do so. This presumption is made possible by the plea voluntariness and knowledge inquiry.

\subsubsection{Factual basis}

In addition to ensuring that defendants enter guilty pleas knowingly and voluntarily, judges must also ensure that there is some factual basis for the plea. Courts should not expect people to be sentenced for criminal charges that they are not or are unlikely to be found guilty of, and prosecutors must not pursue charges that lack a reasonable prospect of conviction.\footnote{See case.} A sentencing judge's obligation to ensure a factual foundation for the plea ensures that the allegations are ``peer-reviewed'' to a degree. Judges are not expected or allowed to operate as independent fact-finders.\footnote{See \textit{R v Woroniuk}, 2019 MBCA 77, where the Manitoba Court of Appeal reversed a ``flagrant error in law" committed by sentencing judge Brian Corrin PCJ who undertook to investigate the factors laid out by counsel at sentencing.} As a result, the scope of the judge's inquiry into the charge's factual foundation is limited, but the inquiry is nevertheless required.

\subsubsection{Collateral consequences}

The term ``collateral consequences" refers to the consequences resulting from a guilty plea that do not directly stem from the sentence the court imposes. These consequences can be legal or extra-legal. Legal collateral consequences result from intersections between different legislative schemes, where a decision made under one legislative scheme has consequences under another. These consequences may include extended driving restrictions imposed by the provincial licensor (rather than the court), restrictions on purchasing certain medications or other substances, and cancelled foreign work permits and permanent resident designations, among others. Many American jurisdictions have codified requirements that courts alert defendants of the inevitable collateral consequences of pleading guilty to an offence.\footnote{Requirements that courts tell defendants about collateral gun ownership and immigration consequences are standard. See eg the National Inventory of Collateral Consequences of Conviction, online: \textless https://niccc.nationalreentryresourcecenter.org/ \textgreater} Where collateral consequence inquiries are required, they are not uniform, with some states requiring that courts inform defendants of some consequences but not others. Extra-legal collateral consequences refer to the non-legal consequences that flow from the defendant's offending behaviour. These may include breakdowns in social or familial relationships, opportunity losses, or physical injuries, among many others.\footnote{For a good example of extra-legal consequences, see \textit{R v Suter}, 2018 SCC 34. This case is discussed in more detail below.}

\subsection{Guilty pleas in Canada}

Before Parliament codified Canadian criminal law in 1892, reported decisions dealing with guilty pleas were rare in Canada.\footnote{I was only able to locate two such decisions while researching this thesis. See \textit{R v Morrison}, 1879 CarswellNB 35 and \textit{R v Morin}, 1890 CarswellQue 17.} However, following codification, reported cases dealing with guilty pleas abounded. This proliferation resulted in early and frequent developments in guilty plea procedures in Canada. Eventually, Parliament implemented the statutory guilty plea that now resolves most Canadian criminal cases. Once codified, Canadian courts began interpreting the plea and helping to shape it into its present form.

\subsubsection{Voluntariness and knowledge}

In the early days of Canadian criminal law, much remained unsettled about guilty pleas, their nature, and what consequences stemmed from entering one. As the 20\textsuperscript{th} century unfolded, reported decisions throughout Canada began insisting that defendants only entered guilty pleas knowingly and voluntarily and started to develop the procedural rights and waivers that guilty pleas entailed. As English and American courts had done, Canadian courts insisted that judges only accept voluntary guilty pleas. Where defendants are induced or threatened to enter a guilty plea, Canadian courts ruled early on that those pleas should not be accepted. 

In the 1908 decision \textit{R v Baird},\footnote{See \textit{R v Baird}, 1908 CarswellSask 23.} the Saskatchewan District Court heard an appeal from a landowner convicted of allowing a fire to spread from his property to a neighbour's. Baird, the landowner, pleaded guilty and was sentenced to a \$200 fine or six months of custody if he defaulted within 30 days. On appeal, Baird raised several issues with the conviction and alleged that he had been forced to enter his guilty plea under oppressive circumstances. The reviewing court noted that it could only give effect to this ground of appeal if there were some evidence that the justice taking Baird's plea had acted oppressively or had induced him to plead guilty. Finding none, the court dismissed his appeal. 

Just as Canadian courts established early on that guilty pleas must be voluntary, they similarly required defendants entering these pleas to know the legal consequences of doing so. Before codification, courts operated on the principle that a person pleading guilty must be presumed to know the law.\footnote{See \textit{Morrison} at paras 9, 13; \textit{Morin} at para 145.} However, developments in the following decades began to shed more light on the importance of ensuring that defendants better knew the implications of their plea before entering it. In \textit{R v Tom},\footnote{\textit{R v Tom}, 1928 CarswellNS 10.} the defendant pleaded guilty to possessing narcotics and was sentenced to six months imprisonment and a \$500 fine. On appeal, Tom argued that he had pleaded guilty but had not understood the consequences. The prosecutor in his case advised him to plead guilty and told him that he would receive a small fine if he did so. The prosecutor and the magistrate recommended that the prosecution be re-opened and the defendant be allowed to plead not guilty. A majority of the Nova Scotia Supreme Court agreed. The dissent agreed that there seemed to be a miscarriage of justice but questioned whether they had the jurisdiction to correct it.

Determining whether a defendant fully knew the consequences of pleading guilty was not a straightforward enterprise. This is partly because the courts did not agree on the consequences of entering a guilty plea. Questions such as whether a guilty plea waived a right to trial\footnote{See \textit{R v Davidson}, 1992 CanLII 2570: ``At common law a plea of guilty was simply an admission of the facts stated in the information"; see also \textit{R v Gillis}, 1914 CarswellYukon 6, where the court treats the subject as a viable but generally settled question of law: `` Counsel for the appellant claims the right to a trial de novo, and that the respondent should be called upon to prove, and that it was incumbent upon him to prove, the appellant guilty of the offence charged. I am of opinion, following the cases of Rex v. Brook, 7 Can. Crim. Cas. 216, and Harrop v. Bayley, 6 E. \& B. 218, that under his plea of ``guilty" the appellant is estopped from calling upon the respondent to produce evidence to establish that he is guilty of the offence with which he is charged; and, so far as the facts relating to his guilt or innocence are concerned, he is not a person who thinks himself aggrieved within the meaning of sec. 749 of the Criminal Code of Canada."} were still live issues when Canadian criminal law was codified. As these issues made their way through the justice system, courts settled on predictable rules and uniform procedures that made it possible for defendants to much better know the consequences of waiving their right to trial. Nevertheless, as seen in the section on collateral consequences below, there is still progress on this front.

\subsubsection{Factual basis}

Canadian criminal courts quickly started requiring some factual foundation as a condition precedent for accepting a guilty plea. An early reported example of this is seen in \textit{R v Herbert}, 1903 CarswellOnt 829. In that case, the Ontario Supreme Court reviewed a case where the defendant, Herbert, pleaded guilty to a murder that nobody suspected he committed. At the time, murder was a capital offence in Canada, and the defendant would have been put to death. In pleading guilty, he implicated a co-accused, who pleaded not guilty and was found such by a jury. The prosecutor advised the court that the Crown would not continue its prosecution if the court rejected the guilty plea. On review, the court found that there was ``no theory that can be suggested upon which [the co-defendant] could be innocent and [Herbert] guilty." Absent a factual foundation for the plea, the court allowed the defendant to withdraw it.

\subsubsection{Collateral consequences}

In Canada, collateral consequences are a valid consideration when sentencing. Canadian courts recognize both legal and extra-legal collateral consequences as valid sentencing considerations. However, Canadian courts are not explicitly required to canvass these consequences with defendants before they enter their guilty pleas. The problems with this approach most frequently appear in cases where a defendant pleads guilty to an offence that results in proceedings to remove them from Canada. Although these cases are all too common, the recent decision in \textit{R v Wong} embodies the problem of legally relevant collateral consequences and the Supreme Court of Canada's approach to resolving it.

In \textit{Wong}, the defendant pleaded guilty to a cocaine trafficking charge but did not know that doing so would negatively impact his permanent residency status in Canada. He appealed for the right to withdraw his guilty plea. The majority and the dissent agreed that defendants must know the ``legally relevant collateral consequences" of pleading guilty for the plea to be valid. However, the court split on which legal test should be applied to defendants looking to do so. The majority held that a defendant must be able to show subjective prejudice to withdraw an uninformed guilty plea of this variety, while the dissent suggested that this could be accomplished through a modified objective test. Although both agreed that Wong had entered his guilty plea without sufficient information, he did not produce evidence of subjective prejudice. His appeal was therefore dismissed.

Canadian courts also recognize extra-legal collateral consequences as a valid consideration when sentencing a defendant. A prime example of this occurred in the \textit{Suter} decision.\footnote{See \textit{R v Suter}, 2018 SCC 34 (CanLII), [2018] 2 SCR 496.} Suter, the appellant and defendant at trial, drove onto a restaurant patio and killed a two-year-old child. The police suspected he was driving impaired and demanded a breath sample. Having spoken with a lawyer who told him not to provide a sample, Suter refused to do so when required. Before he dealt with his charges, a group of vigilantes attacked Suter, cutting off his thumb. His wife was attacked in a separate incident. He ultimately pleaded guilty to refusing to provide a sample in a case causing death. There was no evidence led suggesting that Suter was impaired while driving.

When sentencing Suter, the judge relied heavily on the fact that Suter was the victim of a violent attack and on the fact that the victim's death resulted from ``an accident caused by a non-impaired driving error".\footnote{See \textit{Suter} at paras 3 \& 17.} The judge sentenced Suter to four months of custody and a 30-month driving prohibition. Both he and the Crown appealed. The Alberta Court of Appeal ruled that the sentencing judge erred for several reasons, including relying on the vigilante violence that Suter had suffered. Because the consequence did not ``emanate from state misconduct", it ought to have had no impact on the sentence. The Alberta Court of Appeal re-sentenced Suter to 26 months of custody, which he again appealed. 

Although the majority of the Supreme Court of Canada agreed that the sentencing judge committed several errors that improperly mitigated the sentence, they found that the Alberta Court of Appeal incorrectly dismissed Suter's attack and injuries as valid collateral consequences. The majority instead ruled that collateral consequences may include ``any consequence arising from the commission of an offence, the conviction for an offence, or the sentence imposed for an offence, that impacts the offender." In Suter's case, this included the violent and retributive attack he suffered for his role in the offence. Although the majority felt that the sentencing judge should have imposed a longer custodial sentence in the first place, it allowed the appeal and varied the sentence to time served. The lone dissenting judge would have upheld the original four-month custodial sentence.

\subsubsection{Unequivocal}

Unlike other common law jurisdictions, Canadian common law requires that defendants who enter guilty pleas do so \textit{unequivocally}. Prior to \textit{Wong}, courts usually cited this requirement to an influential dissent in \textit{Adgey v R}, 1973 CanLII 37 (SCC), [1975] 2 SCR 426.\footnote{``[A]n appellate Court should interfere to set aside a conviction made upon a plea of guilty if it be the case that (1) the accused did not understand the nature of the charge before pleading; or (2) the accused did not unequivocally plead guilty to the charge as properly understood; or (3) the accused, on the facts offered in support of the charge, could not in law have been convicted of the offence charged.} The majority in \textit{Wong} affirmed this approach, while the minority dissented on the issue \hl{discussed above}. Although this requirement has been functionally ubiquitous in appellate decisions since \textit{Adgey}, the term ``unequivocal'' is not uniformly understood, and was not included in the \textit{Criminal Code} when Parliament codified the plea inquiry in 2002. How this term is explicitly understood impacts whether Canadian law permits exculpatory no-contest pleas. 

The principle underlying the unequivocal requirement is that the court should only accept the plea that the defendant intended to enter. In cases where a reviewing or first-instance court is left wondering whether a defendant intended to plead guilty, the plea is equivocal and should not be accepted or upheld. Unfortunately, this principle is equivocally understood and applied, such that two main approaches can be identified. The first approach applies the ``unequivocal'' principle to the \textit{plea's content}. Under this approach, the defendant pleads unequivocally when they admit they were responsible for the offence.\footnote{See eg ...} The second approach applies the ``unequivocal'' principle to the \textit{defendant's subjective intent} to bring about the consequences of their plea. Under this approach, the defendant pleads unequivocally when they unreservedly acknowledge their intention to self-convict.\footnote{See eg ...} 

Although Canadian case law supports both approaches, I prefer the latter approach throughout this thesis for several reasons. As will be discussed in greater detail below, the propositional content of a plea is most cogently understood as an admission of proof, not an admission of belief or fact. A defendant who pleads guilty admits that the allegations against them are proven, while a defendant who pleads not guilty advises that the allegations against them must be proven. The former approach reads the unequivocal requirement as a demand that defendants acknowledge the factual accuracy of the allegations. In doing so, it fundamentally misunderstands the nature of a plea.