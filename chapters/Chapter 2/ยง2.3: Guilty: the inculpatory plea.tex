\section{Guilty: the inculpatory plea}

Since the first \textit{Criminal Code of Canada, 1892}, criminal defendants could opt to self-conviction rather than take their charges to trial.\footnote{See \textit{Criminal Code 1892} s 657.} This early provision did not limit a defendant's ability to enter a guilty plea and did not give judges discretion to accept or reject it. Over time, Canadian common law limited a defendant's ability to plead guilty and stipulated that judges may only accept knowing, voluntary, and unequivocal guilty pleas. In Canada, the \textit{Criminal Code} requires that defendants enter their guilty pleas knowingly and voluntarily, while Canadian common law requires that they enter their guilty pleas unequivocally.\footnote{This common law requirement was eventually codified as \textit{Criminal Code} s 606(1.1) in 2002.}

In Canada, the most frequently used uncontested plea is a simple guilty plea.\footnote{In Canada and the United States, courts deal with criminal allegations through guilty pleas far more frequently than through trials. See e.g. https://www.ontariocourts.ca/ocj/files/stats/crim/2021/2021-Offence-Based-Criminal.xlsx} Courts typically accept guilty pleas as a mitigating sentencing factor. While the general contours of a guilty plea are similar across jurisdictions, the specifics will often vary. For example, courts recognize that defendants must plead guilty knowingly and voluntarily. However, the specific conditions that must be satisfied to meet these standards vary.\footnote{See §2.3.2 below for a more detailed discussion of these pleas and their characteristics.} 

\subsection{The guilty plea historically}

The modern guilty plea evolved from a longstanding practice where courts allowed defendants to confess their crimes without requiring the state to prove them.\footnote{Before the 11\textsuperscript{th} century Norman conquest, the Anglo-Saxons of England established a procedure for confessions to criminal charges. See Alschuler, Plea Bargaining and Its History at 7. See also H. ADAMS, H. LODGE, E. YouNG \& J. LAUGHLIN, ESSAYS IN ANGLO-SAXON LAW at 285, \url{https://babel.hathitrust.org/cgi/pt?id=mdp.39015017688113\&view=1up\&seq=314.}} Confessions were rudimentary and risky, lacking the modern guilty plea's procedural protections and finality.\footnote{See \textit{R v Baird}, 1908 CarswellSask 23 at para 2.}

There is some debate over how frequently the courts used these procedures and customs and how well the judiciary historically received them.\footnote{These disagreements tend to form along the fault lines of plea bargaining. Proponents of plea bargaining see guilty pleas as an age-old and venerable institution, while opponents of plea bargaining view them as recent procedural compromises made in the name of expedient justice. See Alschuler, Plea Bargaining, at 2.} However, it is generally agreed that toward the end of the 19\textsuperscript{th} century courts began regularly reporting cases involving guilty pleas.\footnote{See Alschuler at 7 - 8.} By the 1920s, criminal justice surveys in America revealed that many more defendants preferred to plead guilty, and the guilty plea was fixed in American and Canadian criminal proceedings.\footnote{See Alschuler at 26. The percentages of guilty pleas Alschuler reports range between 76 - 90\% in several major American cities at the time.}

\subsection{Plea voluntariness and comprehension}

As guilty pleas became more common, procedural protections around them increased. One of the most significant was the plea voluntariness and comprehension inquiry. To the extent that defendants were presumed innocent, entitled to remain silent, and under no obligation to prove that they were innocent, courts historically sought to ensure that they only waived those generous protections knowingly and voluntarily. This procedural safeguard, referred to here and throughout as the ``plea inquiry," exists to ensure that defendants know what their rights are when they are charged with an offence, and that they only waive those rights of their own free will. Factors the courts consider when deciding these questions include:

\begin{itemize}
\item \textbf{Voluntariness and knowledge.} Defendants entering pleas must do so voluntarily and know the nature and consequences of their plea. As early as the 16\textsuperscript{th} century, English courts required that defendants enter guilty pleas voluntarily,\footnote{See Alschuler at 12 - 15. Alschuler attributes this practice to the co-evolution of the plea alongside the common law confessions rule. As with confessions, guilty pleas allow defendants to self-convict. However, unlike confessions, guilty pleas are presumed valid and admissible.} Whether a plea is entered voluntarily and knowingly impacts whether the defendant was treated \textit{fairly} in being asked to enter it. Both impact the criminal law's \textit{moral efficacy}.
\item \textbf{Factual basis.} In addition to ensuring that defendants enter guilty pleas knowingly and voluntarily, judges must also ensure that there is some factual basis for the plea. Courts must be satisfied that there is enough evidence to sustain a conviction.\footnote{This requirement is similar to the requirement in many (though not all) American states that defendants enter \textit{nolo contendere} pleas with a factual foundation. Similarly, all \textit{Alford} pleas must be supported by a factual foundation.} A sentencing judge's obligation to ensure a factual foundation for the plea ensures that the allegations are reviewed.\footnote{Judges are not expected or allowed to operate as independent fact-finders. See \textit{R v Woroniuk}, 2019 MBCA 77, where the Manitoba Court of Appeal reversed a ``flagrant error in law" committed by sentencing judge Brian Corrin PCJ who undertook to investigate the factors laid out by counsel at sentencing. As a result, the scope of the judge's inquiry into the charge's factual foundation is limited, but the inquiry is nevertheless required.} The factual foundation requirement helps ensure that uncontested pleas are \textit{truthful} and accurate.
\item \textbf{Collateral consequences.} ``Collateral consequences" are consequences resulting from convictions that the court does not impose. These consequences can be legal or extra-legal. Legal collateral consequences result from intersections between different legislative schemes, where a decision made under one legislative scheme has consequences under another.\footnote{These consequences may include extended driving restrictions imposed by the provincial licensor (rather than the court), restrictions on purchasing certain medications or other substances, and cancelled foreign work permits and permanent resident designations, among others. Many American jurisdictions have codified requirements that courts alert defendants of the inevitable collateral consequences of pleading guilty to an offence. Requirements that courts tell defendants about collateral gun ownership and immigration consequences are standard. See e.g. the National Inventory of Collateral Consequences of Conviction, online: \textless https://niccc.nationalreentryresourcecenter.org/ \textgreater} Extra-legal collateral consequences refer to the non-legal consequences that flow from the defendant's offending behaviour.\footnote{These may include breakdowns in social or familial relationships, opportunity losses, or physical injuries, among many others. See \textit{R v Suter}, 2018 SCC 34.} Collateral consequences go to a plea's \textit{fairness}.
\end{itemize}

\subsection{Guilty pleas in Canada}

Before 1892, reported decisions dealing with guilty pleas were rare in Canada,\footnote{I was only able to locate two such decisions while researching this thesis. See \textit{R v Morrison}, 1879 CarswellNB 35 and \textit{R v Morin}, 1890 CarswellQue 17.} but following codification, reported cases dealing with guilty pleas abounded and Canadian guilty plea procedures adapted in turn. In the early days of Canadian criminal law, much remained unsettled about guilty pleas, their nature, and what consequences stemmed from entering one. As the 20\textsuperscript{th} century unfolded, Canadian courts began insisting that defendants may only enter guilty pleas knowingly and voluntarily and to refuse guilty pleas from defendants who were induced or threatened to self-convict.

\subsubsection{Voluntariness and knowledge}
 In \textit{R v Baird},\footnote{See \textit{R v Baird}, 1908 CarswellSask 23.} the Saskatchewan District Court heard an appeal from a landowner convicted of allowing a fire to spread from his land to a neighbour's property. Baird, the landowner, pleaded guilty but argued on appeal that he had entered his guilty plea under oppression. The court reviewed the proceedings below but found no evidence that the justice who took Baird's plea had acted oppressively or had induced him to plead guilty. The court dismissed his appeal. 

Canadian courts similarly required defendants entering guilty pleas to know the legal consequences of doing so. Before codification, Canadian courts operated on the principle that a person pleading guilty must be presumed to know the law.\footnote{See \textit{Morrison} at paras 9, 13; \textit{Morin} at para 145.} However, developments in the following decades moved towards ensuring that defendants better knew the implications of their plea before entering it. 

In \textit{R v Tom},\footnote{\textit{R v Tom}, 1928 CarswellNS 10.} the defendant pleaded guilty to possessing narcotics and was sentenced to six months of jail and a \$500 fine. On appeal, Tom argued that he had not understood the consequences of pleading guilty. This was compounded by the fact that the prosecutor promised Tom he would only receive a small fine if he did so. A majority of the Nova Scotia Supreme Court allowed the appeal. The dissent agreed that there seemed to be a miscarriage of justice but disagreed that they had jurisdiction to order a remedy. As this and similar issues made their way through the justice system, courts settled on predictable rules and uniform procedures that made it possible for defendants to know the consequences of waiving their right to trial much better.\footnote{Early on, questions such as whether a guilty plea waived a right to trial were still live issues when Parliament codified Canadian criminal law. See \textit{R v Davidson}, 1992 CanLII 2570; see also \textit{R v Gillis}, 1914 CarswellYukon 6, where the court treats the subject as a viable but generally settled question of law.}

\subsubsection{Factual basis}

Canadian criminal courts have long required a factual foundation as a prerequisite for accepting a guilty plea. \textit{R v Herbert}, 1903 CarswellOnt 829 provides an early example of this requirement. In that case, the Ontario Supreme Court reviewed a case where the defendant, Herbert, pleaded guilty to a murder nobody suspected he had committed. After pleading guilty, Herbert implicated a co-defendant, who pleaded not guilty and was acquitted. On review, the court found that there was ``no theory that can be suggested upon which [the co-defendant] could be innocent and [Herbert] guilty." Absent any possible factual foundation for the plea, the court allowed the defendant to withdraw it. The factual foundation requirement reflects the common law's longstanding concern for preventing wrongful convictions.

\subsubsection{Collateral consequences}

Canadian courts recognize legal and extra-legal collateral consequences as valid sentencing considerations. Legal collateral consequences are those that arise when a legal result obtained under one statute or legislative scheme has consequences in another. Increased insurance premiums after an impaired driving conviction, losing the ability to purchase firearms after a violent offence conviction, or revoking a work visa after a theft from an employer convictoin are all examples of legal collateral consequences following criminal convictions. Although Canadian courts may consider these consequences when sentencing a defendant, they are not explicitly required to do so. Collateral immigration consequences are an especially well-litigated example of this problem, with \textit{R v Wong} providing a recent and authoritative example. 

In \textit{Wong}, the defendant pleaded guilty to a cocaine trafficking charge but did not know that doing so would affect his Canadian residency. He appealed to withdraw his guilty plea. The court agreed that defendants must know the ``legally relevant collateral consequences" of pleading guilty for the plea to be valid, but split on which legal test they should apply. The majority held that a defendant must be able to show subjective prejudice to withdraw an uninformed guilty plea of this variety. The dissent suggested accomplishing this through a modified objective test. Although both agreed that Wong had entered his guilty plea without sufficient information, the majority dismissed his appeal while the dissent would have allowed it.

Canadian courts also recognize extra-legal collateral consequences as a valid sentencing consideration, as seen in the \textit{Suter} decision.\footnote{See \textit{R v Suter}, 2018 SCC 34 (CanLII), [2018] 2 SCR 496.} While arguing in a parking lot with his wife, Suter drove onto a restaurant patio and accidentally killed a two-year-old child. Although there was no other evidence that Suter was impaired, in the circumstances the police demanded a breath sample. Having spoken with a lawyer who told him not to provide a sample, Suter refused to do so when required, and was charged with refusal in a case involving death. Suter eventually pleaded guilty, but before doing so, a group of vigilantes attacked him and cut off one of his thumbs for his role in the offence.

At sentencing, the judge relied on the fact that Suter was the victim of a violent attack and sentenced him to four months of custody. Both he and the Crown appealed. The Alberta Court of Appeal ruled that the sentencing judge erred for several reasons, including relying on vigilante violence that did not ``emanate from state misconduct", and re-sentenced Suter to 26 months of custody. Suter appealed to the Supreme Court of Canada, who found that the Alberta Court of Appeal incorrectly dismissed Suter's attack and injuries as irrelevant. The majority instead ruled that collateral consequences may include ``any consequence arising from the commission of an offence, the conviction for an offence, or the sentence imposed for an offence, that impacts the offender." In Suter's case, this included the violent retributive attack he suffered for his role in the offence. The majority allowed the appeal and varied the sentence to time served. The lone dissenting judge would have upheld the original four-month custodial sentence.

\subsubsection{Unequivocal}

Unlike other common law jurisdictions, Canadian common law requires that defendants who enter guilty pleas do so \textit{unequivocally}. An unequivocal plea is one that the defendant intended to enter. In cases where a reviewing or first-instance court is left wondering whether a defendant intended to plead guilty, the plea is equivocal and should not be accepted or upheld. Canadian courts have interpreted this requirement in two main ways. The first applies the ``unequivocal'' principle to the \textit{plea's content}. Under this approach, the defendant pleads unequivocally when they admit they were responsible for the offence.\footnote{See e.g. } The second applies the ``unequivocal'' principle to the \textit{defendant's subjective intent} to bring about the consequences of their plea. Under this approach, the defendant pleads unequivocally when they unreservedly acknowledge their intention to self-convict.\footnote{See e.g. Although Canadian case law supports both approaches, I prefer the latter approach throughout this thesis for several reasons. A defendant who pleads guilty admits that the allegations against them are proven, while a defendant who pleads not guilty advises that the allegations against them must be proven. The former approach reads the unequivocal requirement as a demand that defendants acknowledge the factual accuracy of the allegations. In doing so, it fundamentally misunderstands the nature of a plea.} 

