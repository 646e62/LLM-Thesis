\section{Summary}

An uncontested plea is any plea to a criminal charge where the defendant self-convicts without the prosecutor needing to call evidence against them. Two of these pleas, guilty and \textit{nolo contendere}, have been formally authorized and legislated in various jurisdictions throughout America, while Canada only formally recognizes guilty pleas. \textit{Nolo contendere} pleas were developed in Britain and imported to North America alongside British rule and only America appears to have ever historically adopted it any time before the 21\textsuperscript{th} century. Since the 21\textsuperscript{th} century, Canada has begun to explore \textit{nolo contendere} pleas informally. The following section explores the consequences of this forked development and examines whether it would be \textit{possible} to formally merge \textit{nolo contendere} pleas into Canadian criminal law. 