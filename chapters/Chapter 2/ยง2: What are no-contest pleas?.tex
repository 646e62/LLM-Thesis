\chapter{What are no-contest pleas?}

This thesis aims to determine whether no-contest pleas should be permitted and encouraged in Canadian criminal law. To answer that question correctly, ``no-contest pleas" should first be defined. Throughout this thesis, the phrase ``no-contest plea" will refer to any offer made by a criminal defendant and accepted by the court, asking the court to convict them. The common element underlying all no-contest pleas is that the defendant does not contest the Crown's allegations against them but accepts that the allegations are legally proven and relieves the state of its burden of proof.

No-contest pleas differ from situations where a defendant requires the prosecutor to prove some or all of their case. For example, a defendant who enters a guilty plea but still requires the prosecution to prove some or all of the aggravating circumstances would not have entered a no-contest plea.\footnote{In Canadian criminal law, this procedure is usually referred to as a ``\textit{Gardiner} hearing," named after \textit{R v Gardiner}, [1982] 2 SCR 368. Because the defendant contests some elements of the prosecution's case, it does not meet the main criterion of a no-contest plea.} Similarly, a defendant who enters a guilty plea on the condition that a pre-trial motion fails would not be considered to be entering a no-contest plea.\footnote{American criminal law provides for a procedure known as a ``conditional plea". More than the factually contested sentencing procedure described above, a conditional plea is essentially a full \textit{voir dire} with a self-conviction agreement if the motion fails.}