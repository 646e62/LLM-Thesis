\section{\textit{Nolo contendere}: the non-culpatory plea}

Within this plea bargaining context, an obscure common-law plea known as \textit{nolo contendere} experienced a revival in the United States in the early 20\textsuperscript{th} century. \textit{Nolo contendere}, a Latin phrase meaning ``I do not wish to contest,'' had fallen into disuse in England altogether in the early 18\textsuperscript{th} century and is only sparsely recorded in America during the 19\textsuperscript{th} century. However, in the 20\textsuperscript{th} century, the plea made a resurgence and has since become codified law in most jurisdictions in the United States. A defendant who enters a \textit{nolo contendere} plea invites the court to convict them, but neither admits nor denies their involvement. They accept the consequences of a criminal conviction without taking responsibility for the underlying events. At common-law, and in most states, defendants who enter a \textit{nolo contendere} plea are usually not prevented from contesting subsequent civil suits stemming from the same facts. Typically, these same defendants may not have evidence of their \textit{nolo contendere} plea admitted at those subsequent proceedings.\footnote{This is often true even in jurisdictions that do not otherwise recognize or allow \textit{nolo contendere} pleas. Alabama, for example, is among those states that do not allow defendants to enter \textit{nolo contendere} pleas. They are also one of a very few states not to admit evidence of \textit{nolo contendere} pleas for any reason in subsequent proceedings: ``Alabama [...] follows the minority rule that a conviction resulting from a plea of \textit{nolo contendere} is inadmissible not only to prove the conduct underlying the conviction but for all other purposes.” See also \textit{McNair v State}, 653 So. 2d 320 (Ala. Crim. App. 1992).}

\subsection{The common-law origin of \textit{nolo contendere} pleas}

Although \textit{nolo contendere} pleas became integral to American criminal law through common-law usage, their origins are relatively obscure. Although most legal and academic commentators seem to agree that the plea likely emerged in England in the early 15\textsuperscript{th} century and likely disappeared in the early 18\textsuperscript{th} century, they have uncovered little else about the plea's use or origins. Historically and to the present day, academic and judicial authorities frequently cite a brief excerpt from William Hawkins' ``Treatise of Pleas of the Crown,"\footnote{Cite. See also (sources that use this citation).} unchanged since it first appeared in 1716, as their primary source for insight into the origins and nature of the \textit{nolo contendere} plea. The quote, which cites back to the court records of King Henry IV, has arguably provided the foundation for the use and development of the modern \textit{nolo contendere} plea:

\begin{quote}
\singlespacing
An implied confession is when a defendant, in a case not capital, doth not directly own himself guilty, but in a manner admits it by yielding to the King's mercy, and desiring to submit to a small fine: in which case, if the court think fit to accept of such submission, and make an entry that defendant \textit{prosuit se in gratiam regis}, without putting him to a direct confession, or plea (which in such cases seems to be left to discretion), the defendant shall not be estopped to plead not guilty to an action for the same fact, as he shall if the entry is \textit{quod cognovit indictamentum}.\footnote{@hawkinsTreatisePleasCrown1824}
\end{quote}

By the time Hawkins published the first edition of his Treatise of Pleas of the Crown, \textit{nolo contendere} pleas had started to fall into disuse. The last reported case involving a nolo\textit{ contendere} plea in England was \textit{The Queen v Templeman} in 1702:

\begin{quote}
\singlespacing
Upon a motion to submit to a small fine, after a confession of the indictment, which was for an assault, Holt, Ch. J., took a difference where a man confesses an indictment, and where he is found guilty; in the first case a man may produce affidavits to prove son assault upon the prosecutor in mitigation of the fine: otherwise where the defendant is found guilty; for the entry upon a confession is only \textit{non vult contendere cum domina Regina and pon. se in gratiam Curiae}. Defendants may submit to a fine, though absent, if they have a clerk in court that will undertake for the fine (Hill. 2 Ann). \textit{Hickeringil's Case} was that he and his daughters were indicted for trespass, and Hickeringil only appeared on the motion to submit to a small fine. But where a man is to receive any corporal punishment, judgment cannot be given against him in his absence, for there is a capias pro fine; but no process to take a man and put him on the pillory. \textit{Vide tit}. Judgments. \textit{Duke's Case}.
\end{quote}

Following this case, the \textit{nolo contendere} plea lay dormant in English criminal law reports. It would only be a century later that it re-emerged in the United States.

\subsection{\textit{Nolo contendere} pleas in America}
The earliest reported case I located that mentioned \textit{nolo contendere} pleas in the United States was the 1806 decision of \textit{Commonwealth v The Town of Northampton}. In that case, the Supreme Judicial Court of Massachusetts reported throwing out an indictment against the town for failing to provide a schoolmaster. At an earlier proceeding, the town entered a \textit{nolo contendere} plea to the charge. Following that appearance, the Commonwealth returned with a new indictment, alleging that the town's failure to provide schoolmasters was a ``hindrance of that increase of education, which the principles of a free government require, and against the peace and dignity of the commonwealth." The court found that the town had answered the offence proscribed by statute through its \textit{nolo contendere} plea and that the Commonwealth's second allegation had no statutory grounding. It ruled that ``[t]he indictment is clearly bad, and no judgment can be rendered'' from it. Although brief, this decision established that a \textit{nolo contendere} plea was sufficient to trigger the same double jeopardy protections as a guilty plea.

By 1829, the Massachusetts Supreme Judicial Court had developed and deployed a comprehensive definition of the \textit{nolo contendere} plea. This development was apparent in the case of \textit{Commonwealth v James Horton}, where the state charged the defendant with ``a breach of the law relating to retailers.'' When arraigned, Horton said he would not contend with the Commonwealth, effectively pleading \textit{nolo contendere}. As a result, the court fined him \$21 plus costs. On appeal, both parties argued that the plea was ``irregular,'' ``was no answer to the indictment,'' and amounted, at most, to an ``implied confession.'' He claimed that his failure to contest was insufficient to convict him on appeal.

The court held that a plea of \textit{nolo contendere} is an implied confession and that it is within the court's discretion to either receive or reject it. In support of this position, the court referenced its practice of allowing such defendants to argue for a mitigated sentence. Notably, the court remarked that the main material difference between a \textit{nolo contendere} plea and a guilty plea was the ability of a defendant who pleads nolo contendere to contest the facts contained in an indictment at a subsequent civil suit. The court found its ability to enter a conviction upon a plea of nolo contendere "follows by necessary legal inference from the implied confession" and dismissed the appeal. The court cited Hawkins' notation in the Treatise of Pleas of the Crown and the 1702 English decision The \textit{Queen v Templeman} when defining the contours of nolo contendere pleas in their jurisdiction.

The centrality of \textit{Templeman} and Hawkins' notation in these early decisions underscores the role they have played in the preservation and revival of \textit{nolo contendere} pleas in the United States. However, the recurrence of these two sources and the absence of many others also highlights the relatively limited pool of common-law authority for these pleas. Nearly one hundred years after \textit{Horton}, the US Supreme Court would seize on this fact when it was asked to read a common-law limitation into nolo contendere pleas across the country in \textit{Hudson et al v United States}.

\subsubsection{\textit{Hudson et al v United States}}
In \textit{Hudson}, the appellants were indicted on mail fraud charges, which were punishable by either a fine, custody or both. They entered \textit{nolo contendere} pleas and were sentenced to a year plus a day. They appealed on the question of whether the sentencing court could impose a custodial sentence on a \textit{nolo contendere} plea. The appellants cited a string of authorities from the Seventh Circuit Federal Court of Appeals supporting their argument that courts may only accept common-law \textit{nolo contendere} pleas for ``misdemeanours for which punishment may be imposed by fine alone." These decisions, in turn, relied heavily upon Hawkins' citation and its reference to both the ``small fine" a defendant submitted themselves to when pleading \textit{nolo contendere}, as well as the "case not capital" limitation Hawkins referenced.

The court in \textit{Hudson} noted that there were relatively few authorities to cite concerning nolo contendere pleas apart from Hawkins and therefore found an insufficient foundation for the inferences the appellants asked the court to draw. The court in \textit{Hudson} found that the plea of \textit{nolo contendere} had developed over time into its present form and that the language of Hawkins' text itself lacked the normative character the appellants asked the court to give to it. Noting the meagre, inconclusive, and unpersuasive historical background of nolo contendere pleas generally, the \textit{Hudson} court rejected the appellant's argument that custodial sentences were unavailable for nolo contendere pleas. Although most American jurisdictions at the time had already established that courts could sentence defendants pleading \textit{nolo contendere} to custody, the \textit{Hudson} decision cemented this, greatly contributing to the development of the \textit{nolo contendere} plea.

\subsubsection{\textit{Nolo contendere} pleas in America today}

When the United States Supreme Court decided \textit{Hudson}, the American \textit{nolo contendere} plea was primarily a product of American common law. In the nearly one hundred years since most American states have passed legislation recognizing some version of a \textit{nolo contendere} plea. In those states where the legislatures have not adopted the \textit{nolo contendere} plea through statute, courts have generally held that it is unavailable. Although some jurisdictions, such as Wyoming, had historically allowed \textit{nolo contendere} pleas notwithstanding a lack of enabling legislation, most others have expressly forbidden them for that exact reason. 

With most states now formally authorized \textit{nolo contendere} pleas, different jurisdictions have taken different approaches to them. Of the states that do allow \textit{nolo contendere} pleas, several have adopted some version or another of the "federal rules" on \textit{nolo contendere} pleas.\footnote{The federal rules for pleas, including \textit{nolo contendere} pleas, are found in Rule 11 (pleas) of the Federal Rules of Court and Rules 410 (admissibility of pleas, plea discussions, and related statements) and 803 (hearsay exceptions) of the Federal Rules of Evidence. States whose rules on \textit{nolo contendere} pleas closely resemble the federal rules include Arizona, Delaware, Hawai'i, Oregon, Tennessee, Vermont, and West Virginia. While the statutory language these states use is very similar, there are subtle differences between the rules that result. See the table below for details.} Some states allow \textit{nolo contendere} pleas unconditionally, while others only allow them with consent from the prosecutor, the court, or both. In most jurisdictions, evidence of\textit{nolo contendere} pleas is inadmissible in subsequent civil proceedings. However, several states have gone further, extending subsequent inadmissibility to criminal proceedings.\footnote{Georgia, in particular, takes a particularly stark position on admitting evidence of \textit{nolo contendere} pleas:

\begin{quote}
    Except as otherwise provided by law, a plea of nolo contendere shall not be used against the defendant in any other court or proceedings as an admission of guilt or otherwise or for any purpose; and the plea shall not be deemed a plea of guilty for the purpose of effecting any civil disqualification of the defendant to hold public office, to vote, to serve upon any jury, or any other civil disqualification imposed upon a person convicted of any offense under the laws of this state.
    
    \begin{quote}
        See GA ST § 17-7-95(c).
    \end{quote}
    
\end{quote}}

Even amongst states where \textit{nolo contendere} pleas are not permitted, several retain the language of the federal rule. This phenomenon is especially evident in those states that do not allow \textit{nolo contendere} pleas but whose rules of evidence nonetheless restrict the admissibility of \textit{nolo contendere} pleas in subsequent proceedings.

Prior to codification, different courts in different jurisdictions understood the nature and purpose of \textit{nolo contendere} pleas differently. While several of these differences undoubtedly made their way into the final legislated versions of the plea, this was despite early attempts by American jurists to unify common law courts in a correct approach to the plea.

\subsection{The four components of \textit{nolo contendere} pleas}

Since the mid-20\textsuperscript{th} century, American jurists have distinguished the different variations of \textit{nolo contendere} plea from one another using a four-component classification system. This system, first proposed in the 1944 American Law Review annotation "Plea of \textit{nolo contendere} or \textit{non vult contendere}" provided an early, comprehensive overview of \textit{nolo contendere} pleas and the lines along which various states had diverged from one another in their approach to these pleas. The annotation sought to clarify misunderstandings about the plea that had contributed to conflicting court decisions and uneven applications. To that end, it developed a model of the \textit{nolo contendere} plea that understood the plea as being composed of four fundamental aspects:

\begin{itemize}
    \item \textbf{Applicability} considers whether the plea applies to some, all, or no cases and whether the court may only accept the plea in answer to certain classes of offences;
    \item \textbf{Acceptability} assesses what conditions must obtain for the plea to be accepted or withdrawn;
    \item \textbf{Effect in the case} evaluates what effects the plea has on the case after the court accepts it; and
    \item \textbf{Consequences outside the case} gauges the consequences of the plea or the court's decision outside that particular case.
\end{itemize}

The ostensible purpose of these fundamental aspects was to assist the reader in identifying authentic expressions of the \textit{nolo contendere} plea and to assist the courts in implementing the plea correctly. Correctly identified through these fundamental aspects, the hope was that the \textit{nolo contendere} plea could be uniformly understood and applied. However, with the states having since codified \textit{nolo contendere} pleas \textit{en masse}, and with the courts generally refusing to recognize the plea where the legislature has not expressly provided for it, the goal of harmonizing the common-law approach to the plea has become unimportant. Legislative primacy ensures that absent constitutional concerns, the pleas as legislated as \textit{nolo contendere} pleas are valid, notwithstanding differences between jurisdictions or divergence from historical common-law precedents.

Nevertheless, these fundamental aspects still provide contemporary jurists and legal scholars with comprehensive way to categorize the \textit{nolo contendere} plea's distinguishing and essential characteristics. This, in turn, may serve as a valuable rubric for comparing different individually codified instances of the plea with one another. Numerous American authorities have since recognized and utilized these characteristics when considering the nature and purpose of \textit{nolo contendere} pleas. Just as these categories may "clarify misunderstandings" and lead to less conflicting decisions, they can also be used to understand and identify how different specific instances of \textit{nolo contendere} pleas relate to one another more generally. Since their development, the four fundamental aspects of \textit{nolo contendere} pleas have become more critical to understanding the plea than the historical citations from Hawkins, \textit{Templeman}, or any case reported in the 15\textsuperscript{th} time. As will be demonstrated later in this thesis, this classification system also allows emerging phenomena, like the \textit{nolo contendere} plea procedure that is starting to be utilized in Canada, to be compared to and contrasted with their formal counterparts.

\subsubsection{Applicability}

The first component, applicability, addresses the question of which offences, if any, a defendant may enter a \textit{nolo contendere} plea in response to. Historically, American authorities restricted defendants from entering nolo contendere pleas in response to anything but misdemeanour-level offences for which a fine was the expected punishment. Many held that the oft-cited paragraph from Hawkins confined \textit{nolo contendere} pleas to charges that were punished by a fine. Before *Hudson*, dissent on this point was frequent: some courts felt that nolo contendere pleas could only be accepted when a defendant's jeopardy was comparatively low. In contrast, others saw no need to restrict judges from accepting nolo contendere pleas and sentencing offenders to custodial sentences. To this day, some states still only permit \textit{nolo contendere} pleas in answer to specific regulatory or quasi-criminal offences.

However, the Hudson decision confirmed that nolo contendere pleas could be entered, at common-law, for indictable felonies that were either punishable or punished by lengthy prison terms. Despite this expanded view of the common-law scope of \textit{nolo contendere} pleas, most appellate decisions continued to hold that nolo contendere pleas were unavailable for capital offences. Multiple states legislated this restriction as they formally permitted nolo contendere pleas. However, by the early 1960s, several cases held that nolo contendere pleas would not be logically excluded in capital cases. Only twelve states have no statute enabling nolo contendere pleas in answer to criminal charges at the time of this writing.

The applicability of a nolo contendere plea can be further reduced to one of four distinct types, such that it can be said to apply to:

\begin{enumerate}
\item \textit{All} offences;
\item Only \textit{non-capital} offences;
\item \textit{Some} offences;\footnote{While the ``only non-capital offences" category is a subset of the ``some offences" category, the historical importance of only allowing \textit{nolo contendere} pleas to be entered in non-capital cases warrants pointing it out as a special permutation. Others are of course possible, and will be proposed throughout this thesis.} or
\item \textit{No} offences.\footnote{Formally, this may be expressed as: 
\begin{flalign*}
\text{x: a criminal offence} &&\\
\text{y: a criminal offence punishable by death} &&\\
\text{P: a \textit{nolo contendere} plea may be entered} &&\\
(1)        &               &  \forall x P(x)\\
(2)        &               &  \forall x P(x \land\ \lnot y)\\
(3)        &               &  \exists x P(x)\\
(4)        &               &  \forall x \ \lnot P(x)
\end{flalign*}}
\end{enumerate}

Despite having been historically confined to minor criminal infractions at common law, a review of the \textit{nolo contendere} plea today reveals that the plea is usually universally applicable where allowed. Of the 38 states that allow \textit{nolo contendere} pleas in criminal cases, 34 permit defendants to enter them for all criminal offences. Two states with the death penalty explicitly exclude \textit{nolo contendere} pleas in capital cases,\footnote{Namely, Georgia and Louisiana.} while two others allow them only in misdemeanour cases.\footnote{Namely, Mississippi and South Carolina.} The wide availability of \textit{nolo contendere} pleas in the United States today has been a sea change since its limited use as a ``plea for mercy'' in Henry VI's time, or even the "submission to a small fine" alluded to in Hawkins' notation. Even the ``non-capital'' requirement, once thought to be an integral component of \textit{nolo contendere} pleas,\footnote{Cite to the first Drechsler annotation for an early, comprehensive expression of this view.} has been almost entirely legislated out of the plea. Descriptively, the applicability of \textit{nolo contendere} pleas in the United States today is broad.

The scope of this thesis is limited to \textit{nolo contendere} pleas that are entered in response to criminal charges. However, it is worth nothing that although some jurisdictions do not allow \textit{nolo contendere} pleas in \textit{criminal} cases, they may permit them in other legal contexts. In some states, like Illinois or New York, \textit{nolo contendere} pleas may be entered for tax act or liquor law violations, respectively,\footnote{See IL ST CH 725 § 5/113-4.1. Plea of nolo contendere and ... .} but not in response to criminal charges. In Canada, several non-criminal courts and tribunals recognize pleas of "no contest" in answer to violations under various acts, despite being formally prohibited in criminal proceedings.\footnote{Cite some cases from CanLII.} 

\subsubsection{Acceptability}

The second component, acceptability, addresses the questions of whether and under what circumstances the \textit{nolo contendere} plea can be accepted by the court. The “acceptability” of a \textit{nolo contendere} is, at its base, a function of a judge’s discretion to accept or reject a \textit{nolo contendere} plea, and what conditions, if any, they are subject to in doing so. The question of whether and under what circumstances a plea may be accepted by the court may just as easily be understood in terms of what discretion that court has, if any, to accept or reject that plea.

\begin{enumerate}
\item \textit{No} discretion to \textit{accept} the plea;
\item \textit{Some} discretion to \textit{accept} the plea;
\item \textit{Full (or guilty plea equivalent)} discretion to \textit{accept} the plea; or
\item \textit{No} discretion to \textit{reject} the plea.\footnote{Formally, this may be expressed as: 
\begin{flalign*}
\text{x: a \textit{nolo contendere} plea} &&\\
\text{y: a guilty plea}\\
\text{V: a judge validates the plea} &&\\
\text{O, PH, P: obligation, prohibition, and permission} &&\\
(1)        &               &  \forall x O(x)\\
(2)        &               &  \forall x [V(x) \implies P(x)]\\
(3)        &               &  \forall x P(x)\ \lor [P(x) \iff P(y)] \\
(4)        &               &  \forall x PH(x)
\end{flalign*}}
\end{enumerate}

Unless legislation specifically provides for a \textit{nolo contendere} plea, they are typically not allowed. Jurisdictions that allow \textit{nolo contendere} pleas to be entered subject to certain conditions being met may be understood as giving conditional, or partial discretion to the court to accept \textit{nolo contendere} pleas. The ``public interest and effective administration of justice” test required by the federal rule, for example, has been mirrored in the legislation of several states, and serves as a high-level limit on a judge's discretion to accept the plea.\footnote{List some states that do this.} Many states also require that the courts obtain the explicit consent of the prosecutor before accepting \textit{nolo contendere} pleas.\footnote{See, for example, some states that do this.} Other jurisdictions impose no apparent limits on \textit{nolo contendere} pleas beyond those already imposed on guilty pleas.\footnote{See some more examples.} The final type of acceptability, understood as a function of judicial discretion, is that where the court has no discretion to reject a \textit{nolo contendere} plea.\footnote{Only Virginia appears to give judges no discretion to reject a \textit{nolo contendere} plea, despite having discretion to reject a guilty plea. See the statute that authorizes this. As will be seen below, however, the \textit{nolo contendere} procedure used and authorized in Canada may give prosecutors and defendants the power to compel a judge to accept a \textit{nolo contendere}-like plea. This theory is outlined in §4.x below.}

\subsubsection{Procedural effects}

Generally, both \textit{nolo contendere} and best-interest pleas have the same legal effect as a guilty plea within the proceedings themselves. This includes the constitutional right against double jeopardy and a trial waiver. Historically, guilty pleas at common law were considered to be a simple admission of the facts alleged in the information, and not necessarily a waiver of due process or the right to a trial.\footnote{See @1992CanLII2570.} Prior to the widespread codification of criminal law and procedure in Canada, and later the United States, clearly equivocating the procedural effects of a \textit{nolo contendere} plea with the procedural effects of a guilty plea would have been an important distinction to draw. However, for the most part, where states allow defendants to plead \textit{nolo contendere}, the effect of that plea is the same as if the defendant had pleaded guilty.\footnote{Some states make this explicit\footnote{See eg Oregon (O.R.S. § 135.345. No contest plea as conviction), Rhode Island (RI R REV Rule 609), New Mexico (NM ST § 30-1-11. Criminal sentence permitted only upon conviction), and Louisiana (LA C.Cr.P. Art. 552(4). Pleas at the arraignment)}} The defendant waives their right to a trial, a conviction is entered, and the parties proceed to sentencing.

There are, however, some procedural differences in some states. In Massachusetts, defendants who enter not guilty pleas are statutorily prohibited from entering into formal plea agreements with the prosecutors. However, defendants who admit sufficient facts rather than plead *\textit{nolo contendere}* may enter into such agreements.\footnote{See MA ST RCRP Rule 12(b); Rule 12(b)(1) commentary: Rule 12(b)(1) makes it clear that the defendant may tender a guilty plea, a nolo contendere plea, or, in District Court, an admission to sufficient facts, without entering into a plea agreement. See Rule 2(b)(7) (defining “District Court” to include all divisions of the District Court, Boston Municipal Court, and Juvenile Court). However, the rule also provides that the parties may condition a guilty plea (or, in District Court, an admission to sufficient facts) on a plea agreement under Rule 12(b)(5), discussed below. Rule 12(b)(1) omits nolo contendere pleas from those that can be conditioned on a plea agreement, an omission that Rule 12(b)(5) makes explicit, thus limiting the benefits of a plea agreement to those defendants who take responsibility for the crimes to which they are pleading.}.

In Mississippi, judges are required to conduct a plea voluntariness and comprehension inquiry with defendants who enter guilty pleas to any offence that carries a possible jail sentence. No such requirement appears to be in place for defendants who enter \textit{nolo contendere} pleas to the same offences.\footnote{See MS R RCRP Rule 15.3.} By contrast, California requires judges to conduct a special plea inquiry with defendants who enter a \textit{nolo contendere} plea that does not need to be conducted with defendants who plead guilty.\footnote{See West's Ann.Cal.Penal Code § 1016(3).} In Oregon, judges are statutorily required to accept joint recommendations put forward by counsel on a guilty plea, but not similarly required to do the same for \textit{nolo contendere} pleas. Other states, like Ohio, make it clear that \textit{nolo contendere} pleas are not admissions of guilt and distinguish them from guilty pleas accordingly. 

Nevertheless, and apart from these variations, \textit{nolo contendere} pleas generally appear to be the procedural equivalent of a guilty plea by most states that allow defendants to enter them. 

\subsubsection{Subsequent effects}

In most American jurisdictions, the predominant difference between \textit{nolo contendere} and guilty pleas has historically been that while evidence of a defendant's guilty plea may normally be admitted as evidence in a subsequent proceeding, evidence of a defendant's \textit{nolo contendere} plea could not. 

As discussed above, even in jurisdictions that don't permit \textit{nolo contendere} pleas, evidence of these pleas is often considered to be inadmissible in subsequent civil, and even criminal proceedings. While there are 12 states that do not criminal defendants to enter \textit{nolo contendere} pleas, only seven states allow evidence of \textit{nolo contendere} pleas to be used as evidence in subsequent proceedings. Two of these states, Alaska and Arizona, allow defendants to enter \textit{nolo contendere} pleas but also allow evidence of those pleas to be subsequently admitted. Similarly, in New Jersey, \textit{nolo contendere} pleas are both prohibited and afforded no special protections in subsequent civil or criminal proceedings.\footnote{No protective provisions in NJ R EVID N.J.R.E. 410.} However, the court may make an order that a defendant's guilty plea not be admissible in subsequent civil proceedings in New Jersey, upon a defendant's request and at the court's discretion. Illinois, Indiana, Missouri, and New York are the only other state that do not allow defendants to plead \textit{nolo contendere} nor recognize \textit{nolo contendere} pleas entered in other jurisdictions. All other states provide \textit{nolo contendere} defendants with some degree of collateral estoppel, though the degree of protection provided also varies from state to state.

In Pennsylvania, evidence of \textit{nolo contendere} is generally inadmissible. However, when a defendant enters a \textit{nolo contendere} plea to a crime of dishonesty, evidence of that offence must be admitted.\footnote{See PA ST REV Rule 609: **(a) In General.** For the purpose of attacking the credibility of any witness, evidence that the witness has been convicted of a crime, whether by verdict or by plea of guilty or \textit{nolo contendere}, must be admitted if it involved dishonesty or false statement.} In California, defendants who enter \textit{nolo contendere} pleas to misdemeanour offences are protected from having evidence of that admitted in a later court proceeding, but have no such protections for felonies.

Still others, like Michigan, bar admitting evidence of \textit{nolo contendere} pleas in subsequent civil suits, except where the defendant who entered the plea is the one filing the suit.\footnote{See MI R REV MRE 410: (2) A plea of nolo contendere, except that, to the extent that evidence of a guilty plea would be admissible, evidence of a plea of nolo contendere to a criminal charge may be admitted in a civil proceeding to support a defense against a claim asserted by the person who entered the plea.} 

South Carolina allows \textit{nolo contendere} pleas to be admitted for impeachment purposes only, if the conviction was for an offence punishable by death or more than one year imprisonment.\footnote{See SC R REV Rule 609: (a) General Rule. For the purpose of attacking the credibility of a witness, (1) evidence that a witness other than an accused has been convicted of a crime shall be admitted, subject to Rule 403, if the crime was punishable by death or imprisonment in excess of one year under the law under which the witness was convicted, and evidence that an accused has been convicted of such a crime shall be admitted if the court determines that the probative value of admitting this evidence outweighs its prejudicial effect to the accused.} Meanwhile, the approach that some states, like Rhode Island, have stipulated that the benefits of a \textit{nolo contendere} plea may only obtain after a defendant has successfully completed a period of probation.

Most states recognize that some form of collateral estoppel applies to defendants who enter \textit{nolo contendere} pleas. Some of those states have forged exceptions and implemented changes to collateral estoppel to limit its application, or to make it contingent on the defendant doing well in the community. 

\subsection{\textit{Nolo contendere} pleas in Canada}

As discussed above, \textit{nolo contendere} pleas disappeared from the English legal tradition near the beginning of the 18\textsuperscript{th} century, only to re-emerge in the American legal tradition a century later. As a result of this timing, \textit{nolo contendere} pleas did not factor at all in the development of the Canadian criminal law. Canada codified its criminal law early into its history, and functionally excluded every plea but guilty or not guilty in every iteration. For most of Canadian criminal legal history, \textit{nolo contendere} pleas have not been a subject that has warranted or attracted much discussion.

In recent years, however, Canadian courts have begun to recognize an informal \textit{nolo contendere} plea of sorts. Generally referred to by the courts as a ``\textit{nolo contendere} plea procedure,'' it allows a defendant to avoid pleading guilty but still ensure a self-conviction, despite the explicit restriction in \textit{Criminal Code} s 606. Having the same effects as a formal \textit{nolo contendere} plea, the plea procedure may be classified on and understood in light of Dreschler's four-fold classification system for formal \textit{nolo contendere} pleas. The history and contours of this plea procedure will be discussed in more detail in the next chapter.

\newpage
\subsection{\textit{Nolo contendere} data table}

The bottom of the table shows how both formal and informal \textit{nolo contendere} pleas are dealt with in Canada. This will be elaborated on in much greater detail in 3.3: Compatibility with \textit{nolo contendere} pleas below.

The table\footnote{Currently viewable at \url{https://docs.google.com/spreadsheets/d/1NEnrncRwJz6DpluLl7AAXURwdUDINTac3EoBJZOuhGU/edit\#gid=0}} will need to be translated into \LaTeX  sooner or later.