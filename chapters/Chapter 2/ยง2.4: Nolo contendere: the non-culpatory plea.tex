\section{\textit{Nolo contendere}: the non-culpatory plea}

A defendant who enters a \textit{nolo contendere} plea invites the court to convict them but neither admits nor denies their involvement. They accept the consequences of a criminal conviction without taking responsibility for the underlying events. At common-law, and in most states, defendants who enter a \textit{nolo contendere} plea may not have evidence of their \textit{nolo contendere} plea admitted at certain subsequent proceedings.\footnote{This is often true even in jurisdictions that do not otherwise recognize or allow \textit{nolo contendere} pleas. Alabama, for example, is among those states that do not allow defendants to enter \textit{nolo contendere} pleas. They are also one of a very few states not to admit evidence of \textit{nolo contendere} pleas for any reason in subsequent proceedings: ``Alabama [...] follows the minority rule that a conviction resulting from a plea of \textit{nolo contendere} is inadmissible not only to prove the conduct underlying the conviction but for all other purposes.” See also \textit{McNair v State}, 653 So. 2d 320 (Ala. Crim. App. 1992). Georgia, in particular, also takes a particularly stark position on admitting evidence of \textit{nolo contendere} pleas:

\begin{quote}
    Except as otherwise provided by law, a plea of nolo contendere shall not be used against the defendant in any other court or proceedings as an admission of guilt or otherwise or for any purpose; and the plea shall not be deemed a plea of guilty for the purpose of effecting any civil disqualification of the defendant to hold public office, to vote, to serve upon any jury, or any other civil disqualification imposed upon a person convicted of any offense under the laws of this state.
    
    \begin{quote}
        See GA ST § 17-7-95(c).
    \end{quote}
    
\end{quote}}

\subsection{The common-law origin of \textit{nolo contendere} pleas}

Although \textit{nolo contendere} pleas became integral to American criminal law through common-law usage, their origins are relatively obscure. Legal and academic commentators agree that the plea emerged in England in the early 15\textsuperscript{th} century and disappeared from England in the early 18\textsuperscript{th} century, but know little else about the plea's use or origins. Historically and today, authorities frequently cite a brief excerpt from William Hawkins' 1716 edition of ``Treatise of Pleas of the Crown,"\footnote{Cite. See also (sources that use this citation).} unchanged since it first appeared in 1716, as their primary source of information about the plea's origins:

\begin{quote}
\singlespacing
An implied confession is when a defendant, in a case not capital, doth not directly own himself guilty, but in a manner admits it by yielding to the King's mercy, and desiring to submit to a small fine: in which case, if the court think fit to accept of such submission, and make an entry that defendant \textit{prosuit se in gratiam regis}, without putting him to a direct confession, or plea (which in such cases seems to be left to discretion), the defendant shall not be estopped to plead not guilty to an action for the same fact, as he shall if the entry is \textit{quod cognovit indictamentum}.\footnote{@hawkinsTreatisePleasCrown1824}
\end{quote}

By the time Hawkins published the first edition of his Treatise of Pleas of the Crown, \textit{nolo contendere} pleas had started to fall into disuse. The last reported case involving a nolo\textit{ contendere} plea in England was \textit{The Queen v Templeman} in 1702.\footnote{\begin{quote}
\singlespacing
Upon a motion to submit to a small fine, after a confession of the indictment, which was for an assault, Holt, Ch. J., took a difference where a man confesses an indictment, and where he is found guilty; in the first case a man may produce affidavits to prove son assault upon the prosecutor in mitigation of the fine: otherwise where the defendant is found guilty; for the entry upon a confession is only \textit{non vult contendere cum domina Regina and pon. se in gratiam Curiae}. Defendants may submit to a fine, though absent, if they have a clerk in court that will undertake for the fine (Hill. 2 Ann). \textit{Hickeringil's Case} was that he and his daughters were indicted for trespass, and Hickeringil only appeared on the motion to submit to a small fine. But where a man is to receive any corporal punishment, judgment cannot be given against him in his absence, for there is a capias pro fine; but no process to take a man and put him on the pillory. \textit{Vide tit}. Judgments. \textit{Duke's Case}.
\end{quote}} Thereafter, the \textit{nolo contendere} plea lay dormant in English criminal law reports. It would only be a century later that it re-emerged in reported American decisions.

\subsection{\textit{Nolo contendere} pleas in America historically}
One of the earliest reported American cases dealing with \textit{nolo contendere} pleas is the 1806 decision \textit{Commonwealth v The Town of Northampton}. The town of Northampton entered a \textit{nolo contendere} plea to the charge of failing to provide a schoolmaster, only to have the Commonwealth lay the same charge again. The court found that the town had answered the offence proscribed by statute through its \textit{nolo contendere} plea and that the Commonwealth's second allegation had no statutory grounding. It ruled that ``[t]he indictment is clearly bad, and no judgment can be rendered'' from it and established that a \textit{nolo contendere} plea was sufficient to trigger the same double jeopardy protections as a guilty plea.

By 1829, the Massachusetts Supreme Judicial Court had deployed a comprehensive \textit{nolo contendere} plea. This development was apparent in the case of \textit{Commonwealth v James Horton}, where the state charged the defendant with ``a breach of the law relating to retailers.'' When arraigned, Horton said he would not contend with the Commonwealth, effectively pleading \textit{nolo contendere}. As a result, the court fined him \$21 plus costs. On appeal, both parties argued that the plea was ``irregular,'' ``was no answer to the indictment,'' and amounted, at most, to an ``implied confession.''

The court held that a plea of \textit{nolo contendere} is an implied confession and that it is within the court's discretion to either receive or reject it. Notably, the court remarked that the main material difference between a \textit{nolo contendere} plea and a guilty plea was the ability of a defendant who pleads nolo contendere to contest the facts contained in an indictment at a subsequent civil suit. The court cited Hawkins' notation in the Treatise of Pleas of the Crown and the 1702 English decision The \textit{Queen v Templeman} when defining the contours of nolo contendere pleas in their jurisdiction and dismissed the appeal.

The centrality of \textit{Templeman} and Hawkins' notation in these early decisions underscores the role they have played in the preservation and revival of \textit{nolo contendere} pleas in the United States but also highlights the limited pool of common-law authority for these pleas. Nearly one hundred years after \textit{Horton}, the US Supreme Court would seize on this fact when it decided whether to read a common-law limitation into \textit{nolo contendere} pleas across the country in \textit{Hudson et al. v United States}.

\subsubsection{\textit{Hudson et al v United States}}

In \textit{Hudson}, the appellants entered \textit{nolo contendere} pleas to mail fraud charges and were sentenced to a year plus a day. On appeal, they argued that the sentencing court could not impose a custodial sentence on a \textit{nolo contendere} plea. The appellants cited a string of authorities from the Seventh Circuit Federal Court of Appeals that relied heavily upon Hawkins' reference to the ``small fine" a defendant submitted themselves to and the "case not capital" limitation. The court in \textit{Hudson} noted the dearth of authorities apart from Hawkins and therefore found little support for the inferences the appellants asked the court to draw. The court noted that \textit{nolo contendere} pleas had a minimal reliable historical backdrop and was not prepared to create a broad rule based on an obscure citation. The court ruled that custodial sentences were available for \textit{nolo contendere} pleas, echoing decisions already reached by most lower American jurisdictions at the time.

\subsubsection{\textit{Nolo contendere} pleas in America today}

When the United States Supreme Court decided \textit{Hudson}, the American \textit{nolo contendere} plea was primarily available at common law. Nearly one hundred years later, most American states have passed legislation authorizing \textit{nolo contendere} pleas, while those that have not do not allow them. Different approaches have emerged among states formally authorizing \textit{nolo contendere} pleas. Many have adopted some version of the ``federal rules" on \textit{nolo contendere} pleas.\footnote{The federal rules for pleas, including \textit{nolo contendere} pleas, are found in Rule 11 (pleas) of the Federal Rules of Court and Rules 410 (admissibility of pleas, plea discussions, and related statements) and 803 (hearsay exceptions) of the Federal Rules of Evidence. States whose rules on \textit{nolo contendere} pleas closely resemble the federal rules include Arizona, Delaware, Hawai'i, Oregon, Tennessee, Vermont, and West Virginia. While the statutory language these states use is very similar, there are subtle differences between the rules. See the table below for details.} Some states allow \textit{nolo contendere} pleas unconditionally, while others only allow them with consent from the prosecutor, the court, or both. Before codification, different courts in different jurisdictions understood the nature and purpose of \textit{nolo contendere} pleas differently. While several of these differences undoubtedly made their way into the final legislated versions of the plea, this was despite early attempts by American jurists to unify common law courts in a correct approach.

\subsection{The four components of \textit{nolo contendere} pleas}

In the 1944 American Law Review annotation ``Plea of \textit{nolo contendere} or \textit{non vult contendere}," KA Dreschler provided an early, comprehensive overview of \textit{nolo contendere} pleas and the different ways that states approached and implemented them. The annotation proposed a model of the \textit{nolo contendere} plea that broke it down into four fundamental aspects:

\begin{itemize}
    \item \textbf{Applicability} considers whether the plea applies to some, all, or no cases and whether the court may only accept the plea in answer to certain classes of offences;
    \item \textbf{Acceptability} assesses what conditions must obtain for the plea to be accepted or withdrawn;
    \item \textbf{Effect in the case} evaluates what effects the plea has on the case after the court accepts it; and
    \item \textbf{Consequences outside the case} gauges the consequences of the plea or the court's decision outside that particular case.
\end{itemize}

These fundamental aspects still provide contemporary jurists and legal scholars with a comprehensive way to categorize the \textit{nolo contendere} plea's distinguishing and essential characteristics. This classification system is a valuable rubric for comparing different individually codified instances of the plea with one another and can be used to understand and identify how different specific instances of \textit{nolo contendere} pleas relate to one another more generally. As will be demonstrated later in this thesis, this classification system also allows emerging phenomena, like the \textit{nolo contendere} plea procedure that is starting to be utilized in Canada, to be compared to and contrasted with their formal counterparts.

\subsubsection{Applicability}

The first component, applicability, addresses the question of which offences may sustain a \textit{nolo contendere}. The applicability of a nolo contendere plea is further reducible to one of four distinct types:

\begin{enumerate}
\item \textit{All} offences;
\item Only \textit{non-capital} offences;
\item \textit{Some} offences;\footnote{While the ``only non-capital offences" category is a subset of the ``some offences" category, the historical importance of only allowing \textit{nolo contendere} pleas in non-capital cases merits pointing out as a unique permutation.} or
\item \textit{No} offences.%\footnote{Formally: 
%\begin{flalign*}
%text{x: a criminal offence} &&\\
%text{y: a criminal offence punishable by death} &&\\
%\text{P: a \textit{nolo contendere} plea may be entered} &&\\
%(1)        &               &  \forall x P(x)\\
%(2)        &               &  \forall x P(x \land\ \lnot y)\\
%(3)        &               &  \exists x P(x)\\
%(4)        &               &  \forall x \ \lnot P(x)
%\end{flalign*}}
\end{enumerate}

Historically, defendants could only plead \textit{nolo contendere} to misdemeanour-level offences punishable by a fine. Before \textit{Hudson}, dissent on this point was frequent. To this day, some states still only permit \textit{nolo contendere} pleas in answer to specific regulatory or quasi-criminal offences.\footnote{} \textit{Hudson} confirmed that \textit{nolo contendere} pleas could be entered at common law for indictable felonies punishable by prison terms, thus opening the door to courts accepting \textit{nolo contendere} pleas for all manner of offences. Despite being historically confined to minor criminal infractions at common law, a review of the \textit{nolo contendere} plea today reveals that the plea is usually universally applicable where allowed. Of the 38 states that allow \textit{nolo contendere} pleas in criminal cases, 34 permit defendants to enter them for all criminal offences. Two states with the death penalty explicitly exclude \textit{nolo contendere} pleas in capital cases,\footnote{Namely, Georgia and Louisiana.} while two others allow them only in misdemeanour cases.\footnote{Namely, Mississippi and South Carolina.} The wide availability of \textit{nolo contendere} pleas in the United States today has been a sea change since its limited use as a ``plea for mercy'' in Henry VI's time, or even the "submission to a small fine" alluded to in Hawkins' notation. Even the ``non-capital'' requirement, once thought to be an integral component of \textit{nolo contendere} pleas,\footnote{Cite to the first Drechsler annotation for an early, comprehensive expression of this view.} is mainly absent from the statutory plea. Descriptively, the applicability of \textit{nolo contendere} pleas in the United States today is broad.

\subsubsection{Acceptability}

The second component, acceptability, addresses the conditions required for the court to accept the plea. A \textit{nolo contendere} plea's acceptability is a function of a judge’s discretion to accept or reject it and what requirements they are subject to in doing so, and can also be broken down into four types:

\begin{enumerate}
\item \textit{No} discretion to \textit{accept} the plea;
\item \textit{Some} discretion to \textit{accept} the plea;
\item \textit{Full (or guilty plea equivalent)} discretion to \textit{accept} the plea; or
\item \textit{No} discretion to \textit{reject} the plea.%\footnote{Formally, this may be expressed as: 
%\begin{flalign*}
%\text{x: a \textit{nolo contendere} plea} &&\\
%\text{y: a guilty plea}\\
%\text{V: a judge validates the plea} &&\\
%\text{O, PH, P: obligation, prohibition, and permission} &&\\
%(1)        &               &  \forall x O(x)\\
%(2)        &               &  \forall x [V(x) \implies P(x)]\\
%(3)        &               &  \forall x P(x)\ \lor [P(x) \iff P(y)] \\
%(4)        &               &  \forall x PH(x)
%\end{flalign*}}
\end{enumerate}

Unless legislation provides explicitly for a \textit{nolo contendere} plea, they are typically not allowed. Judges in those states have no discretion to accept the plea. Jurisdictions that allow defendants to enter \textit{nolo contendere} pleas subject to meeting certain conditions give the court conditional or partial discretion to accept \textit{nolo contendere} pleas.\footnote{The ``public interest and effective administration of justice” test required by the federal rule, for example, has been mirrored in several states' legislation and serves as a high-level limit on a judge's discretion to accept the plea. See e.g.: \hl{States that do this}. Many states also require that the courts obtain the explicit consent of the prosecutor before accepting \textit{nolo contendere} pleas. See e.g. \hl{the states which do this.}} Other jurisdictions impose no apparent limits on \textit{nolo contendere} pleas beyond those already imposed on guilty pleas.\footnote{See some more examples.} Finally, where judges must accept \textit{nolo contendere} pleas, they have no discretion to reject the plea.\footnote{Only Virginia appears to give judges no discretion to reject a \textit{nolo contendere} plea, despite having the discretion to reject a guilty plea. See the statute that authorizes this. As will be seen below, however, the \textit{nolo contendere} procedure used and authorized in Canada may give prosecutors and defendants the power to compel a judge to accept a \textit{nolo contendere}-like plea.}

\subsubsection{Procedural effects}

\textit{Nolo contendere} generally have the same legal effect as a guilty plea within the proceedings, including the constitutional right against double jeopardy and a trial waiver. For the most part, where states allow defendants to plead \textit{nolo contendere}, the effect of that plea is the same as if the defendant had pleaded guilty.\footnote{Some states make this explicit\footnote{See e.g. Oregon (O.R.S. § 135.345. No contest plea as conviction), Rhode Island (RI R REV Rule 609), New Mexico (NM ST § 30-1-11. Criminal sentence permitted only upon conviction), and Louisiana (LA C.Cr.P. Art. 552(4). Pleas at the arraignment)}} The defendant waives their right to a trial, they are convicted, and the parties proceed to sentencing.\footnote{However, some procedural differences do exist. In Massachusetts, defendants who enter \textit{nolo contendere} pleas may not enter formal plea agreements with the prosecutors. However, defendants who admit sufficient facts rather than plead \textit{nolo contendere} may enter such agreements. See MA ST RCRP Rule 12(b); Rule 12(b)(1) commentary: Rule 12(b)(1) makes it clear that the defendant may tender a guilty plea, a nolo contendere plea, or, in District Court, an admission to sufficient facts, without entering into a plea agreement. See Rule 2(b)(7) (defining “District Court” to include all divisions of the District Court, Boston Municipal Court, and Juvenile Court). However, the rule also provides that the parties may condition a guilty plea (or, in District Court, an admission to sufficient facts) on a plea agreement under Rule 12(b)(5), discussed below. Rule 12(b)(1) omits \textit{nolo contendere} pleas from those conditioned on a plea agreement, an omission that Rule 12(b)(5) makes explicit, thus limiting the benefits of a plea agreement to those defendants who take responsibility for the crimes to which they are pleading. In Mississippi, judges must conduct a plea voluntariness and comprehension inquiry with defendants who enter guilty pleas to any offence that carries a possible jail sentence. No such requirement appears to be in place for defendants who enter \textit{nolo contendere} pleas to the same offences. See MS R RCRP Rule 15.3. By contrast, California requires judges to conduct a particular plea inquiry with defendants who enter a \textit{nolo contendere} plea that is not required with defendants who plead guilty. See West's Ann.Cal.Penal Code § 1016(3). Meanwhile, in Oregon, judges are statutorily required to accept joint recommendations put forward by counsel on a guilty plea but not similarly required to do the same for \textit{nolo contendere} pleas. Other states, like Ohio, make it clear that \textit{nolo contendere} pleas are not admissions of guilt and distinguish them from guilty pleas accordingly.}

\subsubsection{Subsequent effects}

Historically, the predominant difference between \textit{nolo contendere} and guilty pleas has been that \textit{nolo contendere} pleas are usually inadmissible in subsequent proceedings. Even in jurisdictions that do not permit \textit{nolo contendere} pleas, evidence of these pleas is often considered inadmissible in subsequent civil and even criminal proceedings.\footnote{While 12 states do not allow criminal defendants to enter \textit{nolo contendere} pleas, only seven states allow evidence of \textit{nolo contendere} pleas as evidence in subsequent proceedings. Two of these states, Alaska and Arizona, allow defendants to enter \textit{nolo contendere} pleas but also allow admitting evidence of those pleas at subsequent proceedings. Similarly, in New Jersey, \textit{nolo contendere} pleas are prohibited and afforded no special protections in subsequent civil or criminal proceedings. See NJ R EVID N.J.R.E. 410. However, the court may order that a defendant's guilty plea not be admissible in subsequent civil proceedings in New Jersey, upon a defendant's request and at the court's discretion. Illinois, Indiana, Missouri, and New York are the only states that do not allow defendants to plead \textit{nolo contendere} nor recognize \textit{nolo contendere} pleas entered in other jurisdictions. All other states provide \textit{nolo contendere} defendants with some degree of collateral estoppel, though the degree of protection provided also varies from state to state.} 

In Pennsylvania, evidence of \textit{nolo contendere} is generally inadmissible. However, the court must admit evidence of that offence when a defendant enters a \textit{nolo contendere} plea to a crime of dishonesty.\footnote{See PA ST REV Rule 609: (a) \textbf{In General.} For the purpose of attacking the credibility of any witness, evidence that the witness has been convicted of a crime, whether by verdict or by plea of guilty or \textit{nolo contendere}, must be admitted if it involved dishonesty or false statement.} In California, defendants who enter \textit{nolo contendere} pleas to misdemeanour offences are protected from having evidence of that admitted in a later court proceeding but have no such protections for felonies. Still others, like Michigan, bar admitting evidence of \textit{nolo contendere} pleas in subsequent civil suits, except where the defendant who entered the plea is the one filing the suit.\footnote{See MI R REV MRE 410: (2) A plea of nolo contendere, except that, to the extent that evidence of a guilty plea would be admissible, evidence of a plea of nolo contendere to a criminal charge may be admitted in a civil proceeding to support a defence against a claim asserted by the person who entered the plea.} South Carolina allows \textit{nolo contendere} pleas to be admitted for impeachment purposes only if the conviction was for an offence punishable by death or more than one year imprisonment.\footnote{See SC R REV Rule 609: (a) General Rule. For the purpose of attacking the credibility of a witness, (1) evidence that a witness other than an accused has been convicted of a crime shall be admitted, subject to Rule 403, if the crime was punishable by death or imprisonment in excess of one year under the law under which the witness was convicted, and evidence that an accused has been convicted of such a crime shall be admitted if the court determines that the probative value of admitting this evidence outweighs its prejudicial effect to the accused.} Meanwhile, the approach that some states, like Rhode Island, have stipulated that the benefits of a \textit{nolo contendere} plea may only obtain after a defendant has completed a period of probation without violations.

\subsection{\textit{Nolo contendere} pleas in Canada}

Because Canada codified its criminal law early into its history and functionally excluded every plea but guilty or not guilty in every iteration, \textit{nolo contendere} pleas have not been a subject that has warranted or attracted much discussion. In recent years, however, Canadian courts have begun to recognize an informal \textit{nolo contendere} plea of sorts. Generally referred to by the courts as a ``\textit{nolo contendere} plea procedure,'' it allows a defendant to avoid pleading guilty but still ensure a self-conviction, despite the explicit restriction in \textit{Criminal Code} s 606. Because the plea procedure has the same effects as a formal \textit{nolo contendere} plea, it may be classified on and understood in light of Dreschler's four-fold classification system for formal \textit{nolo contendere} pleas.

\newpage
\subsection{\textit{Nolo contendere} data table}

The table\footnote{Currently viewable at \url{https://docs.google.com/spreadsheets/d/1NEnrncRwJz6DpluLl7AAXURwdUDINTac3EoBJZOuhGU/edit\#gid=0}} will need to be translated into \LaTeX  sooner or later.