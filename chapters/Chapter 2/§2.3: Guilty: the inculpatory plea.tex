\section{Guilty: the inculpatory plea}

\subsection{The history of the guilty plea}

The modern guilty plea evolved from a longstanding practice allowing defendants to confess to their crimes. Before the 11\textsuperscript{th} century Norman conquest, the Anglo-Saxons of England established a procedure for confessions to criminal charges.\footnote{See Alschuler, Plea Bargaining and Its History at 7. See also H. ADAMS, H. LODGE, E. YouNG \& J. LAUGHLIN, ESSAYS IN ANGLO-SAXON LAW at 285, \url{https://babel.hathitrust.org/cgi/pt?id=mdp.39015017688113\&view=1up\&seq=314.}} Confessions were rudimentary and fraught with risks to those who entered them. Still, they served as the precursor to the modern-day guilty plea and its procedural protections.

Although self-convictions have long been allowed in common law jurisdictions, there is some debate between jurists, scholars, and legal professionals as to how frequently the courts used these procedures and customs and how well the judiciary historically received them. These disagreements tend to form along the lines of one's view of the practice of plea bargaining. Proponents of plea bargaining see guilty pleas as an age-old institution,\footnote{See example.} while opponents of plea bargaining tend to view them as a more recent procedural compromise made in the name of expedient justice.\footnote{See Alschuler, Plea Bargaining, at 2.} Both may agree that evidence of a modern equivalent of a guilty plea has been largely absent for much of common law history. Fewer reports of common law confessions exist compared to reports of jury trials,\footnote{Alschuler takes the view that guilty pleas were relatively rare until the 19\textsuperscript{th} century. He relies on the fact that relatively few reported decisions cover defendants pleading guilty before this time. There are problems with this approach. Notably, the percentage of reported criminal decisions recording guilty pleas nowadays does not align with the percentage of criminal cases that resolve in a guilty plea. Compare ONCJ stats for 2021 and the 2021 PC survey results. Thus it is hasty to conclude that the low number of reported guilty pleas before the 19\textsuperscript{th} century correlates with a low number of guilty pleas defendants entered. Alschuler's own observation further undermines his position that courts had already begun requiring that all guilty pleas entered be voluntary, a relatively sophisticated development for a supposedly underutilized plea procedure.} and there is some evidence to suggest that some courts ran trials as frequently as contemporary courtrooms hear guilty pleas.\footnote{Find and cite the article that mentions courts where ran dozens of trials daily. Probably Bibas.} 

As common law confessions gave way to the nascent beginnings of a modern guilty plea, there is evidence that judges actively discouraged them and were reluctant to accept them. It was only toward the end of the 19\textsuperscript{th} century that courts began regularly reporting cases involving guilty pleas,\footnote{See Alschuler at 7 - 8.}, and by the 1920s, criminal justice surveys in America revealed that many more defendants opted for self-conviction over a trial.\footnote{See Alschuler at 26. The percentages of guilty pleas Alschuler reports range between 76 - 90\% in several major American cities at the time.} These surveys prompted subsequent investigations into the role that plea bargaining played in these statistics. Many were initially surprised by these outcomes, and plea bargaining began with more critics than advocates. Nevertheless, it continues to be practiced, developed, regulated and refined to the present day.

\subsection{Plea voluntariness and comprehension}

Most, if not all common law jurisdictions require that defendants enter guilty pleas knowingly and voluntarily. However, the way that courts understand and apply the idea of a ``knowing and voluntary'' guilty plea to the criminal cases they try differs from jurisdiction to jurisdiction.  Different jurisdictions require courts to canvass different factors, but some factors commonly arise. These include:

\begin{itemize}
\item \textbf{Voluntariness.} Defendants entering pleas must do so voluntarily and not as the result of undue coercion, threats, or inducements.
\item \textbf{Knowledge.} The defendant must understand the nature and consequences of their plea.
\item \textbf{Factual basis.} Courts must be satisfied that there is enough evidence to sustain a conviction.\footnote{This requirement is similar to the requirement in many (though not all) American states that \textit{nolo contendere} pleas be entered with a factual foundation, and the rule that all \textit{Alford} pleas must be supported by a factual foundation.}
\item \textbf{Collateral consequences.} A criminal conviction may result in automatic penalties that affect other aspects of a person's life, such as their ability to remain in the country, operate a vehicle, or purchase a firearm.
\end{itemize}

Not every jurisdiction requires each of these factors be canvassed with defendants prior to sentencing. Where these factors are canvassed, local legislation, court rules, and customs may require courts to consider additional and different sub-factors, or examine the factors in light of certain and different circumstances. 

\subsubsection{Voluntariness and  knowledge}

Accepting that judges remained wary of guilty pleas for much of their existence, the common law requirement that defendants enter guilty pleas voluntarily, subjectively understood, and based on a factual foundation makes perfect contextual sense. As early as the 16\textsuperscript{th} century, English courts required that defendants entering guilty pleas enter them voluntarily,\footnote{See Alschuler at 12.} a practice that Alschuler attributes to the co-evolution of the plea alongside the common law confessions rule.\footnote{See Alschuler at 12 - 15.} A defendant's right to be presumed innocent until proven guilty beyond a reasonable doubt is the common law's primary tool to prevent state overreach and abuses. To preserve these fundamental rights, defendants must understand that confessing may compromise those rights and increase the likelihood of a subsequent criminal conviction. This is because confessions, like guilty pleas, can operate as a form of self-conviction. While voluntariness and knowledge requirements are more often implied than stated, these deeply- and commonly-held common law principles are usually in the back of the minds of the judges deciding whether a plea was knowingly and voluntarily entered. 

\subsubsection{Factual basis}

In addition to needing to ensure that guilty pleas are entered knowingly and voluntarily, judges must also ensure that there is some factual basis for the plea. People should not be expected to be sentenced for criminal charges that they are not or are unlikely to be found guilty of, and prosecutors have an obligation to not pursue charges that lack a reasonable prospect of conviction.\footnote{See case.} However, the obligation on sentencing judges to ensure that there is a factual foundation for the plea ensures that the allegations are ``peer reviewed'' in a sense. This is especially important when dealing with unrepresented litigants. 

\subsubsection{Collateral consequences}

The term ``collateral consequences" refers to the consequences resulting from a guilty plea that do not directly stem from the sentence the court imposes. 

Collateral consequences can be legal or extralegal. Legal collateral consequences result from 

Not every jurisdiction pays these factors the same attention. In Canada, for example, collateral consequences have been found to be a valid consideration on sentencing, but are not explicitly required to be canvassed with defendants prior to them entering their guilty pleas. ... Where collateral consequence inquiries are required, they are not uniform, with some states requiring that courts inform defendants of some consequences, but not others. 

Many American jurisdictions have codified requirements that courts alert defendants of certain collateral consequences of pleading guilty to an offence. Requirements that courts tell defendants about collateral gun ownership and immigration consequences are common.\footnote{For gun ownership, see: ; for immigration, see: .}

\subsection{Guilty pleas in Canada}

Prior to the codification of the Canadian criminal law in 1892, only two reported cases dealing with guilty pleas appear to have been reported in Canada.\footnote{See \textit{R v Morrison}, 1879 CarswellNB 35 and \textit{R v Morin}, 1890 CarswellQue 17.} However, following codification, reported cases dealing with guilty pleas abounded. This resulted in early and frequent developments to guilty plea procedures in Canada.  

Judicial decisions throughout Canada began to demonstrate the importance of ensuring that defendants only entered guilty pleas knowingly and voluntarily and started to develop the procedural rights and waivers that guilty pleas entailed. In the early days of Canadian criminal law, courts did not agree whether entering a guilty plea waived a defendant's right to have a trial.\footnote{See @1992CanLII2570.} Following codification, it became more explicit at law that guilty pleas amounted to a complete waiver of the right to trial.\footnote{See @1914CarswellYukon6, where the court treats the subject as a viable but generally settled question of law.} Before codification, the law geared towards the principle that a person pleads guilty must be presumed to know the law.\footnote{See \textit{Morrison} at paras 9, 13; \textit{Morin} at para 145.} However, developments in the following decades began to shed more light on the importance of ensuring that defendants better knew the implications of their plea before entering it.

\subsubsection{Voluntariness and knowledge}

See @1908CarswellSask23, where the Saskatchewan District Court investigated whether a guilty plea was entered under oppressive circumstances, or if any inducements were made to have it entered

See @1928CarswellNS10, where the Nova Scotia Supreme Court allowed a defendant to withdraw his guilty plea after it was entered without adequate knowledge of the consequences of doing so

\subsubsection{Factual basis}

See @1903CarswellOnt829, where the Ontario Supreme Court quashed a guilty plea, having found that it was not consistent with facts otherwise accepted as proven.

\subsubsection{Collateral consequences}



Some common law requirements.\footnote{See \textit{R v Wong}, 2018 SCC 25: legally relevant collateral consequences.}

\subsubsection{Unequivocal}

Unlike other common law jurisdictions, Canadian common law requires that defendants who enter guilty pleas do so \textit{unequivocally}. Prior to \textit{Wong}, this requirement was usually cited to an influential dissent. The majority in \textit{Wong} affirmed this approach, while the minority dissented on a different issue. Although this requirement is ubiquitous in appellate decisions since \textit{Adgey} was decided in 1973, the term ``unequivocal'' is not uniformly understood. As will be seen later in this thesis, the way this term is understood specifically has an impact on whether Canadian law permits exculpatory no-contest pleas.

The plea that a court accepts is the plea the defendant intended to enter.


Two approaches in Canada. The first believes that ``unequivocal'' applies to the subjective intent of the defendant entering the plea. They unequivocally admit that they were responsible for the offence. The second believes that ``unequivocal'' applies to the defendant's subjective intent to effect the consequences of their plea. They unequivocally acknowledge that they intend to self-convict.

The latter approach is stronger for several reasons. 

It encompasses and accounts for cases that emerge under the first approach, as well. Take, for example, a

\subsubsection{Codification}

In 2002, these requirements were codified as \textit{Criminal Code} s 606(1.1). This section, which will be examined in greater detail in §§3.1f below, lays out the conditions it expects the court to be satisfied have been met before accepting a guilty plea from a defendant. The \textit{Criminal Code} does not 

The statutory plea inquiry formalizes the legal fact that defendants do not have a right, as such, to enter guilty pleas, as these are always in the discretion of the court to accept or reject.
