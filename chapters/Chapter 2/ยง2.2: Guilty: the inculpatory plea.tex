\section{Guilty: the inculpatory plea}

Since the first \textit{Criminal Code of Canada}, criminal defendants have been able to self-convict rather than take their charges to trial.\footnote{See \textit{The Criminal Code}, 1892 (55-56 Vict), c 29 [\textit{Criminal Code, 1892}], s 657.} This early provision did not limit a defendant's ability to enter a guilty plea and did not give judges discretion to accept or reject it. Over time, Canadian common law delineated a defendant's ability to plead guilty and stipulated that judges may only accept knowing, voluntary, and unequivocal guilty pleas. Canadian Parliament eventually codified a form of this common law requirement as \textit{Criminal Code} s 606(1.1) in 2002.\footnote{See the \textit{Criminal Law Amendment Act}, 2001, SC 2002, c 13, s 49.} In Canada, the most frequently used uncontested plea is the guilty plea.\footnote{See \textit{Ontario Court of Justice Criminal Statistics 2021}, \textit{supra} note 5.}

\subsection{The guilty plea historically}

The modern guilty plea evolved from a longstanding practice where courts allowed defendants to confess their crimes without requiring the state to prove them. Before the 11\textsuperscript{th} century Norman conquest, the Anglo-Saxons of England established a procedure for confessions to criminal charges.\footnote{See Alschuler's ``Plea Bargaining and Its History," \textit{supra} note 4 at 7 — 13.} Confessions were rudimentary and risky, lacking the modern guilty plea's procedural protections and finality.\footnote{See e.g. \textit{R v Baird}, 1908 CarswellSask 23 at para 2 [\textit{Baird}].} There is a question over whether the judiciary viewed these pleas favourably and how often they agreed to accept them.\footnote{See Alschuler, \textit{supra} note 4 at 5 — 6. Although Alschuler asserts that guilty pleas were ``discouraged" and met with ``general disapproval," none of my research into guilty pleas entered in Canada revealed similar concerns among Canadian judges. For his part, Alschuler does not cite any cases, historical studies, or sources to support this claim. This fact does not exclude the possibility that Canadian criminal law actively discouraged guilty pleas once upon a time. However, if it did, this attitude was not reflected in the written decisions discussing guilty pleas.} However, toward the end of the 19\textsuperscript{th} century, courts began regularly reporting cases involving guilty pleas.\footnote{See \textit{ibid}, \textit{supra} note 4 at 7 — 8.} By the 1920s, criminal justice surveys in America revealed that many more defendants preferred to plead guilty, and the guilty plea became fixed in American and Canadian criminal proceedings.\footnote{See \textit{ibid}, \textit{supra} note 4 at 26. The percentages of guilty pleas Alschuler reports range between 76 — 90\% in several major American cities at the time.}

\subsection{Plea voluntariness and comprehension}

As defendants entered guilty pleas more commonly, procedural protections around them solidified. One of the most significant advancements was the \textit{plea voluntariness and comprehension inquiry}. Because defendants were presumed innocent and entitled to remain silent, courts examined defendants to ensure they only waived those procedural protections knowingly and voluntarily. This safeguard, referred to here and throughout as the \textit{plea inquiry}, exists to ensure that defendants know their rights when required to plead to an offence and that they only waive those rights of their own free will. Not all jurisdictions require judges to review the same sets of factors or place the same importance on a judge's review of the same with defendants. But despite these differences, the broad factors that courts consider are generally similar and may include:

\begin{itemize}
\item \textbf{Voluntariness and knowledge.} Defendants entering pleas must do so voluntarily and know the nature and consequences of their plea. As early as the 16\textsuperscript{th} century, English courts required that defendants enter guilty pleas voluntarily,\footnote{See \textit{ibid}, \textit{supra} note 4 at 12 — 15. Alschuler attributes this practice to the co-evolution of the plea alongside the common law confessions rule. As with confessions, guilty pleas allow defendants to self-convict. However, unlike confessions, guilty pleas are presumed valid and admissible as evidence.} Whether defendants enter a plea voluntarily and knowingly impacts whether they were treated \textit{fairly} in being asked to enter it.
\item \textbf{Factual basis.} In addition to ensuring that defendants enter guilty pleas knowingly and voluntarily, local laws may also require judges to verify that there is some factual basis for the plea. This helps ensure there is enough evidence to sustain a conviction.\footnote{Contrast this requirement with the common-law rule that judges need not inquire into the factual circumstances of \textit{nolo contendere} pleas. See George Edward Pickle, ``Nolo Contendere Pleas in Criminal Tax Cases" (1975) 13:2 Am Crim L Rev 249 at 250 - 251.} A sentencing judge's obligation to confirm a factual foundation for the plea ensures that the allegations are reviewed,\footnote{Judges are not expected or allowed to operate as independent fact-finders. See \textit{R v Woroniuk}, 2019 MBCA 77, where the Manitoba Court of Appeal reversed a ``flagrant error in law" committed by sentencing judge Brian Corrin PCJ who took it upon himself to investigate the factors counsel laid out at sentencing. As a result, the scope of the judge's inquiry into the charge's factual foundation is limited, but the inquiry is nevertheless required.} which helps ensure that uncontested pleas are \textit{truthful} and accurate.
\item \textbf{Collateral consequences.} ``Collateral consequences" are consequences the court does not intentionally impose but directly result from criminal convictions. These consequences can be legal or extra-legal. Legal collateral consequences result from intersections between different legislative schemes, where a decision made under one legislative scheme has consequences under another.\footnote{These consequences may include extended driving restrictions imposed by the provincial licensor (rather than the court), restrictions on purchasing certain medications or other substances, and cancelled foreign work permits and permanent resident designations, among others. For a comprehensive overview of American collateral consequences, see e.g. the National Inventory of Collateral Consequences of Conviction, online: \textless \url{https://niccc.nationalreentryresourcecenter.org/}\textgreater.} Extra-legal collateral consequences refer to the non-legal consequences that flow from the defendant's offending behaviour.\footnote{These may include breakdowns in relationships, opportunity losses, or physical injuries, among others. See \textit{R v Suter}, 2018 SCC 34, [2018] 2 SCR 496 [\textit{Suter}] which I discuss immediately below.} Collateral consequences go to a plea's \textit{fairness}. When defendants enter a plea and know the collateral consequences of doing so, their pleas are knowledgeable and voluntary. But when defendants enter pleas without knowing the consequences, the opposite is true.
\end{itemize}

\subsection{Guilty pleas in Canada}

Before Parliament codified Canadian criminal law in 1892, decisions reporting guilty pleas were rare.\footnote{My research on the guilty plea's history in Canada only revealed two such decisions: namely, \textit{R v Morrison} [\textit{Morrison}], 1879 CarswellNB 35 and \textit{R v Morin}, 1890 CarswellQue 17 [\textit{Morin}].} However, following codification, reported cases dealing with guilty pleas abounded and Canadian guilty plea procedures adapted in turn. In the early days of Canadian criminal law, much remained unsettled about guilty pleas, their nature, and what legal consequences stemmed from entering one. As more cases made their way through the justice system, courts settled on predictable rules and uniform procedures that helped defendants understand the consequences of waiving their right to trial.\footnote{Early on, questions such as whether a guilty plea waived a right to trial were still issues when Parliament codified Canadian criminal law. See \textit{R v Davidson}, 1992 CanLII 2570; see also \textit{R v Gillis}, 1914 CarswellYukon 6, where the court treats the subject as a viable but generally settled question of law.}

\subsubsection{Voluntariness and knowledge}

Before codification, Canadian courts apparently operated on the principle that a person pleading guilty must be presumed to know the law.\footnote{See \textit{Morrison}, \textit{supra} note 38 at paras 9, 13; \textit{Morin}, \textit{supra} note 38 at para 145. The courts in both cases cite this principle. However, given that these were the only two Canadian pre-codification decisions I could find that covered guilty pleas, few conclusions should be drawn from this similarity.} However, developments in the following decades moved towards ensuring that defendants better knew the implications of self-convicting before entering uncontested pleas. As the 20\textsuperscript{th} century unfolded, Canadian courts began insisting that defendants only enter guilty pleas knowingly and voluntarily and refusing guilty pleas from defendants induced or threatened to self-convict.

Plea voluntariness ensures that defendants who enter guilty pleas do so of their own free will in an oppression-free environment. Failing to ensure these conditions obtain may warrant appellate intervention. For example, in \textit{R v Baird},\footnote{See \textit{R v Baird}, \textit{supra} note 29.} the Saskatchewan District Court heard an appeal from a landowner convicted of allowing a fire to spread from his land to a neighbour's property. Baird, the landowner, pleaded guilty but argued on appeal that he had entered his guilty plea under oppression. The court reviewed the proceedings below, found no evidence that the justice who took Baird's plea had acted oppressively or had induced him to plead guilty, and dismissed his appeal. 

Around the same time, Canadian courts began requiring defendants entering guilty pleas to know the legal consequences of doing so. In \textit{R v Tom},\footnote{\textit{R v Tom}, 1928 CarswellNS 10.} the defendant pleaded guilty to possessing narcotics and was sentenced to six months of jail and a \$500 fine. On appeal, Tom argued that he had not understood the consequences of pleading guilty. His predicament was compounded by the fact that the prosecuting police officer promised Tom he would only receive a small fine if he did so. A majority of the Nova Scotia Supreme Court allowed the appeal. The dissent agreed that there was a miscarriage of justice but held that the court had no jurisdiction to order a remedy. 

\subsubsection{Factual basis}

Canadian criminal courts have also long required a \textit{factual foundation} as a prerequisite for accepting a guilty plea. This requirement reflects the common law's perennial concern for preventing wrongful convictions by ensuring that the crimes alleged can be made out on the allegations and that there are no allegations that the defendant wishes to contest. At the turn of the 20\textsuperscript{th} century, the Ontario Supreme Court had the opportunity to review an unusual set of facts that exemplified this principle. In \textit{R v Herbert},\footnote{See \textit{R v Herbert}, 1903 CarswellOnt 829.} the court reviewed a case where the defendant, Herbert, pleaded guilty to a murder nobody suspected he had committed. After pleading guilty, Herbert implicated a co-defendant, Sifton, who pleaded not guilty and was acquitted after a trial. 

During this time, Herbert developed misgivings about his guilty plea and looked to withdraw it. In Canada at the time, the death penalty was the mandatory sentence for Herbert's crime. In its written decision, the Ontario Supreme Court did not review the allegations at Sifton's trial. Nonetheless, it concluded that Herbert's guilt was ``absolutely inconsistent" with Sifton's acquittal and that there was, therefore, ``no theory that can be suggested upon which [Sifton] could be innocent and [Herbert] guilty."\footnote{See \textit{ibid} at paras 4 - 6.} That is, the allegations did not logically support the charges. Absent any possible factual foundation for the plea, the court allowed Herbert to withdraw it.

\subsubsection{Collateral consequences}

Canadian courts recognize \textit{legal and extra-legal collateral consequences} as valid sentencing considerations. \textit{Legal collateral consequences} arise when a legal outcome obtained under one statute or legislative scheme triggers a provision in another. Increased insurance premiums after an impaired driving conviction, losing the ability to purchase firearms after a violent offence conviction, or revoking a work visa after a theft from an employer conviction are all legal collateral consequences following criminal convictions. Although Canadian courts may consider these consequences when sentencing a defendant, they are not explicitly required to do so. 

Collateral immigration consequences are an especially well-litigated example of this problem, with \textit{Wong} providing a recent and authoritative example. In \textit{Wong},\footnote{See \textit{Wong}, \textit{supra} note 3.} the defendant pleaded guilty to a cocaine trafficking charge but did not know that doing so would affect his Canadian residency. He appealed to withdraw his guilty plea. The court agreed that defendants must know the ``legally relevant collateral consequences" of pleading guilty for the plea to be valid, but split on which legal test they should apply.\footnote{The majority held that a defendant must be able to show subjective prejudice to withdraw an uninformed guilty plea of this variety. The dissent suggested accomplishing this through a modified objective test.} Although both agreed that Wong entered his guilty plea without sufficient information, the majority dismissed his appeal while the dissent would have allowed it.

Canadian courts also recognize extra-legal collateral consequences as a valid sentencing consideration, as seen in the \textit{Suter} decision.\footnote{See \textit{Suter}, \textit{supra} note 37 at paras 45 — 59.} While arguing in a parking lot with his wife after dinner and a drink, Suter drove onto a restaurant patio, accidentally killing a two-year-old child. Although there was no other evidence that Suter was impaired, the police demanded a breath sample, given the circumstances surrounding the collision. Having spoken with a lawyer who unwisely advised him not to provide a sample, Suter refused to do so when required. As a result, the police charged him with refusing to provide a breath sample in a case involving death. While his case was still pending, a group of vigilantes who were angry about the offence attacked him and cut off one of his thumbs. Suter eventually pleaded guilty and asked the judge to impose a non-custodial sentence.

At sentencing, the judge relied on the fact that Suter was the victim of a violent attack when sentencing him and held that leniency was in order. However, the judge did not feel that a non-custodial sentence was appropriate and sentenced Suter to four months. Both he and the Crown appealed. The Alberta Court of Appeal ruled that the sentencing judge erred for several reasons, including relying on vigilante violence that did not ``emanate from state misconduct" and re-sentenced Suter to 26 months of custody. Suter appealed to the Supreme Court of Canada, which found that the Alberta Court of Appeal incorrectly dismissed Suter's attack and injuries as irrelevant. The Supreme Court of Canada ruled that collateral consequences may include ``any consequence arising from the commission of an offence, the conviction for an offence, or the sentence imposed for an offence, that impacts the offender."\footnote{See \textit{Suter}, \textit{supra} note 37 at para 47.} In Suter's case, this included the violent retributive attack he suffered for his role in the offence. The majority allowed the appeal and varied the sentence to time served. The lone dissenting judge would have upheld Suter's original four-month custodial sentence.\footnote{See \textit{ibid} at paras 105ff.}

\subsubsection{Unequivocal}

Unlike other common law jurisdictions, Canadian common law also requires that defendants who enter guilty pleas do so \textit{unequivocally}. An unequivocal plea is one that the defendant intended to enter. In cases where a reviewing or first-instance court is left wondering whether a defendant intended to plead guilty, the plea is equivocal and should not be accepted or upheld. This requirement appears to trace back to a comment from the influential dissent in \textit{Adgey v R}.\footnote{See \textit{Adgey v R},  1973 CanLII 37 (SCC), [1975] 2 SCR 426 at 433f [\textit{Adgey}].} In that case, the defendant Adgey pleaded guilty to several offences with duty counsel's assistance but wanted to explain some of the charges. Some of Adgey's explanations were consistent with innocence. The judge nonetheless accepted his guilty pleas without conducting a plea inquiry. Adgey argued that the judge should not have done so, given that his explanation for the offences was consistent with innocence. 

The majority dismissed the appeal, finding that the judge was not required to conduct a plea inquiry and that Adgey had formally pleaded guilty. The dissent disagreed and argued that the pleas at first instance were equivocal as it was difficult to tell whether Adgey wanted to self-convict on all the charges he did. The formula expressed in \textit{Adgey} became entrenched in Canadian criminal law, despite being delivered by the dissent. Forty-five years later, the majority in \textit{Wong} endorsed this requirement, solidifying its place in Canadian criminal procedure.

Canadian courts have interpreted this requirement in two main ways. The first applies the unequivocal requirement to the \textit{plea's content}. Under this approach, the defendant pleads unequivocally when they admit they were responsible for the offence.\footnote{See e.g. \textit{R v Smoke}, 2017 SKQB 345 at para 19.} The second applies the unequivocal requirement to the \textit{defendant's subjective intent} to bring about the consequences of their plea. Under this approach, the defendant pleads unequivocally when they unreservedly acknowledge their intention to self-convict.\footnote{See e.g. \textit{R v Singh}, 2014 BCCA 373 at para 43, citing \textit{R v T(R)}, 1992 CanLII 2834 (ON CA) at 521. See also \textit{R v Wiebe}, 2012 BCCA 519 at para 28.} Although the former position is broadly supported, the latter is more cogent. It follows how \textit{Adgey} originally used the term and avoids any definitional overlap with the factual foundation requirement. More importantly, it reflects that a guilty plea's proof-content is primary and that this meaning must be unequivocal. As will be seen throughout this thesis, keeping a plea's proof-content separate from its truth- and belief-content is key to classifying and understanding them.