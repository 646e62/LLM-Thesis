\section{\textit{Nolo contendere}: the non-culpatory plea}

A defendant who enters a \textit{nolo contendere} plea invites the court to convict and punish them without taking responsibility for the allegations. At common-law, and in most states, defendants who enter a \textit{nolo contendere} plea may not have evidence of their \textit{nolo contendere} plea admitted at certain subsequent proceedings. 

\subsection{The common-law origin of \textit{nolo contendere} pleas}

Although \textit{nolo contendere} pleas integrated into American criminal law through common-law usage, their origins are relatively obscure. Legal and academic commentators agree that the plea emerged in England in the early 15\textsuperscript{th} century and disappeared from England in the early 18\textsuperscript{th} century, but know little else about the plea's use or origins. Historically and today, authorities frequently cite a brief excerpt from William Hawkins' 1716 edition of ``Treatise of Pleas of the Crown,"\footnote{Cite. See also (sources that use this citation).} unchanged since it first appeared in 1716, as their primary source of information about the plea's origins:

\begin{quote}
\singlespacing
An implied confession is when a defendant, in a case not capital, doth not directly own himself guilty, but in a manner admits it by yielding to the King's mercy, and desiring to submit to a small fine: in which case, if the court think fit to accept of such submission, and make an entry that defendant \textit{prosuit se in gratiam regis}, without putting him to a direct confession, or plea (which in such cases seems to be left to discretion), the defendant shall not be estopped to plead not guilty to an action for the same fact, as he shall if the entry is \textit{quod cognovit indictamentum}.\footnote{@hawkinsTreatisePleasCrown1824}
\end{quote}

By the time Hawkins published the first edition of his Treatise of Pleas of the Crown, \textit{nolo contendere} pleas had started to fall into disuse. The last reported case involving a nolo\textit{ contendere} plea in England was \textit{The Queen v Templeman} in 1702.\footnote{} Thereafter, the \textit{nolo contendere} plea lay dormant in English criminal law reports until reappearing a century later in reported American decisions.

\subsection{\textit{Nolo contendere} pleas in America}

One of the earliest reported American cases dealing with \textit{nolo contendere} pleas is the 1806 decision \textit{Commonwealth v The Town of Northampton}. The town of Northampton entered a \textit{nolo contendere} plea to the charge of failing to provide a schoolmaster, only to have the Commonwealth lay the same charge again. The court found that the town had answered the offence proscribed by statute through its \textit{nolo contendere} plea and that the Commonwealth's second allegation had no statutory grounding. By doing so, it established that a \textit{nolo contendere} plea was sufficient to trigger the same double jeopardy protections as a guilty plea.

By 1829, the Massachusetts Supreme Judicial Court had deployed a comprehensive \textit{nolo contendere} plea. This development was apparent in the case of \textit{Commonwealth v James Horton}, where the state charged the defendant with ``a breach of the law relating to retailers.'' When arraigned, Horton said he would not contend with the Commonwealth, effectively pleading \textit{nolo contendere}. As a result, the court fined him \$21 plus costs. On appeal, both parties argued that the plea was ``irregular,'' ``was no answer to the indictment,'' and amounted, at most, to an ``implied confession.''\footnote{Cite paragraphs.} The court disagreed, finding that Horton had entered a \textit{nolo contendere} plea. The court held that \textit{nolo contendere} pleas are implied confessions that the court has discretion to accept or reject, and noted that a defendant who pleads \textit{nolo contendere} may contest the same facts at subsequent proceedings. 

The court cited Hawkins' notation in the Treatise of Pleas of the Crown and the 1702 English decision The \textit{Queen v Templeman} to help define the contours of \textit{nolo contendere} pleas in their jurisdiction and dismissed the appeal. The centrality of \textit{Templeman} and Hawkins' notation in these early decisions underscores the role they have played in the preservation and revival of \textit{nolo contendere} pleas in the United States but also highlights the limited pool of common-law authority for these pleas. Nearly one hundred years after \textit{Horton}, the US Supreme Court would seize on this fact when it decided whether to read a common-law limitation into \textit{nolo contendere} pleas across the country in \textit{Hudson et al. v United States}.

\subsubsection{\textit{Hudson et al v United States}}

In \textit{Hudson}, the appellants entered \textit{nolo contendere} pleas to mail fraud charges and were sentenced to a year plus a day. On appeal, they argued that the sentencing court could not impose a custodial sentence on a \textit{nolo contendere} plea. The appellants cited a string of authorities from the Seventh Circuit Federal Court of Appeals that relied heavily upon Hawkins' reference to the ``small fine" a defendant submitted themselves to and the "case not capital" limitation. The court in \textit{Hudson} noted the dearth of authorities apart from Hawkins and therefore found little support for the inferences the appellants asked the court to draw. The court remarked that \textit{nolo contendere} pleas had a minimal reliable historical backdrop and that it was not prepared to create a broad rule that was specifically based on an obscure citation. The court ruled that custodial sentences were available for \textit{nolo contendere} pleas, echoing decisions already reached by most lower American jurisdictions at the time.

\subsection{The four \textit{nolo contendere} criteria}

When the United States Supreme Court decided \textit{Hudson}, the American \textit{nolo contendere} plea was primarily available at common law. Nearly one hundred years later, most American states have passed legislation authorizing \textit{nolo contendere} pleas.\footnote{List the states.} Those that have not do not allow them.\footnote{List these states.} Different states implemented \textit{nolo contendere} pleas differently, and adopted various views about the plea's subsequent effects. In the 1944 American Law Review annotation ``Plea of \textit{nolo contendere} or \textit{non vult contendere}," KA Dreschler provided an early, comprehensive overview of \textit{nolo contendere} pleas and the different ways that states approached and implemented them. The annotation proposed a model of the \textit{nolo contendere} plea that broke it down into four fundamental aspects:

\begin{itemize}
    \item \textbf{Applicability.} A jurisdiction may allow a defendant to plead \textit{nolo contendere} to some, all, or no offences. Where defendants may only plead \textit{nolo contendere} to some offences, the offences they may enter the plea to may vary from state to state. The applicability criterion tracks these differences.
    \item \textbf{Acceptability.} Where a defendant may enter a \textit{nolo contendere} plea, certain preconditions may need to obtain before the plea can be accepted. The acceptability criterion specifies what those conditions are and where they are required.
    \item \textbf{Procedural effects.} \textit{Nolo contendere} pleas, once entered, require the court to take certain actions in the case. The procedural effects criterion covers the immediate effects that entering a \textit{nolo contendere} plea has on a defendant's case.
    \item \textbf{Subsequent effects.} There are implications for defendants who enter \textit{nolo contendere} pleas that extend beyond their conviction. These implications may even arise in jurisdictions that disallow the plea. The subsequent effects criterion identifies these implications.
\end{itemize}

This system enables legal scholars to classify and identify individually implemented instances of the plea with one another. Information gained from this analysis may be used to understand and identify how different specific instances of \textit{nolo contendere} pleas relate to one another more generally. As I will demonstrate later in this thesis, this classification system also allows emerging phenomena, like the \textit{nolo contendere} procedure that is starting to be utilized in Canada, to be compared to and contrasted with their formal counterparts.

\subsubsection{Applicability}

The first component, applicability, addresses the question of which offences may sustain a \textit{nolo contendere}. The applicability of a nolo contendere plea is further reducible to one of four distinct types: namely, (1) \textit{all offences}; (2) only \textit{non-capital} offences; (3) \textit{some offences}; or (4) \textit{no offences}. 

%\footnote{Formally: 
%\begin{flalign*}
%text{x: a criminal offence} &&\\
%text{y: a criminal offence punishable by death} &&\\
%\text{P: a \textit{nolo contendere} plea may be entered} &&\\
%(1)        &               &  \forall x P(x)\\
%(2)        &               &  \forall x P(x \land\ \lnot y)\\
%(3)        &               &  \exists x P(x)\\
%(4)        &               &  \forall x \ \lnot P(x)
%\end{flalign*}}


Although \textit{nolo contendere} pleas were historically thought to be the sole province of minor criminal infractions, \textit{Hudson} confirmed that \textit{nolo contendere} pleas could be entered at common law for indictable felonies punishable by prison terms. This opened the door for courts to accept \textit{nolo contendere} pleas for all offences at common law. A review of the \textit{nolo contendere} plea today reveals that the legislatures followed suit, such that the plea is usually universally applicable where allowed. Of the 38 states that allow \textit{nolo contendere} pleas in criminal cases,\footnote{Alaska, Arizona, Arkansas, California, Colorado, Connecticut, Delaware, Florida, Georgia, Hawai'i, Kansas, Louisiana, Maine, Maryland, Massachusetts, Michigan, Mississippi, Montana, Nebraska, Nevada, New Hampshire, New Mexico, North Carolina, Ohio, Oklahoma, Oregon, Pennsylvania, Rhode Island, South Carolina, South Dakota, Tennessee, Texas, Utah, Vermont, Virginia, West Virginia, Wisconsin, and Wyoming.} all but four permit defendants to enter them for all criminal offences.\footnote{Georgia, Louisiana, Mississippi and South Carolina.} The wide availability of \textit{nolo contendere} pleas in the United States today has been a sea change since its limited use as a ``plea for mercy'' in Henry VI's time, or even the "submission to a small fine" alluded to in Hawkins' notation. 

Even the ``non-capital'' requirement, once thought to be an integral component of \textit{nolo contendere} pleas,\footnote{Cite to the first Drechsler annotation for an early, comprehensive expression of this view.} is mainly absent from the statutory plea. At the time of writing, the death penalty is legal and operational in 21 states,\footnote{Arizona, Arkansas, Florida, Georgia, Idaho, Indiana, Kansas, Kentucky, Louisiana, Mississippi, Missouri, Nebraska, Nevada, Oklahoma, South Carolina, South Dakota, Tennessee, Texas, Utah, and Wyoming} legal but subject to an official moratorium in three states,\footnote{California, Oregon, and Pennsylvania.} legal but subject to a \textit{de facto} moratorium in three others,\footnote{Montana, North Carolina, and Ohio.} and illegal in the remaining 23 states.\footnote{Alaska, Colorado, Connecticut, Delaware, Hawai'i, Illinois, Iowa, Maine, Maryland, Massachusetts, Michigan, Minnesota, New Hampshire, New Jersey, New Mexico, New York, North Dakota, Rhode Island, Vermont, Virginia, Washington, West Virginia, and Wisconsin} 16 states that allow \textit{nolo contendere} pleas have and implement the death penalty.\footnote{Arizona, Arkansas, Florida, Georgia, Kansas, Louisiana, Mississippi, Nebraska, Nevada, Oklahoma, South Carolina, South Dakota, Tennessee, Texas, Utah, and Wyoming.}  Of these states, all but four formally allow \textit{nolo contendere} pleas to be entered in death penalty cases.\footnote{Georgia, Louisiana, South Carolina, and Mississippi. Both Georgia and Louisiana specifically prohibit \textit{nolo contendere} in death penalty cases, while South Carolina and Mississippi only allow \textit{nolo contendere} pleas in misdemeanour cases.} All the states with death penalty moratoriums allow \textit{nolo contendere} pleas. The \textit{nolo contendere} plea is broadly applicable in the United States today. Following \textit{Hudson}, where the plea is allowed, it may generally be entered in response to any type of offence.

\subsubsection{Acceptability}

The second component, acceptability, addresses the conditions required for the court to accept the plea. A \textit{nolo contendere} plea's acceptability is a function of a judge’s discretion to accept or reject it and what requirements they are subject to in doing so, and can also be broken down into four types: namely, (1) \textit{no discretion to accept} the plea; (2) \textit{some discretion to accept or reject} the plea; (3) \textit{full discretion to accept or reject} the plea; and (4) \textit{no discretion to reject} the plea.

%\footnote{Formally, this may be expressed as: 
%\begin{flalign*}
%\text{x: a \textit{nolo contendere} plea} &&\\
%\text{y: a guilty plea}\\
%\text{V: a judge validates the plea} &&\\
%\text{O, PH, P: obligation, prohibition, and permission} &&\\
%(1)        &               &  \forall x O(x)\\
%(2)        &               &  \forall x [V(x) \implies P(x)]\\
%(3)        &               &  \forall x P(x)\ \lor [P(x) \iff P(y)] \\
%(4)        &               &  \forall x PH(x)
%\end{flalign*}}

Unless legislation provides explicitly for a \textit{nolo contendere} plea, they are typically not allowed.\footnote{As an example, cite that superior court case out of Indiana.} Judges in those states have \textit{no discretion to accept} the plea. Jurisdictions that allow defendants to enter \textit{nolo contendere} pleas subject to meeting certain conditions give the court \textit{some discretion to accept or reject} \textit{nolo contendere} pleas.\footnote{The ``public interest and effective administration of justice” test required by the federal rule, for example, has been mirrored in several states' legislation and serves as a high-level limit on a judge's discretion to accept the plea. See e.g.: \hl{States that do this}. Many states also require that the courts obtain the explicit consent of the prosecutor before accepting \textit{nolo contendere} pleas. See e.g. \hl{the states which do this.}} Other jurisdictions impose no apparent limits on \textit{nolo contendere} pleas beyond those already imposed on guilty pleas.\footnote{See some more examples.} Judges in these states have \textit{full discretion to accept or reject} a \textit{nolo contendere} plea. Finally, where judges must accept \textit{nolo contendere} pleas, they have \textit{no discretion to reject} the plea.\footnote{Only Virginia appears to give judges no discretion to reject a \textit{nolo contendere} plea, despite having the discretion to reject a guilty plea. See the statute that authorizes this. As will be seen below, however, the \textit{nolo contendere} procedure used and authorized in Canada may give prosecutors and defendants the power to compel a judge to accept a \textit{nolo contendere}-like plea.}

\subsubsection{Procedural effects}

\textit{Nolo contendere} generally have the same legal effect as a guilty plea within the proceedings, including the constitutional right against double jeopardy and a trial waiver. For the most part, where states allow defendants to plead \textit{nolo contendere}, the effect of that plea is the same as if the defendant had pleaded guilty. The defendant waives their right to a trial, is convicted, and is sentenced.\footnote{Some states make this explicit. See e.g. Oregon: O.R.S. § 135.345. No contest plea as conviction; Rhode Island: RI R REV Rule 609; New Mexico: NM ST § 30-1-11. Criminal sentence permitted only upon conviction, and Louisiana: LA C.Cr.P. Art. 552(4). Pleas at the arraignment. Some exceptions exist. Ohio, for example, makes it clear that \textit{nolo contendere} pleas are not admissions of guilt and distinguish them from guilty pleas accordingly. \hl{Cite this.}}

However, some procedural differences do exist. In Massachusetts, defendants who enter \textit{nolo contendere} pleas may not enter formal plea agreements with the prosecutors.\footnote{See MA ST RCRP Rule 12(b)(1).} In Mississippi, judges must conduct a plea voluntariness and comprehension inquiry with defendants who enter guilty pleas to any offence that carries a possible jail sentence. However, no such requirement appears to be in place for defendants who enter \textit{nolo contendere} pleas to the same offences.\footnote{See MS R RCRP Rule 15.3.} By contrast, California requires judges to conduct a particular plea inquiry with defendants who enter a \textit{nolo contendere} plea that is not required with defendants who plead guilty.\footnote{See West's Ann.Cal.Penal Code § 1016(3).} Meanwhile, in Oregon, judges are statutorily required to accept joint recommendations put forward by counsel on a guilty plea but not similarly required to do the same for \textit{nolo contendere} pleas.\footnote{}

\subsubsection{Subsequent effects}

Historically, the main difference between \textit{nolo contendere} and guilty pleas has been that \textit{nolo contendere} pleas are usually inadmissible in subsequent proceedings. Even in jurisdictions that do not permit \textit{nolo contendere} pleas, evidence of these pleas is often considered inadmissible in subsequent civil and even criminal proceedings. While twelve states do not allow criminal defendants to enter \textit{nolo contendere} pleas,\footnote{} only seven states allow evidence of \textit{nolo contendere} pleas in subsequent proceedings.\footnote{} Two of these states, Alaska and Arizona, allow defendants to enter \textit{nolo contendere} pleas but also allow admitting evidence of those pleas at subsequent proceedings.\footnote{In Alaska, the \textit{nolo contendere} plea is used almost exclusively by defendants wanting to self-convict, having effectively supplanted the guilty plea at some point. \hl{See article about this, case commentary.} It is perhaps unsurprising that it is admissible in subsequent proceedings. Arizona, meanwhile, formally authorizes \textit{nolo contendere} pleas but uses them very sparingly. \hl{See examples}.} Only Illinois, Indiana, Missouri, New Jersey and New York neither allow defendants to plead \textit{nolo contendere} nor recognize \textit{nolo contendere} pleas entered in other jurisdictions. All other states provide \textit{nolo contendere} defendants with some degree of collateral estoppel, though the degree of protection provided also varies from state to state. Most states allow defendants to contest the allegations underlying their convictions at subsequent civil or criminal proceedings,\footnote{Arkansas, California, Colorado, Connecticut, Hawai'i, Louisiana, Michigan, and Virginia make \textit{nolo contendere} pleas inadmissible at civil proceedings. California restricts this to non-felony convictions. Delaware, Florida, Idaho, Iowa, Kansas, Kentucky, Maine, Maryland, Massachusetts, Minnesota, Mississippi, Montana, Nebraska, Nevada New Hampshire, New Mexico, North Carolina, North Dakota, Ohio, Oklahoma, Oregon, Pennsylvania, Rhode Island, South Carolina, South Dakota, Tennessee, Texas, Utah, Vermont, Washington, West Virginia, Wisconsin and Wyoming all make \textit{nolo contendere} proceedings inadmissible at subsequent civil and criminal proceedings.} while Alabama and Georgia allow defendants to contest their \textit{nolo contendere} convictions at virtually any subsequent proceeding.\footnote{}

Other states place specific restrictions on how In Pennsylvania, evidence of \textit{nolo contendere} is generally inadmissible. However, the court must admit evidence of that offence when a defendant enters a \textit{nolo contendere} plea to a crime of dishonesty.\footnote{See PA ST REV Rule 609(a).} In California, defendants who enter \textit{nolo contendere} pleas to misdemeanour offences are protected from having evidence of that admitted in a later court proceeding but have no such protections for felonies. Still others, like Michigan, bar evidence of \textit{nolo contendere} pleas in subsequent civil suits unless the defendant is the one filing the suit.\footnote{See MI R REV MRE 410(2).} South Carolina allows \textit{nolo contendere} pleas to be admitted for impeachment purposes only if the conviction was for an offence punishable by death or more than one year's imprisonment.\footnote{See SC R REV Rule 609(a).} Meanwhile, the approach that some states, like Rhode Island, have stipulated that the benefits of a \textit{nolo contendere} plea may only obtain after a defendant has completed a period of probation without violations.\footnote{} Others, like South Carolina, exclude evidence of \textit{nolo contendere} pleas except when they are used to impeach a defendant on cross-examination. Although the subsequent effects of a \textit{nolo contendere} plea are generally that the plea will be subsequently inadmissible, this inadmissibility can take several forms and be implemented in various ways.

\subsection{\textit{Nolo contendere} pleas in Canada}

Because Canada codified its criminal law early into its history and functionally excluded every plea but guilty or not guilty in every iteration, \textit{nolo contendere} pleas have not attracted much attention or generated much discussion. In recent years, however, Canadian courts have recognized an informal \textit{nolo contendere} plea. Generally referred to by the courts as a ``\textit{nolo contendere} procedure,'' it allows a defendant to avoid pleading guilty but still ensure a self-conviction. Both the statutory loopholes that allow defendants to enter these pleas and the unusual effects that this plea procedure can generate are reviewed in detail in sections 3.3 and 4.4.5 below.