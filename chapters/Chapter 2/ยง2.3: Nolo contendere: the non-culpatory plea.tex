\section{\textit{Nolo contendere}: the Non-culpatory Plea}

\textit{Nolo contendere} pleas began in medieval England as a way for defendants to accept the court's judgment and plead for its mercy without having to confess to their crimes. Like guilty pleas, they have evolved significantly since their earliest days and are nowadays primarily regulated by statute.\footnote{See notes 83 \& 84 below.} Although many aspects of the plea vary widely between the various jurisdictions that allow them, a defendant who enters a \textit{nolo contendere} plea still invites the court to convict and punish them without taking responsibility for the allegations. At common law\footnote{See Jerome Doherty, ``Plea Nolo Contendere" (1943) 31:3 Geo L J 324 at 325.} and in most states,\footnote{See note 113 below.} defendants who enter a \textit{nolo contendere} plea may not have evidence of their plea admitted at certain subsequent proceedings. 

\subsection{The Common Law Origin of \textit{Nolo Contendere} Pleas}

Although \textit{nolo contendere} pleas were integrated into American criminal law through common-law usage, their roots are relatively obscure. Legal and academic commentators agree that the plea emerged in England in the early 15\textsuperscript{th} century and disappeared from England in the early 18\textsuperscript{th} century, but know little else about its use or origins. Historically and today, authorities frequently cite\footnote{See e.g. Henry Clay Moore, ``Conviction upon Plea of Nolo Contendere as Impeaching Evidence" (1967) 21:1 Ark L Rev 124 at 124; Timothy K. Garfield, ``Use of the Nolo Contendere Plea in Subsequent Contexts" (1971) 44:3 S Cal L Rev 737 at 741; Simpson, \textit{supra} note 16 at 27.} a brief excerpt from William Hawkins' ``A Treatise of Pleas of the Crown," unchanged since it first appeared in 1716, as their primary source of information about the plea's origins:

\begin{quote}
\singlespacing
An implied confession is when a defendant, in a case not capital, doth not directly own himself guilty, but in a manner admits it by yielding to the King's mercy, and desiring to submit to a small fine: in which case, if the court think fit to accept of such submission, and make an entry that defendant \textit{prosuit se in gratiam regis}, without putting him to a direct confession, or plea (which in such cases seems to be left to discretion), the defendant shall not be estopped to plead not guilty to an action for the same fact, as he shall if the entry is \textit{quod cognovit indictamentum}.\footnote{See Hawkins, \textit{supra} note 27 at 466.}
\end{quote}
By the time Hawkins published the first edition of his Treatise of Pleas of the Crown, \textit{nolo contendere} pleas had already fallen into disuse.\footnote{See \textit{Templeman}, \textit{supra} note 18.} Thereafter, the \textit{nolo contendere} plea lay dormant\footnote{See \textit{Hudson et al v The United States}, 272 US 451 at 453, 47 Sup Ct 127 (USSC 1926) [\textit{Hudson}].} until reappearing a century later in reported American decisions.

\subsection{\textit{Nolo Contendere} Pleas in America}

One of the earliest reported American cases discussing \textit{nolo contendere} pleas is the 1806 decision \textit{Commonwealth v The Town of Northampton}.\footnote{See \textit{Commonwealth v The Town of Northampton}, 2 Mass 116, 1806 WL 756 (Mass Sup Jud Ct 1806).} The town of Northampton entered a \textit{nolo contendere} plea to the charge of failing to provide a schoolmaster but sought to withdraw the plea at the next court sitting because the original indictment was defective. The town noted that the original indictment charged them with an offence against ``the peace and dignity of the commonwealth" but not with any offence prescribed by statute. The court agreed and set the judgment aside.

Later decisions demonstrated that while \textit{nolo contendere} pleas remained obscure, their legal foundations were nonetheless well-developed. These two phenomena were apparent in the case of \textit{Commonwealth v James Horton},\footnote{See \textit{Commonwealth v James Horton}, 26 Mass 206, 1829 WL 1997 (Mass Sup Jud Ct 1829).} where the state charged the defendant with ``a breach of the law relating to retailers.'' When arraigned, Horton said he would not contend with the Commonwealth. As a result, the court fined him \$21 plus costs. On appeal, both parties argued that the plea was ``irregular,'' ``was no answer to the indictment,'' and amounted, at most, to an ``implied confession.''\footnote{See \textit{ibid} at 210.} 

The court disagreed, finding that Horton had entered a \textit{nolo contendere} plea. The court examined \textit{nolo contendere} pleas in detail and agreed that they were implied confessions but found that the pleas were valid. The court noted that defendants who plead \textit{nolo contendere} may contest the same facts at subsequent proceedings.\footnote{See \textit{ibid}.} The court cited Hawkins' notation in the Treatise of Pleas of the Crown and \textit{Templeman} to help define the contours of \textit{nolo contendere} pleas in their jurisdiction and dismissed the appeal. In this early decision, the centrality of \textit{Templeman} and Hawkins' notation underscores the role they have played in the preservation and revival of \textit{nolo contendere} pleas in the United States but also highlights the limited pool of common-law authority for these pleas. Nearly one hundred years after \textit{Horton}, the US Supreme Court seized on this fact when it decided whether to read a lingering common-law limitation into \textit{nolo contendere} pleas across the country in \textit{Hudson}.

\subsubsection{\textit{Hudson et al v United States}}

In \textit{Hudson},\footnote{See \textit{Hudson}, \textit{supra} note 73.} the appellants entered \textit{nolo contendere} pleas to mail fraud charges and were sentenced to a year plus a day. On appeal, they argued that the sentencing court could not impose a custodial sentence on a \textit{nolo contendere} plea. The appellants cited a string of authorities from the Seventh Circuit Federal Court of Appeals that relied heavily upon Hawkins' reference to the ``small fine" a defendant submitted themselves to and the ``case not capital" limitation.\footnote{See \textit{ibid} at 453.} The Supreme Court in \textit{Hudson} noted the dearth of authorities apart from Hawkins and therefore found little support for the inferences the appellants asked the court to draw.\footnote{See \textit{ibid} at 454, n 2, 457.} The court remarked that \textit{nolo contendere} pleas had a minimal reliable historical backdrop\footnote{See \textit{ibid} at 454.} and that it was not prepared to create a broad rule that was specifically based on an obscure citation. The court ruled that custodial sentences were available for \textit{nolo contendere} pleas, affirming decisions already reached by most lower American jurisdictions at the time.\footnote{See \textit{ibid} at 457.}

\subsection{The Four \textit{Nolo Contendere} Criteria}

When the United States Supreme Court decided \textit{Hudson}, the American \textit{nolo contendere} plea was primarily available at common law. Nearly one hundred years later, the \textit{nolo contendere} plea is available federally, and most American states have passed legislation authorizing it.\footnote{Namely, Alaska: Ala R Crim P rule 11; Arizona: Ariz R Crim P rule 17.1(c); Arkansas: Ark R Crim P rule 24.3; California: Cal Ann Penal Code § 1016 (West 2022); Delaware: Del Ct Com Pl rule 11(b), Del Super Ct rule 11(b); Florida: Fla R Crim P rule 3.170; Georgia: Ga Ann Code § 17-7-95 (West 2022); Hawai'i: Hawaii R Penal P rule 11(b); Kansas: Kan Stat Ann ch 22 art 3208 (West 2022); Louisiana: La C Crim P art 552(4); Maine: Me R Crim P rule 11(a); Maryland: Md R rule 4-242; Massachusetts: Mass R Crim P rule 12(a); Michigan: Mich Comp Law Ann § 767.37; Mississippi: Miss R Crim P rule 15.3; Montana: Mont Code Ann 46-16-105 (West 2022); Nebraska: Neb Rev Stat § 29-1819 (West 2022); Nevada: Nev Rev Stat Ann 174.035; New Hampshire: NH Rev Stat § 605:6; New Mexico: N Mex Stat Ann § 30-1-11; North Carolina: NC Gen Stat Ann § 15A-1022; Ohio: Ohio R Crim P 11(A); Oklahoma: Okla Stat Ann § 513; Oregon: Or Rev Stat Ann § 135.355; Pennsylvania: Pa R Crim P rule 590; Rhode Island: RI Dist Ct R Crim P rule 11, RI Super Ct R Crim P rule 11; South Carolina: SC Code (1976) § 17-23-40; South Dakota: S Dak C Law § 23A-7-2; Tennessee: Tenn R Crim P rule 11(a); Texas: Tex Ann C Crim P art 27.02; Utah: R Crim P rule 11(b); Vermont: Vt R Crim P rule 11(b); Virginia: Va C Ann § 19.2-254; West Virginia: W Va R Crim P rule 11(a); Wisconsin: Wis Stat Ann 971.06; and Wyoming: Wyo R Crim P rule 11(a).} Those that have not do not allow them.\footnote{Namely, Alabama: Ala R Crim P rule 14.2(c); Idaho: Idaho Crim R rule 11(a); Illinois: Ill C Crim P art 113; Indiana: Ind Crim P 35-35-1-1; Iowa: Iowa R Crim P rule 2.8(2); Kentucky: Ky R Crim P rule 8.08; Minnesota: Minn R Crim P rule 14.01; Missouri: Mo R Crim P rule 24.02(a); New Jersey: NJ Stat Ann rule 7:6-2; New York: NY Crim P ch 11-a part 2 tit J § 220.10; North Dakota: N Dak R Crim P rule 11(a); and Washington: Wash Super Ct Crim R rule 4.2(a).} As the states codified their plea procedures, different jurisdictions implemented \textit{nolo contendere} pleas differently and adopted various views about the plea's subsequent effects.

In the 1944 American Law Review annotation ``Plea of \textit{nolo contendere} or \textit{non vult contendere}," KA Drechsler provided an early, comprehensive overview of \textit{nolo contendere} pleas and the different ways that states were approaching and implementing them.\footnote{See Drechsler's Annotation, \textit{supra} note 26.} When KA Drechsler first proposed these four ``aspects" in 1944, they were designed to explain the fundamental nature of a \textit{nolo contendere} plea. At the time, \textit{nolo contendere} pleas were still mostly common-law creations. Drechsler recognized that courts had heavily relied on Hawkins's sparse notation, and that this dearth of precedential information had led to myriad \textit{nolo contendere} variations across different jurisdictions. By examining the \textit{nolo contendere} plea's ``exact nature," Drechsler hoped he could ``clarify certain misunderstandings concerning its use and its consequences which are responsible for many of the conflicting decisions."\footnote{See \textit{ibid} at § I(c)} Nearly twenty years later, CT Drechsler\footnote{Despite repeated attempts to investigate, I have not been able to confirm any relation between the two.} published an extensive updated annotation but left the underlying ``four-aspect" analysis largely undisturbed, noting that the authorities thus far had added very little to that understanding.\footnote{See CT Drechsler, ``Plea of nolo contendere or non vult," Annotation, (1963) 89 ALR 2d 540 at § 2.}

KA Drechsler's essentialist view of \textit{nolo contendere} pleas is problematic. There is good reason to suspect that the plea does not have an ``exact nature," per se, and that what Drechsler normatively refers to as the plea's ``aspects" are better understood as descriptive categories that distinguish one \textit{nolo contendere} variant from another. Some categories, like the ``procedural effects" category, seem to have very limited utility, while other categories, like the ``subsequent effects" category, should be developed more than they have been. But as with other arguably deficient and outdated legal artefacts that become entrenched through habitual quoting, such as the ``WD analysis"\footnote{See \textit{R v W(D)}, [1991] 1 SCR 742 at 749 — 750, 63 CCC (3d) 397 [\textit{W(D)}].} and the ``Lifchus test,"\footnote{See \textit{R v Lifchus}, [1997] 3 SCR 320 at para 36 [\textit{Lifchus}].} KA Drechsler's foundational \textit{nolo contendere} analysis remains popular some 80 years after he first introduced it.\footnote{See e.g. J. R. O'Brien, ``Trade Practices and Consumers: The Plea Non Vult Contendere Cum Domina Regina et Posuit Se in Gratiam Curiae" (1972) 5:1 Fed L Rev 125; \textit{State v Olsen}, 848 NW (2d) 363 (Iowa Sup Ct 2014); \textit{In re People v Darlington}, 105 P 3d 230 at 233 (Colo Sup Ct 2005); \textit{Scott v State}, 928 P 2d 1234 at 1237 (Alaska Ct App 1996).} My goal is not to develop a new or improved classification system for \textit{nolo contendere} pleas but rather to situate my analysis within the existing discourse about them. As these categories remain used and useful, I include and discuss them here, notwithstanding their limitations. 

Drechsler's proposed model defined \textit{nolo contendere} pleas through four aspects, which I read as \textit{criteria}:

\begin{itemize}
    \item \textbf{Applicability.} A jurisdiction may allow a defendant to plead \textit{nolo contendere} to some, all, or no offences. Where defendants may only plead \textit{nolo contendere} to some offences, the offences they may enter the plea to may vary from state to state. The applicability criterion tracks these differences.
    \item \textbf{Acceptability.} Where a defendant may enter a \textit{nolo contendere} plea, certain conditions may need to obtain for the court to accept the plea. The acceptability criterion identifies those conditions and where they are required.
    \item \textbf{Procedural effects.}\footnote{Drechsler refers to these as ``effects in the case."} \textit{Nolo contendere} pleas require the court to follow certain procedures. The procedural effects criterion covers the immediate implications that entering a \textit{nolo contendere} plea has on a defendant's case.
    \item \textbf{Subsequent effects.}\footnote{Drechsler refers to these as ``consequences outside the case."} The implications for defendants who enter \textit{nolo contendere} pleas may extend beyond their conviction. These implications may even arise in jurisdictions that disallow the plea. The subsequent effects criterion considers these implications.
\end{itemize}
Analyzing \textit{nolo contendere} pleas using these categories clarifies how different \textit{nolo contendere} plea variations relate to one another more generally. Later, I will use these criteria to categorize the informal \textit{nolo contendere} procedure used in Canada and compare it with its formal American counterparts.

\subsubsection{Applicability}

The first component, applicability, addresses the question of which offences may sustain a \textit{nolo contendere} plea. A \textit{nolo contendere} plea's applicability is further reducible to one of four distinct types: namely, (1) \textit{all offences}; (2) only \textit{non-capital} offences; (3) \textit{some offences}; or (4) \textit{no offences}. 

Although scholars and jurists historically confined \textit{nolo contendere} pleas to minor criminal infractions, \textit{Hudson} confirmed defendants could enter them at common law for indictable felonies punishable by prison terms. This ruling opened the door for courts to accept \textit{nolo contendere} pleas for all offences at common law. A review of the plea today reveals that the legislatures followed suit, such that \textit{nolo contendere} is usually universally applicable where allowed. Of the 38 states that allow \textit{nolo contendere} pleas in criminal cases, all but four permit defendants to enter them for all criminal offences.\footnote{Namely, Georgia: Ga Ann Code § 17-7-95(a); Louisiana: La C Crim P art 552, Mississippi: Miss R Crim P rule 15.3(b); and South Carolina: SC Code (1976) § 17-23-40.} In some states, such as Alaska, \textit{nolo contendere} pleas have become so prominent that defendants rarely, if ever, enter guilty pleas to self-convict.\footnote{See Jana L Kuss, ``Endangered Species: A Plea for the Preservation of Nolo Contendere in
Alaska" (2005) 41:3 Gonz L Rev 539. In Alaska, defendants do not require consent from the prosecutor or the court to enter \textit{nolo contendere} pleas. As a result, the \textit{nolo contendere} plea has effectively supplanted the guilty plea and is now used almost exclusively by defendants wanting to self-convict.}

Even the ``non-capital'' requirement, once thought to be an integral component of \textit{nolo contendere} pleas,\footnote{See Drechsler's Annotation, \textit{supra} note 26 at § II(b)(3).} is mainly absent from the statutory plea. At the time of writing, capital punishment is legal and enforced in 21 states,\footnote{Namely, Arizona, Arkansas, Florida, Georgia, Idaho, Indiana, Kansas, Kentucky, Louisiana, Mississippi, Missouri, Nebraska, Nevada, Oklahoma, South Carolina, South Dakota, Tennessee, Texas, Utah, and Wyoming. \textit{Nolo contendere} is also available in response to federal charges.} legal but subject to an official moratorium in three states,\footnote{Namely, California, Oregon, and Pennsylvania.} legal but subject to a \textit{de facto} moratorium in three others,\footnote{Namely, Montana, North Carolina, and Ohio.} and illegal in the remaining 23 states.\footnote{Namely, Alaska, Colorado, Connecticut, Delaware, Hawai'i, Illinois, Iowa, Maine, Maryland, Massachusetts, Michigan, Minnesota, New Hampshire, New Jersey, New Mexico, New York, North Dakota, Rhode Island, Vermont, Virginia, Washington, West Virginia, and Wisconsin} Sixteen states that allow \textit{nolo contendere} pleas have and implement the death penalty.\footnote{Namely, Arizona, Arkansas, Florida, Georgia, Kansas, Louisiana, Mississippi, Nebraska, Nevada, Oklahoma, South Carolina, South Dakota, Tennessee, Texas, Utah, and Wyoming.}  Of these states, all but four allow defendants to enter \textit{nolo contendere} pleas in capital cases.\footnote{Namely, Georgia, Louisiana, South Carolina, and Mississippi. Both Georgia and Louisiana expressly prohibit \textit{nolo contendere} in death penalty cases, while South Carolina and Mississippi only allow \textit{nolo contendere} pleas in misdemeanour cases, where the death penalty is presumably unavailable.} All states with death penalty moratoriums, official or otherwise, also allow \textit{nolo contendere} pleas for all offences. The \textit{nolo contendere} plea is broadly applicable in the United States today. Since \textit{Hudson} and widespread codification, where the plea is allowed, defendants may generally enter it for any offence.\footnote{See \textit{May v Lingo}, 167 So (2d) (Ala Sup Ct 1964) at 270.}

\subsubsection{Acceptability}

The second criterion, acceptability, addresses whether the court may accept the plea. A \textit{nolo contendere} plea's acceptability correlates to judicial discretion to accept or reject it, and can also be broken down into four types: namely, (1) \textit{no discretion to accept} the plea; (2) \textit{some discretion to accept or reject} the plea; (3) \textit{full discretion to accept or reject} the plea; and (4) \textit{no discretion to reject} the plea.

Unless legislation provides explicitly for a \textit{nolo contendere} plea, they are typically not allowed.\footnote{See e.g. \textit{Corbin v State}, 713 NE (2d) 906 (Ind Ct App 1999).} Judges in those states have \textit{no discretion to accept} the plea. Jurisdictions that allow defendants to enter \textit{nolo contendere} pleas subject to meeting certain conditions give the court \textit{some discretion to accept or reject} \textit{nolo contendere} pleas.\footnote{The ``public interest and effective administration of justice” test required by the federal rule, for example, has been mirrored in several states' legislation and serves as a high-level limit on a judge's discretion to accept the plea. Arizona, Arkansas, Delaware, Hawai'i, South Dakota, Utah, West Virginia, Wyoming, and the federal rule require courts to apply this test. Several states also require that the courts obtain the explicit consent of the prosecutor before accepting \textit{nolo contendere} pleas. These include Arkansas, Hawai'i, Maine, Montana, North Carolina, and the federal rule.} Others impose no apparent limits on \textit{nolo contendere} pleas beyond those already imposed on guilty pleas. Judges in these states may be said to have \textit{full discretion to accept or reject} a \textit{nolo contendere} plea. Finally, where judges must accept \textit{nolo contendere} pleas, they have \textit{no discretion to reject} the plea.

\subsubsection{Procedural Effects}

\textit{Nolo contendere} pleas generally have the same legal effect as a guilty plea within the proceedings, including the constitutional right against double jeopardy and a trial waiver. For the most part, where states allow defendants to plead \textit{nolo contendere}, the effect of that plea is the same as if the defendant had pleaded guilty. The defendant waives their right to a trial, is convicted, and is sentenced.\footnote{Some states make this explicit. See e.g. Oregon: Or Rev Stat Ann tit 14 § 135.345; Rhode Island: RI State Ct rule 609; New Mexico: N Mex Stat Ann § 30-1-11, and La: LA C Cr P tit 16 art 552(4). Some exceptions exist. Ohio, for example, explicitly states \textit{nolo contendere} pleas are not admissions of guilt and distinguish them from guilty pleas accordingly. See Ohio Rev Ann Crim R rule 11(B)(2).}

However, some procedural differences do exist. In Massachusetts, defendants pleading \textit{nolo contendere} may not enter formal plea agreements with the prosecutors.\footnote{See Mass R Crim P rule 12(b)(1).} In Mississippi, judges must conduct a plea voluntariness and comprehension inquiry with defendants who enter guilty pleas to any offence that carries a possible jail sentence. However, no such requirement appears to be in place for defendants who enter \textit{nolo contendere} pleas to the same offences.\footnote{See Miss R Crim P rule 15.3.} By contrast, California requires judges to conduct a particular plea inquiry with defendants who enter a \textit{nolo contendere} plea that is not required with defendants who plead guilty.\footnote{See Cal Ann Penal Code § 1016(3) (West 2022).} Meanwhile, in Oregon, judges are statutorily required to accept joint recommendations put forward by counsel on a guilty plea but not similarly required to do the same for \textit{nolo contendere} pleas.\footnote{See Or Rev Stat Ann tit 14 § 135.385(2)(e). Although Or Rev Stat Ann tit 14 § 135.432 states that plea deals do not bind judges, § 135.385(2)(e) states that judges are to tell defendants that sentencing recommendations reached through formal disposition recommendations \textit{will be accepted} by the courts.} 

\subsubsection{Subsequent Effects}

Historically, the main difference between \textit{nolo contendere} and guilty pleas has been that the former are usually inadmissible in subsequent proceedings. Even in jurisdictions that do not permit them, evidence that a defendant pleaded \textit{nolo contendere} in another jurisdiction is often inadmissible in subsequent civil and even criminal proceedings. While twelve states do not allow criminal defendants to enter \textit{nolo contendere} pleas, only seven states allow evidence of \textit{nolo contendere} pleas in subsequent proceedings.\footnote{Namely, Alaska, Arizona, Illinois, Indiana, Missouri, New Jersey, and New York.} Two of these states, Alaska and Arizona, allow defendants to enter \textit{nolo contendere} pleas but also allow admitting evidence of those pleas at subsequent proceedings.\footnote{Because the \textit{nolo contendere} has effectively supplanted guilty pleas in Alaska, it is perhaps unsurprising that it is admissible in subsequent proceedings. Meanwhile, Arizona allows \textit{nolo contendere} pleas only sparingly. See \textit{Duran v Maricopa}, 782 P (2d) 324 (Ariz Ct App 1994).} Only Illinois, Indiana, Missouri, New Jersey and New York neither allow defendants to plead \textit{nolo contendere} nor recognize \textit{nolo contendere} pleas entered in other jurisdictions. All other states provide \textit{nolo contendere} defendants with some degree of collateral estoppel.\footnote{Namely, Alabama: Ala R Evid 410; Arkansas: Ark R Evid rule 410; California: Cal Ann Penal Code § 1016(3); Colorado: Colo R Evid rule 410; Connecticut: Conn R Super Ct s 39-25; Delaware: Del Ct Com Pl rule 11(e)(4); Florida: Fla Stat Ann 772.14; Georgia: Ga C Ann § 17-7-95(c); Hawai'i: Hawaii R Evid rule 410; Idaho: Idaho R Evid rule 410, Iowa: Iowa R Evid rule 5.410; Kansas: Kan Stat Ann ch 22 art 3209 (West 2022); Kentucky: Ky R Evid rule 410; Louisiana: La C Crim P art 552(4); Maine: Me R Evid rule 410; Maryland: Md R rule 5-410; Massachusetts: Mass R Evid § 410; Michigan: Mich R Evid rule 410; Minnesota: Minn R Evid rule 410; Mississippi: Miss R Evid rule 410; Montana: Mont Code Ann tit 26 ch 10 art 4 rule 410 (West 2022); Nebraska: Neb Rev Stat § 27-410 (West 2022); Nevada: Nev Rev Stat Ann 48.125; New Hampshire: NH R Evid rule 410; New Mexico: N Mex Stat Ann rule 11-410; North Carolina: NC Gen Stat Ann § 8C-1 rule 410; North Dakota: N Dak R Evid rule 410; Ohio: Ohio R Evid rule 410; Oklahoma: Okla Stat Ann § 2410; Oregon: Or Evid C § 40.200 rule 410; Pennsylvania: Pa R Evid rule 410; Rhode Island: RI R Evid rule 410; South Carolina: SC R Evid rule 410; South Dakota: S Dak C Law § 19-19-410; Tennessee: Tenn R Evid rule 410; Texas: Tex R Evid rule 410; Utah: Utah R Evid rule 410; Vermont: Vt R Crim P rule 11(b), Virginia: Va C Ann § 19.2-254; Washington: Wash R Evid rule 410; West Virginia: W Va R Evid rule 410; Wisconsin: Wis Stat Ann 904.10; and Wyoming: Wyo R Evid rule 410.}

Although a \textit{nolo contendere} plea is usually inadmissible in subsequent proceedings, this inadmissibility can take several forms. In Pennsylvania, evidence of \textit{nolo contendere} is generally inadmissible. However, the court must admit evidence of that offence when a defendant enters a \textit{nolo contendere} plea to a crime of dishonesty.\footnote{See Pa R Evid rule 609(a).} In California, defendants who enter \textit{nolo contendere} pleas to misdemeanour offences are protected from having evidence of that admitted in a later court proceeding but have no such protections for felonies.\footnote{See Cal Ann Penal Code § 1016 (West 2022).} Other states, such as Alabama, Kansas and Georgia, allow defendants to contest their \textit{nolo contendere} convictions at virtually any subsequent proceeding.\footnote{See \textit{McNair v State}, 653 So 2d 320 (Ala Crim App 1992); Kan Stat Ann ch 22 art 3209; Ga C Ann § 17-7-95(c).}

\subsection{\textit{Nolo Contendere} Pleas in Canada}

Because Canada codified its criminal law early into its history and only allowed defendants to enter guilty or not guilty pleas, \textit{nolo contendere} pleas have not attracted much attention or generated much discussion in Canada. In recent years, however, Canadian courts have recognized an informal \textit{nolo contendere} plea procedure that allows defendants to avoid pleading guilty while ensuring self-conviction. I analyze these informal pleas and examine their eccentricities in §§ 3.2.3 and 4.3.5 below.