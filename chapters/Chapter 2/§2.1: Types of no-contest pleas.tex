\section{Types of no-contest pleas}

\subsubsection{No-contest pleas}

As alluded to in Chapter 1, this thesis identifies and examines three types of no-contest pleas: guilty pleas, best-interest pleas, and \textit{nolo contendere} pleas. Because each of these pleas admits the state's case against a defendant, each qualifies as a no-contest plea. What distinguishes these pleas from each other are the admissions they ostensibly ask defendants to make: 

\begin{itemize}

    \item A defendant who enters an honest and unequivocal guilty plea admits that they committed the crime alleged and accepts the consequences of being convicted of the same. This thesis classifies such pleas as \textbf{inculpatory no-contest pleas}, as the defendant who enters one admits their guilt.
    
    \item A defendant who enters a best-interest plea accepts both the consequences of a conviction for the alleged crime and that the offence would have been proven beyond a reasonable doubt if taken to trial but formally maintains their innocence. This thesis classifies such pleas as \textbf{exculpatory no-contest pleaa}, as defendants who enter them actively protest their innocence, notwithstanding clear evidence.
    
    \item A defendant who enters a \textit{nolo contendere} plea accepts the consequences of the crime alleged but neither admits nor denies their actual involvement in the offence. This no-contest plea can be described as a \textbf{non-culpatory no-contest plea}, as the defendant who enters it neither acknowledges nor repudiates responsibility.
    
\end{itemize}

\subsection{Guilty pleas}

Since the Canadian Parliament first codified the criminal law in 1892, criminal defendants have been able to opt for self-conviction rather than take their charges to trial, despite a not guilty plea being the default.\footnote{See \textit{Criminal Code 1892} s 657.} This early provision contained no apparent limitations on a defendant's ability to enter a guilty plea and no apparent discretion for a judge to accept or reject it. Over time, Canadian common law began to limit a defendant's ability to enter a guilty plea, stipulating that judges may only accept guilty pleas in cases where defendants enter them knowingly, voluntarily, and unequivocally. This common law requirement was eventually codified as \textit{Criminal Code} s 606(1.1) in 2002 and has remained an integral part of the guilty plea process.

In many common law jurisdictions, including Canada, the most frequently used no-contest plea is a simple guilty plea.\footnote{In Canada and the United States, courts deal with criminal allegations through guilty pleas far more frequently than through trials. See e.g. https://www.ontariocourts.ca/ocj/files/stats/crim/2021/2021-Offence-Based-Criminal.xlsx} By entering a guilty plea, a defendant accepts responsibility for the offence charged, agrees that the allegations are true, and acknowledges, usually implicitly, that the state complied with all of their procedural rights. Courts typically accept guilty pleas at face value as an admissible expression of remorse, a mitigating factor that the judge must consider during sentencing.

While the general contours of a guilty plea are similar across jurisdictions, the specifics will often vary. For example, courts recognize that a defendant who pleads guilty must do so knowingly and voluntarily. However, the specific conditions that must be satisfied to meet these standards vary.\footnote{See §2.3.2 below for a more detailed discussion of these pleas and their characteristics.} In Canada, the \textit{Criminal Code} requires that defendants enter their guilty pleas knowingly and voluntarily, while Canadian common law requires that they enter their guilty pleas unequivocally.

\subsection{Best-interest pleas}

On the other hand, a best-interest plea refers to any plea that a defendant enters despite taking issue with either the factual foundation of the prosecution's allegations, the state's compliance with their procedural rights, or both. The ``best-interest" label refers to the belief that defendants enter these pleas to further their own best interests, not as an honest admission of their guilt and the adequacy of the state's investigative conduct. The most arguably well-known best-interest plea is the \textit{Alford} plea, named for the 1970 US Supreme Court decision that promulgated best-interest pleas across the United States.

By entering a best-interest plea, a defendant signals to the court that they are willing to accept the consequences of a criminal conviction, notwithstanding their desire to protest their innocence otherwise. Although best-interest pleas are nearly universally treated as being equivalent to guilty pleas, protestations of innocence made during sentencing may diminish the mitigating effect of the plea\footnote{\hl{Should be some example of this somewhere.}} and limit potential rehabilitative options if continued thereafter.\footnote{\hl{Same. Know I've read this before.}}

Although best-interest pleas and \textit{Alford} pleas are not identical to one another, \textit{Alford} pleas are the dominant expression of the genre. While some states have codified best-interest pleas,\footnote{See \hl{Examples of where this happened}.} the widespread availability and notoriety of the \textit{Alford} plea makes it the best candidate for consideration. Where accepted, the \textit{Alford} plea is generally understood to be functionally equivalent to a guilty plea. As such, many academic and judicial authorities do not consider it to be a distinct plea. Instead, these sources argue that \textit{Alford} pleas should be understood as a guilty plea accompanied by a declaration of innocence, as was the case in the \textit{Alford} case itself.\footnote{Cite cases and authorities.}

An \textit{Alford}-style best-interest plea in Canada would therefore be a guilty plea entered per \textit{Criminal Code} s 606(1.1) and accompanied by a declaration of innocence. Whether a plea of this sort would lead to contradictions or perverse results is explored in greater detail in §3.4 below.

\subsection{\textit{Nolo contendere} pleas}

Finally, \textit{nolo contendere} pleas are a form of plea that acknowledges the sufficiency of the prosecution's case without admitting or denying the defendant's factual guilt or the state's procedural compliance. In the United States, \textit{nolo contendere} pleas are \textit{formally} recognized and accepted in most states. In Canada, although \textit{nolo contendere} pleas are not formally recognized, an \textit{informal} \textit{nolo contendere} ``plea procedure" has recently been authorized by appellate courts in several jurisdictions. 

By entering a \textit{nolo contendere} plea, a defendant signals to the court that they are not contesting the allegations against them. In this sense, a \textit{nolo contendere} plea is comparable to a guilty plea, as the defendant invites the court to convict them. By contrast, a defendant entering a formal \textit{nolo contendere} plea does not overtly admit the truth of those charges. In this sense, a \textit{nolo contendere} plea is comparable to a best-interest plea, as the defendant will not or cannot admit their factual guilt.

While formal and informal \textit{nolo contendere} share many key traits, some differences distinguish them from or liken them to other pleas. Generally, these differences arise in the grey areas between the interacting legislative provisions that make informal pleas possible. For example, both formal \textit{nolo contendere} pleas and guilty pleas are typically subject to the same statutory requirement for a plea voluntariness and comprehension inquiry. Informal \textit{nolo contendere} pleas do not have this requirement, though courts may still conduct one as a matter of fairness.\footnote{See \textit{R v DMG}, 2011 ONCA 343.} I will explain the underlying mechanisms and their pertinent interactions in more detail in §3.3.2 below.

\subsection{The relationships between the pleas}

Based on the above, we can draw the following conclusions regarding the relationships between these different no-contest pleas:
\begin{itemize}
\item Guilty pleas and best-interest pleas are related to the extent that the latter is an extension of the former;
\item Formal and informal \textit{nolo contendere} pleas and best-interest pleas are related to the extent that both invite a conviction without admitting guilt; 
\item Guilty pleas and formal \textit{nolo contendere} pleas are related to the extent that both are generally subject to the same knowledge and voluntariness requirements;

    \begin{itemize}
    \item To the extent that best-interest pleas inherit all the traits of a guilty plea, they also inherit this trait; and
    \end{itemize}

\item Guilty pleas, formal and informal \textit{nolo contendere} pleas and best-interest pleas are all linked to the extent that each invites a conviction without requiring the state to discharge its burden of proof.
\end{itemize}