\section{\textit{Nolo contendere}: the non-culpatory plea}

\textit{Nolo contendere} pleas began in medieval England as a way for defendants to plead for the court's mercy without having to confess to their crimes. Like guilty pleas, they have evolved significantly since their earliest days and are largely regulated by statute. Although many aspects of the plea vary widely between the various jurisdictions that allow them, in general a defendant who enters a \textit{nolo contendere} plea still invites the court to convict and punish them without taking responsibility for the allegations. At common-law and in most states, the advantage for defendants who enter a \textit{nolo contendere} plea is that they may not have evidence of their \textit{nolo contendere} plea admitted against them at certain subsequent proceedings. 

\subsection{The common-law origin of \textit{nolo contendere} pleas}

Although \textit{nolo contendere} pleas integrated into American criminal law through common-law usage, their roots are relatively obscure. Legal and academic commentators agree that the plea emerged in England in the early 15\textsuperscript{th} century and disappeared from England in the early 18\textsuperscript{th} century, but know little else about its use or origins. Historically and today, authorities frequently cite a brief excerpt from William Hawkins' ``A Treatise of Pleas of the Crown,"\footnote{See Willaim Hawkins, \textit{A Treatise of the Pleas of the Crown: Or, A System of the Principle Matters Relating to that Subject, Digested Under Proper Heads}, vol 2, 8th ed by John Curwood (London, UK: Sweet, 1824).} unchanged since it first appeared in 1716, as their primary source of information about the plea's origins:

\begin{quote}
\singlespacing
An implied confession is when a defendant, in a case not capital, doth not directly own himself guilty, but in a manner admits it by yielding to the King's mercy, and desiring to submit to a small fine: in which case, if the court think fit to accept of such submission, and make an entry that defendant \textit{prosuit se in gratiam regis}, without putting him to a direct confession, or plea (which in such cases seems to be left to discretion), the defendant shall not be estopped to plead not guilty to an action for the same fact, as he shall if the entry is \textit{quod cognovit indictamentum}.\footnote{See \textit{ibid} at 466.}
\end{quote}

By the time Hawkins published the first edition of his Treatise of Pleas of the Crown, \textit{nolo contendere} pleas had started to fall into disuse. The last reported case involving a nolo\textit{ contendere} plea in England was \textit{The Queen v Templeman} in 1702.\footnote{See \textit{Templeman}, \textit{supra} note XXX.} Thereafter, the \textit{nolo contendere} plea lay dormant\footnote{See \textit{Hudson et al v The United States}, 272 US 451, 47 SCt 127 at para 4 [\textit{Hudson}].} until reappearing a century later in reported American decisions.

\subsection{\textit{Nolo contendere} pleas in America}

One of the earliest reported American cases dealing with \textit{nolo contendere} pleas is the 1806 decision \textit{Commonwealth v The Town of Northampton}.\footnote{See \textit{Commonwealth v The Town of Northampton}, 2 Mass 116, 1806 WL 756.} The town of Northampton entered a \textit{nolo contendere} plea to the charge of failing to provide a schoolmaster but sought to withdraw the plea at the next court sitting because the original indictment was bad. The town noted that the original indictment charged them with an offence against ``the peace and dignity of the commonwealth" but not with any offence prescribed by statute. The court agreed and set the judgment aside.

Although \textit{nolo contendere} pleas had been inuse in Massachusetts for decades, abouts about its legitimacy remained. In 1829, the Massachusetts Supreme Judicial Court had deployed a comprehensive \textit{nolo contendere} plea. This development was apparent in the case of \textit{Commonwealth v James Horton},\footnote{See \textit{Commonwealth v James Horton}, 9 Pick 206, 26 Mass 206, 1829 WL 1997.} where the state charged the defendant with ``a breach of the law relating to retailers.'' When arraigned, Horton said he would not contend with the Commonwealth. As a result, the court fined him \$21 plus costs. On appeal, both parties argued that the plea was ``irregular,'' ``was no answer to the indictment,'' and amounted, at most, to an ``implied confession.''\footnote{See \textit{ibid} at 210.} 

The court disagreed, finding that Horton had effectively entered a \textit{nolo contendere} plea. The court examined \textit{nolo contendere} pleas in detail and agreed that they were implied confessions, but found that the pleas were validin principle., such that court has the discretion to accept or reject them. The court noted that defendants who plead \textit{nolo contendere} may contest the same facts at subsequent proceedings.\footnote{See \textit{ibid}.} The court cited Hawkins' notation in the Treatise of Pleas of the Crown and the 1702 English decision The \textit{Queen v Templeman} to help define the contours of \textit{nolo contendere} pleas in their jurisdiction and dismissed the appeal. In this early decision, the centrality of \textit{Templeman} and Hawkins' notation underscores the role they have played in the preservation and revival of \textit{nolo contendere} pleas in the United States but also highlights the limited pool of common-law authority for these pleas. Nearly one hundred years after \textit{Horton}, the US Supreme Court seized on this fact when it decided whether to read a common-law limitation into \textit{nolo contendere} pleas across the country in \textit{Hudson et al. v United States}.

\subsubsection{\textit{Hudson et al v United States}}

In \textit{Hudson},\footnote{See \textit{Hudson}, \textit{supra} note XXX.} the appellants entered \textit{nolo contendere} pleas to mail fraud charges and were sentenced to a year plus a day. On appeal, they argued that the sentencing court could not impose a custodial sentence on a \textit{nolo contendere} plea. The appellants cited a string of authorities from the Seventh Circuit Federal Court of Appeals\footnote{See \textit{ibid} at para 2.} that relied heavily upon Hawkins' reference to the ``small fine" a defendant submitted themselves to and the "case not capital" limitation.\footnote{See \textit{ibid} at para 2.} The court in \textit{Hudson} noted the dearth of authorities apart from Hawkins and therefore found little support for the inferences the appellants asked the court to draw.\footnote{See \textit{ibid} at paras 9 and 16 and note 2.} The court remarked that \textit{nolo contendere} pleas had a minimal reliable historical backdrop\footnote{See \textit{ibid} at para 9.} and that it was not prepared to create a broad rule that was specifically based on an obscure citation. The court ruled that custodial sentences were available for \textit{nolo contendere} pleas, affirming decisions already reached by most lower American jurisdictions at the time.\footnote{See \textit{ibid} at para 16.}

\subsection{The four \textit{nolo contendere} criteria}

When the United States Supreme Court decided \textit{Hudson}, the American \textit{nolo contendere} plea was primarily available at common law. Nearly one hundred years later, the \textit{nolo contendere} plea is available federally, and most American states have passed legislation authorizing it.\footnote{Namely, Alaska, Arizona, Arkansas, California, Delaware, Florida, Georgia, Hawai'i, Kansas, Louisiana, Maine, Maryland, Massachusetts, Michigan, Mississippi, Montana, Nebraska, Nevada, New Hampshire, New Mexico, North Carolina, Ohio, Oklahoma, Oregon, Pennsylvania, Rhode Island, South Carolina, South Dakota, Tennessee, Texas, Utah, Vermont, Virginia, West Virginia, Wisconsin, and Wyoming.}  Those that have not do not allow them.\footnote{Namely, Alabama, Idaho, Illinois, Indiana, Iowa, Kentucky, Minnesota, Missouri, New Jersey, New York, North Dakota, and Washington.} Different states implemented \textit{nolo contendere} pleas differently and adopted various views about the plea's subsequent effects. In the 1944 American Law Review annotation ``Plea of \textit{nolo contendere} or \textit{non vult contendere}," KA Drechsler provided an early, comprehensive overview of \textit{nolo contendere} pleas and the different ways that states approached and implemented them. The annotation proposed a model that analyzed the \textit{nolo contendere} plea using four criteria:

\begin{itemize}
    \item \textbf{Applicability.} A jurisdiction may allow a defendant to plead \textit{nolo contendere} to some, all, or no offences. Where defendants may only plead \textit{nolo contendere} to some offences, the offences they may enter the plea to may vary from state to state. The applicability criterion tracks these differences.
    \item \textbf{Acceptability.} Where a defendant may enter a \textit{nolo contendere} plea, certain preconditions may need to obtain before the court may accept the plea. The acceptability criterion specifies what those conditions are and where they are required.
    \item \textbf{Procedural effects.}\footnote{Drechsler refers to these as ``effects in the case."} \textit{Nolo contendere} pleas, once entered, require the court to undergo certain procedures or take specific actions concerning a defendant's case. The procedural effects criterion covers the immediate effects that entering a \textit{nolo contendere} plea has on a defendant's case.
    \item \textbf{Subsequent effects.}\footnote{Drechsler refers to these as ``consequences outside the case."} The implications for defendants who enter \textit{nolo contendere} pleas may extend beyond their conviction. These implications may even arise in jurisdictions that disallow the plea. The subsequent effects criterion considers these implications.
\end{itemize}

This classification scheme enables legal scholars to categorize and identify individual instances of the plea with one another. The information gained from this analysis may help scholars and jurists understand and identify how different implementations of \textit{nolo contendere} pleas relate to one another more generally. As I will demonstrate later in this thesis, Dreschler's system also allows scholars and jurists to compare and contrast emerging phenomena, like the \textit{nolo contendere} procedure used in Canada, with their formal counterparts.

\subsubsection{Applicability}

The first component, applicability, addresses the question of which offences may sustain a \textit{nolo contendere}. A \textit{nolo contendere} plea's applicability is further reducible to one of four distinct types: namely, (1) \textit{all offences}; (2) only \textit{non-capital} offences; (3) \textit{some offences}; or (4) \textit{no offences}. 

Although scholars and jurists historically confined \textit{nolo contendere} pleas to minor criminal infractions, \textit{Hudson} confirmed defendants could enter them at common law for indictable felonies punishable by prison terms. This ruling opened the door for courts to accept \textit{nolo contendere} pleas for all offences at common law. A review of the plea today reveals that the legislatures followed suit, such that \textit{nolo contendere} is usually universally applicable where allowed. Of the 38 states that allow \textit{nolo contendere} pleas in criminal cases,\footnote{See note XXX above.} all but four permit defendants to enter them for all criminal offences.\footnote{Namely, Georgia, Louisiana, Mississippi and South Carolina.} In some states, such as Alaska, \textit{nolo contendere} pleas have become so prominent that defendants rarely, if ever, enter guilty pleas to self-convict.\footnote{See Jana L Kuss, "Endangered Species: A Plea for the Preservation of Nolo Contendere in
Alaska" (2005) 41:3 Gonz L Rev 539. In Alaska, defendants do not require consent from the prosecutor or the court to enter \textit{nolo contendere} pleas. As a result, the \textit{nolo contendere} plea has effectively supplanted the guilty plea and is now used almost exclusively by defendants wanting to self-convict.}

Even the ``non-capital'' requirement, once thought to be an integral component of \textit{nolo contendere} pleas,\footnote{See \textit{Drechler's annotation}, \textit{supra} note XXX at \S II(b)(3): ``While there is unanimity among the courts that the (\textit{nolo contendere}) plea cannot be accepted in a case where capital punishment is mandatory, they are divided in regard to the question whether it may be accepted where the penalty is or may be imprisonment."} is mainly absent from the statutory plea. At the time of writing, the death penalty is legal and operational in 21 states,\footnote{Namely, Arizona, Arkansas, Florida, Georgia, Idaho, Indiana, Kansas, Kentucky, Louisiana, Mississippi, Missouri, Nebraska, Nevada, Oklahoma, South Carolina, South Dakota, Tennessee, Texas, Utah, and Wyoming} legal but subject to an official moratorium in three states,\footnote{Namely, California, Oregon, and Pennsylvania.} legal but subject to a \textit{de facto} moratorium in three others,\footnote{Montana, North Carolina, and Ohio.} and illegal in the remaining 23 states.\footnote{Namely, Alaska, Colorado, Connecticut, Delaware, Hawai'i, Illinois, Iowa, Maine, Maryland, Massachusetts, Michigan, Minnesota, New Hampshire, New Jersey, New Mexico, New York, North Dakota, Rhode Island, Vermont, Virginia, Washington, West Virginia, and Wisconsin} Sixteen states that allow \textit{nolo contendere} pleas have and implement the death penalty.\footnote{Namely, Arizona, Arkansas, Florida, Georgia, Kansas, Louisiana, Mississippi, Nebraska, Nevada, Oklahoma, South Carolina, South Dakota, Tennessee, Texas, Utah, and Wyoming.}  Of these states, all but four allow defendants to enter \textit{nolo contendere} pleas in death penalty cases.\footnote{Namely, Georgia, Louisiana, South Carolina, and Mississippi. Both Georgia and Louisiana expressly prohibit \textit{nolo contendere} in death penalty cases, while South Carolina and Mississippi only allow \textit{nolo contendere} pleas in misdemeanour cases, where the death penalty is presumably unavailable.} All states with death penalty moratoriums, official or otherwise, also allow \textit{nolo contendere} pleas. The \textit{nolo contendere} plea is broadly applicable in the United States today. Since \textit{Hudson} and widespread codification, where the plea is allowed, defendants may generally enter it for any offence.\footnote{See \textit{May v Lingo}, 167 So 2d at 270.}

\subsubsection{Acceptability}

The second criterion, acceptability, addresses whether the court may accept the plea. A \textit{nolo contendere} plea's acceptability is a function of a judge’s discretion to accept or reject it and what requirements they are subject to in doing so, and can also be broken down into four types: namely, (1) \textit{no discretion to accept} the plea; (2) \textit{some discretion to accept or reject} the plea; (3) \textit{full discretion to accept or reject} the plea; and (4) \textit{no discretion to reject} the plea.

Unless legislation provides explicitly for a \textit{nolo contendere} plea, they are typically not allowed.\footnote{As an example, cite that superior court case out of Indiana.} Judges in those states have \textit{no discretion to accept} the plea. Jurisdictions that allow defendants to enter \textit{nolo contendere} pleas subject to meeting certain conditions give the court \textit{some discretion to accept or reject} \textit{nolo contendere} pleas.\footnote{The ``public interest and effective administration of justice” test required by the federal rule, for example, has been mirrored in several states' legislation and serves as a high-level limit on a judge's discretion to accept the plea. Arizona, Arkansas, Delaware, Hawai'i, South Dakota, Utah, West Virginia, Wyoming, and the federal rule all require courts to apply this test. Several states also require that the courts obtain the explicit consent of the prosecutor before accepting \textit{nolo contendere} pleas. These include Arkansas, Hawai'i, Maine, Montana, North Carolina, and the federal rule.} Others impose no apparent limits on \textit{nolo contendere} pleas beyond those already imposed on guilty pleas. Judges in these states may be said to have \textit{full discretion to accept or reject} a \textit{nolo contendere} plea. Finally, where judges must accept \textit{nolo contendere} pleas, they have \textit{no discretion to reject} the plea.\footnote{Only Virginia appears to give judges no discretion to reject a \textit{nolo contendere} plea, despite having the discretion to reject a guilty plea. See VA Code Ann. § 19.2-254. As will be seen below, however, the \textit{nolo contendere} procedure used and authorized in Canada may give prosecutors and defendants the power to similarly compel a judge to accept a \textit{nolo contendere}-like plea.}

\subsubsection{Procedural effects}

\textit{Nolo contendere} pleas generally have the same legal effect as a guilty plea within the proceedings, including the constitutional right against double jeopardy and a trial waiver. For the most part, where states allow defendants to plead \textit{nolo contendere}, the effect of that plea is the same as if the defendant had pleaded guilty. The defendant waives their right to a trial, is convicted, and is sentenced.\footnote{Some states make this explicit. See e.g. Oregon: ORS § 135.345; Rhode Island: RI R REV Rule 609; New Mexico: NM ST § 30-1-11, and Louisiana: LA CCrP Art 552(4). Pleas at the arraignment. Some exceptions exist. Ohio, for example, makes it clear that \textit{nolo contendere} pleas are not admissions of guilt and distinguish them from guilty pleas accordingly. See OH ST RCRP Rule 11(B)(2): ``The plea of no contest is not an admission of defendant's guilt, but is an admission of the truth of the facts alleged in the indictment, information, or complaint, and the plea or admission shall not be used against the defendant in any subsequent civil or criminal proceeding."}

However, some procedural differences do exist. In Massachusetts, defendants pleading \textit{nolo contendere} may not enter formal plea agreements with the prosecutors.\footnote{See MA ST RCRP Rule 12(b)(1).} In Mississippi, judges must conduct a plea voluntariness and comprehension inquiry with defendants who enter guilty pleas to any offence that carries a possible jail sentence. However, no such requirement appears to be in place for defendants who enter \textit{nolo contendere} pleas to the same offences.\footnote{See MS R RCRP Rule 15.3.} By contrast, California requires judges to conduct a particular plea inquiry with defendants who enter a \textit{nolo contendere} plea that is not required with defendants who plead guilty.\footnote{See West's Ann.Cal.Penal Code § 1016(3).} Meanwhile, in Oregon, judges are statutorily required to accept joint recommendations put forward by counsel on a guilty plea but not similarly required to do the same for \textit{nolo contendere} pleas.\footnote{See ORS § 135.385(2)(e). Although ORS § 135.432 states that judges are not bound by plea deals made between parties, § 135.385(2)(e) states that judges are to tell defendants that sentencing recommendations reached through formal disposition recommendations \textit{will be accepted} by the courts.} 

\subsubsection{Subsequent effects}

Historically, the main difference between \textit{nolo contendere} and guilty pleas has been that the former are usually inadmissible in subsequent proceedings. Even in jurisdictions that do not permit them, evidence of the fact that a defendant pleaded \textit{nolo contendere} is often considered inadmissible in subsequent civil and even criminal proceedings. While twelve states do not allow criminal defendants to enter \textit{nolo contendere} pleas,\footnote{See note XXX above.} only seven states allow evidence of \textit{nolo contendere} pleas in subsequent proceedings.\footnote{Namely, Alaska, Arizona, Illinois, Indiana, Missouri, New Jersey, and New York.} Two of these states, Alaska and Arizona, allow defendants to enter \textit{nolo contendere} pleas but also allow admitting evidence of those pleas at subsequent proceedings. It is perhaps unsurprising that it is admissible in subsequent proceedings. Arizona, meanwhile, formally authorizes \textit{nolo contendere} pleas but uses them very sparingly. See ....  Only Illinois, Indiana, Missouri, New Jersey and New York neither allow defendants to plead \textit{nolo contendere} nor recognize \textit{nolo contendere} pleas entered in other jurisdictions. All other states provide \textit{nolo contendere} defendants with some degree of collateral estoppel, though the degree of protection provided also varies from state to state.\footnote{Arkansas, California, Colorado, Connecticut, Hawai'i, Louisiana, Michigan, and Virginia make \textit{nolo contendere} pleas inadmissible at civil proceedings. California restricts this to non-felony convictions. Delaware, Florida, Idaho, Iowa, Kansas, Kentucky, Maine, Maryland, Massachusetts, Minnesota, Mississippi, Montana, Nebraska, Nevada, New Hampshire, New Mexico, North Carolina, North Dakota, Ohio, Oklahoma, Oregon, Pennsylvania, Rhode Island, South Carolina, South Dakota, Tennessee, Texas, Utah, Vermont, Washington, West Virginia, Wisconsin and Wyoming all make \textit{nolo contendere} proceedings inadmissible at subsequent civil and criminal proceedings.} Most states allow defendants to contest the allegations underlying their convictions at subsequent civil or criminal proceedings, while Alabama and Georgia allow defendants to contest their \textit{nolo contendere} convictions at virtually any subsequent proceeding.\footnote{See \textit{McNair v State}, 653 So 2d 320 (Ala Crim App 1992): ``Alabama [...] follows the minority rule that a conviction resulting from a plea of nolo contendere is inadmissible not only to prove the conduct underlying the conviction, but for all other purposes"; § 17-7-95(c): ``Except as otherwise provided by law, a plea of nolo contendere shall not be used against the defendant in any other court or proceedings as an admission of guilt or otherwise or for any purpose; and the plea shall not be deemed a plea of guilty for the purpose of effecting any civil disqualification of the defendant to hold public office, to vote, to serve upon any jury, or any other civil disqualification imposed upon a person convicted of any offense under the laws of this state."}

Although a \textit{nolo contendere} plea is generally inadmissible in subsequent proceedings, this inadmissibility can take several forms. In Pennsylvania, evidence of \textit{nolo contendere} is generally inadmissible. However, the court must admit evidence of that offence when a defendant enters a \textit{nolo contendere} plea to a crime of dishonesty.\footnote{See PA ST REV Rule 609(a).} In California, defendants who enter \textit{nolo contendere} pleas to misdemeanour offences are protected from having evidence of that admitted in a later court proceeding but have no such protections for felonies.\footnote{See West's Ann Cal Penal Code § 1016: ``There are six kinds of pleas to an indictment or an information, or to a complaint charging a misdemeanor or infraction:

... 

3. Nolo contendere, subject to the approval of the court. The court shall ascertain whether the defendant completely understands that a plea of nolo contendere shall be considered the same as a plea of guilty and that, upon a plea of nolo contendere, the court shall find the defendant guilty. The legal effect of such a plea, to a crime punishable as a felony, shall be the same as that of a plea of guilty for all purposes. In cases other than those punishable as felonies, the plea and any admissions required by the court during any inquiry it makes as to the voluntariness of, and factual basis for, the plea may not be used against the defendant as an admission in any civil suit based upon or growing out of the act upon which the criminal prosecution is based."} Still others, like Michigan, bar evidence of \textit{nolo contendere} pleas in subsequent civil suits \textit{unless} the defendant is the one filing the suit.\footnote{See MI R REV MRE 410(2).} South Carolina allows \textit{nolo contendere} pleas to be admitted for impeachment purposes only if the conviction was for an offence punishable by death or more than one year's imprisonment.\footnote{See SC R REV Rule 609(a).} Meanwhile, Rhode Island's approach stipulates that a defendant may only receive a \textit{nolo contendere} plea's benefits after completing a probation period without violations.\footnote{RI R REV Rule 410: ``Except as otherwise provided in this rule, evidence of the following is not, in any civil or criminal proceeding, admissible against the defendant who made the plea or was a participant in the plea discussions: ... (3) a plea of nolo contendere where the court defers sentence, places the defendant on probation pursuant to § 12-18-1 of the General Laws, or files the case pursuant to § 12-10-12 of the General Laws, provided that probation is the sole sanction imposed and provided further that said period of deferral, probation, or filing is completed without violation of the terms thereof."} Others, like South Carolina, exclude evidence of \textit{nolo contendere} pleas except when they are used to impeach a defendant on cross-examination. \footnote{See SC R REV Rule 609: ``(a) General Rule. For the purpose of attacking the credibility of a witness, ... (2) evidence that any witness has been convicted of a crime shall be admitted if it involved dishonesty or false statement, regardless of the punishment. For the purposes of this rule, a conviction includes a conviction resulting from a trial or any type of plea, including a plea of nolo contendere or a plea pursuant to \textit{North Carolina v Alford}, 400 US 25 (1970)."}

\subsection{\textit{Nolo contendere} pleas in Canada}

Because Canada codified its criminal law early into its history and only allowed defendants to enter guilty or not guilty pleas, \textit{nolo contendere} pleas have not attracted much attention or generated much discussion here. In recent years, however, Canadian courts have recognized an informal \textit{nolo contendere} plea procedure that allows defendants to avoid pleading guilty while ensuring self-conviction. I review the statutory loopholes that allow defendants to enter these pseudo-pleas and the unusual effects this plea procedure can generate in detail in \S\S 3.3 and 4.4.5 below.