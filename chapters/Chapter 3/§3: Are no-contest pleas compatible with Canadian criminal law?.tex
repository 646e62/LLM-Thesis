\chapter{Are \textit{nolo contendere} pleas compatible with Canadian criminal law?}

\setcounter{footnote}{94}

Having defined pleas generally, outlined the fundamental features of uncontested pleas specifically, and reviewed both guilty and \textit{nolo contendere} pleas, I next ask whether the latter is compatible with Canadian criminal law. I first identify the relevant sections in the \textit{Criminal Code} and then analyze each plea against that framework. 

The first part of my analysis focuses on the statutes authorizing criminal pleas in Canada. Substantive criminal law in Canada has been primarily statutory since the \textit{Criminal Code of Canada 1892} and entirely statutory since Parliament abolished all common-law offences in 1955.\footnote{See Public Prosecution Service of Canada, ``Chapter 1 - FPS Deskbook" (24 December 2008), online: \textless https://www.ppsc-sppc.gc.ca/eng/pub/fpsd-sfpg/fps-sfp/fpd/ch01.html\textgreater.} Although Canadian criminal procedure is still heavily influenced by both the common law and local rules and customs, the \textit{Criminal Code} governs it for the most part. The \textit{Criminal Code} thus is central to understanding what pleas and procedures Canadian criminal law permits and prohibits. Therefore, my compatibility analysis begins by identifying and understanding the legislation's relevant parts. 

With the statutory framework outlined, the second part of this chapter examines whether \textit{nolo contendere} pleas are compatible with Canadian criminal law. To answer this question, I examine how \textit{Criminal Code} sections 606 and 655 allow defendants to self-convict without formally pleading guilty. The intersection between these two sections demonstrates that while Canadian criminal law formally forbids \textit{nolo contendere} pleas, they are nonetheless informally and functionally compatible.