\chapter{ Are no-contest pleas compatible with Canadian criminal law?}
Having outlined the fundamental features of a no-contest plea, the next question I address is whether these pleas are compatible with Canadian criminal law. I first outline and explicate the relevant statutory provisions in the \textit{Criminal Code}. I follow this with an analysis of each type of no-contest plea against that framework and present my conclusions for each.

The first part of this analysis focuses on the Canadian criminal statutory framework and the provisions made for criminal pleas. Substantive criminal law in Canada has been primarily statutory since the first Criminal Code of Canada was adopted in 1892 and entirely statutory since Parliament abolished all common-law offences in 1955.\footnote{See online: \textless https://www.ppsc-sppc.gc.ca/eng/pub/fpsd-sfpg/fps-sfp/fpd/ch01.html\textgreater} Procedural criminal law has similarly become governed by statute, with the current \textit{Criminal Code} containing detailed processes and comprehensive protections throughout its sections and parts. In many ways, Canadian criminal law is reducible to its statutory framework, including Acts like the \textit{Youth Criminal Justice Act}, the \textit{Controlled Drugs and Substances Act}, and most importantly, the \textit{Criminal Code}. Because the \textit{Criminal Code} is central to understanding what pleas Canadian criminal law permits and prohibits, the compatibility analysis must begin by outlining the relevant parts of that legislation.

With the statutory framework outlined, the second part examines whether guilty pleas are compatible with Canadian criminal law. As Canadian criminal law's central text, the \textit{Criminal Code} provides both the bulk of substantive and procedural criminal law in Canada. This includes both the pleas permitted by law and the procedure for entering those pleas.  ... Guilty pleas are compatible with Canadian criminal law, albeit with certain restrictions and no implicit right to enter one. 

The third section examines whether \textit{nolo contendere} pleas are also compatible with Canadian criminal law. This analysis demonstrates that while Canadian criminal law formally forbids \textit{nolo contendere} pleas, a creative approach to \textit{Criminal Code} s 655 makes informal \textit{nolo contendere} pleas possible. Recent appellate decisions in Ontario, Quebec, and Nova Scotia have all affirmed that courts may accept these functional equivalents of \textit{nolo contendere} pleas, and the Supreme Court of Canada has thus far declined to disturb these results. \textit{Nolo contendere} pleas are partially but functionally compatible with Canadian criminal law.

The fourth and final part examines whether best-interest pleas are compatible with Canadian criminal law. Historically, and for the most part, Canadian criminal courts have understood protestations of innocence as being incompatible with guilty pleas. Instead, these courts have held to the principle that they must understand guilty pleas as actual admissions of guilt. However, a plain reading of s 606(1.1) of the \textit{Criminal Code} gives reason to doubt this conventional wisdom, and appellate courts across Canada have upheld best-interest pleas, both explicitly and implicitly. Canadian courts are divided on the issue of whether best-interest pleas are compatible with Canadian criminal law.