\chapter{Are uncontested pleas compatible with Canadian criminal law?}
Having outlined the fundamental features of uncontested pleas and reviewed two examples, I next ask whether these pleas are compatible with Canadian criminal law. I first identify the relevant statutory provisions in the \textit{Criminal Code} and then analyze each plea against that statutory framework.

The first part of my analysis focuses on the statutes authorizing criminal pleas in Canada. Substantive criminal law in Canada has been primarily statutory since the \textit{Criminal Code of Canada 1892} and entirely statutory since Parliament abolished all common-law offences in 1955.\footnote{See online: \textless https://www.ppsc-sppc.gc.ca/eng/pub/fpsd-sfpg/fps-sfp/fpd/ch01.html\textgreater} Although still heavily influenced by both the common law and local rules and customs, procedural criminal law is similarly governed by the \textit{Criminal Code}. The \textit{Criminal Code} is central to understanding what pleas and procedures Canadian criminal law permits and prohibits. Therefore, the compatibility analysis should begin by identifying and understanding the relevant parts of that legislation. 

With the statutory framework outlined, the second part of this chapter examines whether \textit{nolo contendere} pleas are compatible with Canadian criminal law. Specifically, I examine how \textit{Criminal Code} sections 606 and 655 allow defendants to self-convict without formally pleading guilty. This demonstrates that while Canadian criminal law formally forbids \textit{nolo contendere} pleas, they are partially but functionally compatible with Canadian criminal law.