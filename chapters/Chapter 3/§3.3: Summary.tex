\section{Summary}

The \textit{Criminal Code} exclusively accommodates guilty and not guilty pleas. Alternative uncontested pleas like \textit{nolo contendere} are formally excluded. Defendants must set a trial if they cannot or will not plead guilty but may formally admit any of the prosecutor's allegations once they have entered not guilty pleas. Therefore, defendants may effectively enter an informal \textit{nolo contendere} plea through this procedure. Since being identified as such, the \textit{nolo contendere} procedure has become a recognized part of Ontario criminal procedure, and defendants may use it there without apparent limitations. Because the \textit{Criminal Code} is nationally binding, nothing prevents this procedure from taking root in other jurisdictions, as it has apparently begun to do in Nova Scotia and Quebec. However, these pleas are unregulated and risky and have the potential to produce unusual and unexpected results. Having thus determined that a \textit{nolo contendere} near-equivalent is \textit{possible} in Canada, despite statutory language to the contrary, I go on to answer whether having a \textit{nolo contendere} plea in any form is a \textit{good idea}. In the following section, I consider whether this unlegislated and seemingly unintended procedure should be encouraged or discouraged, formalized or left alone, and propose what should be done to address the issues it introduces.