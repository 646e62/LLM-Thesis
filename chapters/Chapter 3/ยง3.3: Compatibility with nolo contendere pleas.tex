\section{Compatibility with \textit{nolo contendere} pleas}

\subsection{\textit{Nolo contendere} pleas are formally excluded}

As discussed above, \textit{Criminal Code} s 606(1) explicitly states that the court may only accept guilty, not guilty, and the special double jeopardy pleas. As a result, formal no-contest pleas such as the \textit{nolo contendere} and \textit{nolo contendere}-like pleas used throughout the United States, are formally excluded from Canadian criminal law.

\subsection{\textit{Nolo contendere} pleas may be constructively entered through \textit{Criminal Code} s 655}

Although \textit{Criminal Code} s 606(1) formally forbids \textit{nolo contendere}, defendants may informally enter the \textit{nolo contendere} plea's constructive equivalent. For this argument, a formal \textit{nolo contendere} plea is any such plea authorized by statute. In contrast, an informal \textit{nolo contendere} plea is any plea or plea procedure that performs a \textit{nolo contendere} plea's equivalent function. Again for this argument, an informal plea or plea procedure performs the same function as a \textit{nolo contendere} plea's equivalent function when it relieves defendants from having to admit guilt formally and spares the prosecution their burden of proof. Procedurally, a defendant may do so by (1) pleading not guilty and admitting the offence's elements at trial; or (2) refusing to enter a plea and admitting the offence's elements at the trial.

Although both \textit{Criminal Code} sections 606(2) and 655 have existed in some form or another since the first \textit{Criminal Code 1892},\footnote{Formerly \textit{Criminal Code 1892} s 562} courts have only recently begun to overtly recognize the extent to which these two sections interact with one another. Specifically, it was only in the 1969 \textit{Castellani} decision that the Supreme Court of Canada began to recognize \textit{Criminal Code} s 655's scope and breadth.

\subsubsection{\textit{Castellani v R}, 1969 CanLII 57}

In \textit{Castellani}, a jury convicted the appellant of murdering his wife. At trial, Castellani sought to admit facts under s 562 [655]. The Crown took issue with one fact he proposed, and the trial judge ruled the appellant could not admit it. He was convicted and appealed. The British Columbia Court of Appeal ruled that the trial judge erred by disallowing the admission. Nevertheless, it ruled that the error did not prejudice Castellani's defence and dismissed his appeal. Castellani appealed to the Supreme Court of Canada.

In dismissing his appeal, the Supreme Court of Canada found that the trial judge had not committed an error and that the appellant was not entitled to invent and agree to allegations unilaterally. Chief Justice Cartwright, for the court, established that a defendant could only admit to facts that the Crown had alleged against them. Defendants cannot, in other words, frame the allegations as they please and then require that the court accept them as proven.

Having established that s 562 [655] allows a defendant to admit any fact the Crown alleges against them, the Supreme Court of Canada set the foundation for the procedure used in the case under review in \textit{R v Cooper}.

\subsubsection{@1977CanLII11}

In \textit{R v Cooper}, the Supreme Court of Canada heard an appeal from a conviction obtained entirely from an agreed statement of facts between the appellant and the Crown. A jury found that the appellant conferred a benefit on a government employee and convicted him. At trial, the Crown entered its entire case through a statement of facts that the defence agreed to. The trial hinged on whether these agreed facts made out the elements of the offence.

The Supreme Court of Canada was divided on whether the rule in Hodge's Case, rather than the procedure adopted at trial. Both the majority and dissent remarked that the procedure the parties adopted at trial was unusual, but neither found that the procedure caused an error. The tacit approval the court gave this trial procedure would prove significant nearly 30 years later when Canadian appellate courts started to draw the connections between this procedure and \textit{nolo contendere} pleas. Before that, however, the Ontario Court of Appeal outlined how prosecutors and defendants may use \textit{Criminal Code} s 655 to create informal conditional pleas\footnote{Define if not already defined.} in the 1993 decision \textit{R v Fegan}.

\subsubsection{\textit{R v Fegan}, 1993 CanLII 8607 (ON CA)}

In \textit{Fegan}, the appellant pleaded guilty to having made harassing and threatening telephone calls after losing a pre-trial evidentiary motion. His lawyer mistakenly believed that Fegan would be allowed to appeal if he pleaded guilty and advised him to do so. The Ontario Court of Appeal rightly noted that the appellant attempted to enter a conditional plea and that conditional pleas are formally unavailable in Canada. The court described how the same procedure could have been accomplished informally through a full factual admission via \textit{Criminal Code} s 655 rather than a not guilty plea. Ultimately, the Court of Appeal dismissed Fegan's appeal on an unrelated evidentiary ruling.

\subsection{\textit{Criminal Code} s 655 and the \textit{nolo contendere} plea procedure's emergence}

Nearly two decades after the \textit{Fegan} decision, the Ontario Court of Appeal, headed by Justice Watt, heard a case centred on an unusual plea procedure. Much like the procedure that \textit{Fegan} described, the defendant in \textit{R v DMG} self-convicted by formally admitting the elements of the offence at trial through \textit{Criminal Code} s 655. Unlike Fegan, however, DMG did not argue a defence at trial and claimed on appeal that his counsel was ineffective and that his plea was involuntary. The court sided with DMG in both respects. For a time following, more than a few practitioners and lower courts interpreted this as the court reproaching the \textit{nolo contendere} procedure underlying the controversy in \textit{DMG}.

An Ontario Court of Appeal, again led by Justice Watt, would later clarify this issue in \textit{R v RP}. In that case, the court specified that the \textit{nolo contendere} plea procedure used in \textit{DMG} was compliant with the \textit{Criminal Code} and that nothing prevented the defendants from using it properly. The problem in \textit{DMG} was that DMG's trial counsel entered the plea for his client without adequately informing him because trial counsel failed to prepare for the hearing properly. When entered knowingly and voluntarily by defendants or their competent counsel, such pleas are lawful.

\subsubsection{\textit{R v DMG}, @2011onca343}

\paragraph{Background\\}

In \textit{R v DMG}, the appellant/defendant DMG faced sexual assault and interference charges involving the same complainant. He denied the allegations and told his lawyer he wanted to plead not guilty but expressed some concern about requiring the complainant to testify. He believed that, by pleading not guilty, he would have the chance to testify at trial. He pleaded not guilty, but the prosecutor read the allegations against him at the trial while his lawyer made no submissions. The judge convicted the appellant and remanded him for sentencing.

At the outset, the parties reserved two days for trial. At the pre-trial, the Crown notified trial counsel that they would seek to admit the complainant's statement via CC 715.1. The trial counsel acknowledged that the appellant's statement was voluntary, despite not having transcribed or reviewed it. The pre-trial judge opined that he would have imposed a 12-month sentence, a proposition the appellant rejected at the time.

Five weeks before trial, trial counsel sent a resolution offer whereby the appellant would not contest the interference charges in exchange for a six-month CSO and 18 months probation. Trial counsel noted that DMG did not want to put the complainant through the stress of a trial. A week later, the Crown countered with a 15 - 18 month jail sentence, not contingent on a joint submission. One week before trial, the Crown readvised trial counsel that she would be looking to adduce the complainant's statement via CC 715.1 and to apply to have her testify from behind a screen. Trial counsel did not respond to either email.

Trial counsel met with the prosecutor on the day of trial, who agreed not to seek a sentence exceeding 15 months if the parties resolved the matter before trial. Trial counsel met with the appellant to discuss. According to trial counsel, he provided the Crown's position to the appellant, who reiterated that he did not wish to put the complainant through a trial and agreed to resolve on the Crown's terms. Crown counsel provided the appellant with written instructions to sign. During this meeting, the appellant told trial counsel that he was not guilty and wanted to testify. He stated that trial counsel asked him whether he thought he was doing a good job and asked him to sign a memo stating the same. He never saw nor signed written instructions.

At trial, trial counsel told the court that the appellant would be entering a not guilty plea but would not dispute the allegations. The Crown read in the allegations, and the trial judge entered a conviction and remanded the appellant for sentencing. The trial judge did not conduct a plea voluntariness or comprehension inquiry with the appellant. At sentencing, the appellant expressed remorse for his actions. The trial judge imposed the jointly recommended sentence.

\paragraph{Analysis\\}

Justice Watt, again writing for a unanimous Ontario Court of Appeal, found that the procedure adopted at trial was similar in many respects to a \textit{nolo contendere} plea, despite Canadian criminal law formally excluding that plea. It noted that the Federal Rules of Criminal Procedure required plea voluntariness and comprehension inquiries for guilty and \textit{nolo contendere} pleas. The Ontario Court of Appeal found that Canadian law permitted the procedure but that it suffered in its execution in two key ways:

\begin{enumerate}
    \item The allegations, as read out by the prosecution, did not constitute an admission under CC 655 or otherwise admissible evidence
    \item As the plea was a functional \textit{nolo contendere} plea, the trial judge should have conducted a plea comprehension and voluntariness inquiry.
\end{enumerate}

The Ontario Court of Appeal hesitantly concluded that the procedure adopted at trial caused a miscarriage of justice due to these irregularities and granted the appeal. The court found that the plea procedure was functionally equivalent to a guilty plea and was not satisfied that the appellant knew what the procedure entailed. As a result, the court found that his plea was involuntary.

\paragraph{Conclusion\\}

The appeal focused on how trial counsel's actions prejudiced the appellant at trial. But despite trial counsel's shortcomings, the court noted several times that ``[n]o statutory provision or common law principle prohibits a procedure similar to what was followed here after an accused has entered a plea of not guilty." The flaw was not in the plea procedure but in how trial counsel employed it.\footnote{See para 51.} This seemingly overlooked portion of the analysis in \textit{DMG} would become the subject of \textit{R v RP} a year and a half later.

\subsubsection{\textit{R v RP}, @2013onca53}

\paragraph{Background\\}

Following the decision in \textit{DMG}, the \textit{nolo contendere} procedure's legality and propriety were uncertain.\footnote{See @2014onlsta50} The \textit{R v RP} decision soon settled these discussions. In \textit{RP}, the appellant faced 19 historical sexual offence allegations involving four family members. The complainants alleged the abuse occurred between ages 3 and 5 and ended between ages 9 - 12.

On the trial's first day, the first complainant testified. Following the first day, trial counsel opined to RP that the witness did well. He noted that RP did not appear to be holding up well physically and expressed concerns about RP's ability to make it through the proceedings. The next day, following a meeting between trial counsel and the prosecutor, the appellant signed written instructions outlining the allegations that he and his counsel expected the remaining witnesses would confirm. The appellant acknowledged that the Crown had a strong case and that neither he nor his witness had a reasonable response to the evidence they expected. The instructions indicated that the appellant would not contest the allegations and acknowledged that doing so would result in a conviction. RP reviewed these instructions with his lawyer and another local lawyer and signed them.

At trial, the appellant re-elected for a judge-alone trial, pleaded not guilty, and formally agreed to the essential elements of the offence under \textit{Criminal Code} s 655. Trial counsel indicated that the appellant was pleading not guilty, but both parties would invite the court to convict the appellant on some charges and stay proceedings on others. The trial judge went along with the recommendation and convicted the appellant.

Prior to his sentencing, the appellant completed a PSR. In the PSR, the appellant stated the complainants invented the allegations and that he could not ``fathom why' the complainants had made their allegations ."\footnote{See para 24.} The sentencing judge inquired about the comments, but ``[n]either counsel suggested that the proceedings were procedurally flawed as a result of the appellant's subsequent rejection of the complainants' accounts."\footnote{See para 24.}

RP maintained his innocence in discussions with trial counsel. He believed he would have recourse to some informal procedure whereby he could continue to contest the allegations, despite his formal agreement to the contrary. When cross-examined on his affidavit, RP testified that he did not want to continue with the trial, did not want to testify, and knew he would be found guilty by participating in the procedure.

The court also heard from trial counsel, who testified that RP had no apparent concerns about being found guilty when counsel discussed the procedure with him and that RP knew that a CSO was unlikely. He anticipated that the appellant might make the comments he did in the PSR, as the appellant had maintained his innocence in discussions with counsel.\footnote{See para 32.}

RP looked to have his conviction overturned because the procedure the trial court followed was fatally flawed and amounted to a miscarriage of justice. Specifically, RP argued that the procedure lacked the safeguards that typically attend guilty pleas, like a plea inquiry or a formal admission of the facts. The appellant argued that the Ontario Court of Appeal should reject the informal \textit{nolo contendere} plea, just as it did in \textit{DMG}.

\paragraph{Analysis\\}

The court began its analysis by reviewing Criminal Code s 606(1) and establishing that a defendant may admit any or all elements of an offence as proven through \textit{Criminal Code} s 655. The court acknowledged that the procedure in the case at bar was very similar to the one that was undertaken in \textit{DMG}, but noted several procedural protections in this case that were, taken together, sufficient to ensure the plea was voluntary:

\begin{enumerate}
    \item RP was a college-educated man who spoke English as a first language; 
    \item Trial counsel obtained detailed written instructions from the appellant;
    \item A plea inquiry was conducted on the record;
    \item The court had an ample factual basis for the plea, having heard a full day's evidence;
    \item Trial counsel clearly articulated that they did not contest the facts alleged by the Crown and that RP accepted them as proven; and
    \item The written instructions disclosed that RP knew the court would sentence him following the procedure.
\end{enumerate}

In summary, the court found that the appellant ``voluntarily participated in a procedure without statutory warrant (or prohibition, except against entry of a formal plea of \textit{nolo contendere}), well aware of the consequences (a finding of guilt and conviction), in the hope of gaining a desired sentencing disposition without having to utter an express admission of guilt of sexual offences". Furthermore, although the court noted that the \textit{nolo contendere} procedure was not prescribed or proscribed by statute, it was commonly used in Ontario. The court cited the \textit{Fegan}, \textit{Ohenhen}, and \textit{Sherret-Robinson} decisions as examples. 

\paragraph{\textit{R v Ohenhen}, 2008 ONCA 838\\}

In \textit{Ohenhen}, the Ontario Court of Appeal reviewed a case where a judge convicted the appellant of making harassing phone calls. At trial, the Crown read in evidence against Ohenhen by consent, and the appellant's lawyer did not conduct any cross-examination or call evidence in response. He was convicted but found not criminally responsible, again by consent. Ohenhen applied to either revoke his consent to the not criminally responsible verdict or have the court rule that the verdict was unreasonable. The Ontario Court of Appeal declined to comment on the trial procedure.

\paragraph{\textit{R v Sherret-Robinson}, 2009 ONCA 886\\}

In \textit{Sherret-Robinson}, the appellant faced a first-degree murder regarding her four-month-old son in 1999. Dr. Charles Smith provided expert testimony claiming that Sherret-Robinson's son died from asphyxiation from suffocation or smothering. Before trial, the Crown offered the defendant a plea to infanticide and a one-year jail sentence. The appellant maintained her not guilty plea but agreed to sufficient facts to make out the infanticide charge. Despite taking the deal, the appellant maintained that she had not harmed her baby.

In 2005, an investigation into Dr. Smith's testimony at several trials, including the appellant's, was conducted. The investigation served as a precursor to the Goudge Inquiry, exposing broader deficiencies with expert evidence in Canadian criminal law. Given the investigation's findings, the Crown and Sherret-Robinson agreed that the Court of Appeal should admit fresh evidence and overturn her conviction. The Ontario Court of Appeal agreed but again did not comment on the procedure used in the first instance.

\paragraph{Conclusion\\}

Despite apparent similarities between \textit{DMG} and this case, the court both upheld the procedure and authorized its use in Ontario. The court explicitly upheld \textit{olo contendere} as lawful, provided parties conduct them with appropriate protections. Convictions obtained through these procedures are equivalent to convictions obtained through guilty pleas, and defendants may only withdraw them where they can demonstrate that a miscarriage of justice occurred. Leave to appeal to the Supreme Court of Canada was denied, thereby tacitly approving the procedure for further use in Canada.

\subsection{Subsequent developments in Ontario}

\subsubsection{\textit{Law Society of Upper Canada v Besant}, 2014 ONLSTA 50}

Following the ruling in \textit{DMG}, as well as the related \textit{R v McKoy} decision,\footnote{See \textit{R v McKoy}, 2011 ONCA 41.} the Law Society of Upper Canada (now the Law Society of Ontario) held misconduct hearings for Charles Besant, the lawyer representing both appellants at trial. In both cases, the Ontario Court of Appeal found that Besant failed to prepare for trial adequately. Despite counsel's unpreparedness, the appellant in \textit{McKoy} proceeded to trial and was convicted. In contrast, the appellant in \textit{DMG} underwent the informal \textit{nolo contendere} procedure that the Ontario Court of Appeal would later explicitly affirm in \textit{RP}. After the tribunal found him guilty on all misconduct counts, Besant appealed its decision. When reviewing DMG's case, the tribunal found that Besant had committed misconduct by participating in the \textit{nolo contendere} procedure, which the tribunal referred to as unsuitable and misleading.

The Appeal Division of the Law Society of Upper Canada considered five appeals issues. The final issue dealt with the informal \textit{nolo contendere} procedure used at DMG's trial. The Appeal Division reviewed the then-recent \textit{RP} decision and agreed with Besant that the procedure was legally available. However, the Appeal Division maintained that Besant's unpreparedness for trial was manifest and that he had prejudiced his client's interests by failing to prepare appropriately. They therefore upheld the finding in the first instance.

As a postscript to their decision, the Appeal Division generally addressed the \textit{nolo contendere} procedure's propriety. While conceding that the procedure used in the \textit{DMG} case was lawful, the Appeal Division warned that lawyers should be ``extremely reluctant to assist" clients who wish to use the procedure. Citing wrongful convictions and the risk that such pleas could mislead the court, the Law Society of Upper Canada sent a strong signal that lawyers who engaged in this procedure did so at their own risk.

\subsubsection{R v Lo, 2020 ONCA 622}

Despite the stark warning Ontario's Law Society issued in \textit{Besant}, informal \textit{nolo contendere} pleas proliferated in the years following that decision. When the Ontario Court of Appeal heard the appeal in \textit{Lo}, the informal \textit{nolo contendere} plea had become accepted in Ontario criminal procedure. In \textit{Lo}, the appellant was a practicing psychologist who treated anxiety patients with relaxation therapy. The College of Psychologists of Ontario charged Lo with disgraceful conduct and sexual abuse, to which he pleaded guilty and no contest, respectively, in proceedings before the College. The Crown used the convictions as evidence against Lo at a subsequent criminal proceeding, leading to the appeal.

On appeal, the court, again led by Justice Watt, briefly discussed the \textit{nolo contendere} procedure used in Ontario and outlined it as being comprised of:

\begin{enumerate}
    \item A not guilty plea;
    \item An agreed statement of facts establishing the elements of the offence or offences charged;
    \item No evidence called by the defence; and
    \item A conviction is entered.
\end{enumerate}

As I will demonstrate in §3.3.5 below, other jurisdictions where defendants now enter similar pleas have already applied this formulation.

\subsubsection{R v Anderson, 2021 ONCA 333}

In \textit{Anderson}, a \textit{per Curiam} Ontario Court of Appeal affirmed that although courts and litigants frequently used the \textit{nolo contendere}  procedure in Ontario for adverse, contested pre-trial motions, its application was not limited to such cases. The appellant, in that case, had initially elected for a trial by judge and jury with a preliminary inquiry on charges that included arson, forcible confinement, and assault with a weapon.

Following the preliminary inquiry, a psychologist assessed Anderson and opined that he was not criminally responsible for the offences he was committed to stand trial. The parties agreed to re-elect to be tried by a judge alone, enter not guilty pleas, and have the evidence from the preliminary inquiry applied to the trial. The defence agreed to make no further submissions, and the Crown recommended that the appellant was not criminally responsible. The trial judge acquitted the appellant on two charges and found him not criminally responsible on the remainder. The appellant later appealed, stating that the trial judge failed to ensure the appellant understood the procedure and its consequences.

The court in Anderson expanded on its earlier decision in \textit{DMG} and clarified that plea inquiries are neither required nor authorized by statute when a defendant pleads not guilty. Although the court found that judges should conduct plea inquiries in such cases, their failure to do so does not automatically entitle an appellant to a reversal on appeal. Instead, the question that appellate courts must answer in all such cases is whether a miscarriage of justice occurred due to unfair proceedings or an unreliable verdict. Having reviewed the case, the court found ample evidence that the appellant understood the proceedings and knew what would entail if the court found him not criminally responsible. The court dismissed Anderson's appeal.

\subsubsection{R v Simpson-Fry, 2022 ONCA 108}

The decision under review in \textit{Simpson-Fry} showcased another use for the informal *nolo contendere* plea. In that case, the appellant had attempted to enter a guilty plea at first instance to sexual assault, forcible confinement, uttering threats, and a probation breach. When entering pleas, the appellant stated that he had been too intoxicated to recall the offence and therefore could not contest the Crown's evidence. The trial judge refused to accept the plea but suggested at a pre-trial meeting that the defendant could instead plead not guilty and decline to contest the evidence. The defendant opted for the latter and was subsequently convicted. Following his conviction, the Crown made a dangerous offender application, which the judge granted.

On appeal, the appellant argued that he had received ineffective assistance of counsel and that the judge misapprehended the evidence by ``rubberstamping" the Crown's uncontested evidence. The Ontario Court of Appeal affirmed that the procedure used in the first instance was appropriate and dismissed the appeal.

\subsection{Subsequent developments outside Ontario}
\subsubsection{Quebec}
\paragraph{Coderre c R, 2013 QCCA 1434\\}
In \textit{Coderre}, the appellant was charged as the directing mind of companies charged with \textit{Income Tax Act} offences. The appellant's counsel and the Crown worked out an intricate plea agreement wherein the appellant would plead not guilty but admit that the Crown had sufficient evidence to prove its case beyond a reasonable doubt. The agreement exempted the Crown from calling witnesses and acknowledged that the appellant would call no evidence. The parties filed a copy of the agreement.

Coderre repudiated the agreement on the trial date and asked the court for a trial on all counts. His lawyers withdrew, and the trial judge rejected his request to repudiate his admissions. The court convicted him, and he appealed, despite the defendant not having entered a plea.

The Quebec Court of Appeal considered two issues on review:
\begin{enumerate}
    \item Was Coderre's acknowledgement that the Crown could prove its case beyond a reasonable doubt an ``admission" under section 655?
    \item Was the procedure equivalent to a guilty plea?
\end{enumerate}

The Quebec Court of Appeal found that an on-the-record admission by defence counsel that the Crown could prove its case beyond a reasonable doubt was not an admission under s 655. Regarding both plea and admission agreements, the Quebec Court of Appeal found that criminal defendants, like the Crown, enjoy a broad right to repudiate agreements made prior to formalization. As the defendant had not yet entered a plea, the court found the agreement could not be considered final.

Citing \textit{RP}, the Quebec Court of Appeal ultimately held that, had the procedure that the appellant and his lawyers initially envisaged gone forward, they would have found that Coderre ``voluntarily participated in a procedure without statutory warrant (or prohibition, except against entry of a formal plea of \textit{nolo contendere}), well aware of the consequences (a finding of guilt and conviction), in the hope of gaining a desired sentencing disposition without having to utter an express admission of guilt." However, having found no formal agreement, the court returned the matter for trial as if the parties had not reached an agreement.

\paragraph{R c Silva, 2022 QCCS 359\\}

Midway through a trial that had already gone on for two months, the defendant opted to concede the Crown's case against him through an informal \textit{nolo contendere} plea, referred to throughout the judgement as a \textit{nolo contendere} procedure. The defendant agreed to an 18-page agreed statement of facts sufficient to establish the elements of four of the five offences charged. Citing both the Quebec Court of Appeal decision in \textit{Coderre} and the Ontario Court of Appeal decisions in \textit{DMG}, \textit{RP}, and \textit{Lo}, the court acknowledged that the defendant wished to preserve his right to appeal. Relying on the procedure outlined in \textit{Lo}, the court conducted a plea inquiry to ensure that Silva voluntarily consented to the procedure and knew it would result in a conviction.

\subsubsection{Nova Scotia}

\paragraph{R v Herritt, 2019 NSCA 92\\}

In \textit{Herritt}, the appellant appealed an ``adverse" pre-trial ruling requiring the appellant's lawyer to review cell phone evidence. While searching the appellant's cell phone, the Crown uncovered documents to the appellant from another lawyer. The Crown applied to have the appellant's trial lawyer review the documents over his objections. The trial judge ruled in the Crown's favour. Following that decision, the appellant sought to enter an agreed statement of facts to resolve the matter at first while preserving the right to appeal the ruling.

On appeal, the court expressed ``serious misgivings" about whether it should entertain the appeal. Although the court noted the informal \textit{nolo contendere} procedure used at trial (without referring to it as such), it reserved its uneasiness for \textbf{the merits of a conviction appeal} based on the ruling made. The appeal was moot as the ruling did not impact any dispositive issues at trial. That concern notwithstanding, the court nonetheless heard the appeal and ruled against the appellant on the merits.

\subsection{Classifying informal \textit{nolo contendere} pleas}

The Canadian \textit{nolo contendere} plea procedure can be classified using the four-part classification system used to identify and distinguish different formal \textit{nolo contendere} pleas from one another. 

The informal ``\textit{nolo contendere} procedure" explicitly authorized in several jurisdictions across Canada is the functional equivalent of a \textit{nolo contendere} plea. In exchange for not being required to enter a formal guilty plea, the defendant admits all of the elements of the offence. It therefore qualifies as a true no-contest plea and, as such, may be compared and contrasted with its American counterparts through the four criteria examined above.

\subsubsection{Applicability}

The first component, applicability, addresses which offences a defendant may plead \textit{nolo contendere}. This component was essential to the American \textit{nolo contendere} plea because, historically, jurists thought that the plea was only available for misdemeanour offences. However, the informal Canadian \textit{nolo contendere} plea does not share this common-law history. Instead, it exists as an incidental, statutory creation. For this reason, it is sometimes (and perhaps more accurately) referred to as a ``\textit{nolo contendere} procedure" rather than as a \textit{nolo contendere} plea proper.

The \textit{nolo contendere} procedure applies to all indictable offences. \textit{Criminal Code} s 795 makes s 655 applicable to summary conviction offences as well as hybrid and indictable ones, in that it is a section listed under \textit{Criminal Code} Part XX. Therefore, informal \textit{nolo contendere} pleas, as developed and used thus far, apply to all criminal offences. Because a defendant may admit facts against them for any offence through CC 655, the informal \textit{nolo contendere} plea may be said to apply to all criminal offences.

The most arguably useful function that the \textit{nolo contendere} procedure performs is as a conditional plea. A defendant enters a conditional plea when they self-convict on the condition that a ruling on a critical pre-trial motion does not go their way. For example, the police charge a defendant with trafficking a controlled substance that they seized from the defendant's vehicle following a controversial search. If the court admits that evidence, they will not have a defence, but if the court excludes it, the prosecutor will not have a case. A defendant in this situation may wish to enter a conditional plea, whereby they would plead guilty on the condition that the court rules the evidence admissible. A conditional plea is not a no-contest plea, as it still requires the state to prove some or all of its case. However, appellate-level decisions in Ontario, Quebec, Nova Scotia, British Columbia, and Alberta have all acknowledged that defendants may also use the \textit{nolo contendere} procedure to enter an informal conditional plea. 

Of these jurisdictions, both Ontario and Quebec have acknowledged that defendants may use the plea in cases that do not involve adverse pre-trial rulings.\footnote{See cases that say that.} That is, defendants may use these pleas in authentic no-contest situations, further confirming its universal applicability.

\subsubsection{Acceptability}

The second component, acceptability, addresses whether and under what circumstances the court may accept a \textit{nolo contendere}. Canadian case law has long established that the Crown must consent to facts admitted under \textit{Criminal Code} s 655. Because \textit{nolo contendere} pleas entered using the procedure described do so through \textit{Criminal Code} s 655, Crown consent is one of the acceptability conditions for the Canadian \textit{nolo contendere} plea procedure.

Although Canadian case history requires Crown consent for facts admitted under \textit{Criminal Code} s 655, case law is less clear about whether and to what extent trial judges are allowed to reject a jointly admitted fact under \textit{Criminal Code} s 655. Most Canadian jurisdictions hold the rule that facts admitted under \textit{Criminal Code} s 655 are unassailable.\footnote{See @2009mbca37 at paras 121 - 122 sees admissions as having "the effect of withdrawing the fact from issue and dispensing wholly with the need for proof of the fact;" @2013onca53 at para 42 lands on the side of stating that agreed facts, once admitted, dispense with the need for any further proof; @2019qccq7061 at paras 161 - 162 understands admissions made under this section to be binding on the court; @2018peca2.} However, appellate law from British Columbia suggests that trial judges may depart from jointly admitted facts, provided they notify counsel and allow them to make submissions. This procedure is similar to the procedure used to depart from a jointly recommended sentence legally.

This interpretive difference leads to unusual results when considering whether the informal \textit{nolo contendere} plea is acceptable. In jurisdictions where appellate level courts have determined that these agreed facts are binding on the court, judges must accept informal \textit{nolo contendere} pleas. In jurisdictions like British Columbia that provide judges with a limited residual discretion to reject jointly-submitted facts at trial, the judge would retain the same minimal discretion they have to reject a jointly-submitted sentence recommendation.

\subsubsection{Procedural effects}

The third component, the plea's procedural effects, considers the \textit{nolo contendere} plea's effects in the case before the court. As discussed above, the \textit{nolo contendere} procedure followed consists of:

\begin{itemize}
    \item A not guilty plea;
    \item An agreed statement of facts admitting the elements of the offence;
    \item No defence evidence; and
    \item A conviction.
\end{itemize}

Convictions following the \textit{nolo contendere} procedure are identical to those following convictions after trial. Because these pleas technically result in a conviction after a trial, the statutory plea inquiry requirements outlined in \textit{Criminal Code} s 606(1.1) do not strictly apply. However, the Ontario Court of Appeal has held that these pleas are substantially similar enough to guilty pleas that the common law plea voluntariness and comprehension requirements must still be met.\footnote{See \textit{R v RP}.}

\subsubsection{Subsequent effects}

Although one of the defining characteristics of the American \textit{nolo contendere} plea's defining characteristics is that they may not be used in some or all subsequent proceedings, no such rule exists in Canada. This is because the informal plea procedure does not share the same common law history as the American \textit{nolo contendere} plea. Whereas the American \textit{nolo contendere} plea stems from a tradition that often included this feature, the likely-unintentional statutory interaction that gave rise to the Canadian \textit{nolo contendere} plea procedure creates nothing comparable. 

However, one unique feature found in the Canadian \textit{nolo contendere} plea procedure is an apparently automatic right to appeal from these pleas. In Canada, a defendant has no common law right to appeal their conviction or sentence. Rather, criminal appeals are created by statute. Indictable appeals are enabled and governed by \textit{Criminal Code} ss 675 and 686 for indictable appeals, while summary conviction appeals are enabled and governed by \textit{Criminal Code} s 813. Appeals in Canada are limited to appeals of convictions entered ``by a trial court" and sentencing appeals. Because guilty pleas formally waive a defendant's right to be tried, they are not contemplated by these sections. However,  because the informal \textit{nolo contendere} plea procedure is technically a conviction entered by a trial court, it may be appealed by right.