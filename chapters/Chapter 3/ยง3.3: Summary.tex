\section{Summary}

The \textit{Criminal Code} was only designed to accommodate guilty and not guilty pleas. Alternative uncontested pleas like \textit{nolo contendere} are formally excluded. However, because defendants must set a trial if they cannot or will not plead guilty, and because defendants may formally admit any of the prosecutor's allegations once not guilty pleas are entered, they may effectively enter an informal plea through this \textit{nolo contendere} procedure. Since being identified, the \textit{nolo contendere} procedure has become a recognized part of Ontario criminal procedure and may be used without apparent limitations. However, these pleas are unregulated and risky and may produce unusual and unexpected results in certain circumstances. In the following section, I examine whether this unlegislated and seemingly unintended procedure should be encouraged or discouraged and propose what should be done to address the issues this plea introduces.