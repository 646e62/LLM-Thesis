\section{Statutory framework}

\textit{Criminal Code} sections 606 and 655 are central to answering whether \textit{nolo contendere} pleas are compatible with Canadian criminal law. Section 606 governs which pleas defendants may enter and how, while section 655 governs admissions a defendant may make. As I will demonstrate in the following sections, defendants may use these two provisions in tandem to enter a surprisingly wide variety of uncontested pleas.

\subsection{\textit{Criminal Code} s 606: pleas permitted and plea procedures}

\textit{Criminal Code} section 606\footnote{ Parliament first introduced this provision as section 657 in 1892:

\begin{quote}
\textbf{657. Plea; refusal to plead.} When the accused is called upon to plead, he may to plead, plead either guilty or not guilty, or such special plea as is hereinbefore provided for.

\textbf{2.} If the accused wilfully refuses to plead, or will not answer directly, the court may order the proper officer to enter a plea of not guilty. 
\end{quote}

Although the wording of the current provision differs slightly from the original version, the substance of \textit{Criminal Code} sections 606(1) and 606(2) remain analogous to \textit{Criminal Code 1892} sections 657(1) \& (2).} governs which pleas defendants may enter to criminal allegations and the procedures that courts must follow when hearing them:

\begin{quote}
    \onehalfspacing
    \textbf{Pleas permitted}
    
    \textbf{606 (1)} An accused who is called on to plead may plead guilty or not guilty, or the special pleas authorized by this Part and no others.\medskip

    \textbf{Refusal to plead}
    
    \textbf{(2)} Where an accused refuses to plead or does not answer directly, the court shall order the clerk of the court to enter a plea of not guilty.
    
\end{quote}

The statutory plea procedure changed in 2002,\footnote{See \textit{Criminal Law Amendment Act}, 2001, SC 2002, c 13.} when Parliament added \textit{Criminal Code} sections 606(1.1) \& (1.2). These sections codified the so-called ``plea inquiry" process discussed in §§ 2.2.2 and 2.2.3 above.\footnote{See §2.2.3 above. See also @1966CanLII252} Plea inquiries developed to ensure courts met this requirement. Sentencing judges were not strictly required to conduct these inquiries, but failure to do so could be considered on an application to withdraw the plea.\footnote{@1973CanLII37} Sections 606(1.1) \& (1.2) codified these principles:

\begin{quote}
    \singlespacing
    \textbf{Conditions for accepting guilty plea}
    
    \textbf{606 (1.1)} A court may accept a plea of guilty only if it is satisfied that

    \begin{quote}
        \textbf{(a)} the accused is making the plea voluntarily;
        
        \textbf{(b)} the accused understands
        \begin{quote}
            \textbf{(i)} that the plea is an admission of the essential elements of the offence,
            
            \textbf{(ii)} the nature and consequences of the plea, and
            
            \textbf{(iii)} that the court is not bound by any agreement made between the accused and the prosecutor; and   
        \end{quote}
        \textbf{(c)} the facts support the charge.
    \end{quote}

\end{quote}

In order to help ensure that defendants do so, \textit{Criminal Code} s 606(1.1) outlines a series of subjective and objective requirements that must obtain for a judge to accept a plea. Subjectively, a defendant must enter the plea voluntarily, know the consequences of doing so, know that one of those consequences is that they are formally admitting the offence, and know that the judge does not have to honour plea bargains reached between counsel. Objectively, the allegations must support the charge. Despite these requirements, the \textit{Criminal Code} does not provide courts guidelines for how to ensure they are met.\footnote{In some jurisdictions, judges regularly conduct this inquiry with every defendant who enters a guilty plea or asks their counsel to do so on the record. See \textit{R v Malaggay}, 2015 BCSC 1250 at para 21. Others may be satisfied with a signed declaration or trust that counsel has undertaken to go through the inquiry with their clients.} Subsection 606(1.2) further complicates matters by disconnecting the validity of the plea from the judicial inquiry:

\begin{quote}
    \onehalfspacing
    \textbf{Validity of plea}
    
    \textbf{606 (1.2)} The failure of the court to fully inquire whether the conditions set out in subsection (1.1) are met does not affect the validity of the plea.
\end{quote}

It is important to note that this section only implies that an otherwise valid plea will not be invalidated by failing to comply with this section, just as an otherwise invalid plea will not be validated by complying with this section. This section does not relieve judges of their obligation to conduct the inquiry,\footnote{See An Offer You Can't Refuse, or some such.} nor does it prevent a defendant from seeking a remedy or asking an appellate court to draw favourable inferences based on a judge's failure to conduct the plea inquiry.

\subsection{\textit{Criminal Code} s 655: admissions after a not guilty plea}

Before Parliament codified s 655,\footnote{Section 655 of the \textit{Criminal Code} has remained virtually unchanged since its inception as s 690 of the 1892 Criminal Code:

\begin{quote}
    \textbf{Admission may be taken on trial}
    
    \textbf{690.} Any accused person on his trial for an indictable offence, or his counselor or solicitor, may admit any fact alleged against the accused so as to dispense with proof thereof.
\end{quote}

The current text of s 655 has been in effect since at least 1955 when it appeared as s 562.} Canadian common law prohibited defendants from making formal admissions in felony cases.\footnote{@2019nsca92} Section 655 of the \textit{Criminal Code} overrides that convention:

\begin{quote}
    \singlespacing
    \textbf{Admissions at trial}
    
    \textbf{655} Where an accused is on trial for an indictable offence, he or his counsel may admit any fact alleged against him for the purpose of dispensing with proof thereof.
\end{quote}

There is a debate amongst Canadian courts over whether judges may depart from agreed facts. Some courts have found that \textit{mutually agreed} facts\footnote{For clarity, Canadian courts have specified that the Crown must allege that fact against a defendant for the defendant to admit something as a fact. A defendant may not have a fact unilaterally admitted if the Crown does not allege it. This holding coincides with the vital principle that the Crown must prove every fact it alleges and that the defendant does not admit it as accurate.} submitted under this section cannot be disturbed by a trial court,\footnote{Cite.} while others have found that judges retain discretion and may reject agreed facts in certain situations.\footnote{Cite.} As discussed in more detail in \S 3.2.6 below, how courts interpret their scope under this section impacts whether they may allow or refuse \textit{nolo contendere} plea procedures.