\section{Statutory framework}

Two statutory provisions are central to whether no-contest pleas are compatible with Canadian criminal law: \textit{Criminal Code} sections 606 and 655. Section 606 governs which pleas defendants may enter and the necessary prerequisite conditions for entering guilty pleas, and section 655 governs what admissions a defendant who has pleaded not guilty may make. As I will demonstrate below, defendants may use these two sections in tandem to enter a surprisingly wide variety of no-contest pleas, both formally and informally.

\subsection{\textit{Criminal Code} s 606: pleas permitted and plea procedures}

\textit{Criminal Code} section 606 currently governs which pleas defendants may enter to criminal allegations and the procedures that courts must follow when hearing them. Parliament first introduced this provision as section 657 in 1892:

\begin{quote}
\textbf{657. Plea; refusal to plead.} When the accused is called upon to plead, he may to plead, plead either guilty or not guilty, or such special plea as is hereinbefore provided for.

\textbf{2.} If the accused wilfully refuses to plead, or will not answer directly, the court may order the proper officer to enter a plea of not guilty. 
\end{quote}

Although the wording of the current provision differs slightly from the original version, the substance of \textit{Criminal Code} sections 606(1) and 606(2) remain analogous to \textit{Criminal Code 1892} sections 657(1) \& (2): 

\begin{quote}
    \onehalfspacing
    \textbf{Pleas permitted}
    
    \textbf{606 (1)} An accused who is called on to plead may plead guilty or not guilty, or the special pleas authorized by this Part and no others.\medskip

    \textbf{Refusal to plead}
    
    \textbf{(2)} Where an accused refuses to plead or does not answer directly, the court shall order the clerk of the court to enter a plea of not guilty.
    
\end{quote}

The most substantial change to the statutory plea procedure was introduced in 2002,\footnote{See \textit{Criminal Law Amendment Act}, 2001, SC 2002, c 13.} when Parliament codified \textit{Criminal Code} sections 606(1.1) \& (1.2). These sections codified the so-called ``plea inquiry" process that courts have long employed to ensure defendants entered their pleas knowingly and voluntarily. 

Before codification, the common law had long recognized that defendants enter guilty pleas freely and voluntarily to be valid.\footnote{@1966CanLII252} Informal plea inquiries had developed as an \textit{ad hoc} means of ensuring that courts met this requirement. Although sentencing judges were not required to conduct plea inquiries as a matter of law, a judge's failure to conduct one could be considered on an application to withdraw the plea.\footnote{@1973CanLII37} Sections 606(1.1) \& (1.2) now statutorily require judges to satisfy themselves that defendants who enter guilty pleas do so knowingly and voluntarily:

\begin{quote}
    \textbf{Conditions for accepting guilty plea}
    
    \textbf{606 (1.1)} A court may accept a plea of guilty only if it is satisfied that

    \begin{quote}
        \textbf{(a)} the accused is making the plea voluntarily;
        
        \textbf{(b)} the accused understands
        \begin{quote}
            \textbf{(i)} that the plea is an admission of the essential elements of the offence,
            
            \textbf{(ii)} the nature and consequences of the plea, and
            
            \textbf{(iii)} that the court is not bound by any agreement made between the accused and the prosecutor; and   
        \end{quote}
        \textbf{(c)} the facts support the charge.
    \end{quote}

\end{quote}

Although \textit{Criminal Code} s 606(1.1) requires judges to be satisfied that defendants enter pleas voluntarily, it does not provide guidelines for doing so. As a result, the "plea comprehension inquiry" is a highly discretionary process. In some jurisdictions, judges regularly conduct this inquiry with every defendant who enters a guilty plea or asks their counsel to do so on the record.\footnote{Manitoba is one such jurisdiction. See also \textit{R v Malaggay}, 2015 BCSC 1250 at para 21: 

\begin{quote}
    ``Reference was made before this Court to a practice that could have been followed and that may have contributed to avoiding the difficulty that arose in this case. This Court appreciates that the Provincial Court is a busy court. There are frequently pressures of time.
    
    ``Nevertheless, to ensure that there is no possible difficulty with accepting a guilty plea, reading the Information to the accused with a personal reply by him, coupled with the inquiry envisioned by s. 606 of the \textit{Criminal Code}, whether or not the accused is represented by counsel would ensure a greater certainty respecting the plea. Where taking of the plea occurs in this manner and the s. 606 inquiry is followed up by a brief factual synopsis of the circumstances relating to the counts for which the accused is pleading guilty, there can be a measure of certainty respecting the plea is a true reflection of the accused intentions."
    
\end{quote}} Others may be satisfied with a signed declaration or trust that counsel has undertaken to go through the inquiry with their clients. As long as the judge is satisfied that the defendant's plea meets the statutory requirements, it also meets the technical requirements of the section.

Subsection 606(1.2) adds further nuance to the statutory plea inquiry process. Subsection 606(1.2) does not change the common-law requirement for a free and voluntary guilty plea,\footnote{@1992CanLII2834} but it may be said to uphold the common-law principle that trial judges need not, as a matter of law, conduct an inquiry at all:

\begin{quote}
    \onehalfspacing
    \textbf{Validity of plea}
    
    \textbf{606 (1.2)} The failure of the court to fully inquire whether the conditions set out in subsection (1.1) are met does not affect the validity of the plea.
\end{quote}

It is important to note that this section only states that whether or not a judge conducted a plea inquiry has no impact on the underlying validity of the plea. An otherwise valid plea will not be invalidated by failing to comply with this section, just as an otherwise invalid plea will not be validated by complying with this section. The section does not disentitle a defendant from seeking a remedy or asking an appellate court to draw an inference based on a judge's failure to conduct the plea inquiry.

\subsection{\textit{Criminal Code} s 655: admissions after a not guilty plea}

Before the codification of s 655 and its predecessors, Canadian criminal common-law prohibited judges from allowing defendants to make formal admissions in felony cases.\footnote{@2019nsca92} Section 655 of the \textit{Criminal Code}\footnote{Section 655 of the \textit{Criminal Code} has remained virtually unchanged since its inception as s 690 of the 1892 Criminal Code:

\begin{quote}
    \textbf{Admission may be taken on trial}
    
    \textbf{690.} Any accused person on his trial for an indictable offence, or his counselor or solicitor, may admit any fact alleged against the accused so as to dispense with proof thereof.
\end{quote}

The current text of s 655 has been in effect since at least 1955 when it appeared as s 562.} overrides that convention:

\begin{quote}
    \textbf{Admissions at trial}
    
    \textbf{655} Where an accused is on trial for an indictable offence, he or his counsel may admit any fact alleged against him for the purpose of dispensing with proof thereof.
\end{quote}

There is a debate amongst Canadian courts over whether facts admitted are facts unequivocally proven. Some courts have found that \textit{mutually agreed} facts\footnote{For clarity, Canadian courts have specified that the Crown must allege that fact against a defendant in order for the defendant to admit something as a fact. A defendant may not have a fact unilaterally admitted if the Crown does not allege it. This holding coincides with the vital principle that the Crown must prove every fact it alleges and that the defendant does not admit it as accurate.} submitted under this section cannot be disturbed by a trial court. In contrast, others have found that judges retain discretion and may reject agreed facts in certain situations. The reasons why this distinction is important will become apparent when I examine \textit{nolo contendere} pleas below.