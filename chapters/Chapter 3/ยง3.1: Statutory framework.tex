\section{Statutory framework}

\textit{Criminal Code} sections 606 and 655 are central to determining whether \textit{nolo contendere} pleas are compatible with Canadian criminal law. Section 606 governs which pleas defendants may enter and how, while section 655 governs admissions a defendant may make. As I demonstrate in the following sections, defendants may use these two provisions in tandem to enter an array of uncontested pleas.

\subsection{\textit{Criminal Code} s 606: pleas permitted and plea procedures}

\textit{Criminal Code} section 606 identifies which pleas defendants may enter to criminal allegations and the procedures that courts must follow when hearing them:

\begin{quote}
    \onehalfspacing
    \textbf{Pleas permitted}
    
    \textbf{606 (1)} An accused who is called on to plead may plead guilty or not guilty, or the special pleas authorized by this Part and no others.\medskip

    \textbf{Refusal to plead}
    
    \textbf{(2)} Where an accused refuses to plead or does not answer directly, the court shall order the clerk of the court to enter a plea of not guilty.
    
\end{quote}
The statutory guilty plea changed in 2002\footnote{See \textit{Criminal Law Amendment Act}, 2001, SC 2002, c 13.} when Parliament added \textit{Criminal Code} sections 606(1.1) \& (1.2). These sections codified the plea inquiry discussed in §§ 2.2.2 and 2.2.3 above. Courts developed plea inquiries to ensure courts met this requirement. Sentencing judges were not strictly required to conduct these inquiries. However, failure to do so could be considered on an application to withdraw the plea.\footnote{See \textit{Adgey}, \textit{supra} note 50.} Sections 606(1.1) \& (1.2) codified these principles:

\break
\begin{quote}
    \singlespacing
    \textbf{Conditions for accepting guilty plea}
    
    \textbf{606 (1.1)} A court may accept a plea of guilty only if it is satisfied that

    \begin{quote}
        \textbf{(a)} the accused is making the plea voluntarily;
        
        \textbf{(b)} the accused understands
        \begin{quote}
            \textbf{(i)} that the plea is an admission of the essential elements of the offence,
            
            \textbf{(ii)} the nature and consequences of the plea, and
            
            \textbf{(iii)} that the court is not bound by any agreement made between the accused and the prosecutor; and   
        \end{quote}
        \textbf{(c)} the facts support the charge.
    \end{quote}

\end{quote}
To help ensure defendants enter guilty pleas knowingly and voluntarily, \textit{Criminal Code} s 606(1.1) outlines the subjective and objective requirements that must be met before a judge may accept a plea. Subjectively, a defendant must enter the plea voluntarily, know the consequences of doing so, know that they are formally admitting the offence, and know that the judge does not have to honour plea deals reached between counsel. Objectively, the allegations must support the charge. Despite these expectations, the \textit{Criminal Code} does not provide guidelines to ensure that judges meet them.\footnote{In some jurisdictions, judges regularly conduct this inquiry with every defendant who enters a guilty plea or asks their counsel to do so on the record. See \textit{R v Malaggay}, 2015 BCSC 1250 at para 21. Others may be satisfied with a signed declaration or trust that counsel has undertaken to go through the inquiry with their clients. See e.g. \textit{R v Fiske}, 2014 SKQB 152 at para 2.} Subsection 606(1.2) further complicates matters by disconnecting the validity of the plea from the judicial inquiry:

\begin{quote}
    \onehalfspacing
    \textbf{Validity of plea}
    
    \textbf{606 (1.2)} The failure of the court to fully inquire whether the conditions set out in subsection (1.1) are met does not affect the validity of the plea.
\end{quote}
It is important to note that this section only implies that an \textit{otherwise valid} plea will not be \textit{invalidated} by failing to comply with this section, just as an \textit{otherwise invalid} plea will not be \textit{validated} by complying with this section.\footnote{This reflects the common law approach to the plea inquiry. See e.g. \textit{Adgey}, \textit{supra} note 50 at 428 — 429.} This section does not relieve judges of their obligation to conduct the inquiry, nor does it prevent a defendant from seeking a remedy on appeal if the judge fails to conduct the plea inquiry.

\subsection{\textit{Criminal Code} s 655: admissions after a not guilty plea}

Before Parliament codified s 655, Canadian common law prohibited defendants from making formal admissions in felony cases.\footnote{See \textit{R v Herritt}, 2019 NSCA 92 at para 74.} Section 655 of the \textit{Criminal Code} overrides that convention:

\begin{quote}
    \singlespacing
    \textbf{Admissions at trial}
    
    \textbf{655} Where an accused is on trial for an indictable offence, he or his counsel may admit any fact alleged against him for the purpose of dispensing with proof thereof.
\end{quote}
There is a debate amongst Canadian courts over whether judges may depart from agreed facts. Some courts have found that \textit{mutually agreed-upon} allegations submitted under this section cannot be disturbed by a trial court, while others have found that judges retain discretion and may reject agreed facts in certain situations. As I discuss in \S 3.2.6 below, how courts interpret their scope under this section impacts whether they may allow or prevent defendants from using the \textit{nolo contendere} procedure and may require more scrupulous judicial attention in jurisdictions where criminal litigants make use of it. But before doing so, I examine several key cases that led to the procedure and outline how the informal Canadian plea procedure stacks against its formal American counterparts.