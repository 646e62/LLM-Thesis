\section{Conclusion}



CC 606 has existed in the *Criminal Code* since its inception

Guilty pleas are clearly allowed

Read alongside CC 606(1.1), it is clear that unequivocal guilty pleas are strongly encourged

Guilty pleas are the only no-contest plea formally allowed in Canada

\subsection{Formal *nolo contendere* pleas}

CC 606 precludes any formal plea other than guilty or not guilty

\subsection{Informal *nolo contendere* pleas}
Through a combination of CC 606 and CC 655, a defendant may

Plead not guilty, and thereby refrain from formally admitting guilt

Admit sufficient facts to require the court to convict them, and thereby refrain from requiring the state to prove its case

In Ontario, this is known as an informal *nolo contendere* procedure, and has been authorized by the Ontario Court of Appeal

As leave to the Supreme Court of Canada was denied in @2013onca53, it can be presumed for now that the plea procedure is tacitly approved throughout the country

Several other jurisdictions have confirmed it, while none have yet to preclude it

\subsection{Best-interest pleas}

Strictly speaking, nothing about CC 606 or CC 606(1.1) prevent defendants from entering, or judges from accepting, best-interest pleas in the style of *Alford*

To the extent that CC 606(1.1) would seem to preclude best-interest pleas, CC 606(1.2) effectively makes that inquiry optional

Strictly construed, nothing about CC 606(1.1) requires a defendant to admit that they are actually guilty of the offence in order for a judge to accept it

But like any other guilty plea, a judge is free to reject it if they believe it's not being entered voluntarily, knowingly, or unequivocally

How courts define these terms is a variable enterprise

Most courts seem disinclined to accept guilty pleas from defendants who protest their innocence, just as most defence lawyers seem disinclined to assist them in entering such pleas

Allowed *de jure*, even if effectively estopped *de facto*