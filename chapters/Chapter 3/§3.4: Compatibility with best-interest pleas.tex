\section{Compatibility with best-interest pleas}

As alluded to in 2.5.3: Best-interest pleas in Canada, whether equivocal best-interest pleas are compatible with Canadian criminal law is an open question. Although England and the United States allow these pleas without as a matter of course, the question has divided academics, judges, and practitioners in both jurisdictions, just as it does in Canada. Some read the plea voluntariness and comprehension inquiry as excluding protestations of innocence during a guilty plea.\footnote{Some lawyers feel that practitioners should never represent a client who wishes to self-convict while maintaining their innocence, regardless of whether the procedure is technically available. See the Emond book on criminal procedure. Similarly, the Law Society of Upper Canada, in the \textit{Besant} decision, recommended that practitioners avoid procedures that increase the risk of wrongful convictions. See also @governmentofcanadaInnocenceStakeNeed2019.} Others do not hold this position so strictly but would still prefer that defendants unequivocally admit fault before being satisfied that they understand that pleading guilty admits all elements of the offence. Others still find no apparent problems with defendants denying guilt after pleading. But in all cases, Canadian decisions have lacked the same enthusiasm for a defendant's ability to exercise their free will on display in cases like \textit{Alford} and \textit{Herbert}. 

\subsection{The differences between the American/English and Canadian approaches to acceptable guilty pleas}

Both \textit{Alford} and \textit{Herbert} focused on whether the appellant made a free and rational choice by pleading guilty.\footnote{See Alford. The \textit{Herbert} court had similar considerations.} The majority of the Supreme Court of the United States took the position that defendants should be able to plead guilty to crimes they are not willing or able to admit to otherwise:

\begin{quote}
    An individual accused of crime may voluntarily, knowingly, and understandingly consent to the imposition of a prison sentence even if he is unwilling or unable to admit his participation in the acts constituting the crime.
\end{quote}\footnote{@400us25 at 37}

By contrast, the approach followed in Canada focuses on whether guilty pleas are voluntary, informed, and unequivocal.\footnote{Canadian courts that have reviewed guilty pleas usually cite this approach to the dissent in @1973CanLII37. Recently, the majority in @2018scc25 endorsed these criteria in a case predominantly focused on other issues.} These distinct approaches are reflected in 

\subsection{Canadian courts generally do not allow explicit best-interest pleas}

Most Canadian cases dealing with best-interest pleas conclude that courts may not accept guilty pleas from defendants who maintain their innocence. 

\subsubsection{\textit{R v M(GO)}, 1989 CanLII 7201 (SK CA)}

In \textit{M(GO)}, the Saskatchewan Court of Appeal heard M(GO)'s conviction appeal. M(GO), a youth, pleaded guilty to setting fire to a church, and the court ordered a pre-sentence report. The Crown and M(GO)'s counsel received the report the morning of the day set for disposition, and neither had time to go through it in detail. In the report, M(GO) gave his version of events, wherein he and another friend accidentally started the fire. 

Although the report's writer recommended that the court delay sentencing and order a further psychological assessment, M(GO) instructed his lawyer to push to complete sentencing that day. The court stood the matter down to the end of the docket but agreed to proceed that day and sentenced M(GO) to 15 months of custody. Neither counsel nor the court mentioned M(GO)'s exculpatory version of events.





\subsubsection{@1995CanLII8926\\}

In \textit{SK}, the appellant pleaded guilty to several sexual offences the Crown alleged he had committed as a youth. SK initially set the allegation down for a trial, and the Crown called its case against him. After the Crown finished calling its evidence, SK's lawyer met with the Crown, SK's father, and SK. Trial counsel had reached a deal to plead guilty to four lesser offences and to recommend a non-custodial sentence jointly. SK pleaded guilty, and the court ordered a pre-sentence report.

SK maintained his innocence to the PSR writer, which the sentencing judge apprehended as a lack of remorse on SK's part. The trial judge rejected the joint submission and sentenced him to 60 days of open custody followed by nearly two years of probation. On appeal, SK filed a letter from his probation officer stating that SK's denials of guilt threatened to preclude him from programming that he was required to take, placing SK at risk of failing to comply with his probation order.

On appeal, the court found no reason to doubt that the appellant had intended to maintain his innocence, finding that his decision to plead guilty was the product of inducement. The court criticized both trial counsel and the trial judge for allowing SK to enter the guilty plea.\footnote{The Crown apparently played no role in the transaction and thus managed to escape appellate reproach.}

The court noted that the trial judge had not complied with the common law requirement for a plea inquiry and that a recent report had come out against Ontario courts accepting guilty pleas where the defendant denied guilt. The court also drew attention to the then-current edition of the Code of Professional Conduct in Ontario, noting that the Code required lawyers to prepare defendants to admit the facts of the offences they plead guilty to before entering guilty pleas.

The court set aside the guilty pleas with the following holding:
\begin{quote}
    I have no hesitation in concluding that the guilty pleas should be set aside. This case presents a graphic example of why it is essential to the plea bargaining process that the accused person is prepared to admit to the facts that support the conviction. The court should not be in the position of convicting and sentencing individuals, who fall short of admitting the facts to support the conviction unless that guilt is proved beyond a reasonable doubt. Nor should sentencing proceed on the false assumption of contrition. That did not happen here, but worse, the sentence became impossible to perform. Plea bargaining is an accepted and integral part of our criminal justice system but must be conducted with sensitivity to its vulnerabilities. A court that is misled, or allows itself to be misled, cannot serve the interests of justice.
\end{quote}

\paragraph{@2015bcsc1250\\}



\paragraph{@2019onsc1869\\}



\paragraph{@2009nltd144\\}

To the extent possible, trial courts should all want only to punish the guilty.

The appellant had a scintilla of a defence, and there was some suggestion that his lawyer pressured him into pleading guilty. Because the court had reasons to believe that innocence was at stake, these were sufficient to allow the defendant the opportunity to withdraw his plea.

\paragraph{@2006CanLII13235}

\begin{quote}
    In these applications the Court must ensure that the accused understood the nature and consequences of his or her plea of guilty. In addition, it must be satisfied that the accused will not suffer any unfairness if he or she is not allowed to withdraw their plea. What I mean by unfairness is that unless a Court concludes that an accused person is using an application to withdraw a plea of guilty for a nefarious purpose, it will be a rare case that such an application should be refused. All accused have the right to a trial and the administration of justice will usually suffer little harm by allowing an accused person to change his or her plea of guilty. The Court cannot conduct a sentence hearing when an accused person that has not had a trial insists that they are innocent. To do so would cause the administration of justice to be brought into serious disrepute. Therefore, I conclude that the Court's emphasis in these applications must be a substantive rather than a procedural one. Form should not triumph over substance. Rather than concentrating on an accused person's comprehension or on whether or not they were represented by counsel, the Court should consider what harm if any will occur if a trial is allowed to proceed. Obviously the adoption of such a test will result in virtually all such applications being granted.
\end{quote}

\subsection{Best interests permitted}

Not many cases will back this up, but there are a few. Recent trends in Ontario and across the country 

\subsubsection{@2000CanLII5725}

In \textit{R v Hector}, the defendant faced three first-degree murder counts. In exchange for pleading guilty, the Crown agreed not to pursue a dangerous offender application against him, to drop all the charges against his wife, and to allow them an hour to visit each other after sentencing. Hector pleaded guilty but later sought to withdraw those pleas on appeal.

On appeal, Hector argued that because he had protested his innocence on several occasions to his counsel and never admitted to his lawyer that he was guilty, his lawyer should not have allowed him to plead guilty. He argued that his pleas were not voluntary and that he had received ineffective assistance from his lawyer. Reviewing Hector's case, the Ontario Court of Appeal found that these issues had no impact on the only substantial question: namely, whether Hector's guilty plea was voluntary, informed, and unequivocal. Reviewing the affidavit evidence from both Hector and his trial counsel, the Ontario Court of Appeal noted:

\begin{quote}
    What is apparent from both affidavits is that the appellant, while professing his innocence at the outset, did not have any defence on the merits. He seemed to think that it was up to his lawyer to provide one. More important, when his counsel presented him with disclosure material from the Crown, he was unable to offer any explanation as to the most damaging evidence that the Crown was prepared to lead. First and foremost, he could give no explanation as to how the gun which forensic testing unequivocally connected to all three murders was found in a duffel bag in his truck, along with the shells of the bullets that matched those which were found in the victims. Indeed, while the appellant never admitted that he had the gun, he never denied it either. Second, when asked to respond to the ``will say" statement of his brother that he would testify that the appellant had described to him in detail the location of the gunshots to one of the murder victims, the appellant said to his lawyer: "You figure it out". Third, he refused to authorize his counsel to retain a private investigator, at the expense of Legal Aid, to investigate possible defences. Fourth, the only alibi that he offered was that of his wife and this alibi was the subject of the obstruction of justice charge against her.
\end{quote}\footnote{\textit{Hector}at para 10.}

Although there is no Canadian \textit{Alford} equivalent, the \textit{Hector} decision comes closest. Although \textit{Hector} does not take the same pro-self-determination stance that \textit{Alford} does, there are strong parallels between the two decisions:

\begin{itemize}
    \item Alford pleaded guilty to avoid the death penalty, whereas Hector pleaded guilty to avoid a dangerous offender designation;
    \item Both were accused of murders and had extensive criminal records;
    \item Both protested their innocence privately;
    \item Hector protested his innocence just prior to entering a plea, while Alford protested his innocence on record; and
    \item Neither had any apparent defence to the charges, and Hector actively impeded his lawyer's attempts to develop one.
\end{itemize}

Because the Ontario Court of Appeal found no evidence that Hector's pleas were involuntary, uninformed, or equivocal, it dismissed his appeal. 

\subsubsection{@2013onca53}

Although the \textit{RP} decision dealt primarily with the validity of the informal \textit{nolo contendere} procedure, it also touched upon best-interest pleas and their propriety. 

After entering his informal *nolo contendere* plea, RP participated in a pre-sentence report, where he protested his innocence to the pre-sentence writer.

\begin{quote}
    Over six months later, the trial judge heard sentencing submissions. Counsel filed a pre-sentence report in which the appellant challenged the complainants' allegations as a "fantasized and dramatized story" and told the pre-sentence reporter that he couldn't "fathom why" the complainants had made their allegations.
\end{quote}

During sentencing submissions by Crown counsel, the trial judge queried the effect of the appellant's comments in the pre-sentence report challenging the complainants' allegations. Neither counsel suggested the proceedings were flawed due to the appellant's subsequent rejection of the complainants' accounts.

Trial counsel for the appellant sought a conditional sentence. Crown counsel submitted that a fit sentence was a term of imprisonment in a federal penitentiary of four to five years, coupled with several ancillary orders to which the appellant took no objection.\footnote{\textit{RP} at paras 23 - 25.}

The trial judge questioned what effects RP's comments might have on the validity of the proceedings.

Neither counsel seemed to suggest that they caused any problems.

The trial judge proceeded as if they didn't

The Ontario Court of Appeal did not comment on these protestations of innocence.

Again, while not overtly encouraging or justifying best-interest pleas, it appears as though the Ontario Court of Appeal is once again giving them its tacit approval.

\subsubsection{@2015skqb101}

\begin{quote}
    Courts must be cognizant of the high incidence of recanting complainants in domestic violence cases and that contact between accused and complainant often continues following the incident, even when the accused is in fact guilty and is convicted. While guarding against wrongful convictions and a miscarriage of justice, we must also take care not to encourage a process which could lead to pressure on victims to recant following a conviction. Provided the plea was voluntary, unequivocal, and informed, perhaps special emphasis should be put on the principle of finality of proceedings in such cases.
    
    I agree with the decisions in Porter and McLeod that a recanting complainant is irrelevant to whether J.A. made his plea voluntarily, unequivocally and informed of the nature of the charges, the legal effect of the plea and the consequences of the plea. (see Leonard) There are many reasons why witnesses recant their previous testimony. An enquiry into the reasons why this complainant may have recanted is not helpful to understanding J.A.'s situation at the time he entered his guilty plea. For this reason, I advised counsel at the hearing that the affidavits of S.C., K.C. and D.C. would be disregarded in their entirety as would paras. 4, 9 and 10 of J.A.'s affidavit.
\end{quote}

\subsection{Technical compliance with \textit{Criminal Code} s 606(1.1)}

Despite this largely adverse reaction to best-interest pleas in Canada, there is nothing at law that would seemingly prevent them from being entered.

A careful (and correct) reading of CC 606(1.1) makes it clear that defendants can legally enter these pleas.

CC 606(1.1) also makes it clear that

The court has the ultimate discretion to accept or reject any plea entered under the section.

Before accepting any plea under the section, the court must be satisfied that the facts support the charge.

These are, perhaps unsurprisingly, the same criteria SCOTUS laid out for **Alford** pleas in @400us25.

\subsection{The equivocal nature of unequivocal pleas}



Reviewing the above decisions, it becomes clear that there is a fundamental disagreement between the courts on whether and to what extent a defendant may protest their innocence and still have their guilty plea accepted by the court.

Alternatively, in many cases, whether and to what extent a defendant may protest their innocence and still have their conviction upheld by an appellate court.

What makes this disagreement confounding is the fact that the courts all appear to be relying on the same authorities and implementing the same tests

Although this test has survived through the continued application of a dissenting decision, a majority of the Supreme Court of Canada in @2018scc25 recently adopted these criteria, with the dissent dividing on other issues.

2022 BCCA 144:

\begin{quote}
    [52]      An unequivocal plea is a plea that is not “qualified, modified or uncertain”: R. v. T.(R.) (1992), 1992 CanLII 2834 (ON CA), 10 O.R. (3d) 514 at 521, 17 C.R. (4th) 247 (C.A.); R. v. Singh, 2014 BCCA 373 at para. 43. In R. v. McNabb (1971), 1971 CanLII 1231 (SK CA), 4 C.C.C. (2d) 316, 1971 CarswellSask 121 (C.A.), for example, the appeal court set aside a plea that was recorded as “Guilty with an explanation”: at para. 5. This was not an “unmistakable and unambiguous” admission to the offence: at para. 9.

    [53]      An unequivocal plea is one that is “certain with respect to the acknowledgement of the essential facts of the crime charged”: R. v. Leonard, 2007 SKCA 128 at para. 17. It is also a plea that, in my view, is certain with respect to an acknowledgement of the essential legal elements of the crime: R. v. Hunt, 2021 ABCA 49 at para. 46. In R. v. Gold, 2004 BCCA 179, for example, this Court set aside a guilty plea to second degree murder on the ground that it was neither unequivocal nor informed. The appellant in that case did not make an unequivocal admission of the specific intent required for murder: at para. 14.
\end{quote}

\subsubsection{Withdrawing guilty pleas} 

The attitudes and comportment of trial judges are essential when determining how the courts treat best-interest pleas in practice.

But an occasionally overlooked source is appellate law.

Arguably, if best-interest pleas may be entered at trial and are not reversible on appeal, the law authorizes them.

Looking at how courts process applications to withdraw guilty pleas, it is clear that a best-interest plea, knowingly and voluntarily entered, is difficult to withdraw in a subsequent appeal process.

What does it mean for a plea to be unequivocal?

The "unequivocal" criterion arguably precludes best-interest pleas.

@1992CanLII2834

However, the object that the word "unequivocal" modifies is not always the same in different cases.

In some cases, it appears that the defendant must unequivocally admit their guilt.

In others, it appears to mean that they must unequivocally intend to enter the plea they did
