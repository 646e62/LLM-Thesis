\section{Compatibility with \textit{nolo contendere} pleas}

\subsection{\textit{Nolo contendere} pleas are formally excluded}

\textit{Criminal Code} s 606(1) explicitly states that the court may only accept guilty, not guilty, and the special double jeopardy pleas. This formally excludes all other pleas from Canadian criminal law, including \textit{nolo contendere}. 

\subsection{\textit{Nolo contendere} pleas may be constructively entered through \textit{Criminal Code} s 655}

Although \textit{Criminal Code} s 606(1) formally forbids \textit{nolo contendere} pleas, defendants may self-convict without admitting guilt through an informal ``\textit{nolo contendere} procedure" by (1) pleading not guilty and admitting the offence's elements at trial; or (2) refusing to enter a plea and admitting the offence's elements at the trial. Criminal pleas are authorized and guided by \textit{Criminal Code} s 606 and admissions by \textit{Crminal Code} s 655. Although both these sections have existed in some form or another since the first \textit{Criminal Code 1892},\footnote{Formerly \textit{Criminal Code 1892} s 562} courts were slow to explore how these sections operated and to recognize how they interact with one another.

\subsubsection{\textit{Castellani v R}, 1969 CanLII 57}

Until relatively recently, Canadian courts had not settled \textit{Criminal Code} section 655's most basic operating principles. The \textit{R v Castellani}\footnote{\textit{Castellani v R}, 1969 CanLII 57 (SCC), [1970] SCR 310 [\textit{Castellani}].} decision was a significant development in this area. In \textit{Castellani}, the appellant sought to make admissions at his trial through \textit{Criminal Code} s 562 (now s 655). The prosecutor disagreed with one of his propositions, and the trial judge ruled the appellant could not admit it as proven. He was convicted and appealed. The British Columbia Court of Appeal ruled that the trial judge erred by disallowing the admission but dismissed his appeal. The Supreme Court of Canada found that Castellani was not entitled to admit evidence as proven unilaterally. The court established that a defendant may only admit to allegations the prosecutor had actually alleged against them.

\subsubsection{\textit{R v Cooper}, 1977 CanLII 11 (SCC), [1978] 1 SCR 860}

Once the Supreme Court of Canada established \textit{Criminal Code} section 655's scope, parties quickly explored its potential applications. In \textit{R v Cooper}, the Supreme Court of Canada heard an appeal from a conviction obtained entirely from an agreed statement of facts between the appellant and the prosecutor. The trial hinged on whether these agreed facts made out the elements of the offence. The Supreme Court of Canada was divided on whether the rule in \textit{Hodge's Case}, rather than the procedure adopted at trial. The majority and dissent remarked that the procedure the parties adopted at trial was unusual, but neither found it caused any error. This tacit approval proved significant nearly 30 years later when the Ontario Court of Appeal adapted this procedure to create the informal Canadian \textit{nolo contendere} plea in \textit{R v DMG}.\footnote{\textit{R v DMG}, 2011 ONCA 343 [\textit{DMG}].}

\subsubsection{\textit{R v Fegan}, 1993 CanLII 8607 (ON CA)}

Before Canadian courts created an \textit{ad hoc} substitute for \textit{nolo contendere} pleas, they outlined a similar procedure for conditional pleas.\footnote{See the definition of a conditional plea in \S 1.5 above.} In \textit{Fegan}, the appellant pleaded guilty to having made harassing and threatening telephone calls after losing a pre-trial evidentiary motion. His lawyer mistakenly believed that Fegan would be allowed to appeal if he pleaded guilty and advised him to do so. The Ontario Court of Appeal rightly noted that the appellant could not and dismissed his appeal, but noted that the appellant could have entered a conditional plea's equivalent through agreements made under \textit{Criminal Code} section 655.

\subsection{\textit{Criminal Code} s 655 and the \textit{nolo contendere} procedure}

Nearly two decades later, the Ontario Court of Appeal decided \textit{DMG}, a case centred on a plea procedure much like the one described in \textit{Fegan}. DMG claimed on appeal that his counsel was ineffective and that his plea was involuntary. Writing for the court, Justice Watt agreed with DMG in both respects, and for a time following, practitioners and lower courts interpreted this as the court reproaching the \textit{nolo contendere} procedure showcased in \textit{DMG}.

A few years later, the Ontario Court of Appeal, again led by Justice Watt, clarified this issue in \textit{R v RP}. There, the court specified that the \textit{nolo contendere} plea procedure used in \textit{DMG} was compliant with the \textit{Criminal Code} and that nothing prevented the defendants from using it properly. When entered knowingly and voluntarily, such pleas are lawful. Since \textit{RP}, the \textit{nolo contendere} procedure has become a recognized practice in Ontario and has been used in at least two other provinces.

\subsubsection{\textit{R v DMG}, 2011 ONCA 343}

In \textit{R v DMG}, the appellant faced sexual assault and interference charges involving the same complainant. He denied the allegations and told his lawyer he wanted to plead not guilty but did not want to force the complainant to testify. He pleaded not guilty. The complainant had made a statement, and the prosecutors told trial counsel they sought to admit it. DMG also made a statement, which trial counsel admitted was made voluntarily. On appeal, the Ontario Court of Appeal learned that trial counsel did not review either statement at any point prior to the hearing. Five weeks before trial, DMG's lawyer offered to resolve for a six-month CSO and 18 months probation. A week later, the prosecutor countered with a 15 - 18 month jail sentence, not contingent on a joint submission. 

At trial, DMG's lawyer provided him with the prosecutor's resolution offer, and DMG agreed to resolve on those terms. During this meeting, the appellant told trial counsel that he was not guilty and wanted to testify. When the court convened, trial counsel told the court that the appellant would plead not guilty plea but not dispute the allegations. The Crown read in the allegations, and the trial judge entered a conviction and remanded the appellant for sentencing. DMG did not testify, and the trial judge did not conduct a plea voluntariness or comprehension inquiry with DMG, who appealed.

The Ontario Court of Appeal likened the procedure adopted at trial to a \textit{nolo contendere} plea. It noted that the Federal Rules of Criminal Procedure in the United States required plea voluntariness and comprehension inquiries for guilty and \textit{nolo contendere} pleas. The Ontario Court of Appeal concluded that the procedure was legal but found that it suffered in its execution in two key ways. 

Firstly, the Ontario Court of Appeal found that the prosecutor could not adduce evidence simply by reading allegations out in open court. Doing so did not constitute a formal admission under \textit{Criminal Code} s 655, and was not admissible evidence. Secondly, the court found that because DMG self-convicted, the trial judge should have conducted a plea comprehension and voluntariness inquiry to ensure DMG knew what the procedure entailed. The court accepted that DMG did not know the consequences of embarking on the \textit{nolo contendere} procedure. 

The Ontario Court of Appeal hesitantly held that the procedure adopted at trial caused a miscarriage of justice due to these irregularities and granted the appeal. Although trial counsel's conduct fell short, the court noted several times that ``[n]o statutory provision or common law principle prohibits a procedure similar to what was followed here after an accused has entered a plea of not guilty." The flaw was not in the plea procedure but in how trial counsel employed it.\footnote{See \textit{DMG} at para 51.} Subsequent confusion on this point became the subject of \textit{R v RP} a year and a half later.

\subsubsection{\textit{R v RP}, 2013 ONCA 53}

Following \textit{DMG}, the \textit{nolo contendere} procedure's legality and propriety were uncertain.\footnote{See @2014onlsta50} \textit{RP} clarified many of these uncertainties. In \textit{RP}, the appellant faced 19 historical sexual offence allegations involving four family members. The first complainant testified on the first day of trial. Trial counsel believed that she testified well and told RP as much. He noted that RP looked unwell. Although he maintained his innocence when speaking with his lawyer, RP signed written instructions the next day outlining the allegations he expected the remaining witnesses would confirm and acknowledged he had no reasonable response to their evidence. RP agreed he would not contest the allegations and would be convicted. He acknowledged that doing so would result in a conviction. RP pleaded not guilty but formally admitted the offence through \textit{Criminal Code} s 655. RP and the prosecutor both invited the court to convict him.

Before his sentencing, RP completed a pre-sentence report. While doing so, he told the pre-sentence report writer that the complainants invented the allegations and that he could not understand why they had done so.\footnote{See para 24. The sentencing judge inquired about the comments, but ``[n]either counsel suggested that the proceedings were procedurally flawed due to the appellant's subsequent rejection of the complainants' accounts."} However, when cross-examined on his affidavit at the appeal, he testified that he did not want to continue with the trial, did not want to take the stand, and knew he would be found guilty by participating in the procedure.

On appeal, RP argued that the procedure the trial court followed was a fatally flawed miscarriage of justice that the Ontario Court of Appeal should denounce, just as it did in \textit{DMG}. The court reviewed \textit{Criminal Code} sections 606(1) and 655 and affirmed that defendants may admit any or all elements of an offence as proven through \textit{Criminal Code} s 655. The court acknowledged that the procedure in the case at bar was very similar to the one undertaken in \textit{DMG} but distinguished the two based on the comparative levels of procedural protections each trial court employed.\footnote{In RP the court noted that the appellant was college-educated, spoke English as a first language, provided detailed written instructions to his lawyer, and knew the consequences of undergoing the procedure. The court also noted that the trial judge conducted a detailed plea inquiry and had an ample factual basis for the charges.} Unlike the appellant in \textit{DMG}, RP ``voluntarily participated in a procedure without statutory warrant (or prohibition, except against entry of a formal plea of \textit{nolo contendere}), well aware of the consequences (a finding of guilt and conviction), in the hope of gaining a desired sentencing disposition without having to utter an express admission of guilt of sexual offences."\footnote{} Leave to appeal to the Supreme Court of Canada was denied.

\subsection{Subsequent developments in Ontario}

\subsubsection{\textit{Law Society of Upper Canada v Besant}, 2014 ONLSTA 50}

Following \textit{DMG}, the Law Society of Upper Canada (now the Law Society of Ontario) held misconduct hearings for Charles Besant, DMG's trial lawyer. The tribunal found that Besant failed to prepare for trial adequately but also found that he committed misconduct by participating in the \textit{nolo contendere} procedure, which it referred to as unsuitable and misleading. Besant appealed on five grounds, the last of which dealt with the propriety of the \textit{nolo contendere} procedure.

The Law Society of Upper Canada's Appeal Division agreed with Besant that the procedure was legally available. However, as a postscript to their decision, the Appeal Division addressed the \textit{nolo contendere} procedure's propriety. While conceding that the procedure used in the \textit{DMG} case was lawful, the Appeal Division warned that lawyers should be ``extremely reluctant to assist" clients who wish to use the procedure.\footnote{} Citing wrongful convictions and the risk that such pleas could mislead the court,\footnote{} the Law Society of Upper Canada sent a strong signal that lawyers who engaged in this procedure did so at their own risk.\footnote{} The Appeal Division dismissed Besant's appeals on other grounds, having found that he failed to prepare for trial adequately and thus prejudiced his client's interests.

\subsubsection{\textit{R v Lo}, 2020 ONCA 622}

Despite the stark warning Ontario's Law Society issued in \textit{Besant}, the \textit{nolo contendere} plea procedures proliferated in Ontario in the years following. By the time the Ontario Court of Appeal decided \textit{Lo}, the \textit{nolo contendere} plea procedure had become entrenched in Ontario criminal procedure. In \textit{Lo}, the appellant was a practicing psychologist who pleaded guilty and no contest to disgraceful conduct and sexual abuse allegations in proceedings before the College of Psychologists in Ontario. The Crown used the convictions as evidence against Lo at a subsequent criminal proceeding, leading to the appeal.

On appeal, the court, again led by Justice Watt, briefly discussed the \textit{nolo contendere} procedure used in Ontario and outlined it as being comprised of: (1) a not guilty plea; (2) an agreed statement of facts establishing the elements of the offences charged; (3) no evidence called by the defence; and (4) a conviction. These procedural steps are broad enough to encompass both the conditional plea that \textit{Fegan} outlined and the \textit{nolo contendere} procedures used in \textit{DMG} and \textit{RP}. Although \textit{Lo} is a recent decision, subsequent decisions from the Ontario Court of Appeal have started to provide additional special use cases for the procedure.

\subsubsection{\textit{R v Anderson}, 2021 ONCA 333}

In \textit{Anderson}, the appellant elected for a trial with a preliminary inquiry on charges that included arson, forcible confinement, and assault with a weapon. Following the preliminary inquiry, a psychologist assessed Anderson as not criminally responsible. The parties agreed to re-elect to be tried by a judge alone, enter not guilty pleas, and have the evidence from the preliminary inquiry applied to the trial. The trial judge acquitted Anderson on two charges and found him not criminally responsible on the remainder. Anderson appealed, arguing that he had not understood the procedure and its consequences. The court found ample evidence that the appellant understood the proceedings and knew what would entail if the court found him not criminally responsible. The court dismissed Anderson's appeal. 

\textit{Anderson} clarified that plea inquiries are neither required nor authorized by statute when a defendant pleads not guilty. Although judges should conduct plea inquiries in such cases, their failure to do so, absent more, does not entitle an appellant to a reversal.\footnote{} \textit{Anderson} did not elaborate on why the appellant sought to self-convict after the expert concluded he was not criminally responsible, but there is no suggestion that he was motivated by an adverse pre-trial evidentiary ruling. This decision also confirmed that although courts and litigants frequently used the \textit{nolo contendere} procedure in Ontario for the contested pre-trial motions contemplated by \textit{Fegan}, its application is not limited to such cases.

\subsubsection{\textit{R v Simpson-Fry}, 2022 ONCA 108}

The recent Ontario Court of Appeal decision in \textit{R v Simpson-Fry}\footnote{\textit{R v Simpson-Fry}, 2022 ONCA 108 [\textit{Simpson-Fry}].} provides another example of a special use case for the \textit{nolo contendere} \textit{procedure}: namely, for those defendants who are hesitant to plead \textit{guilty} because they cannot remember an offence but are willing to self-convict. In \textit{Simpson-Fry}, the appellant attempted to plead guilty to serious violent and sexual offences but stated that he was too intoxicated to recall any details about the offence. The trial judge refused to accept the plea but suggested that the defendant plead not guilty and not contest the evidence. The defendant did so. Following his conviction, the Crown made a dangerous offender application, which the judge granted. On appeal, the appellant argued that he had received ineffective assistance of counsel and that the judge misapprehended the evidence by ``rubberstamping" the Crown's uncontested evidence. The Ontario Court of Appeal affirmed that the procedure used in the first instance was appropriate and dismissed the appeal.

\subsection{Developments outside Ontario}
\subsubsection{\textit{Coderre c R}, 2013 QCCA 1434}
In \textit{Coderre}, the appellant's counsel and the prosecutor agreed that the appellant would plead not guilty but admit that the prosecutor could prove its case beyond a reasonable doubt. The agreement exempted the prosecutor from calling witnesses and acknowledged that the appellant would call no evidence. Coderre repudiated the agreement on the trial date and asked the court for a trial on all counts. The trial judge rejected his request to repudiate his admissions and convicted him. 

The Quebec Court of Appeal found defence counsel's informal ``admission" was not an admission under \textit{Criminal Code} s 655. Criminal defendants, like the prosecutor, enjoy a broad right to repudiate agreements made before formalization. Because Coderre had not yet formally pleaded, the agreement was not final. Citing \textit{RP}, the court added that, had the initially-envisaged\textit{nolo contendere} procedure gone forward, they would have found that Coderre knowingly and voluntarily participated in a legal procedure.

\subsubsection{\textit{R c Silva}, 2022 QCCS 359}

In \textit{Silva}, the accused was a defendant who had set his multiple murder charges down for a lengthy multi-month trial. Two months into the trial, Silva opted to concede the prosecutor's case against him through the \textit{nolo contendere} procedure after losing motions to stay proceedings and exclude key evidence. Silva maintained his not guilty plea but formally agreed that the prosecutor had discharged its burden. Citing \textit{Coderre}, \textit{DMG}, and \textit{RP}, and relying on the procedure outlined in \textit{Lo}, the court conducted a plea inquiry and allowed Silva to self-convict while preserving his right to appeal his unsuccessful pre-trial motions.

\subsubsection{\textit{R v Herritt}, 2019 NSCA 92}

In \textit{Herritt}, the appellant appealed a ruling requiring his lawyer to review cell phone evidence. In response, the appellant entered an agreed statement of facts to self-convict while preserving his right to appeal. The Nova Scotia Court of Appeal was reluctant to hear the matter. The court noted that the \textit{nolo contendere} procedure was used at trial but only expressed misgivings for the appeal's broader lack of merit. The appeal was ultimately dismissed.

\subsection{Classifying informal \textit{nolo contendere} pleas}

The Canadian \textit{nolo contendere} plea procedure can be classified using Dreschler's classification system.\footnote{See §2.3 above.} The Canadian \textit{nolo contendere} procedure is a functional equivalent of a \textit{nolo contendere} plea in many ways but unique in others. Analyzing the plea using Dreschler's criteria highlights these similarities and differences.

\subsubsection{Applicability}

The first criterion, applicability, addresses which offences a defendant may plead \textit{nolo contendere}. By virtue of \textit{Criminal Code} s 795, \textit{Criminal Code} s 655 applies to all offences. Therefore, the \textit{nolo contendere} procedure applies to all criminal offences.

\subsubsection{Acceptability}

The second criterion, acceptability, addresses whether and when the court may accept a \textit{nolo contendere}. Because the \textit{nolo contendere} procedure admits allegations as proven using \textit{Criminal Code} s 655, and because these admissions require prosecutorial consent, it follows that the Canadian \textit{nolo contendere} procedure requires prosecutorial consent. However, case law is ambiguous about whether trial judges may reject allegations admitted under \textit{Criminal Code} s 655. This interpretive difference impacts the acceptability criterion. Where courts acknowledge that they are bound by these agreements, judges must accept informal \textit{nolo contendere} pleas. Where judges believe they retain some discretion to reject jointly-submitted facts at trial, they may exercise that discretion to reject a \textit{nolo contendere} procedure.

The majority opinion is that allegations admitted by prosecutors and defendants bind the court.\footnote{See e.g. Manitoba: @2009mbca37 at paras 121 - 122; Ontario:  @2013onca53 at para 42; Quebec: @2019qccq7061 at paras 161 - 162; Prince Edward Island: @2018peca2.} This position follows from the rule that formally admitted allegations dispense with any need to call evidence on those issues\footnote{See \textit{R v Miljevic}, 2010 ABCA 115 at para 18.} and cannot be disturbed by evidence called on those issues at trial\footnote{See \textit{R v Handy}, 2002 SCC 56 (CanLII), [2002] 2 SCR 908 at para 74.} and the rule that these admissions, once entered, may only be withdrawn in exceptional circumstances.\footnote{See \textit{R v Fertal}, 1993 ABCA 277 at paras 7 - 9.} However, some jurisdictions have suggested that trial judges may have some residual discretion to depart from formal admissions provided they allow counsel to make submissions on the issue before doing so.\footnote{See \textit{R v Duong}, 2019 BCCA 299 at para 52.}

\subsubsection{Procedural effects}

The third component, the plea's procedural effects, considers the \textit{nolo contendere} plea's effects in the case before the court. Because the \textit{nolo contendere} procedure result in a conviction after a trial, the statutory plea inquiry requirements outlined in \textit{Criminal Code} s 606(1.1) do not apply.\footnote{See \textit{Lo}.} The Ontario Court of Appeal has held that these pleas are substantially similar enough to guilty pleas that the common law plea voluntariness and comprehension inquiries are still required,\footnote{See \textit{RP}.} but has also declined to allow appeals simply because this inquiry was not conducted.\footnote{See \textit{Anderson}.} 

\subsubsection{Subsequent effects}

The final criterion, subsequent effects, examines what impact the plea has on the defendant after it has been entered. Although one of the defining characteristics of the American \textit{nolo contendere} plea is its subsequent inadmissibility, no such rule exists in Canada. However, this does not mean that the Canadian plea procedure is without its quirks and potential benefits. One unique feature of the Canadian \textit{nolo contendere} plea procedure is an automatic right to appeal these pleas. This results from the fact that self-convictions entered through the \textit{nolo contendere} procedure qualify as convictions ``by a trial court." Appeals in Canada are limited to trial court convictions and sentencing appeals. Because guilty pleas formally waive a defendant's right to be tried, they are not convictions by a trial court. However, because the informal \textit{nolo contendere} plea procedure is a conviction entered by a trial court, it may be appealed by right. The implications of this are discussed in \S\S 4.4.5 and 4.4.6 below.