\section{Compatibility with \textit{nolo contendere} pleas}

\subsection{\textit{Nolo contendere} pleas are formally excluded}

\textit{Criminal Code} s 606(1) explicitly states that the court may only accept guilty, not guilty, and the special double jeopardy pleas. This section formally excludes all other pleas from Canadian criminal law, including \textit{nolo contendere}. 

\subsection{\textit{Nolo contendere} pleas may be constructively entered through \textit{Criminal Code} s 655}

Although \textit{Criminal Code} s 606(1) formally forbids \textit{nolo contendere} pleas, defendants may self-convict without admitting guilt through an informal ``\textit{nolo contendere} procedure" by (1) pleading not guilty and admitting the offence's elements at trial; or (2) refusing to enter a plea and admitting the offence's elements at the trial. Criminal pleas are authorized and guided by \textit{Criminal Code} s 606 and admissions by \textit{Crminal Code} s 655. Although both these sections have existed in some form or another since the first \textit{Criminal Code, 1892}\footnote{\textit{Criminal Code} s 606(1) originally existed as \textit{Criminal Code, 1892} s 657:

\begin{quote}
\textbf{657. Plea; refusal to plead.} When the accused is called upon to plead, he may to plead, plead either guilty or not guilty, or such special plea as is hereinbefore provided for.

\textbf{2.} If the accused wilfully refuses to plead, or will not answer directly, the court may order the proper officer to enter a plea of not guilty. 
\end{quote}

Similarly, \textit{Criminal Code} s 655 has existed in largely the same form since Parliament codified it as \textit{Criminal Code, 1892} s 690:
\begin{quote}
    \textbf{Admission may be taken on trial}
    
    \textbf{690.} Any accused person on his trial for an indictable offence, or his counselor or solicitor, may admit any fact alleged against the accused so as to dispense with proof thereof.
\end{quote}
} and functioned much the same as they do today, courts were slow to explore how these sections operated and to recognize how they interact.

\subsubsection{\textit{Castellani v R}, 1969 CanLII 57}

Until relatively recently, Canadian courts had not settled \textit{Criminal Code} section 655's most basic operating principles. The \textit{R v Castellani}\footnote{\textit{Castellani v R}, 1969 CanLII 57 (SCC), [1970] SCR 310 [\textit{Castellani}].} decision was a significant development in this area. In \textit{Castellani}, the appellant sought to make admissions at his trial through \textit{Criminal Code} s 562 (now s 655). The prosecutor disagreed with one of his propositions, and the trial judge ruled that the appellant could not admit it as proven. He was convicted and appealed. The British Columbia Court of Appeal ruled that the trial judge erred by disallowing the admission but dismissed his appeal. The Supreme Court of Canada found that a defendant may not have a fact unilaterally admitted if the Crown does not allege it. This holding coincides with the vital principle that the Crown must prove every fact it alleges and that the defendant does not admit it as accurate. It established that defendants may only formally admit to the prosecutor's allegations against them. 

\subsubsection{\textit{R v Cooper}, 1977 CanLII 11 (SCC), [1978] 1 SCR 860}

Once the Supreme Court of Canada delimited \textit{Criminal Code} section 655's scope, prosecutors and defendants quickly began to explore the outer limits of its potential applications. In \textit{R v Cooper}, the Supreme Court of Canada heard an appeal from a conviction obtained entirely from an agreed statement of facts between the appellant and the prosecutor. The trial hinged on whether these agreed facts made out the elements of the offence. The Supreme Court of Canada was divided on whether the rule in \textit{Hodge's Case},\footnote{\textit{R v Hodge}, 1838 CanLII 1 (FOREP).} rather than the procedure adopted at trial. The majority and dissent remarked that the procedure was unusual, but neither found it caused any error. This tacit approval proved significant nearly 30 years later when the Ontario Court of Appeal adapted this procedure to create the Canadian \textit{nolo contendere} procedure in \textit{R v DMG}.\footnote{\textit{R v DMG}, 2011 ONCA 343 [\textit{DMG}].}

\subsubsection{\textit{R v Fegan}, 1993 CanLII 8607 (ON CA)}

Before Canadian courts created an \textit{ad hoc} substitute for \textit{nolo contendere} pleas, they crafted a similar replacement for conditional pleas. In \textit{R v Fegan},\footnote{\textit{R v Fegan}, 1993 CanLII 8607 (ON CA) [\textit{Fegan}].} the appellant made an unsuccessful motion to exclude evidence at trial. After losing that motion, Fegan pleaded guilty to making harassing and threatening telephone calls, as his lawyer mistakenly believed that Fegan would preserve his right to appeal if he did so. The Ontario Court of Appeal rightly noted that Fegan could not and dismissed his appeal. However, they noted that Fegan could have entered a conditional plea's equivalent by not calling any evidence after the prosecutor put in its case. Similarly, because \textit{Criminal Code} s 655 allows defendants to admit facts against them, the court suggested that future defendants willing to plead guilty may agree to facts capable of sustaining a conviction. This suggestion advanced \textit{Criminal Code} s 655's utility and was critical to the \textit{nolo contendere} procedure's development in the subsequent years.

\subsection{\textit{Criminal Code} s 655 and the \textit{nolo contendere} procedure}

Nearly two decades after \textit{Fegan}, the Ontario Court of Appeal decided \textit{DMG}, an appeal from an unusual conviction where the defendant pleaded not guilty but called no evidence and admitted all the elements of the offences charged. On appeal, DMG, the appellant, argued that his counsel was ineffective, his plea was involuntary, and the procedure used at trial was flawed. Writing for the court, Justice Watt agreed with DMG in all respects, likening the procedure to a \textit{nolo contendere} plea. For a time following, practitioners and lower courts interpreted this as the court reproaching the \textit{nolo contendere} procedure showcased in \textit{DMG}.

A few years later, the Ontario Court of Appeal, again led by Justice Watt, clarified this issue in \textit{R v RP}. There, the court specified that the \textit{nolo contendere} plea procedure used in \textit{DMG} was compliant with the \textit{Criminal Code} and that nothing prevented the defendants from using it properly. When entered knowingly and voluntarily, such pleas are lawful. Since \textit{RP}, the \textit{nolo contendere} procedure has become a recognized practice in Ontario and has been acknowledged by appellate courts in at least two other provinces.

\subsubsection{\textit{R v DMG}, 2011 ONCA 343}

In \textit{R v DMG},\footnote{See \textit{DMG}, \textit{supra} note 107.} the appellant faced several sexual assault and interference charges involving the same underaged complainant. He denied the allegations and pleaded not guilty but did not want to force the complainant to testify. The complainant had made a statement, and the prosecutors told DMG's lawyer they wanted to admit it. DMG also made a statement, which his lawyer admitted was voluntary. On appeal, the Ontario Court of Appeal learned that DMG's lawyer did not review either statement before the trial. Five weeks before trial, DMG's lawyer offered to resolve for a six-month CSO and 18 months probation. A week later, the prosecutor countered with a 15 - 18 month jail sentence, not contingent on a joint submission. 

At trial, DMG's lawyer gave him the prosecutor's resolution offer, and DMG agreed to resolve on those terms. During this meeting, DMG told trial counsel that he was not guilty and wanted to testify. When the court convened, trial counsel told the court that the appellant would plead not guilty but also would not dispute the allegations. The Crown read the allegations into the record, and the trial judge convicted DMG. DMG did not testify, and the trial judge did not conduct a plea voluntariness or comprehension inquiry with DMG, who appealed.

The Ontario Court of Appeal likened the procedure adopted at trial to a \textit{nolo contendere} plea.\footnote{See \textit{ibid} at para 60.} It noted that the Federal Rules of Criminal Procedure in the United States required plea inquiries for guilty and \textit{nolo contendere} pleas alike.\footnote{See \textit{ibid} at para 45.} The court concluded that the procedure was legal and complied with the \textit{Criminal Code}, but suffered in its execution in two key ways. First, the Ontario Court of Appeal found that the prosecutor could not adduce evidence simply by reading allegations in open court. Doing so did not constitute a formal admission under \textit{Criminal Code} s 655 and was not otherwise admissible evidence.\footnote{See \textit{ibid} at paras 55 — 70.} Second, the court found that because DMG self-convicted, the trial judge should have conducted a plea inquiry to ensure DMG knew what the procedure entailed.\footnote{See \textit{ibid} at paras 59 — 61.} The court accepted that DMG did not know the consequences of embarking on the \textit{nolo contendere} procedure,\footnote{See \textit{ibid} at para 67.} hesitantly held that the procedure adopted at trial caused a miscarriage of justice due to these irregularities, and granted the appeal. Although trial counsel's conduct fell short, the court emphasized that ``[n]o statutory provision or common law principle prohibits a procedure similar to what was followed here after an accused has entered a plea of not guilty." The flaw was not in the plea procedure but in how trial counsel employed it.\footnote{See \textit{DMG}, \textit{supra} note 107 at para 51.} Subsequent confusion on this point became the subject of the \textit{RP} decision a year and a half later.

\subsubsection{\textit{R v RP}, 2013 ONCA 53}

Following \textit{DMG}, the \textit{nolo contendere} procedure's legality and propriety were uncertain.\footnote{See \textit{Law Society of Upper Canada v Besant}, 2014 ONLSTA 50 [\textit{Bestant}], discussed below.} \textit{RP}\footnote{See \textit{R v RP}, 2013 ONCA 53 [\textit{RP}].} clarified many of these uncertainties. In \textit{RP}, the appellant faced 19 historical sexual offence allegations involving four family members. The first complainant testified on the first day of the trial. Trial counsel believed that she testified well and told RP as much. RP suffered from several health problems during his normal course, and trial counsel noted that RP looked especially unwell after the first trial date and feared RP might not make it through the trial. Although he maintained his innocence when speaking with his lawyer, RP signed written instructions the next day outlining the allegations he expected the remaining witnesses would confirm and acknowledged he had no reasonable response to their evidence. RP agreed he would not contest the allegations and acknowledged he would be convicted. RP pleaded not guilty but formally admitted the offence through \textit{Criminal Code} s 655, and both he and the prosecutor invited the court to convict him.

Before his sentencing, RP completed a pre-sentence report. While doing so, he told the pre-sentence report writer that the complainants invented the allegations and that he could not understand why they had done so.\footnote{See \textit{ibid} at para 24. The sentencing judge inquired about the comments, but ``[n]either counsel suggested that the proceedings were procedurally flawed due to the appellant's subsequent rejection of the complainants' accounts."} However, when cross-examined on his affidavit at the appeal, he confirmed that he did not want to continue with the trial, did not want to take the stand, and knew he would be found guilty by participating in the procedure he, his counsel, and the prosecutor had all agreed upon.

On appeal, RP argued that the procedure the trial court followed was a fatally flawed miscarriage of justice that the Ontario Court of Appeal should denounce, just as it did in \textit{DMG}. The court reviewed \textit{Criminal Code} sections 606(1) and 655 and affirmed that defendants may admit any or all elements of an offence as proven through \textit{Criminal Code} s 655. The court acknowledged that the procedure in the case at bar was similar to the one undertaken in \textit{DMG}. However, it distinguished the two based on the relative levels of procedural protections each trial court employed.\footnote{Notably, the court in RP noted that the appellant was college-educated, spoke English as a first language, provided detailed written instructions to his lawyer, and knew the consequences of undergoing the procedure. See \textit{ibid} at para 29.} Unlike the appellant in \textit{DMG}, RP had 

\begin{quote}
    \singlespacing
    voluntarily participated in a procedure without statutory warrant (or prohibition, except against entry of a formal plea of \textit{nolo contendere}), well aware of the consequences (a finding of guilt and conviction), in the hope of gaining a desired sentencing disposition without having to utter an express admission of guilt of sexual offences.\footnote{See \textit{ibid} at para 65.}
\end{quote} Having found that the \textit{nolo contendere} procedure used was legal and that RP knew his rights, the Ontario Court of Appeal dismissed his appeal. Leave to appeal to the Supreme Court of Canada was subsequently denied.

\subsection{Subsequent developments in Ontario}

\subsubsection{\textit{Law Society of Upper Canada v Besant}, 2014 ONLSTA 50}

Following \textit{DMG}, the Law Society of Upper Canada (now the Law Society of Ontario) held misconduct hearings for Charles Besant, DMG's trial lawyer.\footnote{See \textit{Law Society of Upper Canada v David Charles Besant}, 2013 ONLSHP 76.} The tribunal found that Besant failed to prepare for trial adequately but also found that he committed misconduct by participating in the \textit{nolo contendere} procedure, calling it unsuitable and misleading.\footnote{See \textit{ibid} at paras 127 — 138.} Besant appealed on five grounds, the last of which dealt with the propriety of the \textit{nolo contendere} procedure.

On appeal,\footnote{See \textit{Besant}, \textit{supra} note 116.} the Law Society of Upper Canada's Appeal Division heard and dismissed Besant's appeal, having found that he failed to prepare for trial adequately and thus prejudiced his client's interests. Despite this, the Appeal Division agreed with Besant that the \textit{nolo contendere} procedure used at trial was legally \textit{available}. However, as a postscript to their decision, the Appeal Division addressed the \textit{nolo contendere} procedure's \textit{propriety}. Although lawful, the Appeal Division warned that lawyers should be ``extremely reluctant to assist" clients who wish to use the procedure.\footnote{See \textit{ibid} at para 130.} Citing wrongful convictions and the risk that such pleas could mislead the court, the Law Society of Upper Canada sent a strong signal that lawyers who engaged in this procedure did so at their own risk.\footnote{See \textit{ibid} at para 131.} 

\subsubsection{\textit{R v Lo}, 2020 ONCA 622}

Despite the stark warning Ontario's Law Society issued in \textit{Besant}, the \textit{nolo contendere} plea procedures proliferated there in the years following. By the time the Ontario Court of Appeal decided \textit{Lo},\footnote{See \textit{Lo}, 2020 ONCA 622 [\textit{Lo}].} the \textit{nolo contendere} plea procedure had become entrenched in Ontario's criminal procedure. The \textit{Lo} decision allowed the Ontario Court of Appeal to parenthetically outline the informal plea and note its place in Ontario's criminal procedure. Lo, the appellant, was a practicing psychologist who pleaded guilty and no contest to disgraceful conduct and sexual abuse allegations in proceedings before the College of Psychologists in Ontario. The Crown used the convictions as evidence against Lo at a subsequent criminal proceeding, leading to the appeal.

On appeal, the court, again led by Justice Watt, briefly discussed the \textit{nolo contendere} procedure used in Ontario to contrast it with the formal ``no contest" plea that Lo entered at first instance in his disciplinary proceedings. The court in \textit{Lo} identified four components that together comprise the \textit{nolo contendere} procedure: (1) a not guilty plea; (2) an agreed statement of facts establishing the elements of the offences charged; (3) no evidence called by the defence; and (4) a conviction.\footnote{See \textit{Lo} at para 75.} These procedural steps are broad enough to encompass both the conditional plea procedure that \textit{Fegan} outlined and the \textit{nolo contendere} procedures used in \textit{DMG} and \textit{RP}. Although \textit{Lo} is a recent decision, subsequent decisions from the Ontario Court of Appeal have continued to provide additional special use cases for the procedure.

\subsubsection{\textit{R v Anderson}, 2021 ONCA 333}

In \textit{Anderson},\footnote{See \textit{R v Anderson}, 2021 ONCA 333 [\textit{Anderson}].} the appellant elected for a trial with a preliminary inquiry on arson, forcible confinement, and assault with a weapon charges. Following the preliminary inquiry, a psychologist assessed Anderson as not criminally responsible. The parties agreed to re-elect to a judge-alone trial, enter not guilty pleas, and have the evidence from the preliminary inquiry applied to the trial. The trial judge acquitted Anderson on two charges and found him not criminally responsible on the remainder. Anderson appealed, arguing that he had not understood the procedure and its consequences. The court found ample evidence that Anderson understood the proceedings, knew the consequences of being found not criminally responsible and dismissed his appeal. 

\textit{Anderson} clarified that plea inquiries are neither required nor authorized by statute when a defendant pleads not guilty.\footnote{See \textit{ibid} at paras 40 — 54.} Although judges should conduct plea inquiries in such cases, their failure to do so, absent more, does not entitle an appellant to a reversal.\footnote{See \textit{ibid} at para 50.} \textit{Anderson} did not elaborate on why the appellant sought to self-convict after the expert concluded he was not criminally responsible. However, there is no suggestion that he was motivated by an adverse pre-trial evidentiary ruling. This decision also confirmed that although courts and litigants frequently used the \textit{nolo contendere} procedure in Ontario for the contested pre-trial motions contemplated by \textit{Fegan}, its application is not limited to such cases.

\subsubsection{\textit{R v Simpson-Fry}, 2022 ONCA 108}

The recent Ontario Court of Appeal decision in \textit{R v Simpson-Fry}\footnote{\textit{R v Simpson-Fry}, 2022 ONCA 108 [\textit{Simpson-Fry}].} provides another example of a special use case for the \textit{nolo contendere} \textit{procedure}: namely, for those defendants who are \textit{hesitant to plead guilty} because they cannot remember an offence but are \textit{willing to self-convict}. In \textit{Simpson-Fry}, the appellant attempted to plead guilty to serious violent and sexual offences but told the judge that he was too intoxicated to recall the offence. The trial judge refused to accept a guilty plea but suggested that the Simpson-Fry plead not guilty and not contest the evidence. Simpson-Fry did so. Following his conviction, the Crown made a dangerous offender application, which the judge granted. On appeal, Simpson-Fry argued that he had received ineffective assistance of counsel and that the judge misapprehended the evidence by ``rubberstamping" the Crown's uncontested evidence.\footnote{See \textit{ibid} at para 5.} The Ontario Court of Appeal affirmed that the procedure used in the first instance was appropriate and dismissed his appeal.\footnote{See \textit{ibid} at para 12.}

\subsection{Developments outside Ontario}
\subsubsection{\textit{Coderre c R}, 2013 QCCA 1434}
In \textit{Coderre},\footnote{See \textit{Coderre c R}, 2013 QCCA 1434 [\textit{Coderre}].} the appellant was charged with several \textit{Income Tax Act} violations, and sought to resolve the allegations through a plea deal. Coderre's counsel and the prosecutor agreed that he would plead not guilty but admit that the prosecutor could prove its case beyond a reasonable doubt. The agreement exempted the prosecutor from calling witnesses and acknowledged that the appellant would call no evidence. Coderre repudiated the agreement on the trial date and asked for a trial on all counts. The trial judge rejected his request and convicted him. The Quebec Court of Appeal found defence counsel's nascent agreement was not an admission under \textit{Criminal Code} s 655 and allowed the appeal. Both criminal defendants and prosecutors enjoy a broad right to repudiate agreements before the court finalizes them. Because Coderre had not yet formally pleaded, the agreement was not final.\footnote{See \textit{ibid} at para 41 — 47.} Quoting \textit{RP}, the court added that, had the initially-envisaged \textit{nolo contendere} procedure gone forward, they would have found that Coderre knowingly and voluntarily participated in a legal procedure.\footnote{See \textit{ibid} at para 36.}

\subsubsection{\textit{R c Silva}, 2022 QCCS 359}

In \textit{Silva},\footnote{\textit{R c Silva}, 2022 QCCS 359 [\textit{Silva}].} the defendant was charged with multiple murders and had set the charges down for a lengthy multi-month trial. Two months into the trial, Silva opted to concede the prosecutor's case against him through the \textit{nolo contendere} procedure after losing motions to stay proceedings and exclude key evidence. Silva maintained his not guilty plea but formally agreed that the prosecution had discharged its burden. Citing \textit{Coderre}, \textit{DMG}, and \textit{RP}, and relying on the procedure outlined in \textit{Lo}, the court conducted a plea inquiry.\footnote{See \textit{ibid} at para 40.} It allowed Silva to self-convict while preserving his right to appeal his unsuccessful pre-trial motions.\footnote{See \textit{ibid} at para 38.}

\subsubsection{\textit{R v Herritt}, 2019 NSCA 92}

In \textit{Herritt},\footnote{See \textit{Herritt}, \textit{supra} note 102.} the appellant appealed a ruling requiring his lawyer to review cell phone evidence to determine whether it included privileged information. Trial counsel was adamantly opposed to complying with this order. In response, Herritt entered an agreed statement of facts to self-convict while preserving his right to appeal this interlocutory ruling. The Nova Scotia Court of Appeal was reluctant to hear the matter. The court noted that the \textit{nolo contendere} procedure was used at trial but only expressed misgivings for the appeal's broader lack of merit. The court ultimately dismissed the appeal, finding no merit to trial counsel's underlying objection to the trial court's procedural ruling.\footnote{See \textit{ibid} at paras 6, 86f.}

\subsection{Classifying informal \textit{nolo contendere} pleas}

The Canadian \textit{nolo contendere} plea procedure can be classified using Drechsler's classification system discussed in §2.3 above. The Canadian \textit{nolo contendere} procedure is functionally equivalent to a \textit{nolo contendere} plea in many ways but unique in others. Analyzing the plea using Drechsler's criteria highlights these similarities and differences.

\subsubsection{Applicability}

The first criterion, applicability, addresses which offences a defendant may plead \textit{nolo contendere}. By virtue of \textit{Criminal Code} s 795, \textit{Criminal Code} s 655 applies to all offences.\footnote{\textit{Criminal Code}, \textit{supra} note 2 Part XXVII covers the special rules for summary conviction offences. \textit{Criminal Code} s 795 applies all of the rules from Part XX, where s 655 is found, to Part XXVII.} Because the \textit{Criminal Code} allows defendants charged with any offence to admit any allegation the prosecutors make against them, the \textit{nolo contendere} procedure thus applies to all criminal offences.

\subsubsection{Acceptability}

The second criterion, acceptability, addresses whether and when the court may accept a \textit{nolo contendere}. Because the \textit{nolo contendere} procedure admits allegations through \textit{Criminal Code} s 655, and because admissions under this section require prosecutorial consent, it follows that the Canadian \textit{nolo contendere} procedure requires prosecutorial consent. However, case law is ambiguous about whether trial judges may reject allegations admitted under \textit{Criminal Code} s 655. This interpretive difference impacts the acceptability criterion, as judges \textit{must} accept informal \textit{nolo contendere} pleas where courts acknowledge that these agreements bind them. 

The dominant view is that allegations admitted by prosecutors and defendants bind the court.\footnote{See e.g. Ontario: \textit{RP}, \textit{supra} note 117 at para 42; Manitoba: \textit{R v Korski (CT)}, 2009 MBCA 37 at paras 121 — 122; Quebec: \textit{R v Michael}, 2019 QCCQ 7061 at para 162; Prince Edward Island: \textit{R v Brookfield Gardens Inc}, 2018 PECA 2.} This position follows from the rule that formally admitted allegations dispense with any need to call evidence on those issues,\footnote{See e.g. \textit{R v Handy}, 2002 SCC 56 at para 74 [\textit{Handy}]; \textit{Lo}, \textit{supra} note 126 at para 69.} cannot be disturbed by evidence called on those issues at trial, and that these admissions, once entered, may only be withdrawn in exceptional circumstances.\footnote{See \textit{R v Fertal}, 1993 ABCA 277 at paras 7 — 9.} Where judges retain some discretion to reject jointly-submitted facts at trial, they may exercise that discretion to reject a \textit{nolo contendere} procedure. But a nonstandard approach suggests that trial judges may have some residual discretion to depart from formal admissions provided they allow counsel to make submissions on the issue before doing so.\footnote{See e.g. \textit{R v Duong} 2019 BCCA 299 at para 52.} Under this view, trial judges retain discretion to reject admissions under \textit{Criminal Code} s 655 if they notify counsel and provide reasons for their decision. As a result, defendants appearing in courts that take this view require permission from both the prosecutor and the court to engage the \textit{nolo contendere} procedure.

\subsubsection{Procedural effects}

The third criterion, procedural effects, considers the plea's effects in the case before the court. Defendants utilizing the \textit{nolo contendere} procedure formally plead not guilty but admit sufficient facts to self-convict, per the procedure outlined in \textit{Lo}.\footnote{See \textit{Lo}, \textit{supra} note 126 at para 75.} Formally, this results in a conviction after a trial. Because the \textit{nolo contendere} procedure result in a conviction after a trial, the statutory plea inquiry requirements outlined in \textit{Criminal Code} s 606(1.1) do not apply.\footnote{See \textit{Anderson}, \textit{supra} note 128 at para 51.} The Ontario Court of Appeal has held that these pleas are substantially similar enough to guilty pleas that the common law plea voluntariness and comprehension inquiries are still required,\footnote{See \textit{RP}, \textit{supra} note 117 at 57 — 66.} but has also declined to overturn convictions entered through the \textit{nolo contendere} procedure simply because a judge did not conduct this inquiry.\footnote{See \textit{Anderson}, \textit{supra} note 128.} Because the plea is otherwise unregulated, its procedural effects are indistinguishable from a conviction after trial.

\subsubsection{Subsequent effects}

The final criterion, subsequent effects, examines what impact the plea has on the defendant after they have entered it. Although one of the defining characteristics of the American \textit{nolo contendere} plea is its subsequent inadmissibility, no such rule exists in Canada. However, the unusual statutory interactions that allow this plea procedure give rise to their unique quirks and potential benefits. Chief amongst these is the automatic right to appeal that defendants have when entering these pleas. I discuss this windfall and its implications at the end of the next chapter, where I discuss whether Canada should formally recognize and allow \textit{nolo contendere} pleas.