\chapter{ Are no-contest pleas compatible with Canadian criminal law?}
Having outlined the fundamental features of a no-contest plea and reviewed two examples of the plea, I next ask whether these pleas are compatible with Canadian criminal law. I first identify the relevant statutory provisions in the \textit{Criminal Code} and then analyze each plea against that statutory framework.

The first part of this analysis focuses on the statutes authorizing criminal pleas in Canada. Substantive criminal law in Canada has been primarily statutory since the \textit{Criminal Code of Canada 1892} and entirely statutory since Parliament abolished all common-law offences in 1955.\footnote{See online: \textless https://www.ppsc-sppc.gc.ca/eng/pub/fpsd-sfpg/fps-sfp/fpd/ch01.html\textgreater} Procedural criminal law is similarly governed by the \textit{Criminal Code}. Because the \textit{Criminal Code} is central to understanding what pleas and procedures Canadian criminal law permits and prohibits, the compatibility analysis must begin by identifying the relevant parts of that legislation.

With the statutory framework outlined, the second part of this chapter examines whether guilty pleas are compatible with Canadian criminal law. As Canadian criminal law's central text, the \textit{Criminal Code} provides the pleas permitted by law. Because the \textit{Criminal Code} explicitly authorizes them, guilty pleas are compatible with Canadian criminal law, albeit with certain restrictions and no implicit right to enter one. 

The third and final section examines whether \textit{nolo contendere} pleas are also compatible with Canadian criminal law. This analysis demonstrates that while Canadian criminal law formally forbids \textit{nolo contendere} pleas, they are partially but functionally compatible with Canadian criminal law.